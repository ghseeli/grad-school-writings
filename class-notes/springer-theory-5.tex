\documentclass[springer-theory-notes.tex]{subfiles}

\begin{document}
\section{Geometric Analysis of \(H(Z)\)}
Previously, for the Springer variety \(Z\), we have shown that \(H(Z) \isom \Q(\W)\) and \(\C
\otimes_\Q H(Z) = \C[\W]\) is a semisimple Lie algebra. By the
Artin-Wedderburn theorem, we have \[
  \C \otimes_\Q H(Z) = \bigoplus_\alpha \End_\C(E_\alpha)
\]
where \(\{E_\alpha\}\) is a complete set of \(\g\) irreducible
representations. Our goal for this section is to understand this
decomposition. \\

Recall that, for the Springer resolution \[
  \begin{tikzcd}
    \tilde{\cN} \ar[d, "\mu"] \\
    \cN
  \end{tikzcd} \hspace{1in} Z = \tilde{\cN} \times_\cN \tilde{\cN}
\]
and, for \(Y \subset \cN\), \[
  Z_y = \mu^{-1}(Y) \times_Y \mu^{-1}(Y) \subset \tilde{\cN} \times \tilde{\cN}
\]
Now, \(Z_Y \circ Z = Z_Y = Z \circ Z_Y\).
\begin{prop}
  \(H(Z_Y)\) has a \(H(Z)\)-bimodule structure.
\end{prop}
\end{document}
%%% Local Variables:
%%% TeX-master: "springer-theory-notes.tex"
%%% End: