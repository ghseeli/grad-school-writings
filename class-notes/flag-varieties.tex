\documentclass[11pt,leqno,oneside]{amsbook}
\usepackage{tikz}
\usetikzlibrary{cd}
\usetikzlibrary{decorations.markings}
\usepackage{bbm}
\usepackage{ytableau}
\usepackage{todonotes}
\usepackage{./notes}
\usepackage{../ReAdTeX/readtex-core}
\usepackage{../ReAdTeX/readtex-dangerous}
\usepackage{../ReAdTeX/readtex-abstract-algebra}
\usepackage{../ReAdTeX/readtex-topology}

\newcommand{\bbk}{\mathbbm{k}}
\newcommand{\Class}{\operatorname{Class}}
\newcommand{\Res}{\operatorname{Res}}
\newcommand{\Ind}{\operatorname{Ind}}
\newcommand{\bs}{\textbackslash}
\newcommand{\partitionof}{\vdash}
\newcommand{\T}{\mathsf{T}} % Tableau
\renewcommand{\S}{\mathsf{S}}
\newcommand{\sh}{\operatorname{shape}}
\newcommand{\grdim}{\boldsymbol{\dim}}

\newcommand{\dominatedby}{\trianglelefteq}
\newcommand{\dominates}{\trianglerighteq}
\newcommand{\lexicoleq}{\leq}
\newcommand{\lexicogeq}{\geq}
\newcommand{\covers}{\gtrdot}
\newcommand{\coveredby}{\lessdot}

\newcommand{\cF}{\mathcal{F}}
\renewcommand{\q}{\mathfrak{q}}
\renewcommand{\P}{\mathbb{P}}
\DeclareMathOperator{\Supp}{Supp}
\DeclareMathOperator{\ch}{ch} 
\DeclareMathOperator{\gr}{gr}
\DeclareMathOperator{\Hilb}{Hilb}
\DeclareMathOperator{\wt}{wt}
\DeclareMathOperator{\wts}{\mathcal{W}}
\renewcommand{\vec}[1]{\mathbf{#1}}
\numberwithin{thm}{section}

\newcommand{\Fl}{\mathcal{F\ell}}

\title[Flag Varieties]{Flag Varieties \\ Notes
  from a reading course in Spring 2019}
\author{George H. Seelinger}
\date{Spring 2010}
\begin{document}
\maketitle
\section{Introduction (presented by Weiqiang Wang)}
\section{A Quick Introduction to Vector Bundles (not presented)}
This section will list some facts about vector bundles without
proof. For a more comprehensive treatment, see references such as
\cite{bott-tu} and \cite{milnor-stasheff}. The reader may skip this
section and refer back to it as necessary. 
\begin{defn}\label{defn-of-vector-bundle}
  A rank \(n\) \de{(complex) vector bundle} \(\xi\), consists of
  \begin{enumerate}
  \item a topological space \(E\) called \de{the total space},
  \item a topological space \(B\) called \de{the base space},
  \item and a continuous projection map \(\rho \from E \to B\) such that
    \begin{enumerate}[label=(\roman*)]
    \item for all \(b \in B\), the preimage \(\rho^{-1}(b)\), which is
      called the \de{fiber over \(b\)} is a vector space, and
    \item (local triviality) for all \(b \in B\), there exists a neighborhood \(U \subset
      B\) with \(b \in U\) and a homeomorphism \(h \from U \times \C^n
      \to \rho^{-1}(U)\) (called a \de{local trivialization}) such
      that, for all \(x \in U\), \(h|_{\{x\} 
        \times \C^n} \from \C^n \to \rho^{-1}(x)\) is a vector space
      isomorphism. \[
        \begin{tikzcd}
          U \times \C^n \ar[rd, "\pi_1"] \ar[rr, "h"]& & \rho^{-1}(U)
          \ar[ld, "\rho"]\\
          & U &
        \end{tikzcd}
      \]
    \end{enumerate}
  \end{enumerate}
\end{defn}
\begin{rmk}
  \begin{enumerate}
  \item One can just as easily define a \de{real vector bundle} by
    simply replacing \(\C\) with \(\R\) everywhere in the definition.
  \item One can define a vector bundle where the \(\rho^{-1}(b) \isom
    \C^n\) and \(\rho^{-1}(b') = \C^m\) for \(n \neq m\) if \(b\) and
    \(b'\) are in different connected components of \(B\). However,
    for simplicity, we will usually keep the rank of every fiber the
    same.
  \item If \(b,b' \in B\) are in the same connected component, then
    the local triviality condition forces that \(\rho^{-1}(b) \isom
    \C^n \isom \rho^{-1}(b')\).
  \end{enumerate}
\end{rmk}
\begin{example}\label{first-examples}
  \begin{enumerate}
  \item Perhaps the most familiar example of a (smooth) real vector
    bundle is that of a ``tangent bundle'' of a manifold. That is,
    given an \(n\)-dimensional smooth manifold \(M\), we have a
    \(2n\)-dimensional manifold \(TM := 
    \{(x,v) \in M \times \R^n \st v \in T_xM\}\) where \(T_xM\) is the
    tangent space of \(x\) and map \(\pi_1 \from TM \to M\) given by
    \(\pi_1(x,v) = x\). Then, we have tangent bundle \(\tau(M)\) given
    by \[
      \begin{tikzcd}
        TM \ar[d, "\pi_1"] \\
        M
      \end{tikzcd}
    \]
    with fiber \(\pi_1^{-1}(x) = T_xM \isom \R^n\). Similarly, the
    ``normal bundle'' \(\nu(M)\) is a vector bundle. 
  \item Given a space \(X\), we have the \de{trivial bundle}
    \(\epsilon_n\) given by \[
      \begin{tikzcd}
        X \times \C^n \ar[d, "\pi_1"]\\
        X
      \end{tikzcd}
    \]
    with \(\pi_1^{-1}(x) = \C^n\). If \(X\) is a single point, then
    the trivial bundle is essentially just the vector space
    \(\C^n\). In this way, we see that vector bundles serve as a
    generalization of vector spaces. A particular example of a real
    trivial bundle would be \[
      \begin{tikzcd}
        S^1 \times \R \ar[d, "\pi_1"]\\
        S^1
      \end{tikzcd}
    \]
    where \(S^1 \times \R\) is a cylinder. 
  \item Given a M\"{o}bius strip \(M\), we have a real vector bundle \[
      \begin{tikzcd}
        M \ar[d,"\rho"] \\
        S^1
      \end{tikzcd}
    \]
    where \(\rho \from M \to S^1\) sends a point in the M\"{o}bius
    strip to the circle embedded along the ``middle''. For instance,
    every point on the dashed line is sent by \(\rho\) to the point
    where the 
    dashed line intersects the red line in the picture below.
    \[
      \begin{tikzpicture}[
        decoration={
          markings,
          mark=at position 0.5 with {\arrow{>}}}
        ] 
        \draw[postaction=decorate] (0,0) -- (2,0);
        \draw (2,0) -- (2,-2);
        \draw[postaction=decorate] (2,-2) -- (0,-2);
        \draw (0,0) -- (0,-2);
        \draw[red] (1,0) -- (1,-2);
        \draw[dashed] (0,-0.5) -- (2,-0.5);
      \end{tikzpicture}
    \]
    The fiber is given by \(\rho^{-1}(x) = (0,1)\) which is
    homeomorphic to \(\R\). Furthermore, this example is not equivalent to
    the trivial bundle \(S^1 \times \R \to S^1\). Informally, this can
    be seen by noticing that one can pick a consistent choice of
    orientation on the trivial bundle but not on this bundle.
  \item Given a space \(B = \CP^n\), the points \(x \in \CP^n\) are
    lines. Then, we define the \de{tautological line
    bundle} \(\gamma_n^1\) to be given by total space \(E = \{(x,v) \in \CP^n \times
    \C^{n+1} \st v \text{ lies on }x\}\) and map \(\rho(x,v) = x\). Then,
    \(\rho^{-1}(x) = \{x\} \times \C v \isom \C\).
  \item If we do the tautological line bundle construction over
    \(\RP^1 \isom S^1\), we recover the M\"{o}bius strip example
    above. To see this in detail, see \cite{milnor-stasheff}*{pp 16--17}
  \end{enumerate}
\end{example}
\begin{defn}
  A \de{line bundle} is a rank \(1\) vector bundle.
\end{defn}
\begin{prop}\label{new-bundles-from-old}
  Given a rank \(n\) vector bundle \(\xi\) with base space \(B\) and
  total space 
  \(E\) and \(\rho \from E \to B\), we can construct new vector
  bundles  
  in a variety of ways.
  \begin{enumerate}
  \item Given some map \(f \from B' \to B\), we can construct the
    \de{pullback bundle} \(f^*(\xi)\) over \(B'\) by setting \(f^*(E)
    = \{(x',v) \in B' \times E \st f(x) = \rho(e)\}\) and \(\rho'
    \from f^*(E) \to B\) to be given by \(\rho'(x',e) = x'\) so that
    \((\rho')^{-1}(x') = \rho^{-1}(f(x'))\). This is a
    specific version of a pullback square: 
    \[
      \begin{tikzcd}
        f^*(E) \ar[d,"\rho'"] \ar[r,"\tilde{f}"] & E \ar[d,"\rho"]\\
        B' \ar[r, "f"] & B
      \end{tikzcd}
    \]

  \item We can take the \de{Whitney sum} of two vector bundles over
    \(B\) induced by taking the pullback bundle of the diagonal
    inclusion \(\iota \from B \to B \times B\) that sends \(b \mapsto
    (b,b)\) and this induces fibers 
    \((\rho')^{-1}(b) \isom \rho_1^{-1}(b) \oplus \rho_2^{-1}(b)\).
    \[
      \begin{tikzcd}
       i^*(E_1 \times E_2) \ar[d,"\rho' "] \ar[r,"\tilde{\iota}"]& E_1
       \times E_2 \ar[d,"\rho_1 
       \times \rho_2"]\\ 
       B \ar[r, "\iota"] & B \times B
      \end{tikzcd}
    \]
    We will denote this as \(\xi_1 \oplus \xi_2\).
  \item Any operation that makes sense to be done on a vector space
    can also be done on two vector bundles over \(B\) to get a vector
    bundle over \(B\). \todo{Flush out with some explcit constructions.}
  \item Given a complex vector bundle \(\xi\) of rank \(n\), one can
    get a rank \(2n\) 
    real vector bundle \(\xi_\R\) by simply forgetting the complex
    structure. 
  \end{enumerate}
\end{prop}
\begin{defn}
  Given a vector bundle \(\xi\), a \de{section} is a map \(s \from B
  \to E\) such that \(\rho \circ s = Id_B\).
\end{defn}
\begin{example}
  \begin{enumerate}
  \item For every vector bundle, there exists a \de{zero section}
    \(s_0 \from B \to E\) defined by \(s_0(b) = (b,0)\).
  \item In general, there are uncountably many sections whose images
    intersect 
    the image of the zero section. These are called \de{vanishing
      sections} and are relatively uninteresting.
  \item \todo{Add something not boring.}
  \end{enumerate}
\end{example}
More generally, we wish to define a generalization of vector bundles
that we will use sparringly.
\begin{defn}
   A \de{fiber bundle} \(\xi\) with total space \(E\), base space
   \(B\), projection map \(\rho \from B \to E\), and fiber \(F\) is
   defined analogously to a complex vector bundle where every
   occurrence of \(\C^n\) is replaced by \(F\).
\end{defn}
See \cite{hatcher}*{pp 376--377} for a more precise definition. 
One reason to study vector bundles is that they can give us insight
into the cohomology rings of the spaces involved. A general theorem
which we will state without proof is
\begin{thm}[Leray-Hirsch]
  \cite{hatcher}*{Theorem 4D.1} Let \(\xi\) be a fiber bundle with
  \(\rho \from E \to B\) such that, for some coefficient ring \(R\),
  \begin{enumerate}
  \item \(H^n(F;R)\) is a finitely generated free \(R\)-module for
    each \(n\).
  \item There exist classes \(e_j \in H^{k_j}(E;R)\) who restrictions
    \(u^*(e_j)\) form a basis for \(H^*(F;R)\) in each fiber \(F\),
    where \(i \from F \to E\) is the inclusion.
  \end{enumerate}
  Then, the map \(\Phi \from H^*(B;R) \otimes_R H^*(F;R) \to
  H^*(E;R)\) given by \[
    \sum_{i,j} b_i \otimes i^*(e_j) \mapsto \sum_{i,j} \rho^*(b_i)
    \cupprod e_j
  \]
  is an isomorphism of \(R\)-modules.
\end{thm}
In particular, the Leray-Hirsch theorem tells us that \(H^*(E;R)\) is
a free \(H^*(B;R)\)-module with basis \(\{e_j\}\) where we view
\(H^*(E;R)\) as a module over the ring \(H^*(B;R)\) by defining the
action \[
  b.e = \rho^*(b) \cupprod e
\]
for \(b \in H^*(B;R)\) and \(e \in H^*(E;R)\).
\begin{defn}
  In the discussion that follows, for any vector space \(V\), let
  \(V_0 = V \setminus \{0\}\) and for any total space \(E\), let \(E_0
  = E \setminus s_0(B)\) where \(s_0\) is the zero section.
\end{defn}
\begin{defn}
  \cite{milnor-stasheff}*{p 96} An \de{orientation} of a real vector
  bundle 
  \(\xi\) is a function which assigns an orientation to each fiber
  \(F\) of \(\xi\) such that for every point \(b_0 \in B\), there is a
  neighborhood \(U\) containing \(b_0\) with local
  trivialization \(h \from U \times \C^n \to \rho^{-1}(U)\) such that
  for each fiber \(F = \rho^{-1}(b)\) for \(b \in B\), the
  homomorphism \(h_b \from \C^n \to F\) given by \(h_b(x) := h(b,x)\)
  is orientation preserving. 
\end{defn}
This definition tells us that, for an orientable rank \(n\) complex
vector bundle \(\xi\), for each fiber \(F \isom \R^n\), we can pick a
preferred generator \(u_F \in H^n(F,F_0;\Z) = \Z\). Then, the local
conditions in the definition of orientation imply that, for every
\(b_0 \in B\), there is a neighborhood \(U\) and a cohomology class
\(u \in H^n(\rho^{-1}(U), \rho^{-1}(U)_0;\Z)\) so that for every fiber
\(F = \rho^{-1}(b)\) over \(U\), \[
  \iota^*(u) = u_F \in H^n(F,F_0;\Z)
\]
where \(\iota \from \rho^{-1}(b) \into \rho^{-1}(U)\).
\begin{thm}
  \cite{milnor-stasheff}*{Theorem 9.1} Let \(\xi\) be an oriented
  real vector bundle of rank \(n\) with total space \(E\). Then,
  \begin{enumerate}
  \item \(H^i(E,E_0;\Z) = 0\) for \(i < n\) and
  \item \(H^n(E,E_0;\Z)\) contains a unique cohomology class \(u\),
    called the \de{Thom class}, 
    whose restriction \[
      \iota^*(u) = u_F \in H^n(F,F_0;\Z)
    \]
    for every fiber \(F\) of \(\xi\) where \(\iota \from F \into E\)
    is the standard inclusion.
  \item The correspondence \(y \mapsto y \cupprod u\) maps
    \(H^k(E;\Z)\) isomorphically to \(H^{k+n}(E,E_0;\Z)\) for every
    integer \(k\).
  \end{enumerate}
\end{thm}
\begin{defn}
  For an oriented real vector bundle \(\xi\), the \de{Thom
    isomorphism} \[ 
    \phi \from H^k(B;\Z) \isomto H^k(E,E_0;\Z)
  \]
  is given by the formula \[
    \phi(x) := \rho^* x \cupprod u
  \]
\end{defn}
We will take for granted that this is an isomorphism of abelian
groups. Now, the inclusion \((E, \emptyset) \into (E, E_0)\) induces a
restriction \(H^*(E,E_0;\Z) \to H^*(E;\Z)\) which we will denote by
\(y \mapsto y|_E\). Then, for \(u \in H^n(E,E_0;\Z)\), we can take \[
  u \mapsto u|_E \in H^n(E;\Z) \isom H^n(B;\Z)
\]
via the Thom isomorphism above. Thus, we define
\begin{defn}
  The \de{Euler class} of an oriented rank \(n\) vector bundle \(\xi\)
  is the cohomology class \[
    e(\xi) \in H^n(B;\Z)
  \]
  which corresponds to \(u|_E\) under the Thom isomorphism
  \(H^n(B;\Z) \isom H^n(E;\Z)\).
\end{defn}
\begin{prop}\label{euler-class-props}
  \begin{enumerate}
  \item \cite{milnor-stasheff}*{Property 9.2} If \(f \from B \to B'\)
    is continuous and covered by an orientation preserving map \(f
    \from \xi \to \xi'\), then \(e(\xi) = f^*e(\xi')\).
  \item \cite{milnor-stasheff}*{Property 9.6} The Euler class of a
    Whitney sum is given by \(e(\xi \oplus \xi') = e(\xi) \cupprod
    e(\xi')\).
  \item \(e(\xi \otimes \xi') = e(\xi) + e(\xi')\).
  \item \cite{milnor-stasheff}*{Property 9.7} If the oriented real
    vector bundle \(\xi\) possesses a nowhere 
    zero section, then \(e(\xi)\) must be zero
  \end{enumerate}
\end{prop}
\section{Chern Classes  (presented by
  Thomas Sale)}
The material in this section will roughly correspond to
\cite{bott-tu}*{Sections 20--21}. Because of this, throughout this
section all cohomology is de Rham cohomology and thus over \(\R\). All
spaces can be assumed to be smooth manifolds.
\begin{rmk}
  As one gets more accustomed to using vector bundles, typically one
  stops denoting the bundle by something like \(\xi\) and just denotes
  it by the total space \(E\) when the bundle structure is clear.
\end{rmk}
\subsection{The first Chern class of a line bundle}
\begin{defn}
  Given a complex line bundle \(\xi\), we define the \de{first Chern
    class of \(\xi\)} to be \(c_1(\xi) :=
  e(\xi_\R)\).
\end{defn}
\begin{prop}\label{first-chern-class-of-line-bundle-props}
  Given line bundles \(\xi, \xi'\),
  \begin{enumerate}
  \item \(c_1(\xi \otimes \xi') = c_1(\xi) + c_1(\xi')\).
  \item \(0 = c_1(\xi \otimes \xi^*) = c_1(\xi) + c_1(\xi^*) \implies
    c_1(\xi^*) = -c_1(\xi)\).
  \end{enumerate}
\end{prop}
\begin{proof}
  The first part is immediate from \ref{euler-class-props}(c) and the
  second follows from the fact that \(\xi \otimes \xi^* =
  \Hom(\xi,\xi)\) always has a nonvanishing section sending \(b \in
  B\) to the identity map in \(\Hom(E,E)\) and so \(e(L_\R) = 0\).
\end{proof}
\subsection{Projectivized vector bundles and Chern clases}
\begin{defn}\label{projectivization}
  \begin{enumerate}
  \item Given a rank \(n\) complex vector bundle \(\xi\), we define
    the \de{projectivized bundle \(\P(\xi)\)} to be the
    bundle \[
      \begin{tikzcd}
        \P(E) \ar[d, "\pi"] \\
        B
      \end{tikzcd}
    \]
    where \(\pi^{-1}(b) = \P(\rho^{-1}(b)) \isom \P(\C^n) \isom \CP^{n-1}\). Note that
    \(\pi \circ \P = \rho\) and that a point \(x \in \P(E)\) is a
    line in \(\rho^{-1}(\pi(x))\). \[
      \begin{tikzcd}
        & E \ar[d, "\rho"] \ar[ld, dashed, "\P"]\\
      \P(E) \ar[r, "\pi"] & B
      \end{tikzcd}
    \]
   \item We define the pullback vector bundle \(\pi^*(\xi)\) of the
     projectivization map 
    \(\pi \from \P(E) \to M\) and the
    fiber \((\rho')^{-1}(x) = \rho^{-1}(\pi(x))\). \[
      \begin{tikzcd}
        \pi^*(E) \ar[r,"\tilde{\pi}"] \ar[d, "\rho'"] & E
        \ar[d,"\rho"] \\ 
        \P(E) \ar[r, "\pi"] & B
      \end{tikzcd}
    \]
  \item The \de{universal sub-bundle} of \(\pi^*(\xi)\) is defined by
    taking the total space \[
      S = \{(x,v) \in \pi^*(E) \st v \in x\} 
    \]
    and is a copy of the tautological line bundle over \(\P(E)\) (see
    \ref{first-examples}(d)) and so \(S\) is a rank \(1\) vector
    bundle.
  \item The \de{universal quotient bundle} \(Q\) of \(\pi^*(E)\) is
    defined by taking the short exact sequence \[
      0 \to S \to \pi^*E \to Q \to 0
    \]
    Then, \(Q\) is a rank \(n-1\) vector bundle.
  \item If we restrict the unviversal sub-bundle \(S\) to be over a
    fiber \(\pi^{-1}(b) \isom \CP^{n-1}\), then we have a bundle with
    total space \(\tilde{S}\) which is a copy of the tautological
    line bundle over \(\CP^{n-1}\).
  \end{enumerate}
\end{defn}
Now, recall the fact that \(H^*(\P(\pi^{-1}(b));\R) = H^*(\CP^{n-1};\R) \isom \R[a]/(a^{n})\) as
rings. If we set \(x = c_1(\tilde{S}^*) = -c_1(\tilde{S}) \in
H^2(\P(E))\) and \(i \from \P(\pi^{-1}(b)) \into \P(E)\), then
\(-i^*(x) = a \in H^2(\CP^{n-1};\R)\) \todo{Explain why}. Since this works for every
fiber simultaneously, we can apply
\begin{cor}[Leray-Hirsch (special case)]
  The cohomology of \(H^*(\P(E);\R)\) is a free module over \(H^*(B;\R)\) with basis
  \(\{1,x, \ldots, x^{n-1}\}\) with action \(y.x^k = \pi^* (y)
  \cupprod x\).
\end{cor}
From this fact, we can write \(x^n \in H^*(P(E);\R)\) as a linear
combination with coefficients in \(H^*(B)\).
\begin{defn}\label{general-def-of-chern-classes}
  \begin{enumerate}
  \item The \de{Chern classes} of the bundle \(\xi\) with data
    \(\rho \from E \to B\) are elements \(c_i(\xi) \in H^*(B;\R)\)
    which are the unique coefficients such that
    \[ x^n + \pi^*(c_1(\xi)) x^{n-1} + \cdots + \pi^*(c_n(\xi)) = 0 \in
      H^*(\P(E);\R)
    \]
    Thus, \(c_i(\xi) \in H^{2i}(B;\R)\).
  \item We define the \de{total Chern class} to be \[
      c(\xi) := 1 + c_1(\xi) + \cdots + c_n(\xi) \in H^*(B;\R)
    \]
  \end{enumerate}
\end{defn}
\begin{prop}[Main properties of Chern classes]\label{chern-class-props}
  \begin{enumerate}
  \item Given a map \(f \from B' \to B\), Chern classes have
    naturality \(c(f^*(\xi)) = f^*(c(\xi))\). As an immediate
    corollary, this menas that Chern classes are an invariant of
    vector bundles, that is, \(\xi \isom \xi' \implies c(\xi) =
    c(\xi')\). 
  \item Given \(\gamma^1\) to be the tautological complex line bundle
    over \(\P^n\), then \(H^*(\P^n;\R) \isom
    \R[c_1(\gamma^1)]/((c_1(\gamma^1)))^{n+1}\).
  \item \(c(\xi \oplus \xi') = c(\xi) \cupprod c(\xi')\), called the
    \de{Whitney product formula}.
  \item If \(\xi\) has rank \(n\), then \(c_i(\xi) := 0\) for \(i >
    n\).
  \item If \(\xi\) has a non-vanishing section, then \(c_n(\xi) = 0\).
  \item The top Chern class of a complex vector bundle \(\xi\) is the
    Euler class of \(\xi_\R\). In other words, \(c_n(\xi) =
    e(\xi_\R)\). 
  \end{enumerate}
\end{prop}
\begin{cor}\label{total-class-of-line-bundle}
  As a consequence of the above properties, the total Chern class of a
  line bundle \(\lambda\) is always of the form \[
    c(\lambda) = 1 + c_1(\lambda)
  \]
  where \(c_1(\lambda) = e(\lambda_\R)\) by definition.
\end{cor}
\subsection{The Splitting Principle}
\begin{defn}\label{split-manifold}
  Let \(\xi\) with data \(\tau \from E \to M\) be a smooth complex
  vector bundle of rank 
  \(n\) over a manifold \(M\). Then, we define a \de{split manifold}
  of \(\tau\) to be a space \(F(E)\) with map \(\sigma \from F(E) \to
  M\) such that
  \begin{enumerate}
  \item \(\sigma^*\xi\) is a direct sum of line bundles and
  \item the induced map on cohomology \(\sigma^* \from H^*(M;\R) \to
    H^*(F(E);\R)\) is injective.
  \end{enumerate}
\end{defn}
\begin{example}
  Given a rank \(2\) complex vector bundle \(\xi\) with \(\rho \from E
  \to M\), we can take \(F(E) = \P(E)\) because \(\pi^*(\xi)\)
  decomposes as a direct sum of \(S \oplus Q\), its universal sub and
  quotient bundles. \[
    \begin{tikzcd}
      E \ar[d,"\rho"] & \pi^*(E) \isom S \oplus Q \ar[d]  \\
      M & \P(E) \ar[l,"\pi"]
    \end{tikzcd}
  \]
\end{example}
\begin{example}
  Given a rank \(3\) complex vector bundle \(\xi\) with \(\rho \from E
  \to M\), we can iterate the construction above where the
  \(2\)-dimensional quotient bundle \(Q_E\) can be split into a direct sum of
  line bundles when pulled back to \(\P(Q_E)\).
  \[
    \begin{tikzcd}
      E \ar[d,"\rho"] & \alpha^*(E) \isom S_E \oplus Q_E \ar[d] &
      \beta^*(S_E \oplus Q_E) \isom \beta^* S_e \oplus S_{Q_E} \oplus
      Q_{Q_E} \ar[d] \\
      M & \P(E) \ar[l,"\alpha"] & \ar[l, "\beta"] \P(Q_E) 
    \end{tikzcd}
  \]
  Thus, \(\P(Q_E)\) is the split manifold \(F(E)\).
\end{example}
This leads us to the general idea
\begin{prop}[The Splitting Principle]\label{splitting-principle}
  \cite{bott-tu}*{p 275} To prove a polynomial identity in Chern
  classes, it suffices to 
  assume that the vector bundles are direct sums of line bundles.
\end{prop}
\begin{proof}
  Given a polynomial \(f\), we want to show \(f(c(E)) = 0\). Then,
  with setup \[
    \begin{tikzcd}
      E \ar[d] & \sigma^*E \ar[d] \\
      M & \ar[l, "\sigma"] F(E)
    \end{tikzcd}
  \]
  we get \(\sigma^*(f(c(E))) = f(c(\sigma^*E))\) and so, because
  \(\sigma^*\) is injective by definition of \(F(E)\),
  \(f(c(\sigma^*(E))) = 0 \implies f(c(E)) = 0\). Since
  \(\sigma^*(E)\) is a direct sum of line bundles by construction, we
  are done. 
\end{proof}
\section{Chern class computations, flag manifolds, and the
  Grassmannian (presented by George H. Seelinger)}
\subsection{Proof of the Whitney Product Formula}
\begin{lem}\label{whitney-product-formula-on-line-bundles}
  Let \(\xi\) be a direct sum of line bundles, that is \(\xi =
  \lambda_1 \oplus \cdots \oplus \lambda_n\) where each \(\lambda_i\)
  is a line bundle over \(B\). Then, \[
    c(\xi) = c(\lambda_1) c(\lambda_2) \cdots c(\lambda_n)
  \]
\end{lem}
\begin{proof}
  First, let us take vector bundle \(\xi\) to be a a direct sum of
  line bundles. Then, if we take pullback bundle \(\xi\) by the
  projectivization map (see \ref{projectivization}) \(\pi \from \P(E) \to M\), \(\pi^* \xi\) is a
  direct sum of line bundles, let us say \(\pi^*\xi = L_1 \oplus
  \cdots \oplus L_n\).
  \[
      \begin{tikzcd}
       S \ar[r, hookrightarrow] &\pi^* E \isom  L_1 \oplus \ldots \oplus L_n 
        \ar[r,"\tilde{\pi}"] \ar[d, "\rho'"] & E
        \ar[d,"\rho"] \\ 
        & \P(E) \ar[r, "\pi"] & M
      \end{tikzcd}
  \]
  Now, let \(s_i \from S \to L_i\) be the projection of vector
  bundles. Thus, \(s_i\) induces a section of the bundle \(\Hom(S,L_i)
  \to M\). \footnote{This is a special case of a canonical fact. The map \(s_i\) induces
    a vector space isomorphism on each fiber over \(b\), say
    \((s_i)_b\), and so the
    canonical associated section of the \(\Hom\) bundle sends each
    \(b\) to the map \((s_i)_b\).} Furthermore, since the fiber of
  \(S\) over every point \(y \in \P(E)\) is a \(1\)-dimensional
  subspace of the fiber of \(y\) in \(\pi^* \xi\), that is
  \((\rho')^{-1}(y)\), the projections \(s_1, \ldots, s_n\) cannot be
  simultaneously zero and so the open sets \[
    U_i := \{y \in \P(E) \st s_i(y) \neq 0\}
  \]
  form an open cover of \(\P(E)\).

  Now, it is useful to note that the bundle \(\Hom(S,L_i) \isom S^*
  \otimes L_i\). Then, for each \(i\), \(c_1(S^* \otimes L_i) \in
  H^2(\P(E))\) restricts to zero in \(H^2(U_i)\) by construction of
  \(U_i\) (\(L_i\) is a trivial line bundle over \(U_i\)), and so we
  can lift this element to \(H^2(\P(E),U_i)\). So, if we take the
  product, we can lift to \(H^2(\P(E), U_1 \union \cdots \union U_n) =
  H^2(\P(E),\P(E)) = 0\). Therefore, \[
    0 = \prod_{i=1}^n c_1(S^* \otimes L_i)
    \overset{\ref{first-chern-class-of-line-bundle-props}(a)}{=}
    \prod_{i=1}^n(c_1(S^*) + c_1(L_i))
  \]
  and if we set \(x = c_1(S^*)\), we get \[
    0 = \prod_{i=1}^n(x + c_1(L_i)) = x^n + e_1(c_1(L_1), \ldots,
    c_1(L_n)) x^{n-1} + \cdots + e_n(c_1(L_1), \ldots, c_1(L_n))
  \]
  where \(e_i\) is the \(i\)th symmetric polynomial. However, this is
  precisely how we defined \(c_i(\xi)\), so we get \[
    c_i(\xi) = e_i(c_1(L_1), \ldots, c_1(L_n)) \implies c(E) = \prod
    (1+c_1(L_i)) = \prod c(L_i)
  \]
\end{proof}
\begin{cor}[Corollary of proof]\label{chern-classes-are-elementary-symmetric-functions}
  Given a complex vector bundle \(\xi\) of rank \(n\), \[
    c(E) = \prod_{i=1}^n (1+c_1(L_i))
  \]
  where \(\pi^* \xi = L_1 \oplus \cdots \oplus L_n\) is a direct sum
  of line bundles and, as before, \(\pi \from \P(E) \to M\).
\end{cor}
\begin{proof}[Proof of \ref{chern-class-props}(c)]
  We now use the lemma above and the splitting principle
  \ref{splitting-principle} to prove the
  formula in general. Given complex vector bundles \(\xi, \xi'\) over
  \(M\) of rank \(n\) and \(m\), respectively,
  we can iterate the splitting constructions to get a direct sum of
  line bundles for which the Chern class identites are equivalent. \[
    \begin{tikzcd}
      E \oplus E' \ar[d] & L_1 \oplus \cdots \oplus L_n \oplus \pi^* E
      \ar[d] & L_1
      \oplus \cdots \oplus L_n \oplus L_1' \oplus \cdots \oplus L_m' \ar[d]
      \\
      M & \ar[l, "\pi"] F(E) & \ar[l,"\pi'"] F(\pi^*E)
    \end{tikzcd}
  \]
  Now, if \(\sigma = \pi' \circ \pi\), then
  \begin{align*}
    \sigma^*(c(\xi \oplus \xi'))
    & = c(\sigma^*(\xi \oplus \xi'))
    & \text{By naturality of Chern classes \ref{chern-class-props}(a)}\\
    & = c(L_1 \oplus \cdots \oplus L_n \oplus L_1' \oplus \cdots
      \oplus L_m')
    & \text{By splitting principle construction above}\\
    & = c(L_1) \cdots c(L_n) c(L_1') \cdots c(L_m')
    & \text{By Lemma \ref{whitney-product-formula-on-line-bundles}} \\
    & = c(L_1 \oplus \cdots \oplus L_n) c(L_1' \oplus \cdots \oplus
      L_m')
    & \text{By Lemma \ref{whitney-product-formula-on-line-bundles}}\\
    & = c(\sigma^* \xi) c(\sigma^* \xi') 
    & \text{By splitting principle construction above} \\
    & = \sigma^* c(\xi)c(\xi')
    & \text{By naturality of Chern classes}
  \end{align*}
  However, also by construction (since we have a split manifold
  \ref{split-manifold}), \(\sigma^*\) is injective, and thus \(c(\xi
  \oplus \xi') = c(\xi)c(\xi')\).
\end{proof}
\begin{rmk}
  \begin{enumerate}
  \item Given a short exact sequence of smooth complex vector
    bundles \[
      0 \to A \to B \to C \to 0
    \]
    one can prove that \(c(B) = c(A)c(C)\) using the fact that \(B \isom A
    \oplus C\) as smooth bundles and the Whitney product formula.
  \item Using the splitting principle and the Whitney product formula,
    one can directly prove \ref{chern-class-props}(f), that is
    \(c_n(\xi) = e(\xi_\R)\). See \cite{bott-tu}*{p 278} for a
    complete proof.
  \end{enumerate}
\end{rmk}
\subsection{Computing some Chern classes}
Recall from Corollary \ref{chern-classes-are-elementary-symmetric-functions}
that the Chern classes of a rank \(n\) complex 
vector bundle \(\xi\) are precisely the elementary symmetric functions in the
first Chern classes of the line bundles into which \(\xi\) splits when
pulled back to the splitting manifold \(F(E)\). Thus, by the
fundamental theorem of symmetric functions, any symmetric polynomial
in these variables is a polynomial in \(c_1(\xi), \ldots, c_n(\xi)\).
\subsection{Flag manifolds}
\begin{defn}
  \begin{enumerate}
  \item Given a complex vector space \(V\) of dimension \(n\), a
    \de{(complete) flag} in \(V\) is a sequence of subspaces
    \[ A_1 \propsubset A_2 \propsubset \cdots \propsubset A_n = V,
      \dim_\C A_i = i
    \]
    Note that a flag on \(V\) induces a \(\C\) basis on \(V\) by
    picking basis vector \(b_i \in A_{i}/A_{i-1}\) (take \(A_0 = \{0\}\)).
  \item Let \(\Fl(V)\) be the collection of all flags of \(V\), called
    the \de{flag manifold}. 
  \end{enumerate}
\end{defn}
\begin{rmk}\label{flag-manifold-is-a-manifold}
  In fact, \(\Fl(V)\) is a manifold because \(GL(n,\C)\) acts
  transitively on \(\Fl(V)\) and the stabilizer of a flag is the
  closed subgroup of all upper triangular matrices (with respect to
  the induced basis), say \(H\), and so \(\Fl(V) \isom GL(n,\C)/H\) as
  a \(G\)-space, but \(GL(n,\C)/H\) can be given a manifold structure
  since \(H\) is a closed subgroup and \(GL(n,\C)\) is a Lie group.
\end{rmk}
\begin{defn}
  Given a vector bundle \(\xi\) with \(\rho \from E \to M\), one can form the
  \de{associated flag bundle} \(\Fl(\xi)\) with \(f \from \Fl(E) \to M\)
  where \(f^{-1}(x) = \Fl(\rho^{-1}(x))\) with the natural transition maps.
\end{defn}
Since the fiber is not a vector space, the associated flag bundle is
not a vector bundle, but it is still a fiber bundle.
\begin{prop}
  \cite{bott-tu}*{Proposition 21.15} The associated flag bundle \(\Fl(\xi)\) of a vector bundle \(\xi\)
  is the split manifold constructed in the examples after
  \ref{split-manifold}. 
\end{prop}
\begin{proof}
  Let \(V \to \{*\}\) be a rank \(3\) vector bundle over a point (ie a
  \(3\) dimensional vector space.) Then, we have \[
    \begin{tikzcd}
      V \ar[d]& S_V \oplus Q_V \ar[d]& S_V \oplus S_{Q_V} \oplus Q_{Q_V} \ar[d]\\
      \{*\} &\ar[l] \P(V) & \ar[l] \P(Q_V) = F(V)
    \end{tikzcd}
  \]
  However, we have correspondences
  \begin{center}
    \begin{tabular}{c|c}
      A point in \(x \in \P(V)\)&  a line \(\ell_x\) in \(V\)\\
      A point in \(S_V \oplus Q_V\) & \((x,(v_1,v_2))\) such that \(
      x \in \P(V), v_1 \in \ell_x, v_2 \in V / \ell_x\) \\
      A point in \(\P(Q_V)\)& a line \(\ell\) in \(V\)
      and a line \(\ell'\) in \(V/\ell\).
    \end{tabular}
  \end{center}
  However, this means that we can similarly regard \(\ell'\) as a
  \(2\)-dimensional plane in \(V\) containing \(\ell\), so \[
    F(V) = \P(Q_V) = \{\ell \subset \ell \oplus \ell' \subset V\} = \Fl(V)
  \]

  More generally, given rank \(n\) vector bundle \(E \to M\), the
  split manifold \(F(E)\) is obtained by a sequence of \(n-1\)
  projectivizations (see \ref{projectivization}). Then,\\
  \begin{center}
    \noindent
    \begin{tabular}{@{}c|c}
      A point in \(\P(E)\)
      & \((p,\ell)\) where \(p \in M\) and
        \(\ell\) is a line in \(\rho^{-1}(p)\)\\
      A point in \(\P(Q_1)\) over \((p,\ell_1) \in \P(E)\)
      & \((p,\ell_1,\ell_2)\) where \(\ell_2 \in \rho^{-1}(p) /
        \ell_1\)\\
      A point in \(\P(Q_2)\) over \((p,\ell_1,\ell_2) \in \P(Q_1)\)
      & \((p,\ell_1,\ell_2,\ell_3)\) where \(\ell_3 \in \rho^{-1}(p) /
        (\ell_1 \oplus \ell_2)\) \\
      \vdots & \vdots \\
      A point in \(F(E) = \P(Q_{n-1})\) over \((p,\ell_1, \ell_2,
      \ldots, \ell_{n-1})\)
      & \((p, \ell_1, \ldots, \ell_n)\) where \(\ell_n \in
        \rho^{-1}(p)/(\ell_1 \oplus \cdots \oplus \ell_{n-1})\)
    \end{tabular}
  \end{center}
  But this last point naturally identifies with the flag \[
    (p, \ell_1 \subset \ell_1 \oplus \ell_2 \subset \cdots \subset \rho^{-1}(p))
  \]
  Thus, \(F(E)\) and \(\Fl(E)\) are the same under this equivalence.
\end{proof}
Recall from before (\ref{general-def-of-chern-classes}) that we
essentially constructed the Chern classes 
of \(M\) such that \[
  H^*(\P(E)) \isom H^*(M)[x]/(x^n + c_1(\xi) x^{n-1} + \cdots
  + c_n(\xi)), \ \ x = c_1(S^*)
\]
as rings. However, we have another useful characterization given by
\begin{prop}
  \cite{bott-tu}*{Proposition 21.16} Given the setup \[
    \begin{tikzcd}
      E \ar[d] & 0 \to S \to \pi^*E \to Q \to 0 \ar[d]\\
      M & \ar[l, "\pi"] \P(E)
    \end{tikzcd}
  \]
  we have
  \[
    H^*(\P(E);\R) = H^*(M;\R)[c_1(S),c_1(Q), \ldots,c_{n-1}(Q)]/(c(S)c(Q) = \pi^* c(E))
  \]  
\end{prop}
\begin{proof}
  We simply eliminate the generators \(c_1(Q), \ldots, c_{n-1}(Q)\)
  using the relation \(c(S)c(Q) = \pi^*c(E)\) by equating terms of
  equal degrees in \[
    (1-c_1(S^*))(1+c_1(Q) + \cdots + c_{n-1}(Q)) = 1+\pi^*c_1(E) +
    \cdots + \pi^* c_{n}(E)
  \]
  gives us
  \begin{align*}
      c_1(Q) - c_1(S^*) & = \pi^*c_1(E) \\
      c_2(Q) - c_1(S^*)c_1(Q) & = \pi^*c_2(E)\\
      c_3(Q) - c_1(S^*)c_2(Q) & = \pi^*c_3(E)\\
       \vdots \\
       c_{n-1}(Q) - c_1(S^*) c_{n-2}(Q) & = \pi^* c_{n-1}(E)\\
       -c_1(S^*) c_{n-1}(Q) & = \pi^* c_{n}(E)
    \end{align*}
  This shows that \(c_i(Q), 1 \leq i \leq n-1\) can be
  expressed in terms 
  of \(c_1(S^*)\) and elements of \(H^*(M)\) and so they can be
  eliminated as generators. The one remaining relation transformed by
  the other relations will give \[
    - c_1(S^*) c_{n-1}(Q) = \pi^* c_n(E) \iff c_1(S^*)^n + \pi^* c_1(E)
    c_1(S^*)^{n-1} + \cdots + \pi^* c_n(E) = 0
  \]
  and so we have established the equivalence.
\end{proof}
\begin{prop}\label{cohomology-of-flag-manifold}
  \cite{bott-tu}*{Proposition 21.17 (enhanced)} Let \(E \to M\) be a
  complex rank \(n\) vector bundle. Then, cohomology ring of the flag
  manifold \(\Fl(E)\) is \[
    H^*(\Fl(E);\R) = \R[x_1, \ldots, x_n] / \left( \prod_{i=1}^n
      (1+x_i) = c(E) \right)
  \]
  where \(x_i = c_1(S_i)\) for \(1 \leq i \leq n-1\) amd \(x_n =
  c_1(Q_{n-1})\) where the \(S_i\)'s and \(Q_i\)'s come from iterating the
  projectivization construction to acheive the splitting
  principle. Furthermore, in the special case where our vector 
  bundle is \(V \to \{*\}\), 
  we get the identity above with \(c(E) = 1\).
\end{prop}
\begin{proof}
  Since the flag bundle is obtained by a sequence of \(n-1\)
  projectivizations, we can compute first
  \begin{align*}
    H^*(\P(Q_1);\R)
    & = H^*(\P(E);\R)[c(S_2), c(Q_2)]/(c(S_2) c(Q_2) = c(Q_1)) \\
    & = H^*(M;\R)[c(S_1), c(Q_1), c(S_2), c(Q_2)]/(c(S_1) c(Q_1) =
      c(E), c(S_2) c(Q_2) = c(Q_1)) \\
    & = H^*(M;\R)[c(S_1), c(S_2), c(Q_2)]/(c(S_1) c(S_2) c(Q_2) = c(E))
  \end{align*}
  Then, using induction, \[
    H^*(\P(Q_{n-2});\R) = H^*(M;\R)[c(S_1), \ldots, c(S_{n-1}),
    c(Q_{n-1})]/(c(S_1) \cdots c(S_{n-1})c(Q_{n-1}) = c(E))
  \]
  Then, since \(c(S_i) = (1+x_i)\) and \(c(Q_{n-1}) = (1+x_n)\), we
  are done.
\end{proof}
\begin{defn}
  Given a manifold \(M\), the \de{Poincar\'{e} series} of a manifold
  \(M\) is \[
    P_t(M) := \sum_{i=0}^\infty \dim H^i(M;\R) t^i
  \]
  More generally, if \(A = \bigoplus_{i=0}^\infty A_i\)  is a graded
  algebra over a field \(K\), then the Poincar\'{e} series is given
  by \[
    P_t(A) := \sum_{i=0}^\infty (\dim_K A_i) t^i
  \]
\end{defn}
\begin{example}
  Since \(H^*(\CP^{n-1};\R) = \R[x]/(x^n)\) with \(\deg x = 2\), we
  get immediately that \[
    P_t(\CP^{n-1}) = 1 + t^2 + \cdots + t^{2(n-1)} = \frac{1-t^{2n}}{1-t^2}
  \]  
\end{example}
\begin{lem}
  Since \[
    H^*(\P(E);\R) \isom H^*(M;\R) \otimes H^*(\CP^{n-1};\R)
  \]
  as modules, we have that \[
    P_t(\P(E)) = P_t(M) \frac{1-t^{2n}}{1-t^2}
  \]
\end{lem}
\begin{proof}
  The result follows immediately from dimension counting of the
  tensor product and the example above.
\end{proof}
\begin{cor}\label{poincare-of-flag-manifold}
  Let \(V\) be a complex complex vector space of dimension
  \(n\). Then, \[
    P_t(\Fl(V)) = \frac{(1-t^2)(1-t^4) \cdots (1-t^{2n})}{(1-t^2)^n}
  \]
  and, more generally, if \(E \to M\) is a rank \(n\) vector bundle,
  then \[
    P_t(\Fl(E)) = P_t(M) \frac{(1-t^2)(1-t^4) \cdots (1-t^{2n})}{(1-t^2)^n}
  \]
\end{cor}
\begin{proof}
  The flag manifold is constructed by a sequence of projectivizations
  and, since each time we projectivize a rank \(k\) vector bundle we
  multiply the Poincar\'{e} polynomial by \((1-t^{2k})/(1-t^2)\), we
  get \[
    P_t(\Fl(E)) = \frac{1-t^{2n}}{1-t^2} P_t(\P(Q_{n-3})) = \cdots =
    \frac{1-t^{2n}}{1-t^2} \frac{1-t^{2n-2}}{1-t^2} \cdots \frac{1-t^2}{1-t^2} P_t(M)
  \]
\end{proof}
\subsection{The Grassmannian}
\begin{defn}
  The \de{Grassmannian} of a complex vector space \(V\), denoted \(G_k(V)\) is the set of
  all subspaces of codimension \(k\) in \(V\).
\end{defn}
\begin{example}
  Note that for \(V\) a complex vector space of dimension \(n\),
  \(G_{n-1}(V) = \P(V) \isom \CP^{n-1}\).
\end{example}
\begin{rmk}
  The Grassmannian is a manifold because it can be represented as the
  space \[
    G_k(V) = \frac{U(n)}{U(k)\times U(n-k)}
  \]
  The argument follows a similar one to \ref{flag-manifold-is-a-manifold}.
\end{rmk}
\begin{defn}
  Given a complex vector space \(V\) of dimension \(n\), we define the
  following vector bundles over \(G_k(V)\).
  \begin{enumerate}
  \item The \de{universal subbundle} which has total space \[
      S := \{(x,v) \in G_k(V) \times V \st v \in x\}
    \]
    and thus the fiber of \(x \in G_k(V)\) is the plane \(x\) defines
    in \(V\).
  \item The \de{product bundle} \(\hat{V} := G_k(V) \times V\).
  \item The \de{universal quotient bundle} \(Q\) defined by the exact
    sequence \[
      0 \to S \to \hat{V} \to Q \to 0
    \]
  \end{enumerate}
\end{defn}
\begin{prop}
  \cite{bott-tu}*{Proposition 23.1} The cohomology of the complex
  Grassmannian \(G_k(V)\)has Poincar\'{e} polynomial \[
    P_t(G_k(V)) = \frac{(1-t^2) \cdots (1-t^{2n})}{(1-t^2) \cdots
      (1-t^{2k})(1-t^2) \cdots (1-t^{2(n-k)})}
  \]
\end{prop}
\begin{proof}
  Consider the setup over the Grassmannian
  \[
    \begin{tikzcd}
      S \oplus Q \ar[d]& f^*Q \ar[d]\\
      G_k(V) & \ar[l,"f"] F(S) & \ar[l] F(f^*Q)
    \end{tikzcd}
  \]
  Then, we consider
  \begin{center}
    \begin{tabular}{c|c}
      A point in \(F(S)\)&\((P, L_1 \subset \cdots \subset P)\) for
                           \((n-k)\) plane \(P \subset V\) and flag in
                           \(P\)\\
      A point in \(F(f^*Q)\) & A point \((P,L_1 \subset \cdots \subset
                               P)\) in \(F(S)\) together with a flag
                               in \(V/P\)\\
      & ie \((P, L_1 \subset \cdots \subset L_{n-k-1} \subset P
        \subset L_{n-k+1} \subset \cdots \subset V)\)
    \end{tabular}
  \end{center}
  Thus, \(F(f^*Q)\) is the flag manifold \(F(V)\) and is obtained from
  the Grassmannian by \(2\) flag constructions. Thus, by our

  computations of the flag manfold Poincar\'{e} polynomials \ref{poincare-of-flag-manifold}, \[
    P_t(\Fl(V)) = P_t(F(f^*Q)) = P_t(G_k(V)) \frac{(1-t^2) \cdots (1-t^{2(n-k)})(1-t^2)
    \cdots (1-t^{2k})}{(1-t^2)^{n}}
  \]
\end{proof}
\begin{prop}
  \cite{bott-tu}*{Proposition 23.2} Let \(V\) be a complex vector
  space of dimension \(n\).
  \begin{enumerate}
  \item As a ring, \[
      H^*(G_k(V);\R) = \frac{\R[c_1(S), \ldots, c_{n-k}(S), c_1(Q),
        \ldots, c_k(Q)]}{(c(S)c(Q)=1)}
    \]
  \item The Chern classes \(c_1(Q) ,\ldots, c_k(Q)\) of the quotient
    bundle generate the cohomology ring \(H^*(G_k(V))\).
  \item For a fixed \(k\) and a fixed \(i\), there are no polynomial
    relations of degree \(i\) among \(c_1(Q), \ldots, c_k(Q)\) if the
    dimension of \(V\) is large enough.
  \end{enumerate}
\end{prop}
To prove this, we will need the following lemma
\begin{lem}
  \cite{bott-tu}*{p 297} If \(I\) is an ideal in \(A = \R[x_1, \ldots,
  x_{n-k}, y_1, \ldots, 
  y_k]\) generated by the homogeneous terms of \[
    (1+x_1+\cdots + x_{n-k})(1+y_1+\cdots+y_k)-1
  \]
  where \(\deg x_i = 2i\) and \(\deg y_i = 2i\), then the Poincar\'{e}
  series of \(A/I\) is given by \[
    P_t(A/I) = \frac{(1-t^2) \cdots (1-t^{2n})}{(1-t^2) \cdots
      (1-t^{2(n-k)})(1-t^2) \cdots (1-t^{2k})}
  \]
\end{lem}
\begin{proof}[Proof of Proposition]
  Consider once again the setup \[
    \begin{tikzcd}
      S \oplus Q \ar[d]& f^*Q \ar[d]\\
      G_k(V) & \ar[l,"f"] F(S) & \ar[l] F(f^*Q) = \Fl(V)
    \end{tikzcd}
  \]
  Then, we have by repeated application of
  \ref{cohomology-of-flag-manifold} for \(H^*(F(f^*Q))\) that
  \begin{align*}
    H^*(\Fl(V)\R)
    & = H^*(F(S);\R)[y_1, \ldots, y_k]/\left( \prod (1+y_j) = c(Q)
      \right)\\
    & = H^*(G_k(V);\R)[x_1, \ldots, x_{n-k}, y_1, \ldots, y_k]/\left(
      \prod (1+x_i) = c(S), \prod (1+y_j) = c(Q) \right)
  \end{align*}
  but also by the special case of \ref{cohomology-of-flag-manifold},
  we have directly that \[
    H^*(\Fl(V);\R) = \R[x_1, \ldots, x_{n-k}, y_1, \ldots, y_k] / \left(
      \prod (1-x_i) \prod (1+y_j) = 1 \right)
  \]
  and so \(c(S), c(Q)\) have no other relations besides \(c(S)c(Q) =
  1\) otherwise, they would be present in the presentation above
  presentation. Then, it follows \todo{how?} that there is an
  injection of algebras \[
    \frac{\R[c(S), c(Q)]}{(c(S)c(Q) = 1)} \into H^*(G_k(V);\R)
  \]
  However, by the lemma abovem \[
    P_t\left(  \frac{\R[c(S), c(Q)]}{(c(S)c(Q) = 1)} \right) =
    \frac{(1-t^2)^n}{(1-t^2) \cdots (1-t^{2(n-k)})(1-t^2) \cdots
      (1-t^{2k})} = P_t(G_k(V))
  \]
  Thus, the injection is an isomorphism by dimension counting, proving
  part (a).

  For part (b), we write \[
    c(S) \overset{(a)}{=} \frac{1}{c(Q)} = \frac{1}{\prod (1+y_j)} =
    \prod_j \left( \sum_{r=0}^N y_j^r \right)
  \]
  since there exists an \(N\) such that \(y_j^N = 0\) for all
  \(j\) since the Grassmannian is finite dimensional. Thus, \(c(S)\)
  can be written in terms of 
  \(c_i(Q)\)'s and it follows that the \(c_i(S)\) terms are
  unnecessary generators.

  However, this process also induces polynomial relations of degrees
  \(2(n-k+1), \ldots, 2n\) among the \(c_1(Q), \ldots, c_k(Q)\). So,
  if \(2(n-k+1) > i\), there exists no polynomial relations of degree
  \(i\) among the \(c_j(Q)\)'s. 
\end{proof}
\begin{bibdiv}
  \begin{biblist}
    \bib{bott-tu}{book}{
      author={Bott, Raoul}
      author={Tu, Loring W.}
      title={Differential Forms in Algebraic Topology}
      year={1982}
    }
    \bib{hatcher}{book}{
      author={Hatcher, Allen}
      title={Algebraic Topology}
      year={2001}
    }
    \bib{milnor-stasheff}{book}{
      author={Milnor, John W.}
      author={Stasheff, James D.}
      title={Characteristic Classes}
      year={1974}
    }
  \end{biblist}
\end{bibdiv}
\end{document}