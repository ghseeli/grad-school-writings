\documentclass[11pt,leqno,oneside]{amsbook}
\usepackage{tikz}
\usetikzlibrary{cd}
\usetikzlibrary{decorations.markings}
\usepackage{bbm}
\usepackage{ytableau}
\usepackage{todonotes}
\usepackage{./notes}
\usepackage{../ReAdTeX/readtex-core}
\usepackage{../ReAdTeX/readtex-dangerous}
\usepackage{../ReAdTeX/readtex-abstract-algebra}
\usepackage{../ReAdTeX/readtex-topology}

\newcommand{\bbk}{\mathbbm{k}}
\newcommand{\Class}{\operatorname{Class}}
\newcommand{\Res}{\operatorname{Res}}
\newcommand{\Ind}{\operatorname{Ind}}
\newcommand{\bs}{\textbackslash}
\newcommand{\partitionof}{\vdash}
\newcommand{\T}{\mathsf{T}} % Tableau
\renewcommand{\S}{\mathsf{S}}
\newcommand{\sh}{\operatorname{shape}}
\newcommand{\grdim}{\boldsymbol{\dim}}

\newcommand{\dominatedby}{\trianglelefteq}
\newcommand{\dominates}{\trianglerighteq}
\newcommand{\lexicoleq}{\leq}
\newcommand{\lexicogeq}{\geq}
\newcommand{\covers}{\gtrdot}
\newcommand{\coveredby}{\lessdot}

\newcommand{\cF}{\mathcal{F}}
\renewcommand{\q}{\mathfrak{q}}
\renewcommand{\P}{\mathbb{P}}
\DeclareMathOperator{\Supp}{Supp}
\DeclareMathOperator{\ch}{ch} 
\DeclareMathOperator{\gr}{gr}
\DeclareMathOperator{\Hilb}{Hilb}
\DeclareMathOperator{\wt}{wt}
\DeclareMathOperator{\wts}{\mathcal{W}}
\renewcommand{\vec}[1]{\mathbf{#1}}
\numberwithin{thm}{section}

\title[Flag Varieties]{Flag Varieties \\ Notes
  from a reading course in Spring 2019}
\author{George H. Seelinger}
\date{Spring 2010}
\begin{document}
\maketitle
\section{Introduction (presented by Weiqiang Wang)}
\section{A Quick Introduction to Vector Bundles (not presented)}
This section will list some facts about vector bundles without
proof. For a more comprehensive treatment, see references such as
\cite{bott-tu} and \cite{milnor-stasheff}. The reader may skip this
section and refer back to it as necessary. 
\begin{defn}\label{defn-of-vector-bundle}
  A rank \(n\) \de{(complex) vector bundle} \(\xi\), consists of
  \begin{enumerate}
  \item a topological space \(E\) called \de{the total space},
  \item a topolocial space \(B\) called \de{the base space},
  \item and a continuous projection map \(\rho \from E \to B\) such that
    \begin{enumerate}
    \item for all \(b \in B\), the preimage \(\pi^{-1}(b)\), which is
      called the \de{fiber over \(b\)} is a vector space, and
    \item (local triviality) for all \(b \in B\), there exists a neighborhood \(U \subset
      B\) with \(b \in U\) and a homeomorphism \(h \from U \times \C^n
      \to \rho^{-1}(U)\) (called a \de{local trivialization}) such
      that, for all \(x \in U\), \(h|_{\{x\} 
        \times \C^n} \from \C^n \to \rho^{-1}(x)\) is a vector space
      isomorphism. \[
        \begin{tikzcd}
          U \times \C^n \ar[rd, "\pi_1"] \ar[rr, "h"]& & \rho^{-1}(U)
          \ar[ld, "\rho"]\\
          & U &
        \end{tikzcd}
      \]
    \end{enumerate}
  \end{enumerate}
\end{defn}
\begin{rmk}
  One can just as easily define a \de{real vector bundle} by simply
  replacing \(\C\) with \(\R\) everywhere in the definition.
\end{rmk}
\begin{example}\label{first-examples}
  \begin{enumerate}
  \item Perhaps the most familiar example of a (smooth) real vector
    bundle is that of a ``tangent bundle'' of a manifold. That is,
    given an \(n\)-dimensional smooth manifold \(M\), we have a
    \(2n\)-dimensional manifold \(TM := 
    \{(x,v) \in M \times \R^n \st v \in T_xM\}\) where \(T_xM\) is the
    tangent space of \(x\) and map \(\pi_1 \from TM \to M\) given by
    \(\pi_1(x,v) = x\). Then, we have tangent bundle \(\tau(M)\) given
    by \[
      \begin{tikzcd}
        TM \ar[d, "\pi_1"] \\
        M
      \end{tikzcd}
    \]
    with fiber \(\pi_1^{-1}(x) = T_xM \isom \R^n\). Similarly, the
    ``normal bundle'' \(\nu(M)\) is a vector bundle. 
  \item Given a space \(X\), we have the \de{trivial bundle}
    \(\epsilon_n\) given by \[
      \begin{tikzcd}
        X \times \C^n \ar[d, "\pi_1"]\\
        X
      \end{tikzcd}
    \]
    with \(\pi_1^{-1}(x) = \C^n\). If \(X\) is a single point, then
    the trivial bundle is essentially just the vector space
    \(\C^n\). In this way, we see that vector bundles serve as a
    generalization of vector spaces.
  \item Given a M\"{o}bius strip \(M\), we have a real vector bundle \[
      \begin{tikzcd}
        M \ar[d,"\rho"] \\
        S^1
      \end{tikzcd}
    \]
    where \(\rho \from M \to S^1\) sends a point in the M\"{o}bius
    strip to the circle embedded along the ``middle''. \[
      \begin{tikzpicture}[
        decoration={
          markings,
          mark=at position 0.5 with {\arrow{>}}}
        ] 
        \draw[postaction=decorate] (0,0) -- (2,0);
        \draw (2,0) -- (2,-2);
        \draw[postaction=decorate] (2,-2) -- (0,-2);
        \draw (0,0) -- (0,-2);
        \draw[red] (1,0) -- (1,-2);
      \end{tikzpicture}
    \]
    The fibers \(\rho^{-1}(x) = \R\) which is homeomorphic to
    \((0,1)\). 
  \item Given a space \(B = \CP^n\), the points \(x \in \CP^n\) are
    lines. Then, we define the ``tautological line
    bundle'' to be given by total space \(E = \{(x,v) \in \CP^n \times
    \C^n \st v \text{ lies on }x\}\) and map \(\rho(x,v) = x\). Then,
    \(\rho^{-1}(x) = \Span_\C\{v\} \isom \C\).
  \item If we do the tautological line bundle construction over
    \(\RP^1 \isom S^1\), we recover the M\"{o}bius strip example
    above. 
  \end{enumerate}
\end{example}
\begin{defn}
  A \de{line bundle} is a rank \(1\) vector bundle.
\end{defn}
\begin{prop}\label{new-bundles-from-old}
  Given a rank \(n\) vector bundle \(\xi\) with base space \(B\) and
  total space 
  \(E\) and \(\rho \from E \to B\), we can construct new vector
  bundles  
  in a variety of ways.
  \begin{enumerate}
  \item Given some map \(f \from B' \to B\), we can construct the
    \de{pullback bundle} \(f^*(\xi)\) over \(B'\) by setting \(f^*(E)
    = \{(x',v) \in B' \times E \st f(x) = \rho(e)\}\) and \(\rho'
    \from f^*(E) \to B\) to be given by \(\rho'(x',e) = x'\) so that
    \((\rho')^{-1}(x') = \rho^{-1}(f(x'))\). This is a
    specific version of a pullback square: 
    \[
      \begin{tikzcd}
        f^*(E) \ar[d,"\rho'"] \ar[r,"\tilde{f}"] & E \ar[d,"\rho"]\\
        B' \ar[r, "f"] & B
      \end{tikzcd}
    \]

  \item We can take the \de{Whitney sum} of two vector bundles over
    \(B\) induced by taking the pullback bundle of the diagonal
    inclusion \(\iota \from B \to B \times B\). 
  \item Any operation that makes sense to be done on a vector space
    can also be done on two vector bundles over \(B\) to get a vector
    bundle over \(B\).
  \item Given a complex vector bundle \(\xi\) of rank \(n\), one can
    get a rank \(2n\) 
    real vector bundle \(\xi_\R\) by simply forgetting the complex
    structure. 
  \end{enumerate}
\end{prop}
\begin{defn}
  Given a vector bundle \(\xi\), a \de{section} is a map \(s \from B
  \to E\) such that \(\rho \circ s = Id_B\).
\end{defn}
\begin{example}
  \begin{enumerate}
  \item For every vector bundle, there exists a \de{zero section}
    \(s_0 \from B \to E\) defined by \(s_0(b) = (b,0)\).
  \item In general, there are uncountably many sections whose images
    intersect 
    the image of the zero section. These are called \de{vanishing
      sections} and are relatively uninteresting.
  \item \todo{Add something not boring.}
  \end{enumerate}
\end{example}
\begin{defn}
  In the discussion that follows, for any vector space \(V\), let
  \(V_0 = V \setminus \{0\}\) and for any total space \(E\), let \(E_0
  = E \setminus s_0(E)\) where \(s_0\) is the zero section.
\end{defn}
\begin{defn}
  \cite{milnor-stasheff}*{p 96} An \de{orientation} of a real vector
  bundle 
  \(\xi\) is a function which assigns an orientation to each fiber
  \(F\) of \(\xi\) such that for every point \(b_0 \in B\), there is a
  neighborhood \(U\) containing \(b_0\) with local
  trivialization \(h \from U \times \C^n \to \rho^{-1}(U)\) such that
  for each fiber \(F = \rho^{-1}(b)\) for \(b \in B\), the
  homomorphism \(h_b \from \C^n \to F\) given by \(h_b(x) := h(b,x)\)
  is orientation preserving. 
\end{defn}
This definition tells us that, for an orientable rank \(n\) complex
vector bundle \(\xi\), for each fiber \(F \isom \R^n\), we can pick a
preferred generator \(u_F \in H^n(F,F_0;\Z) = \Z\). Then, the local
conditions in the definition of orientation imply that, for every
\(b_0 \in B\), there is a neighborhood \(U\) and a cohomology class
\(u \in H^n(\rho^{-1}(U), \rho^{-1}(U)_0;\Z)\) so that for every fiber
\(F = \rho^{-1}(b)\) over \(U\), \[
  \iota^*(u) = u_F \in H^n(F,F_0;\Z)
\]
where \(\iota \from \rho^{-1}(b) \into \rho^{-1}(U)\).
\begin{thm}
  \cite{milnor-stasheff}*{Theorem 9.1} Let \(\xi\) be an oriented
  real vector bundle of rank \(n\) with total space \(E\). Then,
  \begin{enumerate}
  \item \(H^i(E,E_0;\Z) = 0\) for \(i < n\) and
  \item \(H^n(E,E_0;\Z)\) contains a unique cohomology class \(u\),
    called the \de{Thom class}, 
    whose restriction \[
      \iota^*(u) = u_F \in H^n(F,F_0;\Z)
    \]
    for every fiber \(F\) of \(\xi\) where \(\iota \from F \into E\)
    is the standard inclusion.
  \item The correspondence \(y \mapsto y \cupprod u\) maps
    \(H^k(E;\Z)\) isomorphically to \(H^{k+n}(E,E_0;\Z)\) for every
    integer \(k\).
  \end{enumerate}
\end{thm}
\begin{defn}
  The \de{Thom isomorphism} \[
    \phi \from H^k(B;\Z) \isomto H^k(E,E_0;\Z)
  \]
  is given by the formula \[
    \phi(x) := \rho^* x \cupprod u
  \]
\end{defn}
We will take for granted that this is an isomorphism of abelian
groups. Now, the inclusion \((E, \emptyset) \into (E, E_0)\) induces a
restriction \(H^*(E,E_0;\Z) \to H^*(E;\Z)\) which we will denote by
\(y \mapsto y|_E\). Then, for \(u \in H^n(E,E_0;\Z)\), we can take \[
  u \mapsto u|_E \in H^n(E;\Z) \isom H^n(B;\Z)
\]
via the Thom isomorphism above. Thus, we define
\begin{defn}
  The \de{Euler class} of an oriented rank \(n\) vector bundle \(\xi\)
  is the cohomology class \[
    e(\xi) \in H^n(B;\Z)
  \]
  which corresponds to \(u|_E\) under the Thom isomorphism
  \(H^n(B;\Z) \isom H^n(E;\Z)\).
\end{defn}
\begin{prop}\label{euler-class-props}
  \begin{enumerate}
  \item \cite{milnor-stasheff}*{Property 9.2} If \(f \from B \to B'\)
    is continuous and covered by an orientation preserving map \(f
    \from \xi \to \xi'\), then \(e(\xi) = f^*e(\xi')\).
  \item \cite{milnor-stasheff}*{Property 9.6} The Euler class of a
    Whitney sum is given by \(e(\xi \oplus \xi') = e(\xi) \cupprod
    e(\xi')\).
  \item \(e(\xi \otimes \xi') = e(\xi) + e(\xi')\).
  \item \cite{milnor-stasheff}*{Property 9.7} If the oriented real
    vector bundle \(\xi\) possesses a nowhere 
    zero section, then \(e(\xi)\) must be zero
  \end{enumerate}
\end{prop}
\section{Chern Classes  (presented by
  Thomas Sale)}
The material in this section will roughly correspond to
\cite{bott-tu}*{Sections 20--21}. Because of this, throughout this
section all cohomology is de Rham cohomology and thus over \(\R\). All
spaces can be assumed to be smooth manifolds.
\begin{rmk}
  As one gets more accustomed to using vector bundles, typically one
  stops denoting the bundle by something like \(\xi\) and just denotes
  it by the total space \(E\) when the bundle structure is clear.
\end{rmk}
\begin{defn}
  Given a complex line bundle \(\xi\), we define the \de{first Chern
    class of \(\xi\)} to be \(c_1(\xi) :=
  e(\xi_\R)\).
\end{defn}
\begin{prop}
  Given line bundles \(\xi, \xi'\),
  \begin{enumerate}
  \item \(c_1(\xi \otimes \xi') = c_1(\xi) + c_1(\xi')\).
  \item \(0 = c_1(\xi \otimes \xi^*) = c_1(\xi) + c_1(\xi^*) \implies
    c_1(\xi^*) = -c_1(\xi)\).
  \end{enumerate}
\end{prop}
\begin{proof}
  The first part is immediate from \ref{euler-class-props}(c) and the
  second follows from the fact that \(\xi \otimes \xi^* =
  \Hom(\xi,\xi)\) always has a nonvanishing section sending \(b \in
  B\) to the identity map in \(\Hom(E,E)\) and so \(e(L_\R) = 0\).
\end{proof}
\begin{defn}
  \begin{enumerate}
  \item Given a rank \(n\) complex vector bundle \(\xi\), we define
    the \de{projectivized bundle \(\P(\xi)\)} to be the
    bundle \[
      \begin{tikzcd}
        \P(E) \ar[d, "\pi"] \\
        B
      \end{tikzcd}
    \]
    where \(\pi^{-1}(b) = \P(\rho^{-1}(b)) \isom \P^n\). Note that
    \(\pi \circ \P = \rho\). \[
      \begin{tikzcd}
        & E \ar[d, "\rho"] \ar[ld, dashed, "\P"]\\
      \P(E) \ar[r, "\pi"] & B
      \end{tikzcd}
    \]
   \item We define the pullback vector bundle \(\pi^*(\xi)\) of the
     projectivization map 
    \(\pi \from \P(E) \to M\) and the
    fiber \((\rho')^{-1}(x) = \rho^{-1}(\pi(x))\). \[
      \begin{tikzcd}
        \pi^*(E) \ar[r,"\tilde{\pi}"] \ar[d, "\rho'"] & E
        \ar[d,"\rho"] \\ 
        \P(E) \ar[r, "\pi"] & B
      \end{tikzcd}
    \]
  \item The \de{universal sub-bundle} of \(\pi^*(\xi)\) is defined by
    taking the total space \[
      S = \{(x,v) \in \pi^*(E) \st v \in x\}
    \]
    and is a copy of the tautological line bundle over \(\P(E)\) (see
    \ref{first-examples}(d)). Note that \(S\) is a rank \(1\) vector
    bundle. 
  \item The \de{universal quotient bundle} \(Q\) of \(\pi^*(E)\) is
    defined by taking the short exact sequence \[
      0 \to S \to \pi^*E \to Q \to 0
    \]
    Then, \(Q\) is a rank \(n-1\) vector bundle.
  \item If we restrict the unviversal sub-bundle \(S\) to be over a
    fiber \(\pi^{-1}(b) \isom \P^n\), then we have a bundle with
    total space \(\tilde{S}\) which is a copy of the tautological
    line bundle over \(\P^n\).
  \end{enumerate}
\end{defn}
Now, using the fact that \(H^*(\P^n;\R) \isom \R[x]/(x^n)\) with
generator \(x = c_1(\tilde{S}^*) = -c_1(\tilde{S})\), we have
\begin{thm}[Leray-Hirsch (special case)]
  Given the setup above, \[
     H^*(B;\R) \otimes \R[1,x,\ldots,x^{n-1}] \isom H^*(\P(E);\R)
  \]
  as \(\R\)-modules.
  % given by \[
  %   \sum_{i,j} b_i \otimes i^* c_1(S)^j \mapsto \sum_{i,j} \pi^* b_i
  %   \cupprod 
  % \]
  In other words, the cohomology of
  \(H^*(\P(E);\R)\) is a free module over \(H^*(M;\R)\) with basis
  \(\{1,x, \ldots, x^{n-1}\}\). 
\end{thm}
\begin{rmk}
  For a gentle treatment on Leray-Hirsch in generality, see
  \cite{hatcher}*{Section 4.D}.
\end{rmk}
From this fact, we can write \(x^n \in H^*(P(E);\R)\) as a linear
combination with coefficients in \(H^*(M)\).
\begin{defn}
  \begin{enumerate}
  \item The \de{Chern classes} of the bundle \(\xi\) with data
    \(\rho \from E \to B\) are elements \(c_i(\xi) \in H^*(B;\R)\)
    which are the unique coefficients such that
    \[ x^n + \pi^* c_1(\xi) x^{n-1} + \cdots + \pi^* c_n(\xi) = 0 \in
      H^*(\P(E);\R)
    \]
    Thus, \(c_i(\xi) \in H^{2i}(B;\R)\). \todo{Understand this better.}
  \item We define the \de{total Chern class} to be \[
      c(\xi) := 1 + c_1(\xi) + \cdots + c_n(\xi) \in H^*(B;\R)
    \]
  \end{enumerate}
\end{defn}

\begin{bibdiv}
  \begin{biblist}
    \bib{bott-tu}{book}{
      author={Bott, Raoul}
      author={Tu, Loring W.}
      title={Differential Forms in Algebraic Topology}
      year={1982}
    }
    \bib{hatcher}{book}{
      author={Hatcher, Allen}
      title={Algebraic Topology}
      year={2001}
    }
    \bib{milnor-stasheff}{book}{
      author={Milnor, John W.}
      author={Stasheff, James D.}
      title={Characteristic Classes}
      year={1974}
    }
  \end{biblist}
\end{bibdiv}
\end{document}