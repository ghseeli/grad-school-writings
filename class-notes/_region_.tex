\message{ !name(sheaves.tex)}\documentclass[11pt,leqno,oneside]{amsbook}
\usepackage{tikz}
\usetikzlibrary{cd}
\usepackage{bbm}
\usepackage{ytableau}
\usepackage{todonotes}

\usepackage{./notes}
\usepackage{../ReAdTeX/readtex-core}
\usepackage{../ReAdTeX/readtex-abstract-algebra}

\newcommand{\bbk}{\mathbbm{k}}
\newcommand{\Class}{\operatorname{Class}}
\newcommand{\Res}{\operatorname{Res}}
\newcommand{\Ind}{\operatorname{Ind}}
\newcommand{\bs}{\textbackslash}
\newcommand{\partitionof}{\vdash}
\newcommand{\T}{\mathsf{T}} % Tableau
\renewcommand{\S}{\mathsf{S}}
\renewcommand{\F}{\mathcal{F}}

\numberwithin{thm}{section}

\title[Theory of Sheaves]{Theory of Sheaves \\ Notes
  inspired by a class taught by Andrei Rapinchuk in Fall 2018}
\author{George H. Seelinger}
\date{Fall 2018}
\begin{document}

\message{ !name(sheaves.tex) !offset(-3) }

\maketitle
\section{Presheaves and Sheaves}
\begin{defn}
  Let \(X\) be topological space. A \de{presheaf of sets} \(\F\) on
  \(X\) is given by the following data.
  \begin{enumerate}
  \item For each open set \(U \subset X\), \(\F(U)\) is a set.
  \item If \(V \subset U\), there exists a map \(\rho^U_V \from \F(U)
    \to \F(V)\) such that
    \begin{enumerate}
    \item \(\rho_U^U = id_{\F(U)}\)
    \item If \(W \subset V \subset U\), then \(\rho_W^U = \rho_W^V
      \circ \rho_V^U\). In other words, \[
        \begin{tikzcd}
          \F(U) \ar[rd, "\rho_V^U"] \ar[rr, "\rho_W^U"] & & \F(W)\\
          & \F(V) \ar[ur, "\rho_W^V"]&
        \end{tikzcd}
      \]
    \end{enumerate}
  \end{enumerate}
\end{defn}

\message{ !name(sheaves.tex) !offset(7) }

\end{document}