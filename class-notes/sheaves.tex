\documentclass[11pt,leqno,oneside]{amsbook}
\usepackage{tikz}
\usetikzlibrary{cd}
\usepackage{bbm}
\usepackage{ytableau}
\usepackage{todonotes}

\usepackage{./notes}
\usepackage{../ReAdTeX/readtex-core}
\usepackage{../ReAdTeX/readtex-abstract-algebra}

\newcommand{\bbk}{\mathbbm{k}}
\newcommand{\Class}{\operatorname{Class}}
\newcommand{\Res}{\operatorname{Res}}
\newcommand{\Ind}{\operatorname{Ind}}
\newcommand{\bs}{\textbackslash}
\newcommand{\partitionof}{\vdash}
\newcommand{\T}{\mathsf{T}} % Tableau
\renewcommand{\S}{\mathsf{S}}
\renewcommand{\F}{F}
\newcommand{\cF}{\mathcal{F}}
\newcommand{\m}{\mathfrak{m}}
\newcommand{\Dom}{\operatorname{Dom}}

\numberwithin{thm}{section}

\title[Theory of Sheaves]{Theory of Sheaves \\ Notes
  inspired by a class taught by Andrei Rapinchuk in Fall 2018}
\author{George H. Seelinger}
\date{Fall 2018}
\begin{document}
\maketitle
\section{Presheaves and Sheaves}
\begin{defn}
  Let \(X\) be topological space. A \de{presheaf of sets} \(\F\) on
  \(X\) is given by the following data.
  \begin{enumerate}
  \item For each open set \(U \subset X\), \(\F(U)\) is a set.
  \item If \(V \subset U\), there exists a map \(\rho^U_V \from \F(U)
    \to \F(V)\) such that
    \begin{enumerate}
    \item \(\rho_U^U = id_{\F(U)}\)
    \item If \(W \subset V \subset U\), then \(\rho_W^U = \rho_W^V
      \circ \rho_V^U\). In other words, \[
        \begin{tikzcd}
          \F(U) \ar[rd, "\rho_V^U"] \ar[rr, "\rho_W^U"] & & \F(W)\\
          & \F(V) \ar[ur, "\rho_W^V"]&
        \end{tikzcd}
      \]
    \end{enumerate}
  \end{enumerate}
\end{defn}
\begin{rmk}
  \begin{enumerate}
  \item Sometimes \(\F(\emptyset) = \{e\}\), the singleton set, is
    included in the 
    definition, but it is also a formal consequence of the above
    statements.
  \item A presheaf of sets can also be defined as a contravariant
    functor from \(\catname{Top}(X) \to \catname{Set}\) where
    \(\catname{Top}(X)\) is the category with objects being the open
    sets of \(X\) and the morphisms being \[
      \Hom(V,U) =
      \begin{cases}
        \{\iota \from V \into U\} & V \subset U\\
        \emptyset & V \not\subset U
      \end{cases}
    \]
  \end{enumerate}
\end{rmk}
Informally, we can think of presheaves as a collection of
functions. Historically, sheaves were thought in the context of some
``\'{e}tale space'', \(\cF\), mapping to \(X\) and then considering
the set of sections over \(U \subset X\). \[ 
  \begin{tikzcd}
    \cF \ar[r] & X\\
    & \ar[ul, "\sigma \in \F(U)"] U
  \end{tikzcd}
\]
This leads to the terminology of a ``sheaf of sections.'' Thus, we
often call elements of \(\F(u)\) ``sections of \(\F\) over \(U \subset
X\).'' Similarly, one can think of \(\rho_V^U\) as a restriction map.
\begin{example}
  Throughout, let \(X\) be a topological space.
  \begin{enumerate}
  \item Fix another topological space \(Y\). Then, for \(U \subset
    X\), let \[
      F(U) = \{\phi \from U \to Y \st \phi \text{ is continuous} \}
      \text{ and } F(\emptyset) = \{e\} 
    \]
    Then, for \(V \subset U\), we define \(\rho_V^U \from F(U) \to
    F(V)\) via \[
      (\phi \from U \to Y) \mapsto (\phi |_V \from V \to Y)
    \]
    In fact, if we take \(Y=\R\) or \(\C\), then we would get a
    presheaf of groups or even rings.
  \item If we take a set \(S\) and set \[
      F(U) = \{\text{Constant functions }\phi \from U \to S\}
    \]
    we get the \de{constant presheaf}. We can modify this slightly by
    instead letting \(F(U)\) be all locally constant functions from
    \(U\).
  \item Given \(Y\) as another topological space, recall that, given a
    map \(\pi \from Y \to X\), a section \(\sigma \from X \to Y\) is a
    continuous map such that \(\sigma \circ \pi = id_X\). Then, we can
    define the \de{presheaf of sections} by defining, for
    \(U \subset X\), \[
      F(U) = \{\phi \from U \to Y \text{ continuous} \st \pi \circ
      \phi = id_U\}
    \]
  \item Given any presheaf \(F\) on \(X\), we can restrict it to an
    open subset \(W \subset X\) by taking only the \(F(U)\) where \(U
    \subset W\).
  \item Fix a point \(p \in X\) and a set \(S\). Then, we can define
    the \de{skyscraper presheaf} by saying, for open \(U \subset
    X\), \[
      F(U) =
      \begin{cases}
        S & \text{if }p \in U\\
        \{e\} & \text{if }p \not\in U
      \end{cases}
    \]
    Then, if \(p \in V \subset U\), \(\rho^U_V = id_S\) and if \(p
    \not \in V\), then there is a unique map \(\rho^U_V \from F(U) \to
    \{e\}\).
  \end{enumerate}
\end{example}
Let \(K\) be an algebriacally closed field (eg \(K=\C\)).
\begin{defn}
  Given an ideal \(I \ideal K[x_1, \ldots, x_n]\), define the
  \de{vanishing set} \[
    V(I) := \{(a_1, \ldots, a_n) \in K^n \st f(a_1, \ldots, a_n) = 0
    \forall f \in I\}
  \]
  We call a set \(S \subset K^n\) for which there exists an ideal \(I
  \ideal K[x_1, \ldots, x_n]\) such that \(S = V(I)\) an \de{algebraic
  set}.
\end{defn}
Then, there is a topology called the \de{Zariski topology} on \(K^n\)
for which sets of the form 
\(V(I)\) form the family of all closed sets.
\begin{defn}
  If \(X = V(I)\) is an algebraic set, a function \(f \from X \to K\)
  is \de{regular} if there exists a polynomial \(p(x_1, \ldots, x_n)
  \in K[x_1, \ldots, x_n]\) such that \(f(x) = p(x)\) for all \(x \in
  X\). (In other words, there is a polynomial \(p\) such that \(p|_X =
  f\).)
\end{defn}
All such functions form a ring, called the \de{coordinate ring},
denoted \(K[X]\). There is a natural map
\begin{align*}
  K[x_1, \ldots, x_n] & \onto K[X] \\
  p & \mapsto p|_X
\end{align*}
which gives us
\begin{prop}
  The coordinate ring \(K[X] \isom K[x_1, \ldots, x_n]/I(X)\) where
  \(I(X) = \{p \st p|_X = 0\}\).
\end{prop}
Recall, however, that \(X = V(I)\) for some \(I \ideal K[x_1, \ldots,
x_n]\). One can check that \(I \subset I(X) = I(V(I))\), but this
containement is not necessarily an equality due to the following example.
\begin{example}
  Let \(I = (x_1^2)\). Then, \(X = V(I) = \{(a_1, \ldots, a_n) \in K^n \st
  a_1 = 0\}\). From this, we see that any \(p \in x_1 K[x_1, \ldots,
  x_n]\) has the property that \(p|_X = 0\). Thus, \(I(X) = (x_1)\).
\end{example}
One may then ask how \(I(X)\) is related to \(X\). The answer is given
by the following famous theorem.
\begin{thm}[Hilbert's Nullstellensatz]
  Let \(K\) be an algebraically closed field. Then, for \(J \ideal
  K[x_1, \ldots, x_n]\), we have that \[
    I(V(J)) = r(J)
  \]
  where \(r(J) = \{s \in K[x_1, \ldots, x_n] \st s^m
  \in J \text{ for some }m \in \N\}\) is the ``radical'' of \(J\).
\end{thm}
There are other equivalent formulations of Hilbert's Nullstellensatz,
such as \todo{Check that this is actually equivalent.}
\begin{prop}
 Given proper ideal \(I \propsubset K[x_1, \ldots, x_n]\), we have
 that \(V(I) \neq 
 \emptyset\). 
\end{prop}
Furthermore, \(K[X]\) allows us to recover \(X\) in a functorial way,
since we have
\begin{cor}
  For \(K\) an algebraically closed field, we have that
  \begin{enumerate}
  \item all maximal ideals \(\m \ideal K[x_1, \ldots, x_n]\) are of
    the form \(\m = (x_1-a_1, \ldots, x_n - a_n)\) and
  \item \(V(\m) = \{(a_1, \ldots, a_n)\}\).
  \end{enumerate}
\end{cor}
Thus, we can recover all the points of \(X\) by applying \(V(\cdot)\)
to the maximal ideals of \(X\).
\begin{defn}
  A topological space \(X\) is called \de{irreducible} if \(X \neq X_1
  \union X_2\) for \(X_1,X_2\) proper closed sets of \(X\).
\end{defn}
\begin{rmk}
  Note that this notion is relatively uninteresting for \(T_2\)
  topological spaces since any non-trivial \(T_2\) space is
  reducible. As such, this notion is rarely used outside algebraic
  geometry. 
\end{rmk}
\begin{prop}
  A space \(X \subset K^n\) is irreducible if and only if \(I(X)\) is
  a prime 
  ideal if and only if \(K[X]\) is an integral domain.
\end{prop}
Using Hilbert's Basis Theorem, we also get 
\begin{prop}
  Every algebraic set is a finite union of irreducible algebraic sets.
\end{prop}
\begin{prop}
  If \(f \in K[x_1, \ldots, x_n]\) is irreducible, then \(X = V((f))\)
  is irreducible.
\end{prop}
\begin{proof}
  By Hilbert's Nullstellensatz, we have that \[
    I(V((f))) = r((f))
  \]
  and so, \(g \in r((f)) \implies f \divides g^m\) and because \(f\)
  is irreducible, this gives that \(f \divides g \implies g \in
  (f)\). Thus, it must be that \((f)\) is a prime ideal and so
  \(V((f))\) is irreducible.
\end{proof}
\begin{defn}
  If \(X \subset K^n\) is irreducible, then \(K[X]\) is an integral
  domains and so we define \(K(X)\) to be the fraction field of
  \(K[X]\), also referred to as the \de{field of rational functions}.
\end{defn}
While it is nice that we have this definition, we have a fundamental
problem because \(f \in K(X)\) does not have a \emph{canonical}
presentation as \(f = \frac{g}{h}\), and so we can run into problems
if one choice of \(h\) is zero at a point \(x\). Thus, we must do a
little work to 
address this problem.
\begin{defn}
  Given \(f \in K(X)\) and \(x \in X\), we say \(f\) is \de{defined at
  \(x\)} if there is \(g_x, h_x \in K[X]\) such that \(f =
  \frac{g_x}{h_x}\) and \(h_x \neq 0\). Furthermore, we define \[
    \Dom(f) := \{x \in X \text{ where }f\text{ is defined}\}
  \]
\end{defn}
However, it is not clear that \(f\) is well-defined on the
domain. Indeed, we check
\begin{proof}
  Let \(f = \frac{g_1}{h_1} = \frac{g_2}{h_2}\) such that \(h_1(x),
  h_2(x) \neq 0\). Then, this gives\[
    g_1 h_2 = g_2 h_1 \implies g_1(x) h_2(x) = g_2(x) h_1(x) \implies
    \frac{g_1(x)}{h_1(x)} = \frac{g_2(x)}{h_2(x)}
  \]
\end{proof}
\begin{prop}
  The domain of a rational function \(f\) is a non-empty Zariski open set.
\end{prop}
This follows immediately by considering that a Zariski closed set is
of the form \(V(I)\) and so an open set is of the form \(K^n \setminus
V(I) = \Union_{p \in I} D(p)\) where
\begin{defn}
  \[
    D(p) := \{x \in K^n \st p(x) \neq 0\}
  \]
  is called the \de{principal (distinguished) open set defined by \(p\)}.
\end{defn}
\begin{prop}
  \begin{enumerate}
  \item \(D(p_1) \union D(p_2) = D(p_1 p_2)\)
  \item The \(D(p)\)'s form a base for the Zariski topology.
  \end{enumerate}
\end{prop}
\begin{bibdiv}
  \begin{biblist}
  \end{biblist}
\end{bibdiv}
\end{document}