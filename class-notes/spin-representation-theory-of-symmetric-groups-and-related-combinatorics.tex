\documentclass[11pt,leqno,oneside]{amsbook}
\usepackage{tikz}
\usetikzlibrary{cd}
\usepackage{bbm}
\usepackage{ytableau}
\usepackage{todonotes}

\usepackage{./notes}
\usepackage{../ReAdTeX/readtex-core}
\usepackage{../ReAdTeX/readtex-dangerous}
\usepackage{../ReAdTeX/readtex-abstract-algebra}

\newcommand{\bbk}{\mathbbm{k}}
\newcommand{\Class}{\operatorname{Class}}
\newcommand{\Res}{\operatorname{Res}}
\newcommand{\Ind}{\operatorname{Ind}}
\newcommand{\bs}{\textbackslash}
\newcommand{\partitionof}{\vdash}
\newcommand{\T}{\mathsf{T}} % Tableau
\renewcommand{\S}{\mathsf{S}}
\newcommand{\sh}{\operatorname{shape}}
\newcommand{\grdim}{\boldsymbol{\dim}}

\newcommand{\dominatedby}{\trianglelefteq}
\newcommand{\dominates}{\trianglerighteq}
\newcommand{\lexicoleq}{\leq}
\newcommand{\lexicogeq}{\geq}
\newcommand{\covers}{\gtrdot}
\newcommand{\coveredby}{\lessdot}

\renewcommand{\H}{\mathcal{H}}
\renewcommand{\Q}{\mathcal{Q}}
\newcommand{\Cl}{\mathcal{C\ell}}
\numberwithin{thm}{section}

\title[Spin Representation Theory of Symmetric Groups]{Spin
  Representation Theory of Symmetric Groups and Related Combinatorics \\ Notes
  from a reading course in Fall 2018}
\author{George H. Seelinger}
\date{Fall 2018}
\begin{document}
\maketitle
\section{Introduction (presented by Jinkui Wan)}
When discussing the representation theory of the symmetric group, one
considers \emph{linear representations} which are group
homomoprhisms \[
  \Sym_n \to GL(V)
\]
In 1911, Schur started considering projective representations \[
  \Sym_n \to PGL(V) = GL(V)/\C^*
\]
leading to the projective representation theory of \(\Sym_n\). It
turns out that this corresponds to the linear representation theory of
an extension of \(\Sym_n\), denoted \(\tilde{\Sym}_n\) and referred to as
the \de{double cover of the symmetric group}, fitting into the
short exact sequence \[
  1 \to \Z/2\Z \to \tilde{\Sym}_n \to \Sym_n \to 1
\]
where, if \(\Z/2\Z = \{1,z\}\), then \(z\) is central in
\(\tilde{\Sym}_n\), which gives us that \(z = 1\) or \(z=-1\).

When \(z=1\), we have the representation theory of \(\Sym_n\). When,
\(z=-1\), we have the representation theory of the \emph{spin
  symmetric group algebra}
\[
\C\Sym_n^- = \C\Sym_n/\langle z+1 \rangle = \left\langle t_1, \ldots, t_n
\st
\begin{array}{l}
  t_i^2=1\\
  t_i t_{i+1} t_i = t_{i+1} t_i t_{i+1}\\
  t_i t_j = -t_j t_i \text{ when }|i-j|>1
\end{array}
\right\rangle
\]
which is equipped with a \(\Z/2\Z\)-grading. So, when we discuss
spin representations of \(\Sym_n\), we are discussing linear
representations of \(\C\Sym_n^-\). Our program to establish these
ideas is as follows. \\

Part I
\begin{enumerate}[label=(\arabic*)]
\item Basics of associative superalgebras
\item Connection to Hecke-Clifford (or Sengeev) algebra, \(\H_n\)
\item Split conjugacy classes in a finite supergroup
\item Characteristic map
\item Schur-\(Q\) functions
\item Schur-Sergeev duality
\item Seminormal form of irreducible representations
\end{enumerate}
Part II
\begin{enumerate}[label=(\arabic*)]
\item Centers of \(\C\Sym_n^-\) (analog of Farahat-Higman theory for
  \(\C\Sym_n\))
\item Coinvariant theory for \(\C\Sym_n^-\)
\item Spin Kostka polynomials
\item Quantum deformation (in particular, Olshanki-Sergeev duality)
\end{enumerate}
\section{Generalities for Associative Superalgebras}
\subsection{Definitions and Examples}
\begin{defn}
  \begin{enumerate}
  \item   A \de{vector superspace} (over \(\C\)) is a
    \(\Z/2\Z\)-graded vector space 
    \(V = V_{\ov{0}} \oplus V_{\ov{1}}\), where elements of
    \(V_{\ov{0}}\) are called \de{even} and elements of \(V_{\ov{1}}\)
    are called \de{odd}. For \(v \in V_i\), \(i \in \Z/2\Z\), we say
    \(|v| = i\).
  \item If \(V\) is a vector superspace with \(\dim V_{\ov{0}} = m\)
    and \(\dim V_{\ov{1}} = n\), we say the \de{graded dimension} of
    \(V\) is \((m,n)\), denoted \(\grdim V = (m,n)\).
  \item A \de{superalgebra} is a \(\C\)-algebra \(A\) with a
    \(\Z/2\Z\)-grading \(A = A_{\ov{0}} \oplus A_{\ov{1}}\) such that
    \(A_i A_j \subset A_{i+j}\) for all \(i,j \in \Z/2\Z\).
  \item A \de{superalgebra ideal} is a homogeneous ideal, that is, a
    subset \(I \subset A\) such that \(I = I_{\ov{0}} \oplus
    I_{\ov{1}}\) as vector spaces and \(A_i I_j \subset I_{i+j}\) for all \(i,j \in
    \Z/2\Z\). 
  \item A superalgebra that has no non-trivial ideals is called
    \de{simple}.
  \item A \de{superalgebra homomorphism} \(\theta \from A \to B\) is
    an even algebra homomorphism, that is, an algebra homomorphism
    sending \(A_i \to B_i\) for all \(i \in \Z/2\Z\).
  \item Given superalgebras \(A\) and \(B\), the tensor product \(A
    \otimes B\) is a superalgebra with multiplication \[
      (a \otimes b)(a' \otimes b') = (-1)^{|a||b|} aa' \otimes bb'
    \]
    for homogeneous elements and extended by linearity.
  \item A \de{commutative superalgebra} is one that is graded
    commutative, that is \[
      yx = (-1)^{|x||y|}xy
    \]
    Thus, the \de{supercommutator} of a superalgebra is given by \[
      [x,y] = xy-(-1)^{|x||y|}yx
    \]
    and the \de{supercenter} is given by \[
      Z(A) = \{a \in A \st [a,x] = 0 \text{ for all }x \in A\}
    \]
    which is different than the center of an ungraded algebra.
  \item Given a superalgebra \(A\), we let \(|A|\) be the associative
    algebra where we forget the grading on \(A\). 
  \end{enumerate}
\end{defn}
\begin{example}
  \begin{enumerate}
  \item Let \(V = V(m|n)\), the vector superspace with \(\grdim V =
    (m,n)\). Then, \(\End_\C(V)\) is a superalgebra and is isomorphic
    to the matrix superalgebra \[
      M(m|n) := \left\{ \left(
          \begin{array}{cc}
            a&b\\
            c&d
          \end{array}
\right) \st
\begin{array}{c}
  a \text{ is an }m \times m \text{ matrix}\\
  b \text{ is an }m \times n \text{ matrix}\\
  c \text{ is an }n \times m \text{ matrix}\\
  d \text{ is an }n \times n \text{ matrix}
\end{array}
\right\}
\]
or, in other words, \(M(m|n)\) consists of all \(m|n\)-block matrices
and has \(\grdim M(m|n) = (m^2+n^2, 2mn)\). Furthermore, \(M(m|n)\) is
a simple superalgebra since \(|M(m|n)|\) is simple as a \(\C\)-algebra.
\item Let \(V = V(n|n)\) and \(p \in \End_\C(V)\) be an odd involution
  (that is, it sends \(V_i \to V_{i+1}\) for \(i \in \Z/2\Z\)). Then,
  we define \[
    \Q(V) := \{f \in \End_\C(V) \st fp = (-1)^{|f|}pf\} =
    \Q(V)_{\ov{0}} \oplus \Q(V)_{\ov{1}}
  \]
  \(\Q(V)\) is also a superalgebra. Moreover, if we pick a basis
  \(\{v_1, \ldots, v_n\}\) of \(V_{\ov{0}}\) and let \(v_i' = p(v_i)\)
  for \(1 \leq i \leq n\), we have that, with respect to the basis
  \(\{v_1, \ldots, v_n, v_1', \ldots, v_n'\}\), \(\Q(V)\) is
  isomorphic to \[
    \Q(n) := \left\{ \left(
        \begin{array}{cc}
          a&b\\
          -b&a
        \end{array}
      \right) \in M(n|n) \right\}
  \]
  and it is simple.
  \item The Clifford algebra \(\Cl_n\) is the superalgebra generated
    by the odd elements \(c_1, \ldots, c_n\) subject to the
    relations \[
      \begin{cases}
        c_i^2 = 1\\
        c_i c_j = - c_j c_i & \forall 1 \leq i \neq j \leq n
      \end{cases}
    \]
  \end{enumerate}
\end{example}
\begin{lem}
  There exist isomorphisms of superalgebras
  \begin{enumerate}
  \item \(M(m|n) \otimes M(k|l) \isom M(mk+nl|mk+nl)\)
  \item \(M(m|n) \otimes \Q(k) \isom Q((m+n)k)\)
  \item \(\Q(m) \otimes \Q(n) \isom M(mn|mn)\)
  \end{enumerate}
\end{lem}
\begin{proof}
  For part (a), we note that \[
    \End_\C(V(m|n)) \otimes \End_\C(V(k|l)) \isom \End_\C(V(mk+ml | mk+nl))
  \]
  under the isomorphism sending \(f \otimes g\) to the endomorphism of
  \(V(mk+ml | mk+nl)\) mapping \(v \otimes w\) to \((-1)^{|g||v|}f(v)
  \otimes g(w)\).

  For part (b), we have \[
    \End(V(m|n)) \otimes \Q(V(k|k),p) \isom Q(V(m|n) \otimes V(k|k), id
    \otimes p)
  \]

  For (c), one explicitly checks that \(\Q(1) \otimes \Q(1) \isom
  M(1|1)\) and then inductively applies (a) and (b) above.
\end{proof}
\begin{cor}
   Since \(\Cl_{m+n} \isom \Cl_m \otimes \Cl_n\) under the isomorphism
    sending generators \(c_1, \ldots, c_n\) to \(c_1 \otimes 1,
    \ldots, c_n \otimes 1\) and \(c_{n+1}, \ldots, c_{n+m}\) to \(1
    \otimes c_1, \ldots, 1 \otimes c_m\), we have the corollaries
  \begin{enumerate}
  \item \(\Cl_1 \isom \Q(1)\) under the isomorphism \(c_1 \mapsto
    p(v_1)\)
  \item \(\Cl_2 \isom M(1|1)\) since \(\Cl_2 \isom \Cl_1 \otimes \Cl_1
    \isom \Q(1) \otimes \Q(1) 
    \isom M(1|1)\)
  \item \(\Cl_{2k} \isom M(2^{k-1}|2^{k-1})\)
  \item \(\Cl_{2k-1} \isom \Q(2^{k-1})\)
  \item and thus, \(\Cl_n\) is simple by parts (c) and (d).
  \end{enumerate}
\end{cor}
\subsection{Classification of Simple Superalgebras}
\begin{thm}
  There are two types of finite dimensional simple associative
  superalgebras over \(\C\):
  \begin{enumerate}
  \item \(M(m|n)\)
  \item \(\Q(n)\)
  \end{enumerate}
\end{thm} 
\subsection{Wedderburn Theorem and Schur's Lemma}
\begin{defn}
  \begin{enumerate}
  \item   A \de{(super)module} over a superalgebra \(A\) is a vector space \(M
  = M_{\ov{0}} \oplus M_{\ov{1}}\) with a left action of \(A\) on
  \(M\) such that \(A_i M_j \subset M_{i+j}\) for all \(i,j \in \Z/2\Z\).
  \item A homomorphism between \(A\)-modules \(M\) and \(N\) is a
    linear map \(f \from M \to N\) such that \[
      f(am) = (-1)^{|f||a|} a f(m) \text{ for all }a \in A, m \in M
    \]
    and \[
      \Hom_A(M,N) := \Hom_A(M,N)_{\ov{0}} \oplus \Hom_A(M,N)_{\ov{1}}
    \]
    where \(f \in \Hom_A(M,N)_{\ov{1}} \subset \Hom_\C(M,N)\) is such that
    \(f(M_i) \subset N_{i+1}\) for all \(i \in \Z/2\Z\).
  \end{enumerate}
\end{defn}
\begin{defn}
  An \(A\)-module is said to be \de{simple} if it is nonzero and has no
  proper \(A\)-submodules. An \(A\)-module \(M\) is said to be
  \de{semisimple} if every \(A\)-submodule of \(M\) is a direct
  summand of \(M\).
\end{defn}
\begin{thm}[Super Wedderburn Theorem]
  The following are equivalent for a finite dimensional superalgebra \(A\).
  \begin{enumerate}
  \item Every \(A\)-module is semisimple
  \item \(A\) is a finite direct sum of left simple superideals
  \item \(A\) is a direct product of a finite number of simple algebras
  \end{enumerate}
\end{thm}
\begin{defn}
  Thus, we say a superalgebra \(A\) is \de{semisimple} if it satisfies
  one of the three conditions.
\end{defn}
\begin{example}
  \begin{enumerate}
  \item \(M(m|n) = I_1 \oplus I_2 \oplus \cdots \oplus I_m \oplus
    I_{m+1} \oplus \cdots \oplus I_{m+n}\) where \(I_k =
    M(m|n)E_{k,k}\) for \(1 \leq k \leq m+n\).
  \item \(\Q(n) = J_1 \oplus \cdots \oplus J_n\) where \[
      J_k = \Q(n) (E_{k,k} + E_{n+k,n+k})
    \]
    \todo{Check this. Most likely depends on your choice of basis and
      involution.}
  \item \(\Hom_{M(m|n)}(I_k, I_k) \isom \C\) and \(\Hom_{\Q(n)}(J_k,
    J_k) \isom \C \oplus \C p\). Importantly, the latter space is not
    \(1\)-dimensional despite \(J_k\) being \(1\)-dimensional!
  \end{enumerate}
\end{example}
\begin{cor}
   A finite dimensional semisimple superalgebra \(A\) is isomorphic
   to \[
     A \isom \bigoplus_{i=1}^m M(r_i|s_i) \oplus \bigoplus_{j=1}^n \Q(n_j)
   \]
   where \(m = m(A)\) and \(q=q(A)\) are invariants of \(A\).
 \end{cor}
 \begin{defn}
   A simple \(A\)-module \(V\)is said to be of type M (resp. type Q)
   if it is annihilated by all but one summand of the form
   \(M(r_i|s_i)\) (resp. \(\Q(n_j)\)).
 \end{defn}
 \begin{cor}
   \begin{enumerate}
   \item The number of non-isomorphic simple \(A\)-modules is given by
     \(m(A)+q(A) = \dim (Z(|A|) \intersect A_{\ov{0}})\).
   \item The number of non-isomorphic simple \(A\)-modules of type
     \(Q\) is given by \(q(A) = \dim(Z(|A|) \intersect A_{\ov{1}})\).
   \end{enumerate}
 \end{cor}
 \begin{thm}[Schur's Lemma]
   If \(M\) and \(L\) are simple \(A\)-modules, then \[
     \dim\Hom_A(M,L) =
     \begin{cases}
       1 & \text{ if }M \isom L\text{ of type M}\\
       2 & \text{ if }M \isom L\text{ of type Q}\\
       0 & \text{ otherwise}
     \end{cases}
   \]
\end{thm}
\begin{rmk}
  \begin{enumerate}
  \item A simple \(A\)-module \(M\) is of type M if and only if
    \(|M|\) is a simple \(|A|\)-module
  \item A simple \(A\)-module \(M\) is of type Q if and only if
    \(|M|\) is a direct sum of two non-isomorphic simple \(|A|\)-modules.
  \end{enumerate}
\end{rmk}
\section{Split conjugacy classses in a finite supergroup}
Throughout, let \(G\) be a finite group with index \(2\) subgroup
\(G_0 \subgroup G\).
\begin{defn}
  \begin{enumerate}
  \item We say that the elements of \(G_0\) are \de{even elements} and the
  elements of \(G_1 := G \setminus G_0\) are \de{odd elements},
  \item \(\C G\) is a superalgebra, which we will denote \(\C[G,G_0]\)
  \end{enumerate}
\end{defn}
\begin{thm}[Super MAschke's Theorem]
  \(\C[G,G_0]\) is semisimple
\end{thm}
\begin{prop}
  \begin{enumerate}
  \item If \(g \in G_i\), \(h \in G\), then \(hgh^{-1} \in G_i\) for
    all \(i \in \Z/2\Z\).
  \item The number of non-isomorphic simple \(\C[G,G_0]\)-modules is
    equal to the number of even conjugacy classes in \(G\).
  \item The number of non-isomorphic simple \(\C[G,G_0]\)-module of
    type \(Q\) is equal to the number of odd conjugacy classes in \(G\).
  \end{enumerate}
\end{prop}
\begin{proof}
  We note that, by the usual Artin-Wedderburn theorem, \(\C G\)
  decomposes into a direct sum of simple matrix algebras, each of
  which has a \(1\)-dimensional center and can be indexed by a
  conjugacy class of \(G\) via \(c_i = \sum_{g \in C_i} g\) where
  \(C_i\) is a conjugacy class of \(G\). In fact, this shows in the
  classical theory that the number of conjugacy classes of \(G\) equal
  the number of irreducible representations.
  
  Since conjugacy classes are either even or odd by (a), which is left
  as an exercise, (b) follows
  because \(\dim (Z(\C G) \intersect 
  \C[G,G_0]_{\ov{0}})\) is equal to the number of non-isomorphic
  simple \(\C[G,G_0]\)-modules and (c) follow from the fact that \(\dim(Z(\C G)
  \intersect \C[G,G_0]_{\ov{1}})\) gives the number of those of type \(Q\).
\end{proof}
Now, consider the following situation. Let \(\tilde{G}\) be a group
such that there exists an index \(2\) subgroup \(\tilde{G}_0 \subgroup
\tilde{G}\) and there exists a short exact sequence \[
  1 \to \{1,z\} \to \tilde{G} \to[\theta] G \to 1
\]
where \(z^2 = 1\) and \(z\) is central in \(\tilde{G}\). Then,
\begin{prop}
  For \(C\) a conjugacy class of \(G\), the preimage \[
    \theta^{-1}(C) = \{g, gz \st g \in C\} \subset \tilde{G}
  \]
  has that 
  \begin{enumerate}
  \item \(\theta^{-1}(C)\) is a single conjugacy class in \(\tilde{G}\) if \(g\) is
    conjugate to \(zg\) in \(\tilde{G}\) or
  \item \(\theta^{-1}(C)\) splits into two conjugacy classes in
    \(\tilde{G}\) if there exists a \(g \in C\) such that \(g\) is not
    conjugate to \(zg\). In this case, we call \(C\) \de{split}.
  \end{enumerate}
\end{prop}
\begin{defn}
  We set \[
    \C\tilde{G}^- := \C[G,G_0]/\langle z+1 \rangle
  \]
  and call a \(\C\tilde{G}^-\)-module a \de{spin
    \(\C[\tilde{G},\tilde{G}_0]\)}-module. 
\end{defn}
\begin{prop}
  \begin{enumerate}
  \item   We have the isomorphism of superalgebras 
  \[\C[\tilde{G},\tilde{G}_0] \isom \underbrace{\C[G,G_0]}_{(z=1)}
    \oplus \underbrace{\C\tilde{G}^-}_{(z=-1)} \]
  \item The number of non-isomorphic simple spin
    \(\C[\tilde{G},\tilde{G}_0]\)-modules is equal to the number of
    even split conjugacy classes of \(G\).
  \item The number of non-isomorphic simple spin
    \(\C[\tilde{G},\tilde{G}_0]\)-modules of type Q is equal to the
    number of odd split conjugacy classes of \(G\).
  \end{enumerate}
\end{prop}
\begin{proof}
  Part (a) follows from the semisimplicity of
  \(\C[\tilde{G},\tilde{G}_0]\).

  Now, (a) tells us that \[
    Z(|\C\tilde{G}^-|) = \{a \in Z(|\C[\tilde{G},\tilde{G}_0]|) \st za
    = -a\}
  \]
  So let \[
    \underbrace{D_1, zD_1, D_2, zD_2, \ldots, D_r, zD_r}_{\text{split}}, \underbrace{D_{r+1}, \ldots,
  D_{r+s}}_{\text{non-split}}
  \] be the conjugacy classes of \(\tilde{G}\) where \(r\) is the
  number of split conjugacy classes in \(G\),  \(D_i \intersect zD_i =
  \emptyset\) for \(1 \leq i \leq r\) and \(zD_j = D_j\) for \(r+1
  \leq j \leq r+s\). Then, \[
    Z(|\C\tilde{G}|) \intersect \C\tilde{G}_{\ov{0}} = \{a \in Z(|\C
    \tilde{G}|) \st a \text{ is even and }za = -a\}
  \]
  has basis \(d_{i_1}-zd_{i_1}, d_{i_2}-zd_{i_2}, \ldots,
  d_{i_k}-zd_{i_k}\) for \(d_i\) even
  and \todo{Check this part} \[
    Z(|\C\tilde{G}|) \intersect \C\tilde{G}_{\ov{1}} = \{a \in Z(|\C
    \tilde{G}|) \st a \text{ is odd and }za = -a\}
  \]
  has basis \(d_{j_1}-zd_{j_1}, d_{j_2}-zd_{j_2}, \ldots,
  d_{j_k}-zd_{j_\ell}\) for \(d_j\) odd.
\end{proof}
\begin{example}
  We have \[
    1 \to \{1,z\} \to \tilde{\Sym}_n \to \Sym_n \to 1
  \]
  where \(z \in \tilde{\Sym}_n\) is even and central, the subgroup of
  index \(2\) is \(\tilde{A}_n\), and \[
    \C\Sym_n^- := \C \tilde{\Sym}_n/\langle z+1 \rangle
  \]
  is the spin symmetric group algebra.
\end{example}
\begin{bibdiv}
  \begin{biblist}
    \bib{macdonald}{book}{
      author={Macdonald, I.G.}
      title={Symmetric Functions and Hall Polynomials}
      year={1979}
      note={2nd Edition, 1995}
    }
    \bib{sagan}{article}{
      author={Sagan, Bruce E.}
      title={Shifted Tableaux, Schur \(Q\)-functions, and a conjecture
      of R. Stanley}
      year={1987}
      journal={J. Combin. Theory Ser. A}
      pages={62--103}
    }
    \bib{wan-wang}{article}{
      author={Wan, Jinkui}
      author={Wang, Weiqiang}
      title={Lectures on Spin Representation Theory of Symmetric
        Groups}
      year={2012}
      journal={Bull. Inst. Math. Acad. Sin.}
      pages={91--164}
    }
  \end{biblist}
\end{bibdiv}
\end{document}