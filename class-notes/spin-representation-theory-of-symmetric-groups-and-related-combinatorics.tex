\documentclass[11pt,leqno,oneside]{amsbook}
\usepackage{tikz}
\usetikzlibrary{cd}
\usepackage{bbm}
\usepackage{ytableau}
\usepackage{todonotes}

\usepackage{./notes}
\usepackage{../ReAdTeX/readtex-core}
\usepackage{../ReAdTeX/readtex-abstract-algebra}

\newcommand{\bbk}{\mathbbm{k}}
\newcommand{\Class}{\operatorname{Class}}
\newcommand{\Res}{\operatorname{Res}}
\newcommand{\Ind}{\operatorname{Ind}}
\newcommand{\bs}{\textbackslash}
\newcommand{\partitionof}{\vdash}
\newcommand{\T}{\mathsf{T}} % Tableau
\renewcommand{\S}{\mathsf{S}}
\newcommand{\sh}{\operatorname{shape}}
\newcommand{\grdim}{\boldsymbol{\dim}}

\newcommand{\dominatedby}{\trianglelefteq}
\newcommand{\dominates}{\trianglerighteq}
\newcommand{\lexicoleq}{\leq}
\newcommand{\lexicogeq}{\geq}
\newcommand{\covers}{\gtrdot}
\newcommand{\coveredby}{\lessdot}

\renewcommand{\H}{\mathcal{H}}
\renewcommand{\Q}{\mathcal{Q}}
\newcommand{\Cl}{\mathcal{C\ell}}
\numberwithin{thm}{section}

\title[Spin Representation Theory of Symmetric Groups]{Spin
  Representation Theory of Symmetric Groups and Related Combinatorics \\ Notes
  from a reading course in Fall 2018}
\author{George H. Seelinger}
\date{Fall 2018}
\begin{document}
\maketitle
\section{Introduction (presented by Jinkui Wan)}
When discussing the representation theory of the symmetric group, one
considers \emph{linear representations} which are group
homomoprhisms \[
  \Sym_n \to GL(V)
\]
In 1911, Schur started considering projective representations \[
  \Sym_n \to PGL(V) = GL(V)/\C^*
\]
leading to the projective representation theory of \(\Sym_n\). It
turns out that this corresponds to the linear representation theory of
an extension of \(\Sym_n\), denoted \(\tilde{S}_n\) and referred to as
the \de{double cover of the symmetric group}, fitting into the
short exact sequence \[
  1 \to \Z/2\Z \to \tilde{S}_n \to \Sym_n \to 1
\]
where, if \(\Z/2\Z = \{1,z\}\), then \(z\) is central in
\(\tilde{S}_n\), which gives us that \(z = 1\) or \(z=-1\).

When \(z=1\), we have the representation theory of \(\Sym_n\). When,
\(z=-1\), we have the representation theory of the \emph{spin
  symmetric group algebra}
\[
\C\Sym_n^- = \C\Sym_n/\langle z+1 \rangle = \left\langle t_1, \ldots, t_n
\st
\begin{array}{l}
  t_i^2=1\\
  t_i t_{i+1} t_i = t_{i+1} t_i t_{i+1}\\
  t_i t_j = -t_j t_i \text{ when }|i-j|>1
\end{array}
\right\rangle
\]
which is equipped with a \(\Z/2\Z\)-grading. So, when we discuss
spin representations of \(\Sym_n\), we are discussing linear
representations of \(\C\Sym_n^-\). Our program to establish these
ideas is as follows. \\

Part I
\begin{enumerate}[label=(\arabic*)]
\item Basics of associative superalgebras
\item Connection to Hecke-Clifford (or Sengeev) algebra, \(\H_n\)
\item Split conjugacy classes in a finite supergroup
\item Characteristic map
\item Schur-\(Q\) functions
\item Schur-Sergeev duality
\item Seminormal form of irreducible representations
\end{enumerate}
Part II
\begin{enumerate}[label=(\arabic*)]
\item Centers of \(\C\Sym_n^-\) (analog of Farahat-Higman theory for
  \(\C\Sym_n\))
\item Coinvariant theory for \(\C\Sym_n^-\)
\item Spin Kostka polynomials
\item Quantum deformation (in particular, Olshanki-Sergeev duality)
\end{enumerate}
\section{Generalities for Associative Superalgebras}
\subsection{Definitions and Examples}
\begin{defn}
  \begin{enumerate}
  \item   A \de{vector superspace} (over \(\C\)) is a
    \(\Z/2\Z\)-graded vector space 
    \(V = V_{\ov{0}} \oplus V_{\ov{1}}\), where elements of
    \(V_{\ov{0}}\) are called \de{even} and elements of \(V_{\ov{1}}\)
    are called \de{odd}. For \(v \in V_i\), \(i \in \Z/2\Z\), we say
    \(|v| = i\).
  \item If \(V\) is a vector superspace with \(\dim V_{\ov{0}} = m\)
    and \(\dim V_{\ov{1}} = n\), we say the \de{graded dimension} of
    \(V\) is \((m,n)\), denoted \(\grdim V = (m,n)\).
  \item A \de{superalgebra} is a \(\C\)-algebra \(A\) with a
    \(\Z/2\Z\)-grading \(A = A_{\ov{0}} \oplus A_{\ov{1}}\) such that
    \(A_i A_j \subset A_{i+j}\) for all \(i,j \in \Z/2\Z\).
  \item A \de{superalgebra ideal} is a homogeneous ideal, that is, an
    ideal of either \(A_{\ov{0}}\) or \(A_{\ov{1}}\).
  \item A superalgebra that has no non-trivial ideals is called
    \de{simple}.
  \item A \de{superalgebra homomorphism} \(\theta \from A \to B\) is
    an even algebra homomorphism, that is, an algebra homomorphism
    sending \(A_i \to B_i\) for all \(i \in \Z/2\Z\).
  \item Given superalgebras \(A\) and \(B\), the tensor product \(A
    \otimes B\) is a superalgebra with multiplication \[
      (a \otimes b)(a' \otimes b') = (-1)^{|a||b|} aa' \otimes bb'
    \]
    for homogeneous elements and extended by linearity.
  \end{enumerate}
\end{defn}
\begin{example}
  \begin{enumerate}
  \item Let \(V = V(m|n)\), the vector superspace with \(\grdim V =
    (m,n)\). Then, \(\End_\C(V)\) is a superalgebra and is isomorphic
    to the matrix superalgebra \[
      M(m|n) := \left\{ \left(
          \begin{array}{cc}
            a&b\\
            c&d
          \end{array}
\right) \st
\begin{array}{c}
  a \text{ is an }m \times m \text{ matrix}\\
  b \text{ is an }m \times n \text{ matrix}\\
  c \text{ is an }n \times m \text{ matrix}\\
  d \text{ is an }n \times n \text{ matrix}
\end{array}
\right\}
\]
or, in other words, \(M(m|n)\) consists of all \(m|n\)-block matrices
and has \(\grdim M(m|n) = (m^2+n^2, 2mn)\). Furthermore, \(M(m|n)\) is
a simple superalgebra.
\item Let \(V = V(n|n)\) and \(p \in \End_\C(V)\) be an odd involution
  (that is, it sends \(V_i \to V_{i+1}\) for \(i \in \Z/2\Z\)). Then,
  we define \[
    \Q(V) := \{f \in \End_\C(V) \st fp = (-1)^{|f|}pf\} =
    \Q(V)_{\ov{0}} \oplus \Q(V)_{\ov{1}}
  \]
  \(\Q(V)\) is also a superalgebra. Moreover, if we pick a basis
  \(\{v_1, \ldots, v_n\}\) of \(V_{\ov{0}}\) and let \(v_i' = p(v_i)\)
  for \(1 \leq i \leq n\), we have that, with respect to the basis
  \(\{v_1, \ldots, v_n, v_1', \ldots, v_n'\}\), \(\Q(V)\) is
  isomorphic to \[
    \Q(n) := \left\{ \left(
        \begin{array}{cc}
          a&b\\
          -b&a
        \end{array}
      \right) \in M(n|n) \right\}
  \]
  and it is simple.
  \item The Clifford algebra \(\Cl_n\) is the superalgebra generated
    by the odd elements \(c_1, \ldots, c_n\) subject to the
    relations \[
      \begin{cases}
        c_i^2 = 1\\
        c_i c_j = - c_j c_i & \forall 1 \leq i \neq j \leq n
      \end{cases}
    \]
  \end{enumerate}
\end{example}
\begin{lem}
  There exist isomorphisms of superalgebras
  \begin{enumerate}
  \item \(M(m|n) \otimes M(k|l) \isom M(mk+nl|mk+nl)\)
  \item \(M(m|n) \otimes \Q(k) \isom Q((m+n)k)\)
  \item \(\Q(m) \otimes \Q(n) \isom M(mn|mn)\)
  \end{enumerate}
\end{lem}
\begin{proof}
  For part (a), we note that \[
    \End_\C(V(m|n)) \otimes \End_\C(V(k|l)) \isom \End_\C(V(mk+ml | mk+nl))
  \]
  under the isomorphism sending \(f \otimes g\) to the endomorphism of
  \(V(mk+ml | mk+nl)\) mapping \(v \otimes w\) to \((-1)^{|g||v|}f(v)
  \otimes g(w)\).

  For part (b), we have \[
    \End(V(m|n)) \otimes \Q(V(k|k),p) \isom Q(V(m|n) \otimes V(k|k), id
    \otimes p)
  \]

  For (c), one explicitly checks that \(\Q(1) \otimes \Q(1) \isom
  M(1|1)\) and then inductively applies (a) and (b) above.
\end{proof}
\begin{cor}
   Since \(\Cl_{m+n} \isom \Cl_m \otimes \Cl_n\) under the isomorphism
    sending generators \(c_1, \ldots, c_n\) to \(c_1 \otimes 1,
    \ldots, c_n \otimes 1\) and \(c_{n+1}, \ldots, c_{n+m}\) to \(1
    \otimes c_1, \ldots, 1 \otimes c_m\), we have the corollaries
  \begin{enumerate}
  \item \(\Cl_1 \isom \Q(1)\) under the isomorphism \(c_1 \mapsto
    p(v_1)\)
  \item \(\Cl_2 \isom M(1|1)\) since \(\Cl_2 \isom \Cl_1 \otimes \Cl_1
    \isom \Q(1) \otimes \Q(1) 
    \isom M(1|1)\)
  \item \(\Cl_{2k} \isom M(2^{k-1}|2^{k-1})\)
  \item \(\Cl_{2k-1} \isom \Q(2^{k-1})\)
  \item and thus, \(\Cl_n\) is simple by parts (c) and (d).
  \end{enumerate}
\end{cor}
\subsection{Classification of Simple Superalgebras}
\begin{thm}
  There are two types of finite dimensional simple associative
  superalgebras over \(\C\):
  \begin{enumerate}
  \item \(M(m|n)\)
  \item \(\Q(n)\)
  \end{enumerate}
\end{thm}
\subsection{Wedderburn Theorem and Schur's Lemma}
\begin{defn}
  \begin{enumerate}
  \item   A \de{(super)module} over a superalgebra \(A\) is a vector space \(M
  = M_{\ov{0}} \oplus M_{\ov{1}}\) with a left action of \(A\) on
  \(M\) such that \(A_i M_j \subset M_{i+j}\) for all \(i,j \in \Z/2\Z\).
  \item A homomorphism between \(A\)-modules \(M\) and \(N\) is a
    linear map \(f \from M \to N\) such that \[
      f(am) = (-1)^{|f||a|} a f(m) \text{ for all }a \in A, m \in M
    \]
    and \[
      \Hom_A(M,N) := \Hom_A(M,N)_{\ov{0}} \oplus \Hom_A(M,N)_{\ov{1}}
    \]
    where \(\Hom_A(M,N)_{\ov{1}} \subset \Hom_\C(M,N)\) such that
    \(M_i \to N_{i+1}\) for all \(i \in \Z/2\Z\).
  \end{enumerate}
\end{defn}
\begin{bibdiv}
  \begin{biblist}
    \bib{macdonald}{book}{
      author={Macdonald, I.G.}
      title={Symmetric Functions and Hall Polynomials}
      year={1979}
      note={2nd Edition, 1995}
    }
    \bib{sagan}{article}{
      author={Sagan, Bruce E.}
      title={Shifted Tableaux, Schur \(Q\)-functions, and a conjecture
      of R. Stanley}
      year={1987}
      journal={J. Combin. Theory Ser. A}
      pages={62--103}
    }
    \bib{wan-wang}{article}{
      author={Wan, Jinkui}
      author={Wang, Weiqiang}
      title={Lectures on Spin Representation Theory of Symmetric
        Groups}
      year={2012}
      journal={Bull. Inst. Math. Acad. Sin.}
      pages={91--164}
    }
  \end{biblist}
\end{bibdiv}
\end{document}