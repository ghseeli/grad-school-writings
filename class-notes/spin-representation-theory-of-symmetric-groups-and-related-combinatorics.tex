\documentclass[11pt,leqno,oneside]{amsbook}
\usepackage{tikz}
\usetikzlibrary{cd}
\usepackage{bbm}
\usepackage{ytableau}
\usepackage{todonotes}
\usepackage{./notes}
\usepackage{../ReAdTeX/readtex-core}
\usepackage{../ReAdTeX/readtex-dangerous}
\usepackage{../ReAdTeX/readtex-abstract-algebra}

\newcommand{\bbk}{\mathbbm{k}}
\newcommand{\Class}{\operatorname{Class}}
\newcommand{\Res}{\operatorname{Res}}
\newcommand{\Ind}{\operatorname{Ind}}
\newcommand{\bs}{\textbackslash}
\newcommand{\partitionof}{\vdash}
\newcommand{\T}{\mathsf{T}} % Tableau
\renewcommand{\S}{\mathsf{S}}
\newcommand{\sh}{\operatorname{shape}}
\newcommand{\grdim}{\boldsymbol{\dim}}

\newcommand{\dominatedby}{\trianglelefteq}
\newcommand{\dominates}{\trianglerighteq}
\newcommand{\lexicoleq}{\leq}
\newcommand{\lexicogeq}{\geq}
\newcommand{\covers}{\gtrdot}
\newcommand{\coveredby}{\lessdot}

\renewcommand{\H}{\mathcal{H}}
\renewcommand{\Q}{\mathcal{Q}}
\newcommand{\Cl}{\mathcal{C\ell}} % Clifford algebra
\newcommand{\OP}{\mathcal{OP}} % Odd partitions
\newcommand{\EP}{\mathcal{EP}} % Odd partitions
\newcommand{\SP}{\mathcal{SP}} % Strict partitions
\newcommand{\CC}{\mathcal{C}} % Conjugacy class
\newcommand{\partitions}{\mathcal{P}} % Partitions
\newcommand{\B}{\mathbb{B}}
\newcommand{\K}{\mathcal{K}} % Farahat-Higman ring
\renewcommand{\D}{\mathcal{D}} % Even split conjugacy class
\renewcommand{\q}{\mathfrak{q}}
\DeclareMathOperator{\Supp}{Supp}
\DeclareMathOperator{\ch}{ch} 
\DeclareMathOperator{\gr}{gr}
\DeclareMathOperator{\Hilb}{Hilb}
\DeclareMathOperator{\wt}{wt}
\renewcommand{\vec}[1]{\mathbf{#1}}
\numberwithin{thm}{section}

\title[Spin Representation Theory of Symmetric Groups]{Spin
  Representation Theory of Symmetric Groups and Related Combinatorics \\ Notes
  from a reading course in Fall 2018}
\author{George H. Seelinger}
\date{Fall 2018}
\begin{document}
\maketitle
\section{Introduction (presented by Jinkui Wan)}
When discussing the representation theory of the symmetric group, one
considers \emph{linear representations} which are group
homomoprhisms \[
  \Sym_n \to GL(V)
\]
In 1911, Schur started considering projective representations \[
  \Sym_n \to PGL(V) = GL(V)/\C^*
\]
leading to the projective representation theory of \(\Sym_n\). It
turns out that this corresponds to the linear representation theory of
an extension of \(\Sym_n\), denoted \(\tilde{\Sym}_n\) and referred to as
the \de{double cover of the symmetric group}, fitting into the
short exact sequence \[
  1 \to \Z/2\Z \to \tilde{\Sym}_n \to \Sym_n \to 1
\]
where, if \(\Z/2\Z = \{1,z\}\), then \(z\) is central in
\(\tilde{\Sym}_n\), which gives us that \(z = 1\) or \(z=-1\).

When \(z=1\), we have the representation theory of \(\Sym_n\). When,
\(z=-1\), we have the representation theory of the \emph{spin
  symmetric group algebra}
\[
\C\Sym_n^- = \C\Sym_n/\langle z+1 \rangle = \left\langle t_1, \ldots, t_n
\st
\begin{array}{l}
  t_i^2=1\\
  t_i t_{i+1} t_i = t_{i+1} t_i t_{i+1}\\
  t_i t_j = -t_j t_i \text{ when }|i-j|>1
\end{array}
\right\rangle
\]
which is equipped with a \(\Z/2\Z\)-grading. So, when we discuss
spin representations of \(\Sym_n\), we are discussing linear
representations of \(\C\Sym_n^-\). Our program to establish these
ideas is as follows. \\

Part I
\begin{enumerate}[label=(\arabic*)]
\item Basics of associative superalgebras
\item Connection to Hecke-Clifford (or Sengeev) algebra, \(\H_n\)
\item Split conjugacy classes in a finite supergroup
\item Characteristic map
\item Schur-\(Q\) functions
\item Schur-Sergeev duality
\item Seminormal form of irreducible representations
\end{enumerate}
Part II
\begin{enumerate}[label=(\arabic*)]
\item Centers of \(\C\Sym_n^-\) (analog of Farahat-Higman theory for
  \(\C\Sym_n\))
\item Coinvariant theory for \(\C\Sym_n^-\)
\item Spin Kostka polynomials
\item Quantum deformation (in particular, Olshanki-Sergeev duality)
\end{enumerate}
\section{Generalities for Associative Superalgebras (presented by
  Jinkui Wan)}
\subsection{Definitions and Examples}
\begin{defn}
  \begin{enumerate}
  \item   A \de{vector superspace} (over \(\C\)) is a
    \(\Z/2\Z\)-graded vector space 
    \(V = V_{\ov{0}} \oplus V_{\ov{1}}\), where elements of
    \(V_{\ov{0}}\) are called \de{even} and elements of \(V_{\ov{1}}\)
    are called \de{odd}. For \(v \in V_i\), \(i \in \Z/2\Z\), we say
    \(|v| = i\).
  \item If \(V\) is a vector superspace with \(\dim V_{\ov{0}} = m\)
    and \(\dim V_{\ov{1}} = n\), we say the \de{graded dimension} of
    \(V\) is \((m,n)\), denoted \(\grdim V = (m,n)\).
  \item A \de{superalgebra} is a \(\C\)-algebra \(A\) with a
    \(\Z/2\Z\)-grading \(A = A_{\ov{0}} \oplus A_{\ov{1}}\) such that
    \(A_i A_j \subset A_{i+j}\) for all \(i,j \in \Z/2\Z\).
  \item A \de{superalgebra ideal} is a homogeneous ideal, that is, a
    subset \(I \subset A\) such that \(I = I_{\ov{0}} \oplus
    I_{\ov{1}} = (I \intersect A_{\ov{0}}) \oplus (I \intersect
    A_{\ov{1}})\) as vector spaces and \(A_i I_j \subset I_{i+j}\) for
    all \(i,j \in 
    \Z/2\Z\). 
  \item A superalgebra that has no non-trivial ideals is called
    \de{simple}.
  \item A \de{superalgebra homomorphism} \(\theta \from A \to B\) is
    an even algebra homomorphism, that is, an algebra homomorphism
    sending \(A_i \to B_i\) for all \(i \in \Z/2\Z\).
  \item Given superalgebras \(A\) and \(B\), the tensor product \(A
    \otimes B\) is a superalgebra with multiplication \[
      (a \otimes b)(a' \otimes b') = (-1)^{|a||b|} aa' \otimes bb'
    \]
    for homogeneous elements and extended by linearity.
  \item A \de{commutative superalgebra} is one that is graded
    commutative, that is \[
      yx = (-1)^{|x||y|}xy
    \]
    Thus, the \de{supercommutator} of a superalgebra is given by \[
      [x,y] = xy-(-1)^{|x||y|}yx
    \]
    and the \de{supercenter} is given by \[
      Z(A) = \{a \in A \st [a,x] = 0 \text{ for all }x \in A\}
    \]
    which is different than the center of an ungraded algebra.
  \item Given a superalgebra \(A\), we let \(|A|\) be the associative
    algebra where we forget the grading on \(A\). 
  \end{enumerate}
\end{defn}
\begin{example}
  \begin{enumerate}
  \item Let \(V = V(m|n)\), the vector superspace with \(\grdim V =
    (m,n)\). Then, \(\End_\C(V)\) is a superalgebra and is isomorphic
    to the matrix superalgebra \[
      M(m|n) := \left\{ \left(
          \begin{array}{cc}
            a&b\\
            c&d
          \end{array}
\right) \st
\begin{array}{c}
  a \text{ is an }m \times m \text{ matrix}\\
  b \text{ is an }m \times n \text{ matrix}\\
  c \text{ is an }n \times m \text{ matrix}\\
  d \text{ is an }n \times n \text{ matrix}
\end{array}
\right\}
\]
or, in other words, \(M(m|n)\) consists of all \(m|n\)-block matrices
and has \(\grdim M(m|n) = (m^2+n^2, 2mn)\). Furthermore, \(M(m|n)\) is
a simple superalgebra since \(|M(m|n)|\) is simple as a \(\C\)-algebra.
\item Let \(V = V(n|n)\) and \(p \in \End_\C(V)\) be an odd involution
  (that is, it sends \(V_i \to V_{i+1}\) for \(i \in \Z/2\Z\)). Then,
  we define \[
    \Q(V) := \{f \in \End_\C(V) \st fp = (-1)^{|f|}pf\} =
    \Q(V)_{\ov{0}} \oplus \Q(V)_{\ov{1}}
  \]
  \(\Q(V)\) is also a superalgebra. Moreover, if we pick a basis
  \(\{v_1, \ldots, v_n\}\) of \(V_{\ov{0}}\) and let \(v_i' = p(v_i)\)
  for \(1 \leq i \leq n\), we have that, with respect to the basis
  \(\{v_1, \ldots, v_n, v_1', \ldots, v_n'\}\), \(\Q(V)\) is
  isomorphic to \[
    \Q(n) := \left\{ \left(
        \begin{array}{cc}
          a&b\\
          -b&a
        \end{array}
      \right) \in M(n|n) \right\}
  \]
  and it is simple.
  \item The Clifford algebra \(\Cl_n\) is the superalgebra generated
    by the odd elements \(c_1, \ldots, c_n\) subject to the
    relations \[
      \begin{cases}
        c_i^2 = 1\\
        c_i c_j = - c_j c_i & \forall 1 \leq i \neq j \leq n
      \end{cases}
    \]
  \end{enumerate}
\end{example}
\begin{lem}
  There exist isomorphisms of superalgebras
  \begin{enumerate}
  \item \(M(m|n) \otimes M(k|l) \isom M(mk+nl|mk+nl)\)
  \item \(M(m|n) \otimes \Q(k) \isom Q((m+n)k)\)
  \item \(\Q(m) \otimes \Q(n) \isom M(mn|mn)\)
  \end{enumerate}
\end{lem}
\begin{proof}
  For part (a), we note that \[
    \End_\C(V(m|n)) \otimes \End_\C(V(k|l)) \isom \End_\C(V(mk+ml | mk+nl))
  \]
  under the isomorphism sending \(f \otimes g\) to the endomorphism of
  \(V(mk+ml | mk+nl)\) mapping \(v \otimes w\) to \((-1)^{|g||v|}f(v)
  \otimes g(w)\).

  For part (b), we have \[
    \End(V(m|n)) \otimes \Q(V(k|k),p) \isom Q(V(m|n) \otimes V(k|k), id
    \otimes p)
  \]

  For (c), one explicitly checks that \(\Q(1) \otimes \Q(1) \isom
  M(1|1)\) and then inductively applies (a) and (b) above.
\end{proof}
\begin{cor}
   Since \(\Cl_{m+n} \isom \Cl_m \otimes \Cl_n\) under the isomorphism
    sending generators \(c_1, \ldots, c_n\) to \(c_1 \otimes 1,
    \ldots, c_n \otimes 1\) and \(c_{n+1}, \ldots, c_{n+m}\) to \(1
    \otimes c_1, \ldots, 1 \otimes c_m\), we have the corollaries
  \begin{enumerate}
  \item \(\Cl_1 \isom \Q(1)\) under the isomorphism \(c_1 \mapsto
    p(v_1)\)
  \item \(\Cl_2 \isom M(1|1)\) since \(\Cl_2 \isom \Cl_1 \otimes \Cl_1
    \isom \Q(1) \otimes \Q(1) 
    \isom M(1|1)\)
  \item \(\Cl_{2k} \isom M(2^{k-1}|2^{k-1})\)
  \item \(\Cl_{2k-1} \isom \Q(2^{k-1})\)
  \item and thus, \(\Cl_n\) is simple by parts (c) and (d).
  \end{enumerate}
\end{cor}
\subsection{Classification of Simple Superalgebras}
\begin{thm}
  There are two types of finite dimensional simple associative
  superalgebras over \(\C\):
  \begin{enumerate}
  \item \(M(m|n)\)
  \item \(\Q(n)\)
  \end{enumerate}
\end{thm} 
\subsection{Wedderburn Theorem and Schur's Lemma}
\begin{defn}
  \begin{enumerate}
  \item   A \de{(super)module} over a superalgebra \(A\) is a vector space \(M
  = M_{\ov{0}} \oplus M_{\ov{1}}\) with a left action of \(A\) on
  \(M\) such that \(A_i M_j \subset M_{i+j}\) for all \(i,j \in \Z/2\Z\).
  \item A homomorphism between \(A\)-modules \(M\) and \(N\) is a
    linear map \(f \from M \to N\) such that \[
      f(am) = (-1)^{|f||a|} a f(m) \text{ for all }a \in A, m \in M
    \]
    and \[
      \Hom_A(M,N) := \Hom_A(M,N)_{\ov{0}} \oplus \Hom_A(M,N)_{\ov{1}}
    \]
    where \(f \in \Hom_A(M,N)_{\ov{1}} \subset \Hom_\C(M,N)\) is such that
    \(f(M_i) \subset N_{i+1}\) for all \(i \in \Z/2\Z\).
  \end{enumerate}
\end{defn}
\begin{defn}
  An \(A\)-module is said to be \de{simple} if it is nonzero and has no
  proper \(A\)-submodules. An \(A\)-module \(M\) is said to be
  \de{semisimple} if every \(A\)-submodule of \(M\) is a direct
  summand of \(M\).
\end{defn}
\begin{thm}[Super Wedderburn Theorem]
  The following are equivalent for a finite dimensional superalgebra \(A\).
  \begin{enumerate}
  \item Every \(A\)-module is semisimple
  \item \(A\) is a finite direct sum of left simple superideals
  \item \(A\) is a direct product of a finite number of simple algebras
  \end{enumerate}
\end{thm}
\begin{defn}
  Thus, we say a superalgebra \(A\) is \de{semisimple} if it satisfies
  one of the three conditions.
\end{defn}
\begin{example}
  \begin{enumerate}
  \item \(M(m|n) = I_1 \oplus I_2 \oplus \cdots \oplus I_m \oplus
    I_{m+1} \oplus \cdots \oplus I_{m+n}\) where \(I_k =
    M(m|n)E_{k,k}\) for \(1 \leq k \leq m+n\).
  \item \(\Q(n) = J_1 \oplus \cdots \oplus J_n\) where \[
      J_k = \Q(n) (E_{k,k} + E_{n+k,n+k})
    \]
    \todo{Check this. Most likely depends on your choice of basis and
      involution.}
  \item \(\Hom_{M(m|n)}(I_k, I_k) \isom \C\) and \(\Hom_{\Q(n)}(J_k,
    J_k) \isom \C \oplus \C p\). Importantly, the latter space is not
    \(1\)-dimensional despite \(J_k\) being \(1\)-dimensional!
  \end{enumerate}
\end{example}
\begin{cor}
   A finite dimensional semisimple superalgebra \(A\) is isomorphic
   to \[
     A \isom \bigoplus_{i=1}^m M(r_i|s_i) \oplus \bigoplus_{j=1}^n \Q(n_j)
   \]
   where \(m = m(A)\) and \(q=q(A)\) are invariants of \(A\).
 \end{cor}
 \begin{defn}
   A simple \(A\)-module \(V\)is said to be of type M (resp. type Q)
   if it is annihilated by all but one summand of the form
   \(M(r_i|s_i)\) (resp. \(\Q(n_j)\)).
 \end{defn}
 \begin{cor}
   \begin{enumerate}
   \item The number of non-isomorphic simple \(A\)-modules is given by
     \(m(A)+q(A) = \dim (Z(|A|) \intersect A_{\ov{0}})\).
   \item The number of non-isomorphic simple \(A\)-modules of type
     \(Q\) is given by \(q(A) = \dim(Z(|A|) \intersect A_{\ov{1}})\).
   \end{enumerate}
 \end{cor}
 \begin{thm}[Schur's Lemma]
   If \(M\) and \(L\) are simple \(A\)-modules, then \[
     \dim\Hom_A(M,L) =
     \begin{cases}
       1 & \text{ if }M \isom L\text{ of type M}\\
       2 & \text{ if }M \isom L\text{ of type Q}\\
       0 & \text{ otherwise}
     \end{cases}
   \]
\end{thm}
\begin{rmk}
  \begin{enumerate}
  \item A simple \(A\)-module \(M\) is of type M if and only if
    \(|M|\) is a simple \(|A|\)-module
  \item A simple \(A\)-module \(M\) is of type Q if and only if
    \(|M|\) is a direct sum of two non-isomorphic simple \(|A|\)-modules.
  \end{enumerate}
\end{rmk}
\section{Split Conjugacy Classses in a Finite Supergroup (presented by
  Jinkui Wan)}
Throughout, let \(G\) be a finite group with index \(2\) subgroup
\(G_0 \subgroup G\).
\begin{defn}
  \begin{enumerate}
  \item We say that the elements of \(G_0\) are \de{even elements} and the
  elements of \(G_1 := G \setminus G_0\) are \de{odd elements},
  \item \(\C G\) is a superalgebra, which we will denote \(\C[G,G_0]\)
  \end{enumerate}
\end{defn}
\begin{thm}[Super MAschke's Theorem]
  \(\C[G,G_0]\) is semisimple
\end{thm}
\begin{prop}
  \begin{enumerate}
  \item If \(g \in G_i\), \(h \in G\), then \(hgh^{-1} \in G_i\) for
    all \(i \in \Z/2\Z\).
  \item The number of non-isomorphic simple \(\C[G,G_0]\)-modules is
    equal to the number of even conjugacy classes in \(G\).
  \item The number of non-isomorphic simple \(\C[G,G_0]\)-module of
    type \(Q\) is equal to the number of odd conjugacy classes in \(G\).
  \end{enumerate}
\end{prop}
\begin{proof}
  We note that, by the usual Artin-Wedderburn theorem, \(\C G\)
  decomposes into a direct sum of simple matrix algebras, each of
  which has a \(1\)-dimensional center and can be indexed by a
  conjugacy class of \(G\) via \(c_i = \sum_{g \in \CC_i} g\) where
  \(\CC_i\) is a conjugacy class of \(G\). In fact, this shows in the
  classical theory that the number of conjugacy classes of \(G\) equal
  the number of irreducible representations.
  
  Since conjugacy classes are either even or odd by (a), which is left
  as an exercise, (b) follows
  because \(\dim (Z(\C G) \intersect 
  \C[G,G_0]_{\ov{0}})\) is equal to the number of non-isomorphic
  simple \(\C[G,G_0]\)-modules and (c) follow from the fact that \(\dim(Z(\C G)
  \intersect \C[G,G_0]_{\ov{1}})\) gives the number of those of type \(Q\).
\end{proof}
Now, consider the following situation. Let \(\tilde{G}\) be a group
such that there exists an index \(2\) subgroup \(\tilde{G}_0 \subgroup
\tilde{G}\) and there exists a short exact sequence \[
  1 \to \{1,z\} \to \tilde{G} \to[\theta] G \to 1
\]
where \(z^2 = 1\) and \(z\) is central in \(\tilde{G}\). Then,
\begin{prop}
  For \(C\) a conjugacy class of \(G\), the preimage \[
    \theta^{-1}(C) = \{g, gz \st g \in C\} \subset \tilde{G}
  \]
  has that 
  \begin{enumerate}
  \item \(\theta^{-1}(C)\) is a single conjugacy class in \(\tilde{G}\) if \(g\) is
    conjugate to \(zg\) in \(\tilde{G}\) or
  \item \(\theta^{-1}(C)\) splits into two conjugacy classes in
    \(\tilde{G}\) if there exists a \(g \in C\) such that \(g\) is not
    conjugate to \(zg\). In this case, we call \(C\) \de{split}.
  \end{enumerate}
\end{prop}
\begin{defn}
  We set \[
    \C\tilde{G}^- := \C[G,G_0]/\langle z+1 \rangle
  \]
  and call a \(\C\tilde{G}^-\)-module a \de{spin
    \(\C[\tilde{G},\tilde{G}_0]\)}-module. 
\end{defn}
\begin{prop}
  \begin{enumerate}
  \item   We have the isomorphism of superalgebras 
  \[\C[\tilde{G},\tilde{G}_0] \isom \underbrace{\C[G,G_0]}_{(z=1)}
    \oplus \underbrace{\C\tilde{G}^-}_{(z=-1)} \]
  \item The number of non-isomorphic simple spin
    \(\C[\tilde{G},\tilde{G}_0]\)-modules is equal to the number of
    even split conjugacy classes of \(G\).
  \item The number of non-isomorphic simple spin
    \(\C[\tilde{G},\tilde{G}_0]\)-modules of type Q is equal to the
    number of odd split conjugacy classes of \(G\).
  \end{enumerate}
\end{prop}
\begin{proof}
  Part (a) follows from the semisimplicity of
  \(\C[\tilde{G},\tilde{G}_0]\).

  Now, (a) tells us that \[
    Z(|\C\tilde{G}^-|) = \{a \in Z(|\C[\tilde{G},\tilde{G}_0]|) \st za
    = -a\}
  \]
  So let \[
    \underbrace{D_1, zD_1, D_2, zD_2, \ldots, D_r, zD_r}_{\text{split}}, \underbrace{D_{r+1}, \ldots,
  D_{r+s}}_{\text{non-split}}
  \] be the conjugacy classes of \(\tilde{G}\) where \(r\) is the
  number of split conjugacy classes in \(G\),  \(D_i \intersect zD_i =
  \emptyset\) for \(1 \leq i \leq r\) and \(zD_j = D_j\) for \(r+1
  \leq j \leq r+s\). Then, \[
    Z(|\C\tilde{G}|) \intersect \C\tilde{G}_{\ov{0}} = \{a \in Z(|\C
    \tilde{G}|) \st a \text{ is even and }za = -a\}
  \]
  has basis \(d_{i_1}-zd_{i_1}, d_{i_2}-zd_{i_2}, \ldots,
  d_{i_k}-zd_{i_k}\) for \(d_i\) even
  and \todo{Check this part} \[
    Z(|\C\tilde{G}|) \intersect \C\tilde{G}_{\ov{1}} = \{a \in Z(|\C
    \tilde{G}|) \st a \text{ is odd and }za = -a\}
  \]
  has basis \(d_{j_1}-zd_{j_1}, d_{j_2}-zd_{j_2}, \ldots,
  d_{j_k}-zd_{j_\ell}\) for \(d_j\) odd.
\end{proof}
\begin{example}
  We have \[
    1 \to \{1,z\} \to \tilde{\Sym}_n \to[\theta_n] \Sym_n \to 1
  \]
  where \(z \in \tilde{\Sym}_n\) is even and central, the subgroup of
  index \(2\) is \(\tilde{A}_n\), and \[
    \C\Sym_n^- := \C \tilde{\Sym}_n/\langle z+1 \rangle
  \]
  is the spin symmetric group algebra.
\end{example}
\begin{defn}
  Throughout the remainder of these notes, we define \(\theta_n \from
  \tilde{\Sym}_n \to \Sym_n\) to be the double covering may above.
\end{defn}
\section{A Morita Superequivalence (presented by Jinkui Wan)}
Since \(\Sym_n\) acts on \(\Cl_n\) via \(\sigma.c_i = c_{\sigma(i)}\),
we can define the semidirect product \(\Cl_n \rtimes \C \Sym_n\) with
multiplication \[
  (x,\sigma)(y,\tau) = (x \sigma(y), \sigma \tau)
\]
\begin{defn}
  We define the \de{Hecke-Clifford superalgebra} as \[
    \H := \Cl_n \rtimes \C \Sym_n
  \]
  with the \(c_i\) having odd parity and the \(s_j\) having even
  parity. 
\end{defn}
\begin{lem}
  There exists a superalgebra isomoprhism
  \begin{align*}
    \C\Sym_n^- \otimes \Cl_n & \isomto \H_n\\
    c_i & \mapsto c_i \\
    t_j & \mapsto \frac{1}{\sqrt{-2}} s_j(c_j-c_{j+1})
  \end{align*}
  \todo{Check this last map.}
\end{lem}
Recall that \(\Cl_n\) is a simple superalgebra and it has a unique
simple module \(U_n\). If \(n\) is even, \(U_n\) is of type M and
if \(n\) is odd, \(U_n\) is of type Q. This leads us to define two
functors
\begin{align*}
  F_n := - \otimes U_n \from \C\Sym_n\catname{-Mod} \to
  \H_n\catname{-Mod}\\
  G_n := \Hom_{\Cl_n}(U_n,-) \from \H_n\catname{-Mod} \to \C\Sym_n^-\catname{-Mod}
\end{align*}
\begin{lem}
  \cite{kleshchev}*{Prop 13.2.2}
  \begin{enumerate}
  \item If \(n\) is even, then \(F_n \circ G_n \isom id\) and \(G_n
    \circ F_n \isom id\).
  \item If \(n\) is odd, then \(F_n \circ G_n \isom id \oplus \pi\)
    and \(G_n \circ F_n \isom id \oplus \pi\) where \(\pi(M)_i =
    M_{i+1}\) for all \(i \in \Z/2\Z\). 
  \end{enumerate}
\end{lem}
Thus, because \((F_n \circ G_n)(M) = \Hom_{\Cl_n}(U_n,M) \otimes
U_n\), we have a (super)Morita equivalence between \(\C\Sym_n^-
\otimes \Cl_n\) and \(\H_n\).
\section{A Double Cover \(\tilde{B}_n\) (presented by Jinkui Wan)}
Recall that \(B_n = \Z_2^n \rtimes \Sym_n\). We define
\begin{defn}
  \[
    \Pi_n := \left\langle z,a_1,\ldots,a_n \st
    \begin{array}{c}
      z^2 = a_i^2 = 1, \ \forall 1 \leq i \leq n \\
      a_i a_j = z a_j a_i, \ i \neq j
    \end{array}
\right\rangle
  \]
\end{defn}
Then, \(\Sym_n\) acts on \(\Pi_n\) via \(\sigma(z) = z\) and
\(\sigma(a_i) = a_{\sigma(i)}\). This gives us the short exact
sequence
\begin{align*}
  1 \to \{1,z\} \to \Pi_n \rtimes \Sym_n & \to B_n \to 1\\
  a_i & \to b_i
\end{align*}
and so we define \todo{The flow here is not great.}
\begin{defn}
  Let \(\tilde{B}_n\) be the supergroup on \(\Pi_n \rtimes \Sym_n\)
  with the \(a_i\) odd, \(z\) even, and \(\sigma \in \Sym_n\) even.
\end{defn}
Since \(\C\tilde{B}_n/ \langle z+1 \rangle = \H_n\), we wish to
understand conjugacy classes in \(B_n\). We will do so by example.
\begin{example}
  Consider \[
    x = ((+++-+++-+-),(1234)(567)(89)) \in B_{10}
  \]
  As an element of \(\Sym_{10}\), \((1234)(567)(89)\) has cycle type
  \((4,3,2,1)\), but we wish to assign a parity to each of these
  cycles. To do so, we look at the \((+,-)\)-array in the first
  coordinate and take the product of the entries corresponding to the
  cycle. So, \((1234)\) gets cycle type \(+ \times + \times + \times
  - = -\) since those are entries \(1,2,3\), and \(4\) in the
  array. This gives the cycle type as a tuple of partitions \(\rho =
  (\rho^+, \rho^-)\) and so \(\rho(x) = ((3),(4,2,1))\). Similarly,
  if \[
    y = ((+---+---+-),(1386)(279)(45)) \in B_{10}
  \]
  then the first cycle has parity \(+ \times - \times - \times - = -\)
  since those are the \(1,3,6,\) and \(8\) entries of the array. One
  can check that \(y\) has the same cycle type as \(x\).
\end{example}
\begin{lem}
  Two elements of \(B_n\) are conjugate if and only if their cycle
  types are the same.
\end{lem}
\begin{cor}
  The number of conjugacy classes in \(B_n\) is \[
    \#\{(\rho^+,\rho^-) \st |\rho^+|+|\rho^-| = n\}
  \]
\end{cor}
Now, the conjugacy class \(\CC_{\rho^+,\rho^-}\) is even if \(k\) is
even for \(\underbrace{b_{i_1} b_{i_2} \ldots b_{i_k}}_{\in \Z_2^n} \sigma \in
C_{\rho^+,\rho^-}\).
\begin{thm}[Read]
  \cite{cheng-wang}*{Theorem 3.31}
  \begin{enumerate}
  \item Even \(\CC_{\rho^+,\rho^-}\) splits if and only if \(\rho^+ \in
    \OP_n\) and \(\rho^- = \emptyset\)
  \item Odd \(\CC_{\rho^+,\rho^-}\) splits if and only if \(\rho^+ =
    \emptyset\) and \(\rho^- \in \SP_n^- \)
  \end{enumerate}
  where \(\SP_n^-\) is all partitions of \(n\) with strict parts and
  odd length.
\end{thm}
\begin{defn}
  For \(\alpha \in \OP_n\), let \(\CC_\alpha^+\) be the split
  conjugacy class in \(\tilde{B}_n\) satisfying
  \begin{enumerate}
  \item \(\CC_\alpha^+ = \theta_n^{-1}(\CC_{\alpha,\emptyset})\)
  \item There exists \(\sigma \in \CC_\alpha^+\) such that \(\sigma \in
    \Sym_n\) with cycle type \(\alpha\).
  \end{enumerate}
\end{defn}
\section{A ring structure on \(R^-\) (presented by Jinkui Wan)}
\begin{defn}
  We give the following definition
  \begin{enumerate}
  \item Let \(R_n^- := [\H_n\catname{-Mod}]\), the Grothendieck group
    of \(\H_n\catname{-Mod}\).
  \item Let \(R^- := \bigoplus_{n=0}^\infty R_n^-\) where \(R_0^- =
    \Z\)
  \item Let \(R^-_{\mathbb{Q}} = \mathbb{Q} \otimes_\Z R^-\)
  \item Let \(\H_{m,n}\) be the subalgebra of \(\H_{m+n}\) generated
    by \(\Cl_{m+n}\) and \(S_m \times S_n\). Note that \(\H_{m,n}
    \isom \H_m \otimes \H_n\) as a superalgebra.
  \item Given \(M \in \H_m\catname{-Mod}\) and \(N \in
    \H_n\catname{-Mod}\), we define \[
      [M] \cdot [N] := [\Ind_{\H_m \otimes \H_n}^{\H_{m+n}} M \otimes N]
    \]
  \end{enumerate}
\end{defn}
\begin{prop}
  \(R^-\) is commutative with respect to the above multiplication.
\end{prop}
\begin{defn}
  Define a bilinear form via \[
    \langle [M],[N] \rangle := \dim \Hom_{\H_n}(M,N)
  \]
  for \(M,N \in \H_n\)
\end{defn}
\begin{lem}
  For \(\phi \in R_n^-\) (viewed as a character of \(\tilde{B}_n\)),
  set \(\phi_\alpha := \phi(x)\) for any \(x \in \CC_\alpha\). Then,
  \begin{enumerate}
  \item For \(\phi \in R_m^-, \psi \in R_n^-\), and \(\gamma \in
    \OP_{m+n}\), \[
      (\phi \cdot \psi)_\gamma = \sum_{\substack{\alpha \in \OP_m,
          \beta \in \OP_n \\ \alpha \union \beta = \gamma}}
      \frac{z_\gamma}{z_\alpha z_\beta} \phi_\alpha \psi_\beta
    \]
    where \(z_\alpha\) is the order of the centralizer of \(\sigma\)
    of cycle type \(\alpha\) in \(\Sym_n\).
  \item \[
      \langle \phi, \psi \rangle = \sum_{\alpha \in \OP}
      2^{-\ell(\alpha)} z_\alpha^{-1} \phi_\alpha \psi_\alpha
    \]
  \end{enumerate}
\end{lem}
\begin{prop}
  The character value vanishes unless you are in an even split
  conjugacy class.
\end{prop}
\section{The ring \(\Gamma\) (presented by Jinkui Wan)}
\begin{defn}
  Let \(x = \{x_1, x_2, \ldots\}\).
  \begin{enumerate}
  \item Define \(q_r = q_r(x)\) via the generating function
    \[ Q(t) = \sum_{r \geq 0} q_r(x) t^r = \prod_{i \geq 1}
      \frac{1+tx_i}{1-tx_i}
    \]
  \item Let \(\Gamma\) be the \(\Z\)-subring of the ring of symmetric
    functions generated by \(q_r\), \(r \geq 0\).
  \item \(\Gamma_{\mathbb{Q}} := \mathbb{Q} \otimes_\Z \Gamma\)
  \end{enumerate}
\end{defn}
\begin{prop}
  \begin{enumerate}
  \item \(\sum_{r+s=n} (-1)^r q_r q_s = 0\) because \(Q(t)Q(-t) = 1\)
  \item \(q_n = \sum_{\alpha \in \OP_n} 2^{\ell(\alpha)} z_\alpha^{-1}
    p_\alpha\) where \(p_\alpha = P_{\alpha_1} \cdots
    p_{\alpha_\ell}\) because \(\ln Q(t) = \sum_{r \text{ odd}}
    \frac{2 p_r(x) t^r}{r}\).
  \end{enumerate}
\end{prop}
\begin{thm}
  \begin{enumerate}
  \item \(\Gamma_{\mathbb{Q}}\) is a polynomial algebra with
    polynomial generators \(p_{2r-1}\) for \(r \geq 1\).
  \item \(\{p_\mu \st \mu \in \OP\}\) is a basis for \(\Gamma_{\mathbb{Q}}\)
  \end{enumerate}
\end{thm}
\begin{defn}
  Let us define inner product on \(\Gamma_{\mathbb{Q}}\) via \[
    \langle p_\alpha, p_\beta \rangle := 2^{-\ell(\alpha)} z_\alpha
    \delta_{\alpha \beta}, \forall \alpha, \beta \in \OP
  \]
\end{defn}
\begin{defn}
  Define the \de{(spin) characteristic map} to be
  \begin{align*}
    \ch^- \from R_{\mathbb{Q}}^- & \to \Gamma_{\mathbb{Q}}\\
    \phi & \mapsto \sum_{\alpha \in \OP_n} z_\alpha^{-1} \phi_\alpha
           p_\alpha 
  \end{align*}
\end{defn}
\begin{prop}
  \begin{enumerate}
  \item \(\ch^-\) is an algebra isomorphism.
  \item \(\ch^-\) is an isometry (that is, \(\langle \phi,\psi \rangle
    = \langle \ch^-(\phi), \ch^-(\psi) \rangle\)).
  \end{enumerate}
\end{prop}
Now, we seek to construct the basic spin module.
\begin{prop}
  \(\H_n = \Cl_n \rtimes \C \Sym_n\) acts on \(\Cl_n = \Span\{c_I \st
  I \subset \{1,2,\ldots,n\}\}\) via \[
    \begin{cases}
      c_i.(c_{i_1} \cdots c_{i_k}) = c_i c_{i_1} \cdots c_{i_k} \\
      \sigma.(c_{i_1} \cdots c_{i_k}) = c_{\sigma(i_1)} \cdots
      c_{\sigma(i_k)} 
    \end{cases}
  \]
  where \(c_i \in \Cl_n\), \(\sigma \in \Sym_n\), and the action is
  extended by linearity. 
\end{prop}
\begin{prop}
  Let \(\sigma = \sigma_1 \cdots \sigma_\ell\) be a cycle
  decomposition of \(\sigma\). Then \[
    \sigma c_I =
    \begin{cases}
      \pm c_I & \text{ if }I\text{ is a union of some supports of
      }\sigma_1, \ldots, \sigma_\ell \\
      \pm c_J (J \neq I) & \text{ otherwise}
    \end{cases}
  \]
\end{prop}
\begin{example}
  Let \(\sigma = (134)(25) \in \Sym_5\). Then, \[
    \sigma c_3 c_5 = c_4 c_2
  \]
  but \[
    \sigma c_1 c_3 c_4 = c_3 c_4 c_1 = c_1 c_3 c_4
  \]
\end{example}
Thus, the character of this action, say \(\xi^n\), satisfies \[
  \xi^n(\alpha) = 2^{\ell(\alpha)}, \alpha \in \OP_n
\]
and thus \(\ch^-(\xi^n) = \sum_{\alpha \in \OP_n} z_\alpha^{-1}
2^{\ell(\alpha)} p_\alpha = q_n\).
\begin{defn}
 For \(\lambda \in \SP\), we define \(\xi^\lambda\) via the recursive
 formulas
 \begin{align*}
   \xi^{(\lambda_1, \lambda_2)} & = \xi^{\lambda_1} \xi^{\lambda_2} + 2
   \sum_{i=1}^{\lambda_2}(-1)^i \xi^{\lambda_1 + i} \xi^{\lambda_2 -
   i} \\
   \xi^\lambda & =
   \begin{cases}
     \sum_{j=2}^k (-1)^j \xi^{(\lambda_1,\lambda_j)} \xi^{(\lambda_2,
       \ldots, \hat{\lambda}_j, \ldots, \lambda_k)} & k =
     \ell(\lambda) \text{ is even} \\
     \sum_{j=1}^k (-1)^{j-1} \xi^{\lambda_j} \xi^{(\lambda_1, \ldots,
       \hat{\lambda}_j, \ldots, \lambda_k)} & k = \ell(\lambda) \text{
     is odd}
   \end{cases}
 \end{align*}
\end{defn}
\begin{thm}
  \begin{enumerate}
  \item \(\ch^-(\xi^\lambda) = Q_\lambda\), the Schur-\(Q\) function
    (to be defined in the next lecture).
  \item \(\left\{ \zeta^\lambda :=
      2^{-\frac{\ell(\lambda)-\delta(\lambda)}{2}} \xi^\lambda \st
      \lambda \in \SP_n \right\}\), where \(\delta(\lambda) =
    \chi\{\ell(\lambda) \text{ is odd}\}\), is a complete list of
    simple characters.
  \item \(\zeta^\lambda\) is of type M if \(\ell(\lambda)\) is even
    and of type Q if \(\ell(\lambda)\) is odd.
  \item The degree of \(\zeta^\lambda\) is \[
      2^{n-\frac{\ell(\lambda)-\delta(\lambda)}{2}}
      \frac{n!}{\lambda_1! \cdots \lambda_\ell!}\left( \prod_{i<j}
      \frac{\lambda_i-\lambda_j}{\lambda_i+\lambda_j}\right) 
    \]
  \end{enumerate}
\end{thm}
\section{Schur-\(Q\) functions and related combinatorics (presented by
  George H. Seelinger)}
\section{Center of Symmetric Group Algebras and Spin Symmetric Group
  Algebras (presented by Jinkui Wan)}
\subsection{Farahat-Higman's Construction for \(\Sym_n\)}
Given a permutation \(\sigma\), we note that its cycle type is not
stable under inclusion from \(\Sym_n \into \Sym_{n+1}\).
\begin{example}
  Let \(\sigma = (134)(2576) \in \Sym_8\). Then, \(\sigma\) has cycle
  type \[
    (4,3,1) = \ydiagram{4,3,1}
  \]
  but, when included into \(\Sym_n9\), \(\sigma\) has cycle type \[
    (4,3,1,1) = \ydiagram{4,3,1,1}
  \]
\end{example}
\begin{defn}
  Given a cycle \(\sigma \in \Sym_n\), its \de{modified cycle type},
  \(\lambda\), is given by removing the first column from its cycle
  type. 
\end{defn}
\begin{example}
  The modified cycle type of \(\sigma = (134)(2576) \in \Sym_8\) is
  \(\lambda = (3,2)\). Note that this is stable with respect to
  \(\Sym_8 \into \Sym_9\).
\end{example}
\begin{prop}
  \begin{enumerate}
  \item If \(\sigma\) is of modified type \(\lambda\), then
    \(|\lambda|\) is the minimal length for \(\sigma\) as a product of
    (not necessarily simple) transpositions.
  \item If \[
      \begin{cases}
        \sigma \text{ is of modified type }\lambda\\
        \tau \text{ is of modified type }\mu\\
        \sigma\tau \text{ is of modified type }\nu
      \end{cases},
      \text{ then } |\nu| \leq |\lambda| + |\mu|
    \]
  \end{enumerate}
\end{prop}
\begin{example}
  \((134) = (13)(34)\) and has modified type \((2)\).
\end{example}
\begin{defn}
  \begin{enumerate}
  \item Let \(\CC_\lambda(n)\) be the conjugacy class of \(\Sym_n\) of
    modified type \(\lambda\). Note \(\CC_\lambda(n) = \emptyset\) if
    \(n < |\lambda|+\ell(\lambda)\).
  \item Let \[
      C_\lambda(n) :=
      \begin{cases}
        \text{Class sum of }\CC_\lambda(n) & \text{ if } n \geq
        |\lambda|+\ell(\lambda) \\
        0 & \text{ otherwise}
      \end{cases}
    \]
  \end{enumerate}
\end{defn}
\begin{example}
  \(\CC_0(n) = \{id\}\) and \(\CC_{(1)}(n)\) contains all transpositions
  of \(\Sym_n\). Thus, \(C_{(1)}(n) = \sum_{1 \leq i < j \leq n} (ij)\).
\end{example}
\begin{prop}
  \(\{C_\lambda(n) \st |\lambda|+\ell(\lambda) \leq n\}\) is a basis
  for \(Z(\Z \Sym_n)\). 
\end{prop}
\begin{defn}
  Write \[
    C_\lambda(n) C_\mu(n) = \sum A_{\lambda \mu}^\nu(n) C_\nu(n)
  \]
\end{defn}
\begin{example}
  \[
    C_{(1)}(n) C_{(1)}(n) = 3 C_{(2)}(n) + 2 C_{(1,1)}(n) +
    \frac{1}{2}n(n-1) C_0(n)
  \]
  since \(C_{(1)}(n)^2 = \sum (ij)(kl)\) for all transpositions
  \((ij),(kl)\) in \(\Sym_n\).
\end{example}
\begin{thm}[Farahat-Higman]
  Let \(\lambda,\mu,\nu\) be partitions. Then,
  \begin{enumerate}
  \item There is a unique polynomial \(f_{\lambda \mu}^\nu(x) \in
    \Q[x]\) such that \(a_{\lambda \mu}^\nu(n) = f_{\lambda
      \mu}^\nu(n)\) for all \(n \geq |\nu|+\ell(\nu)\).
  \item \(f_{\lambda \mu}^\nu(x) = 0\) unless \(|\nu| \leq
    |\lambda|+|\mu|\)
  \item If \(|\nu| = |\lambda|+|\mu|\), then \(f_{\lambda
      \mu}^\nu(x)\) is a constant. In other words, \(a_{\lambda
      \mu}^\nu(n)\) is independent of \(n\).
  \end{enumerate}
\end{thm}
\begin{proof}[Proof Idea]
  Let \(\Gamma = \{(\sigma,\tau) \st \sigma \in \CC_\lambda(n), \tau
  \in \CC_\mu(n), \sigma \tau \in \CC_\nu(n)\}\). Then, once
  computes \[
    a_{\lambda \mu}^\nu(n) = \frac{\# \Gamma}{\# \CC_\lambda(n)}
  \]
  If we let \(\Sym_n\) act on \(\Gamma\) by conjugation, that is
  \(\gamma.(\sigma,\tau) = (\gamma \sigma \gamma^{-1}, \gamma \tau
  \gamma^{-1})\), then we get that \[
    \Gamma = \Gamma_1 \union \Gamma_2 \union \cdots \union \Gamma_k
  \]
  Without loss of generality, take \((\sigma_1, \tau_1) \in
  \Gamma_1\). Then, \[
    \# \Gamma_1 = \frac{n!}{\#\text{ of centralizers of
      }(\sigma_1,\tau_1)\text{ in }\Sym_n}
  \]
  Suppose \(\gamma.(\sigma_1,\tau_1) = (\sigma_1,\tau_1)\). Then, \[
    \gamma \sigma_1 \gamma^{-1} = \sigma_1 \text{ and } \gamma \tau_1
    \gamma^{-1} = \tau_1
  \]
  Thus, \[
    \gamma \in \Sym_{\Supp(\sigma_1,\tau_1)} \times
    \Sym_{\{1,\ldots,n\} \setminus \Supp(\sigma_1, \tau_1)}
  \]
  where \(\Supp(\sigma_1,\tau_1) = \{j \in \{1,\ldots,n\} \st
  \sigma_1(j) \neq j \text{ or }\tau_1(j) \neq j\}\). So, \[
    \# \text{ centralizers of }(\sigma_1, \tau_1)\text{ in }\Sym_n =
    \# \text{ centralizers of }(\sigma_1, \tau_1)\text{ in
    }\Sym_{\Supp(\sigma_1,\tau_1)} \times (n - \# \Supp(\sigma_1,\tau_1))!
  \]
  Thus, using our formula above for \(a_{\lambda \mu}^\nu\), we arrive
  at \todo{Finish this formula}\[
    a_{\lambda \mu}^\nu(n) = \frac{\sum_i \# \Gamma_i}{\#
      \CC_\lambda(n)}
  \]
\end{proof}
\begin{defn}
  \begin{enumerate}
  \item  Let \(\B\) be the ring of polynomials \(f(x) \in \mathcal{Q}[x]\) such that
  \(f(n) \in \Z\) for all \(n \in \Z\).
  \item Let \(\K\) be the \(\B\)-algebra with basis \(\{c_\lambda \st
    \lambda \in \partitions\}\) such that \[
      c_\lambda c_\mu := \sum_{\nu \in \partitions} f_{\lambda
        \mu}^\nu(x) c_\nu
    \]
    We call this ring the \de{Farahat-Higman ring}.
  \end{enumerate}
\end{defn}
\begin{prop}
  We have the following facts
  \begin{enumerate}
  \item \(\K\) is commutative and associative.
  \item \(\K\) is filtered via \(\deg(c_\lambda) = |\lambda|\) for all
    \(\lambda \in \partitions\).
  \end{enumerate}
\end{prop}
\begin{rmk}
  \(\K\) is not graded because \(\sum_{\nu \in \partitions} f_{\lambda
  \mu}^\nu(x) c_\nu\) is not homogeneous. However, if we say \(\K_r =
  \Span\{c_\lambda \st |\lambda| \leq r\}\), then \(\K_r \K_s \subset
  \K_{r+s}\), making \(\K\) filtered. 
\end{rmk}
\begin{defn}
  Let \(\gr \K\) be the associated graded algebra, that is, \(\gr \K\)
  is defined by \((\gr
  \K)_r = \K_r/\K_{r-1}\) and then \(\gr \K = \bigoplus_{r \geq 0}
  (\gr \K)_r\).
\end{defn}
\begin{lem}
  We have the following facts.
  \begin{enumerate}
  \item   If \(|\lambda|+s = m\), then \[
    a_{\lambda,(s)}^{(m)} =
    \begin{cases}
      \frac{(m+1)s!}{\prod_{i \geq 0} m_i(\lambda)!} & \text{ if
      }\ell(\lambda) \leq s+1\\
      0 & \text{otherwise}
    \end{cases}
  \]
  where \(m_0(\lambda) = r+1-\ell(\lambda)\).
\item If \(|\lambda|+s = |\nu|\), then \[
    a_{\lambda,(s)}^\nu = \sum_{\substack{(i,\mu) \in \N \times
        \partitions  \\
        1 \leq i \leq \ell(\lambda) \\ \mu \union \nu = \lambda \union
      (\nu_i)}} a_{\mu,(s)}^{(\nu_i)}
  \]
  
  \end{enumerate}
\end{lem}
\begin{prop}
  Let \(\lambda\) be a partition and let \(m_i(\lambda)\) be the
  number of \(i\)'s in \(\lambda\) such 
  that \(\lambda = (i^{m_i(\lambda)})_{i \geq 1}\). Then, the top
  degree of \(c_\lambda c_{(s)}\) is given by \[
    (c_\lambda c_{(s)})^* = \sum_{|\nu|=|\lambda|+s}
    a^\nu_{\lambda,(s)} c_\nu = \sum_{\mu \subset \lambda, \ell(\mu) \leq
      s+1} \frac{(m_{s+|\mu|}(\lambda)+1)(s+|\mu|+1)s!}{(s+1-\ell(\mu))!
    \prod_{i \geq 1} m_i(\mu)!} c_{\lambda \union(s+|\mu|) - \mu}
  \]
\end{prop}
\begin{rmk}
  \(\mu = (i^{m_i(\mu)}) \subset \lambda \iff m_i(\mu) \leq
  m_i(\lambda) \ \forall i \geq i\).
\end{rmk}
\begin{example}
  To illustrate the formula \(\lambda \union (s+|\mu|)-\mu\), consider
  cycles \(\sigma = (134)(2567)(8)\) and \(\tau = 
  (28)\). Then, \(\lambda = (3,2)\) is the modified cycle type of
  \(\sigma\) and \(s=1\) gives the modified cycle type of
  \(\tau\). Then, \[
    \sigma \tau = (134)(28567)
  \]
  which has modified cycle type \((4,2) = (3,2) \union (1+|(3)|) -
  (3)\).
\end{example}
\begin{cor}
  \[
    c_{\lambda_1} c_{\lambda_2} \cdots c_{\lambda_\ell} = \sum_{\mu
      \dominates \lambda} d_{\lambda \mu} c_\mu
  \]
  and \(d_{\lambda \lambda} > 0\) in \(\gr \K\). Thus, \(c_{1}, c_2,
  \ldots\) are algebraically independent elements of
  \(\Span_\Z\{c_\lambda\}\).
\end{cor}
\begin{prop}
  \(\mathbb{Q} \otimes_\Z \gr \K\) is a polynomial algebra generated
  by \(c_1, c_2, \ldots\) (Note that \(\Z \into \B\) as constants.)
\end{prop}
\begin{rmk}
  \begin{enumerate}
  \item There exists a ring isomorphism \(\Lambda \isomto
    \gr\K\) sending duals of \(h_\lambda^*\) (images of \(h_\lambda\)
    under a certain automorphism), called \(g_\lambda\), to
    \(c_\lambda\). See  
    \cite{macdonald}*{p 132--3}.
  \item \(\gr Z(\Z \Sym_n) \isom H^*(\Hilb^n(\C^2);\Z)\), the
    cohomology ring of the Hilbert Scheme of points on \(\C^2\), as a
    \(\Z\)-algebra.
  \end{enumerate}
\end{rmk}
\begin{thm}
  The homomorphism given by
  \begin{align*}
    \Pi_n \from \K & \to Z(\Z\Sym_n)\\
    \sum f_\lambda(x)c_\lambda & \mapsto \sum f_\lambda(n) c_\lambda(n)
  \end{align*}
  is a surjective homomorphism.
\end{thm}
\begin{prop}
  \(\K\) is generated by \(K_m := \sum_{|\lambda|=m} c_\lambda\) for
  \(m \geq 0\).
\end{prop}
This tells us that \[
  \Pi_m(\K) = \sum_{|\lambda|=m} c_\lambda(n) = \sum_{\substack{\sigma
      \in \Sym_n \\ \# \text{ of cycles in } \sigma = n-m}} \sigma
\]
and so \(Z(\Z\Sym_n)\) is generated by \(\Pi_n(\K_0), \Pi_n(\K_1),
\ldots, \Pi_n(\K_{n-1})\). 
\section{Double cover of \(\tilde{\Sym}_n\) and even split conjugacy
  classes}
Recall the short exact sequence \[
  1 \to \{1,z\} \to \tilde{\Sym}_n \to \Sym_n \to 1
\]
where \[
  \tilde{\Sym}_n = \left\langle z,t_1,t_2,\ldots,t_{n-1} \st
  \begin{cases}
    z \text{ is central}\\
    z^2 = 1, t_i^2 = z\\
    t_i t_{i+1} t_i = t_{i+1} t_i t_{i+1} \\
    t_i t_j = z t_j t_i & |i-j|>1
  \end{cases}
\right\rangle
\]
\begin{defn}
  Define the element \[
    x_i := t_i t_{i+1} \cdots t_{n-1} t_n t_{n-1} \cdots t_{i+1} t_i
  \]
  which gets mapped to the transposition \((i,n) \in \Sym_n\) under
  the map \(\theta_n\) \todo{The map \(\theta_n\) was never properly
    defined}. Then, we let \[
    [i_1 i_2 \cdots i_m] :=
    \begin{cases}
      z & \text{ if } m=1\\
      x_{i_1} x_{i_m} x_{i_{m-1}} \cdots x_{i_2} x_{i_1} & \text{ if }
      m \geq 2
    \end{cases}
  \]
\end{defn}
\begin{prop}
  Every element of \(\tilde{\Sym}_n\) is of the form \[
    z^q \underbrace{[i_1 i_2 \cdots i_m][j_1 j_2 \cdots j_k] \cdots}_{\text{disjoint}}
  \]
  where \(q=0,1\).
\end{prop}
\begin{lem}
  For \(\lambda\) such that \(|\lambda|+\ell(\lambda) \leq n\), \(
  \theta_n^{-1}(\CC_\lambda(n)) \) splits if and only if
  \begin{enumerate}
  \item \(\lambda\) has only even parts or
  \item \(\lambda \in \SP\), \(|\lambda|\) odd,
    \(|\lambda|+\ell(\lambda) = n\) or \(n-1\).
  \end{enumerate}
\end{lem}
\begin{prop}
  \(\sigma \in \Sym_n\) of modified type \(\lambda\) is even if and
  only if \(|\lambda|\) is even.
\end{prop}
\begin{defn}
  Let \(\D_\lambda(n)\) be the even split conjugacy class in
  \(\tilde{\Sym}_n\) containing
  \([1,2,\ldots,\lambda_1+1][\lambda_1+2,\ldots,
  \lambda_1+\lambda_2+2] \ldots\).
\end{defn}
\begin{prop}
  \begin{enumerate}
  \item \(\theta^{-1}(\CC_\lambda(n)) = \D_\lambda(n) \union z
    \D_\lambda(n)\).
  \item \(\{d_\lambda(n) \st \lambda \in \EP, |\lambda|+\ell(\lambda)
    \leq n\}\) is a basis for the even center of \(\Z\Sym_n^- = \Z
    \tilde{\Sym}_n/\langle z+1 \rangle\). 
  \end{enumerate}
\end{prop}
\begin{defn}
  Define \(b_{\lambda \mu}^\nu(n)\) by \[
    d_{\lambda}(n) d_{\mu}(n) = \sum_{\nu \in \EP} b_{\lambda \mu}^\nu
    (n) c_\nu(n)
  \]
\end{defn}
\begin{example}
  \[
    d_{(4)}(8) d_{(2)}(8) = 13 d_{(4)}(8) - 35 d_{(2)}(8) - 18
    d_{(2,2)}(8) - 7d_{(6)}(8) + 2d_{(4,2)}(8) \in Z(\Z\Sym_8^-)
  \]
\end{example}
\begin{thm}[Tysse-Wang]
  Let \(\lambda,\mu,\nu \in \EP\).
  \begin{enumerate}
  \item There exists a unique \(g_{\lambda \mu}^\nu(x) \in \Q[x]\)
    such that \(b_{\lambda \mu}^\nu(n) = g_{\lambda \mu}^\nu(n)\) for
    all \(n \geq |\nu|+\ell(\nu)\).
  \item \(g_{\lambda \mu}^\nu(x) = 0\) unless \(|\nu| \leq
    |\lambda|+|\mu|\)
  \item If \(|\nu| = |\lambda|+|\mu|\), then \(g_{\lambda
      \mu}^\nu(x)\) is a constnat. 
  \end{enumerate}
\end{thm}
\begin{defn}
  Let the \de{spin Farahat-Higman algebra} \(\F\) be a \(\B\)-algebra with basis \(\{d_\lambda \st \lambda
  \in \EP\}\) and \[
    d_\lambda d_\mu = \sum_{\nu \in \EP} g_{\lambda \mu}^\nu(x) d_\nu
  \]
  which is filtered with respect to \(\deg(d_\lambda) = |\lambda|\).
\end{defn}
\begin{prop}
  Let \(\lambda\) be a partition and rewrite \(\lambda =
  (i^{m_i(\lambda)}_{i \geq 1})\). Let \(s \geq 0\) be event. Then, \[
    (d_\lambda d_{(s)})^* = \sum_\mu (-1)^{\ell(\mu)}
    \frac{(m_{s+|\mu|}(\lambda)+1)(s+|\mu|+1)s!}{(s+1-\ell(\mu))!
      \prod_{i \geq 1} m_i(\mu)!} d_{\lambda \union(s+|\mu|)-\mu}
  \]
  \todo{What is this summation over?}
\end{prop}
\begin{cor}
  \begin{enumerate}
  \item \(\mathbb{Q} \otimes_{\Z[\frac{1}{2}]} \gr \F\) is generated
    by \(d_2, d_4 ,\ldots\)
  \item There exists an injective homomorphism \(\Q
    \otimes_{\Z[\frac{1}{2}]} \gr \F \into \Q \otimes_\Z \K\) via (for
    \(\lambda \in \EP\)) 
    \(d_\lambda \mapsto (-1)^{\ell(\lambda)c_\lambda}\). 
  \end{enumerate}
\end{cor}
\subsection{Connections to odd Jucys-Murphy elements}
\begin{defn}
  Let us define \[ M_k := \sum_{i=1}^{k-1} [i,k] \in \Z \Sym_n^-
  \]
\end{defn}
\begin{prop}
  We have
  \begin{enumerate}
  \item \(M_k M_l = - M_l M_k\) for \(k \neq l\)
  \item \[
      M_k^2 = -(k-1) - \sum_{1 \leq i \neq j \leq k-1} [i,j,k] \in \Z\Sym_n^-
    \]
  \end{enumerate}
\end{prop}
\begin{defn}
  we define \[
    e_{r,n} := \sum_{1 \leq i_1 < i_2 < \cdots < i_r \leq n} M_{i_1}^2
    M_{i_2}^2 \cdots M_{i_r}^2 \in Z(\Z \Sym_n^-)
  \]
\end{defn}
\begin{prop}
  \begin{enumerate}
  \item \(e_{r,n}\) has top degree \(2r\)
  \item \(e_{r,n} = \sum_{\lambda \in \EP, |\lambda|+\ell(\lambda)
      \leq n} A_\lambda(n) d_\lambda(n)\) for some \(A_\lambda(n)\).
  \item \(A_\lambda(n)\) is the coefficient of \([1,2,\ldots,
    \lambda_1+1][\lambda_1+2,\ldots,\lambda_1+\lambda_2+2]\cdots\) in
    \(e_{r,n}\) and is independent of \(n\). 
  \end{enumerate}
\end{prop}
\begin{defn}
  In light of the proposition above, we write \(A_\lambda :=
  A_\lambda(n)\) and define \[
    e_r^* := \sum_{\lambda \in \EP, |\lambda|=2r} A_\lambda d_\lambda
    \in \F
  \]
\end{defn}
\begin{example}
  \(e_1^* = -d_2\) and \(e_2^* = d_{(2,2)}-2d_4\).
\end{example}
\begin{prop}
  \(A_\lambda = (-1)^{\ell(\lambda) \prod_{i \geq 1}
    c_{\frac{\lambda_i}{2}}}\) where \(c_0 = 1\) and \(c_r =
  \frac{1}{r+1}\binom{2r}{r}\) are the Cartan numbers.
\end{prop}
\begin{thm}
  \(\B\left[\frac{1}{2}\right] \otimes_\B \F\) is generated by
  \(e_1^*, e_2^*, e_3^*, \ldots\)
\end{thm}
\begin{cor}
  Via the surjective homomorphism
  \begin{align*}
    \B\left[\frac{1}{2}\right] \otimes_\B \F
    & \to Z(\Z\left[ \frac{1}{2} \right] \Sym_n^-) \\
    \sum_{\lambda \in \EP} f_\lambda(x) d_\lambda
    & \mapsto \sum_{\lambda \in \EP} f_\lambda(n) d_\lambda(n)
  \end{align*}
  the even center of \(\Z\left[ \frac{1}{2} \right] \Sym_n^-\) is
  generated by (the top degree of) \(e_{r,n}\).
\end{cor}
\section{Schur-Sergeev duality for \(\q(n)\) (presented by Chris Chung)}
\section{Seminormal form construction for irreducible \(\H_n\)-modules
  (presented by Jinkui Wan)}
A review for the symmetric group case was presented, but not written
up here yet.
\begin{defn}
  We define the \de{Jucys-Murphy elements} in \(\H_n = \Cl_n \rtimes
  \C\Sym_n\) as \[
    J_k := \sum_{1 \leq j < k \leq n} (1+c_j c_k) (jk), 
  \]
\end{defn}
\begin{prop}
  The Jucys-Murphy elements have the following properties.
  \begin{enumerate}
  \item \(J_k J_l = J_l J_k\) for \(l \leq k \neq l \leq n\).
  \item \(c_k J_k = -J_k c_k\) and \(c_l J_k = J_k c_l\) for \(k \neq
    l\).
  \item \(s_k J_k = J_{k+1}s_k - (1+c_k c_{k+1})\) for \(1 \leq k \leq
    n-1\) and \(s_l J_k = J_k s_l\) for \(k \neq l,l+1\).
  \end{enumerate}
\end{prop}
\begin{defn}
  The \de{degenerate affine Hecke-Clifford algebra} is given by \[
    \hat{\H}_n = \langle s_1, \ldots, s_{n-1}, c_1, \ldots, c_n, x_1
    ,\ldots, x_n\rangle
  \]
  with additional relations \todo{Additional to the Hecke-Clifford
    algebra?} \[
    \begin{cases}
      x_k x_l = x_l x_k\\
      s_k x_k = x_{k+1} s_k - (1+c_k c_{k+1})\\
      s_i x_k = x_k s_i & k \neq i,i+1\\
      x_k c_k = - c_k x_k \\
      x_k c_l = c_l x_k & l \neq k
    \end{cases}
  \]
\end{defn}
\begin{prop}
  There exists a projection \(\pi \from \hat{\H}_n \onto \H_n\) such
  that \[
    \pi(s_i) = s_i, \pi(c_k) = c_k, \pi(x_l) = J_l
  \]
  In particular, \(\pi(x_1) = J_1 = 0\), so \(x_1 \in \ker \pi\). 
\end{prop}
In fact, \(\ker \pi = \langle x_1 \rangle\) and so
\catname{\(\H_n\)-Mod} can be identified as the subcategory of
\catname{\(\hat{\H}_n\)-Mod} on which \(x_1 = 0\).
\begin{thm}[PBW Theorem]
  We have that \[
    \{x_1^{\alpha_1} \cdots x_n^{\alpha_n} c_1^{\beta_1}\cdots
    c_n^{\beta_n}w \st \alpha_i \in \Z_+, \beta_i \in \{0,1\}, 1 \leq
    i \leq n, w \in \Sym_n\}
  \]
  is a basis for \(\hat{\H}_n\).
\end{thm}
\begin{cor}[Corollary of PBW Theorem]
  The subalgebra of \(\hat{\H}_n\) generated by \(z_1,\ ldots, x_n,
  c_1, \ldots, c_n\) is isomorphic to \[
    \Cl_n \otimes \C[x_1, \ldots, x_n] / \langle x_k c_k = -c_k x_k,
    x_k c_l = c_l x_k, l \neq k \rangle = \underbrace{P_1^c \otimes P_1^c \otimes
    \cdots \otimes P_1^c}_{n \text{ copies}}
\]
where \(P_1^c = \langle x_1,c_1 \rangle\). 
\end{cor}
\begin{prop}
  The \(\Sym_n\) fixed points \(\C[x_1^2,x_2^2,\ldots,x_n^2]^{\Sym_n}
  \subset \text{Center of }\hat{\H}_n\).
\end{prop}
\begin{prop}
  \begin{enumerate}
  \item The eigenvalues of \(x_1^2, \ldots, x_n^2\) are of the form
    \(q(i) := i(i+1)\) for \(i \in \Z_+\).
  \item If all the eigenvalues of \(x_j^2\) on a finite dimensional
    \(\hat{\H}_n\)-module \(M\) for a fixed \(j\) are of the form
    \(q(i)\), then \(M\) is integral.
  \end{enumerate}
\end{prop}
The second part of the proposition follows from the intertwining
elements.
\begin{defn}
  Let \de{intertwining element} \(\Phi_k \in \hat{\H}_n\) be given by
  \begin{align*}
    \Phi_k
    & := s_k(x_k^2-x_{k+1}^2) + (x_k + x_{k+1}) + c_k
      c_{k+1}(x_k-x_{k+1}) \\
    & = (x_{k+1}^2-x_k^2)s_k - (x_k + x_{k+1}) - c_k c_{k+1}(x_k-x_{k+1})
  \end{align*}
\end{defn}
Note, the second equality follows from the fact that \[
  s_k x_k^2 = x_{k+1} s_k x_k - (1+c_k c_{k+1}) x_k = x_{k+1}^2 s_k -
  (1+c_k c_{k+1})(1+x_k)
\]
and, using conjugation by \(s_k\), \[
  x_k^2 s_k = s_k x_{k+1}^2 - s_k (1+c_k c_{k+1})(1+x_k) s_k \implies
  - s_k x_{k+1}^2 = - x_k^2 s_k - s_k(1+c_k c_{k+1})(1+x_k) s_k
\]
Therefore, we get \[
  s_k(x_k^2-x_{k+1}^2) = (x_{k+1}^2-x_k^2)s_k - (1+c_k c_{k+1})(1+x_k)
  - s_k (1+c_k c_{k+1})(1+x_k) s_k
\]
\begin{prop}
We have the following useful relations for the intertwining elements.
\[
  \begin{cases}
    \Phi_k \Phi_l = \Phi_l \Phi_k & |k-l| > 1\\
    \Phi_k \Phi_{k+1} \Phi_k = \Phi_{k+1} \Phi_k \Phi_{k+1}\\
    \Phi_k^2 = 2(x_k^2+x_{k+1}^2) - (x_k^2 - x_{k+1}^2)^2\\
    \Phi_k x_k = x_{k+1} \Phi_k, \Phi_k x_{k+1} = x_k \Phi_k, \Phi_k
    x_l = x_l \Phi_k & l \neq k,k+1\\
    \Phi_k c_k = c_{k+1} \Phi_k, \Phi_k c_{k+1} = c_k \Phi_k, \Phi_k
    c_l = c_l \Phi_k & l \neq k,k+1
  \end{cases}
\]
\end{prop}
\begin{prop}
  If \(v\) is some eigenvector of \(x_j^2\), that is, if \(x_j^2 v
= av\), then \(x_j^2 \Phi_j v = a \Phi_j v\), so \(\Phi_j v\) is an
\(x_j^2\) eigenvector with the same eigenvalue.
\end{prop}
\begin{proof}
  Since \(x_j^2 \Phi_j = \Phi_j x_{j+1}^2\), we get that \[
    x_j^2 \Phi_j x = \Phi_j x_{j+1}^2 v =
  \]
  \todo{Finish this}
\end{proof}
\begin{defn}
  A finite dimensional \(\hat{\H}_n\)-module \(M\) is called
  \de{completely splittable (CS)} if \(x_1, x_2, \ldots, x_n\) act
  semisimply, ie the actions of \(x_1, \ldots, x_n\) can be
  diagonalized simultaneously.
\end{defn}
\begin{prop}
  Every integral CS \(\hat{\H}_n\)-module \(M\) can be decomposed
  as \[
    M = \bigoplus_{\vec{i} \in \Z_+^n} M_{\vec{i}}
  \]
  where \[
    M_{\vec{i}} = \{v \in M \st x_k^2 v = q(i_k)v, 1 \leq k \leq n\}
  \]
  is the common eigenspace of \(x_1^2, x_2^2, \ldots, x_n^2\) with
  eigenvalues \(q(i_1), q(i_2), \ldots, q(i_k)\). Furthermore,
  define \[
    \wt(M) := \{\vec{i} \in \Z_+^n \st M_{\vec{i}} \neq 0\}
  \]
\end{prop}
Our goal is to describe \(\wt(M)\) for integral irreducible CS
\(\hat{\H}_n\)-modules.
\begin{lem}
  If \(\vec{i} = (i_1, \ldots, i_n) \in \Z_+^n\) is in \(\wt(M)\) for
  some integral irreducible CS \(\hat{\H}_n\)-module, then \(i_k \neq
  i_{k+1}\) for all \(1 \leq k \leq n-1\).
\end{lem}
\begin{proof}
  Suppose \(i_k = i_{k+1}\) so that \(x_k^2 v = x_{k+1}^2 v\) for \(v \in 
  M_{\vec{i}}\). Since \[
    x_k^4 s_k - 2 q(i_k) x_k^2 s_k  + q(i_k)^2 s_k  = x_k^2 (s_k
    x_{k+1}^2 - s_k(1+c_k c_{k+1})(1+x_k)s_k) - 2 q(i_k) x_k^2 s_k +
    q(i_k)^2 s_k
  \]
  Somehow we get \todo{figure this out} that \(s_k v \in
  M_{\vec{i}}\). Then, we get
  \begin{align*}
    & \implies x_k^2 s_k v = q(i_k)s_k v\\
    & \implies (x_k^2 - q(i_k))s_k v = 0 \\
    & \implies (s_k x_{k+1}^2 - x_k(1-c_k c_{k+1}) - (1-c_k
      c_{k+1})x_{k+1})v-q(i_k)s_k v = 0
  \end{align*}
  However, \(x_{k+1}^2 v = q(i_{k+1})v = q(i_k)v\) by assumption, so
  \begin{align*}
    & \implies (x_k(1-c_k c_{k+1})+(1-c_k c_{k+1})x_k)v = 0 \\
    & \implies 2(x_k^2+x_{k+1}^2) v = 0 \\
    & \implies i_k = i_{k+1} = 0\\
    & \implies x_k^2 v = 0 = x_{k+1}^2 v\\
    & \implies x_k v = 0 = x_{k+1}v & \text{ since }M\text{ is CS}\\
    & \implies v = 0 & \text{ since }
  \end{align*}
  \todo{Finish this proof.}
\end{proof}
\begin{lem}
  Suppose \(\vec{i} = (i_1, \ldots, i_n) \in \wt(M) \subset \Z_+^n\)
  for some integral, irreducible, CS \(\hat{\H}_n\)-module. Fix \(1
  \leq k \leq n-1\).
  \begin{enumerate}
  \item If \(i_k \neq i_{k+1}\pm 1\), then \(\Phi_k z \neq 0\) for all
    \(0 \neq z \in M_{\vec{i}}\).
  \item If \(i_k = i_{k+1} \pm 1\), then \(\Phi_k = 0\) on
    \(M_{\vec{i}}\). 
  \end{enumerate}
\end{lem}
\begin{proof}
  Since \(\Phi_k^2 = 2(x_k^2+x_{k+1}^2) - (x_k^2-x_{k+1}^2)^2\), then \[
    \Phi_k^2 z = (2(q(i_k)+q(i_{k+1})) - q(i_k)^2 + 2q(i_k)q(i_{k+1}) -
    q(i_{k+1})^2)z = 0
  \]
  if and only if \[
    (q(i_k)-q(i_{k+1}))^2 = 2(q(i_k)+q(i_{k+1}))
  \]
  Now, if we write \(i_k = i_{k+1}+c\), then \[
    \begin{cases}
      q(i_k) = i_{k+1}^2+2c i_{k+1}+c^2 + i_{k+1}+c\\
      q(i_{k+1}) = i_{k+1}^2+i_{k+1}
    \end{cases}
    \implies
    \begin{cases}
      q(i_k)-q(i_{k+1}) = 2 c i_{k+1} + c^2 + c \\
      q(i_k)+q(i_{k+1}) = 2i_{k+1}^2+2c i_{k+1}+c^2 + 2i_{k+1}+c
    \end{cases}
  \]
  From here, one checks that \(c = \pm 1\) certainly gives 
  solutions independent of \(i_{k+1}\). Thus, \(i_k = i_{k+1} \pm 1
  \implies \Phi_k^2 z = 0\). \todo{However, there are other solutions
    to these equations, namely \(c = -2 i_{k+1}\) and
    \(c=-2(i_{k+1}+1)\). So, how do I prove the first staent?}

  So, to prove the second part, since \(\Phi_k^2 z = 0\), it must be
  that if \(\Phi_k z \neq 0\) and so \(\Phi_k z \in M_{s_k \vec{i}}\)
  \todo{Why is this true?}. Then, there exists a minimal sequence \(\Phi_{j_1}, \ldots,
  \Phi_{j_r}\) such that \(\Phi_{j_1} \cdots \Phi_{j_r} \Phi_k z \in
  M_{\vec{i}}\) since \(M\) is irreducible \todo{Also not clear
    why}. Then, if \(\sigma = s_{j_1} \cdots s_{j_r} s_k \in \Sym_n\),
  it must be that \(\sigma \cdot \vec{i} = \vec{i}\). If one assumes
  \(\sigma \neq 1\), this leads to a violation of Lemma \ref{} with
  some work. Then, using the exchange condition for Coxeter groups,
  one shows that \(r=1\) which gives \(j_1 = k\), so \(\Phi_k^2 z \neq
  0\), contradicting what we showed above.
\end{proof} 
\begin{bibdiv}
  \begin{biblist}
    \bib{cheng-wang}{book}{
      author={Cheng, Shun-Jen}
      author={Wang, Weiqiang}
      title={Dualities and Representations of Lie Superalgebras}
      year={2012}
    }
    \bib{kleshchev}{book}{
      author={Kleshchev, Alexander}
      title={Linear and Projective Representations of Symmetric
        Groups}
      year={2005}
    }
    \bib{macdonald}{book}{
      author={Macdonald, I.G.}
      title={Symmetric Functions and Hall Polynomials}
      year={1979}
      note={2nd Edition, 1995}
    }
    \bib{sagan}{article}{
      author={Sagan, Bruce E.}
      title={Shifted Tableaux, Schur \(Q\)-functions, and a conjecture
      of R. Stanley}
      year={1987}
      journal={J. Combin. Theory Ser. A}
      pages={62--103}
    }
    \bib{wan-wang}{article}{
      author={Wan, Jinkui}
      author={Wang, Weiqiang}
      title={Lectures on Spin Representation Theory of Symmetric
        Groups}
      year={2012}
      journal={Bull. Inst. Math. Acad. Sin.}
      pages={91--164}
    }
  \end{biblist}
\end{bibdiv}
\end{document}