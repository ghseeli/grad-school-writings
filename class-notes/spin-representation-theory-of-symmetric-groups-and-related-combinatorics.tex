\documentclass[11pt,leqno,oneside]{amsbook}
\usepackage{tikz}
\usetikzlibrary{cd}
\usepackage{bbm}
\usepackage{ytableau}
\usepackage{todonotes}

\usepackage{./notes}
\usepackage{../ReAdTeX/readtex-core}
\usepackage{../ReAdTeX/readtex-abstract-algebra}

\newcommand{\bbk}{\mathbbm{k}}
\newcommand{\Class}{\operatorname{Class}}
\newcommand{\Res}{\operatorname{Res}}
\newcommand{\Ind}{\operatorname{Ind}}
\newcommand{\bs}{\textbackslash}
\newcommand{\partitionof}{\vdash}
\newcommand{\T}{\mathsf{T}} % Tableau
\renewcommand{\S}{\mathsf{S}}
\newcommand{\sh}{\operatorname{shape}}

\newcommand{\dominatedby}{\trianglelefteq}
\newcommand{\dominates}{\trianglerighteq}
\newcommand{\lexicoleq}{\leq}
\newcommand{\lexicogeq}{\geq}
\newcommand{\covers}{\gtrdot}
\newcommand{\coveredby}{\lessdot}

\renewcommand{\H}{\mathcal{H}}
\numberwithin{thm}{section}

\title[Spin Representation Theory of Symmetric Groups]{Spin
  Representation Theory of Symmetric Groups and Related Combinatorics \\ Notes
  from a reading course in Fall 2018}
\author{George H. Seelinger}
\date{Fall 2018}
\begin{document}
\maketitle
\section{Introduction (presented by Jinkui Wan)}
When discussing the representation theory of the symmetric group, one
considers \emph{linear representations} which are group
homomoprhisms \[
  \Sym_n \to GL(V)
\]
In 1911, Schur started considering projective representations \[
  \Sym_n \to PGL(V) = GL(V)/\C^*
\]
leading to the projective representation theory of \(\Sym_n\). It
turns out that this corresponds to the linear representation theory of
an extension of \(\Sym_n\), denoted \(\tilde{S}_n\) and referred to as
the \de{double cover of the symmetric group}, fitting into the
short exact sequence \[
  1 \to \Z/2\Z \to \tilde{S}_n \to \Sym_n \to 1
\]
where, if \(\Z/2\Z = \{1,z\}\), then \(z\) is central in
\(\tilde{S}_n\), which gives us that \(z = 1\) or \(z=-1\).

When \(z=1\), we have the representation theory of \(\Sym_n\). When,
\(z=-1\), we have the representation theory of the \emph{spin
  symmetric group algebra}
\[
\C\Sym_n^- = \C\Sym_n/\langle z+1 \rangle = \left\langle t_1, \ldots, t_n
\st
\begin{array}{l}
  t_i^2=1\\
  t_i t_{i+1} t_i = t_{i+1} t_i t_{i+1}\\
  t_i t_j = -t_j t_i \text{ when }|i-j|>1
\end{array}
\right\rangle
\]
which is equipped with a \(\Z/2\Z\)-grading. So, when we discuss
spin representations of \(\Sym_n\), we are discussing linear
representations of \(\C\Sym_n^-\). Our program to establish these
ideas is as follows. \\

Part I
\begin{enumerate}[label=(\arabic*)]
\item Basics of associative superalgebras
\item Connection to Hecke-Clifford (or Sengeev) algebra, \(\H_n\)
\item Split conjugacy classes in a finite supergroup
\item Characteristic map
\item Schur-\(Q\) functions
\item Schur-Sergeev duality
\item Seminormal form of irreducible representations
\end{enumerate}
Part II
\begin{enumerate}[label=(\arabic*)]
\item Centers of \(\C\Sym_n^-\) (analog of Farahat-Higman theory for
  \(\C\Sym_n\))
\item Coinvariant theory for \(\C\Sym_n^-\)
\item Spin Kostka polynomials
\item Quantum deformation (in particular, Olshanki-Sergeev duality)
\end{enumerate}
\begin{bibdiv}
  \begin{biblist}
    \bib{macdonald}{book}{
      author={Macdonald, I.G.}
      title={Symmetric Functions and Hall Polynomials}
      year={1979}
      note={2nd Edition, 1995}
    }
    \bib{sagan}{article}{
      author={Sagan, Bruce E.}
      title={Shifted Tableaux, Schur \(Q\)-functions, and a conjecture
      of R. Stanley}
      year={1987}
      journal={J. Combin. Theory Ser. A}
      pages={62--103}
    }
    \bib{wan-wang}{article}{
      author={Wan, Jinkui}
      author={Wang, Weiqiang}
      title={Lectures on Spin Representation Theory of Symmetric
        Groups}
      year={2012}
      journal={Bull. Inst. Math. Acad. Sin.}
      pages={91--164}
    }
  \end{biblist}
\end{bibdiv}
\end{document}