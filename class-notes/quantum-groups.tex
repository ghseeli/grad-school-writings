\documentclass[11pt,leqno,oneside]{amsbook}
\usepackage{tikz}
\usetikzlibrary{cd}
\usepackage{bbm}
\usepackage{ytableau}
\usepackage{todonotes}

\usepackage{./notes}
\usepackage{../ReAdTeX/readtex-core}
\usepackage{../ReAdTeX/readtex-dangerous}
\usepackage{../ReAdTeX/readtex-abstract-algebra}
\usepackage{../ReAdTeX/readtex-lie-algebras}

\newcommand{\bbk}{\mathbbm{k}}
\newcommand{\Class}{\operatorname{Class}}
\newcommand{\Res}{\operatorname{Res}}
\newcommand{\Ind}{\operatorname{Ind}}
\newcommand{\bs}{\textbackslash}
\newcommand{\partitionof}{\vdash}
\newcommand{\T}{\mathsf{T}} % Tableau
\renewcommand{\S}{\mathsf{S}}
\newcommand{\sh}{\operatorname{shape}}

\newcommand{\dominatedby}{\trianglelefteq}
\newcommand{\dominates}{\trianglerighteq}
\newcommand{\lexicoleq}{\leq}
\newcommand{\lexicogeq}{\geq}
\newcommand{\covers}{\gtrdot}
\newcommand{\coveredby}{\lessdot}
\numberwithin{thm}{section}

\newcommand{\rootlattice}{Q}
\newcommand{\weightlattice}{P}
\renewcommand{\simpleroots}{\Pi}
\newcommand{\U}{\mathcal{U}}
\renewcommand{\b}{\mathfrak{b}}
\newcommand{\ch}{\operatorname{ch}}
\newcommand{\halfsum}{\rho}
\newcommand{\numposrootcombos}{\mathfrak{P}}
\newcommand{\qfactorial}[1]{[#1]_q!}
\newcommand{\qbinom}[3][q]{\left[\begin{array}{c}#2\\#3\end{array}\right]_{#1}}
\newcommand{\A}{\mathbb{A}}
\DeclareMathOperator{\diag}{diag}
\DeclareMathOperator{\wt}{wt}
\renewcommand{\k}{\mathbbm{k}}

\title[Quantum Groups]{Quantum Groups \\ Notes
  from a class taught by Weiqiang Wang in Fall 2019}
\author{George H. Seelinger}
\date{Fall 2019}
\begin{document}
\maketitle
\section{\(q\)-Numbers}
Let \(q\) be an indeterminate. Then, we will work in any of the
following rings \[
  \Z[q] \subset \Z[q,q^{-1}] \subset \Q(q) \subset \C(q)
\]
\begin{defn}
  For an indeterminate \(q\) and \(n \in \Z\), we define
  \begin{enumerate}
  \item \([n]_q := \frac{q^n-q^{-n}}{q-q^{-1}} = q^{n-1} + q^{n-3} +
    \cdots + q^{1-n}\)
  \item \(\qfactorial{0} := 1\) and \(\qfactorial{n} := [n]_q [n-1]_q
    \cdots [1]_q\) for \(n \in \Z_{\geq 0}\)
  \item If \(m \in \Z, n \geq 0\), then \[
      \qbinom{m}{n} = \frac{[m]_q [m-1]_q \cdots
        [m-n+1]_q}{\qfactorial{n}} \overset{\text{If }m \geq 0}{=} \frac{\qfactorial{m}}{\qfactorial{n}\qfactorial{m-n}}
    \]
  \end{enumerate}
\end{defn}
\begin{rmk}
  When the \(q\) is clear, we will drop the \(q\) from the notation
  and say \([n] := [n]_q\), etc.
\end{rmk}
\begin{example}
  We compute some examples of \(q\)-numbers.
  \begin{enumerate}
  \item \([0] = 0\)
  \item \([1] = 1\)
  \item \([2] = q+q^{-1}\)
  \end{enumerate}
\end{example}
\begin{prop}
  We have the following simple identities on \(q\)-numbers.
  \begin{enumerate}
  \item \([-n] = -[n]\) for any \(n \in \Z\).
  \item \(\qbinom{m}{0} = 1 = \qbinom{m}{m}\) for all \(m \in \Z\).
  \item \(\qbinom{m}{n} = 0\) for \(0 \leq m < n\).
  \end{enumerate}
\end{prop}
\begin{prop}\label{q-pascal-identity}
  We have the identity \[
    \qbinom{m+1}{n} = q^{-n} \qbinom{m}{n} + q^{m-n+1} \qbinom{m}{n-1}
  \]
  and also that both \([n]_q\) and \(\qbinom{m}{n}\) are elements of
  \(\Z[q,q^{-1}]\) 
\end{prop}
\begin{proof}
  We compute directly that
  \begin{align*}
     \hspace{-0.5in} q^{-n}\qbinom{m}{n}+q^{m-n+1}\qbinom{m}{n-1}
    & =
    q^{-n}\frac{[m][m-1]\cdots[m-n+1]}{\qfactorial{n}} +
      q^{m-n+1}\frac{[m][m-1]\cdots [m-n+2]}{\qfactorial{n-1}}
      \\
    & =
    q^{-n}\frac{[m][m-1]\cdots[m-n+1]}{\qfactorial{n}} +
    q^{m-n+1}\frac{[n][m][m-1]\cdots [m-n+2]}{[n]\qfactorial{n-1}}\\
    & = (q^{-n}[m-n+1] +
    q^{m-n+1}[n])\frac{[m][m-1]\cdots[m-n+2]}{\qfactorial{n}}\\
    & =
      \left(\frac{q^{m-2n+1}-q^{-m-1}}{q-q^{-1}}+\frac{q^{m+1}-q^{m-2n+1}}{q-q^{-1}}\right)\frac{[m]\cdots[m-n+2]}{\qfactorial{n}}
    \\
    & =
      \left(\frac{q^{m+1}-q^{-(m+1)}}{q-q^{-1}}\right)\frac{[m]\cdots[m-n+2]}{\qfactorial{n}}
    \\
    & = \frac{[m+1][m] \cdots [(m+1)-n+1]}{\qfactorial{n}} = \qbinom{m+1}{n}
  \end{align*}
  Now, observe that \[
    [n]_q = \frac{q^n-q^{-n}}{q-q^{-1}} = \frac{q}{q^n}\cdot
    \frac{q^{2n}-1}{q^2-1} =
    \frac{1}{q^{n-1}}(q^{2n-2}+q^{2n-4}+\cdots+q^2+1) \in \Z[q,q^{-1}]
  \]
  This immediately gives that \(\qfactorial{n} \in \Z[q,q^{-1}]\). To
  show \(\qbinom{m}{n} \in \Z[q,q^{-1}]\), we proceed by
  induction on \(m\). Namely, \(\qbinom{m}{0} = 1 \in \Z[q,q^{-1}]\) for all
  \(m \in \Z\). Then, \[
    \qbinom{m+1}{n} = q^{n}\underbrace{\qbinom{m}{n}}_{\in
      \Z[q,q^{-1}]} + q^{-m+n-1}\underbrace{\qbinom{m}{n-1}}_{\in
      \Z[q,q^{-1}]} \in \Z[q,q^{-1}]
  \]
\end{proof}
\begin{thm}{\(q\)-Binomial Theorem}\label{q-binomial-theorem}
  For an indeterminate \(z\) and \(r \geq 0\), 
  \[
    \prod_{j=0}^{r-1} (1+q^{2j}z) = \sum_{i=0}^{r-1} q^{i(r-1)}
    \qbinom{r}{i} z^i
  \]
\end{thm}
\begin{proof}
  This follows by induction. If \(r = 0\), then we simply have \(1 =
  1\). Now, proceed by induction. Then, \[
    \prod_{j=0}^r (1+q^{2j}z) = (1+q^{2r}z)\left( \sum_{i=0}^{r-1}
      q^{i(r-1)} \qbinom{r}{i} z^i \right) = \sum_{i=0}^{r-1}
      q^{i(r-1)} \qbinom{r}{i} z^i + \sum_{i=0}^{r-1} q^{i(r-1)+2r} \qbinom{r}{i}z^{i+1}
  \]
  Then, if we fix the \(z\) power for some \(1 \leq k \leq r-1\), we
  get coefficient
  \begin{align*}
    q^{k(r-1)} \qbinom{r}{k} + q^{(k-1)(r-1)+2r} \qbinom{r}{k-1}
    & =
      q^{k(r-1)} \qbinom{r}{k} + q^{k(r-1)+r+1} \qbinom{r}{k-1} \\
    & = q^{kr}
      \left( q^{-k} \qbinom{r}{k} + q^{-k+r+1} \qbinom{r}{k-1}\right) \\
    & = q^{kr} \qbinom{r+1}{k}
  \end{align*}
  where the last equality follows from~\ref{q-pascal-identity}.
\end{proof}
\begin{cor}
  As consequences to~\ref{q-binomial-theorem}, we get
  \begin{enumerate}
  \item For \(r \geq 1\),
    \[
      \sum_{i=0}^r (-1)^i q^{-i(r-1)} \qbinom{r}{i} = 0
    \]
  \item Assume \(xy = q^2yx\). Then, \[
      (x+y)^n = \sum_{t=0}^n q^{t(n-t)}\qbinom{n}{t}y^t x^{n-t}
    \]
  \end{enumerate}
\end{cor}
Sometimes in the literature, \(q\)-numbers are encoded slightly
differently. We present the alternate definition here.
\begin{defn}
  \(\{n\}_v := 1+v+v^2 + \cdots + v^{n-1} = \frac{v^n-1}{v-1}\)
\end{defn}
Then, the two definitions are related as follows.
\begin{prop}
  Setting \(v=q^2\), \[
    \{n\}_v = q^{n-1}[n]_q
  \]
\end{prop}
\section{The Quantum Group \(\U_q(\sl_2)\)}
Throughout this section, we will let \(\U := \U_q(\sl_2)\). Let \(\k\)
be a field of characteristic \(0\) with \(q \in \k\), \(q \neq 0\),
and \(q\) is not a root of \(1\).
\begin{defn}
  We define the \de{quantum group \(\U := \U_q(\sl_2)\)} to be the \(\k\)-algebra
  generated by elements \(E,F,K,K^{-1}\) with relations
  \begin{enumerate}
  \item \(KK^{-1} = 1 = K^{-1}K\)
  \item \(KE = q^2 EK\)
  \item \(KF = q^{-1}FK\)
  \item \(EF-FE = \frac{K-K^{-1}}{q-q^{-1}}\)
  \end{enumerate}
\end{defn}
\begin{defn}
  We define the \de{Drinfield double} \(\tilde{\U} = \langle E,F,K,K' \rangle\) to be the
  \(\k\)-algebra with relations
  \begin{enumerate}
  \item \(K'E = q^{-2}EK'\)
  \item \(K'F = q^2 EK'\)
  \item \(EF-FE = \frac{K-K'}{q-q^{-1}}\)
  \end{enumerate}
\end{defn}
\begin{rmk}
  Note that \(\tilde{U}/\langle KK'-1 \rangle \isom \U\) and that
  \(KK'\) is central in \(\tilde{U}\).
\end{rmk}
\begin{defn}
  We define the following maps.
  \begin{enumerate}
  \item The \(\k\)-linear involution \(\omega\) acts on \(\U\) by
    \[
      \omega(E) = F, \omega(F) = E, \omega(K) = K^{-1}
    \]
  \item The \(\k\)-linear anti-involution \(\tau\) (ie \(\tau(xy) =
    \tau(y)\tau(x)\)) acts on \(\U\) by \[
      \tau(E) = E, \tau(F) = F, \tau(K) = K^{-1}
    \]
  \end{enumerate}
\end{defn}
\begin{defn}
  For making computations more compact, we define the short hand
  \begin{enumerate}
  \item \([K;n] = \frac{q^n K-q^{-n}K^{-1}}{q-q^{-1}}\)
  \item For \(n \in \Z_{\geq 0}\), \(E^{(n)} =
    \frac{E^n}{\qfactorial{n}}\) and \(F^{(n)} =
    \frac{F^n}{\qfactorial{n}}\). 
  \end{enumerate}
\end{defn}
\begin{thm}[PBW Theorem]
  \(\{F^sK^nE^r \st s,r \in \Z_{\geq 0}, n \in \Z\}\) forms a basis
  for \(\U\). 
\end{thm}
\begin{proof}[Sketch of Proof]
  \begin{enumerate}
  \item Use the commutation relations of \(\U\) to show that this is a
    spanning set; when commuting an \(E\) past an \(F\), one only
    picks up lower degree correction terms.
  \item Construct a ``regular representation'' \(M =
    \k[\tilde{F},\tilde{E},\tilde{K},\tilde{K^{-1}}]\) on which \(\U\) acts to show linear
    independence. This argument is more sophisticated, but since this
    is a faithful representation, you get that the map \(\theta \from \U
    \to \End_\k(M)\) is injective and since \(\theta(F^sK^nE^r)(1) =
    \tilde{F}^s\tilde{K}^n\tilde{E}^r\), which is known to be linearly
    independent, then the set \(\{\theta(F^s K^N E^r)\}\) is linearly
    independent, thus giving us the desired linear independence by the
    injectivity of \(\theta\). See~\cite{jantzen}*{Theorem 1.5}.
  \end{enumerate}
\end{proof}
\begin{lem}[Useful Identities]\label{useful-identities}
  \begin{enumerate}
  \item \([K;n]E = E[K;n+2]\)
  \item \([K;n]F = F[K;n-2]\)
  \item \(EF^s = F^s E + [s]F^{s-1}[K;1-s]\) for \(s \geq 0\)
  \item \(E^rF^s = \sum_{i=0}^{\min(r,s)} \qbinom{r}{i}\qbinom{s}{i}
    [i]! F^{s-i}\left( \prod_{j=1}^i [K;i-(r+s)+j] \right)E^{r-i}\)
  \item[d'] \(E^{(r)}F^{(s)} = \sum_{i=0}^{\min(r,s)} F^{(s-i)}
    \qbinom{K;2i-(r+s)}{i} E^{(r-i)}\) where \(\qbinom{K;c}{i} :=
    \frac{[K;c][K;c-1] \cdots [K;c-i+1]}{[i]!}\).
  \end{enumerate}
\end{lem}
Identity (d') gives one reason why divided powers are sometimes more
convenient; writing identities with them can sometimes be simpler.
\begin{rmk}
  \(\U_q(\sl_2)\) has no zero-divisors.
\end{rmk}
\subsection{Finite-dimensional Representations of \(\U_q(\sl_2)\)}
\begin{example}
  Let \(M = \k m_0 \oplus \k m_1\) with \(K m_0 = qm_0\) and \(Km_1 =
  q^{-1}m_1\) and \(E,F\) actions given by \[
    \begin{tikzcd}
      0 & \ar[l, "F"] m_1 \ar[r, bend left, "E"] & \ar[l, bend left,
      "F"] m_0 \ar[r, "E"] & 0
    \end{tikzcd}
  \]
\end{example}
\begin{lem}
  Let \(M\) be a finite-dimensional \(\U\)-module. Then, there exists
  an \(r > 0\) such that \(E^r M = 0\) and \(F^r M = 0\).
\end{lem}
\begin{defn}
  For \(M \in \U\catname{-mod}\), \(\lambda \in \k^\times\), let
  \(M_\lambda := \{m \in M \st Km = \lambda m\}\) be called the
  \de{\(\lambda\)-weight subspace} of \(M\).
\end{defn}
\begin{lem}
  \begin{enumerate}
  \item \(E M_\lambda \subset M_{q^2 \lambda}\) and
    \(F M_{\lambda} \subset M_{q^{-2} \lambda}\).
  \item If \(M_\lambda \neq 0\) and \(M\) is simple, then \[
      M = \bigoplus_{n \in \Z} M_{q^{2n} \lambda}
    \]
  \end{enumerate}
\end{lem}
\begin{prop}
  Let \(M\) be a finite-dimensional \(\U\)-module. Then, \[
    M = \bigoplus_{a \in \Z} M_{+q^a} \oplus M_{-q^a}
  \]
\end{prop}
\begin{proof}
  It is equivalent to show that the minimal polynomial of \(K\) on
  \(M\) is of the form \(\prod_i (K-\lambda_i)\) with distinct
  \(\lambda_i \in \pm q^\Z\). To do this, set \[
    h_r := \prod_{j=1-r}^{r-1} [K;r-s+j], \ \ r > 0, h_0 = 1
  \]
  Now, for \(s > 0\), if \(F^s M = 0\), then \(F^{s-r}h_r M = 0\) for
  all \(0 \leq r \leq s\) because \[
    \left(E^r F^s \prod_{j=1}^{r-1} [K;r-s+j]\right) M = \left(
      \sum_{i=0}^r a_i F^{s-i} h_i \prod_{j=0}^{r-i-1}[K;-s-j] E^{r-i}\right)M
  \]
  for \(a_i = \qbinom{r}{i}\qbinom{s}{i} [i]!\)
  by~\ref{useful-identities}(d) allows us to use induction. Then, we
  have \[
    0 = h_s M = \prod_{j=1-s}^{s-1} \left[ \underbrace{(q-q^{-1})^{-1}
        q^j K^{-1}}_{\text{Invertible scalar}}
      \underbrace{(K^2-q^{-2j})}_{\text{Minimal polynomial divides this}} \right]M
  \]
  and thus we have distinct \(\lambda_i \in \pm q^{\Z}\)
\end{proof}
\begin{bibdiv}
  \begin{biblist}
    \bib{jantzen}{book}{
      author={Jantzen, Jens Carsten}
      title={Lectures on Quantum Groups}
      year={1995}
    }
  \end{biblist}
\end{bibdiv}
\end{document}