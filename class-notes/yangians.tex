\documentclass[11pt,leqno,oneside]{amsbook}
\usepackage{tikz}
\usetikzlibrary{cd}
\usepackage{bbm}
\usepackage{ytableau}
\usepackage{todonotes}

\usepackage{./notes}
\usepackage{../ReAdTeX/readtex-core}
\usepackage{../ReAdTeX/readtex-dangerous}
\usepackage{../ReAdTeX/readtex-abstract-algebra}
\usepackage{../ReAdTeX/readtex-lie-algebras}

\newcommand{\bbk}{\mathbbm{k}}
\newcommand{\Class}{\operatorname{Class}}
\newcommand{\Res}{\operatorname{Res}}
\newcommand{\Ind}{\operatorname{Ind}}
\newcommand{\bs}{\textbackslash}
\newcommand{\partitionof}{\vdash}
\newcommand{\T}{\mathsf{T}} % Tableau
\renewcommand{\S}{\mathsf{S}}
\newcommand{\sh}{\operatorname{shape}}

\newcommand{\dominatedby}{\trianglelefteq}
\newcommand{\dominates}{\trianglerighteq}
\newcommand{\lexicoleq}{\leq}
\newcommand{\lexicogeq}{\geq}
\newcommand{\covers}{\gtrdot}
\newcommand{\coveredby}{\lessdot}
\numberwithin{thm}{section}

\newcommand{\rootlattice}{Q}
\newcommand{\weightlattice}{P}
\renewcommand{\simpleroots}{\Pi}
\newcommand{\U}{\mathcal{U}}
\renewcommand{\b}{\mathfrak{b}}
\newcommand{\ch}{\operatorname{ch}}
\newcommand{\halfsum}{\rho}
\newcommand{\numposrootcombos}{\mathfrak{P}}
\newcommand{\qfactorial}[1]{[#1]_q!}
\newcommand{\qbinom}[3][q]{\left[\begin{array}{c}#2\\#3\end{array}\right]_{#1}}
\newcommand{\A}{\mathbb{A}}
\DeclareMathOperator{\diag}{diag}
\DeclareMathOperator{\wt}{wt}
\renewcommand{\k}{\mathbbm{k}}
\newcommand{\associatedGraded}{\operatorname{gr}}
\newcommand{\qdet}{\operatorname{qdet}}

\title[Yangians]{Yangians \\ Notes
  from a reading course in Fall 2019}
\author{George H. Seelinger}
\date{Fall 2019}
\begin{document}
\maketitle
\section{Yangians for \(\gl_N\)}
For some motivation, consider the standard basis elements \(E_{ij}\)
of \(\gl_N\) satisfy the relation \[
  [E_{ij}, E_{kl}] = \delta_{kj} E_{il} - \delta_{il}E_{kj}
\]
Now, if we consider an \(N\times N\) matrix \(E\) whose \(ij\)th entry
is \(E_{ij}\), ie \[
  E := \sum_{i,j} e_{ij} \otimes E_{ij} \in \End \C^N \otimes \U(\gl_N)
\]
we have the less well known relation for any \(r,s \in \Z_{\geq 0}\), \[
  [(E^{r+1})_{ij}, (E^s)_{kl}] - [(E^r)_{ij}, (E^{s+1})_{kl}] =
  (E^r)_{kl}(E^s)_{il} - (E^s)_{kj}(E^r)_{il}
\]
\begin{defn}
  We define the \de{Yangian} for \(\gl_N\), denoted \(Y(\gl_N)\) or
  \(Y_N\) for short, to be the associative algebra generated by
  elements \(t_{ij}^{(r)}\) for \(r \in \Z_{\geq 0}\) and \(1 \leq
  i,j \leq N\) with relations
  \begin{equation}
    \label{eq:yangian-relns1}
        [t_{ij}^{(r+1)},t_{kl}^{(s)}] - [t_{ij}^{(r)}, t_{kl}^{(s+1)}] =
        t_{kj}^{(r)}t_{il}^{(s)} - t_{kj}^{(s)}t_{il}^{(r)} 
  \end{equation}
  where \(r,s \in \Z_{\geq 0}\) and \(t_{ij}^{(0)} = \delta_{ij}\).
\end{defn}
If we define the formal power series \(t_{ij}(u) := \sum_{r \geq 0}
t^{(r)}_{ij}u^{-r} = \delta_{ij} + t^{(1)}_{ij}u^{-1} +
t^{(2)}_{ij}u^{-2} + \cdots \in Y_N[[u^{-1}]]\), then
\begin{equation}
  \label{eq:yangian-relns2}
  (u-v)[t_{ij}(u),t_{kl}(v)] = t_{kj}(u)t_{il}(v) - t_{kj}(v)t_{il}(u)
\end{equation}
where the indeterminants \(u,v\) commute with each other. One then
recovers~\ref{eq:yangian-relns1} by taking the coefficient of 
\(u^{-r}v^{-s}\).
\begin{defn}
  Let us define the following elements.
  \begin{enumerate}
  \item \(T(u) := \sum_{i,j} e_{ij} \otimes t_{ij}(u) \in \End \C^N
    \otimes Y_N[[u^{-1}]]\)
  \item \(P = \sum_{i,j} e_{ij} \otimes e_{ji} \in (\End
    \C^N)^{\otimes 2}\), ie \(P\) is the endomorphism given by \[
      P(x \otimes y) = y \otimes x, \ \ x,y \in \C^N
    \]
  \item \(P_{ab} := \sum 1^{a-b} \otimes e_{ij} \otimes a^{b-a-1}
    \otimes e_{ji} \otimes 1^{m-b} \in (\End \C^N)^{\otimes m}\)
  \item \(R(u) := 1-Pu^{-1}\)
  \item \(R_{ab}(u) := 1-P_{ab}u^{-1}\)
  \end{enumerate}
\end{defn}
\begin{prop}[Yang-Baxter Equation]
  \(R_{12}(u)R_{13}(u+v)R_{23}(v) = R_{23}(v)R_{13}(u+v)R_{12}(u)\) in
  \((\End \C^N)^{\otimes 3}(u,v)\).
\end{prop}
\begin{proof}
  Multiply the desired formula above by \(uv(u+v)\) to get \[
    (u-P_{12})(u+v-P_{13})(v-P_{23}) = (v-P_{23})(u+v-P_{13})(u-P_{12})
  \]
  Then, it suffices to show that \(P_{12}P_{13}P_{23} =
  P_{23}P_{13}P_{12}\) which we can do by regarding these as elements
  of \(\Sym_3\) by their action on \((\C^N)^{\otimes 3}\). In
  particular, \(P_{ij}\) acts as \((i,j)\), and so the result is
  immediate from the relations in \(\C[\Sym_3]\).
\end{proof}
\begin{thm}[RTT Relations]\label{rtt-relations}
  We have that 
  \[
    R(u-v)T_1(u)T_2(v) = T_2(v)T_1(u)R(u-v)
  \] in \((\End
  \C^N)^{\otimes 2} \otimes Y_N[[u^{-1}]]\) where \(T_1(u) = \sum
  e_{ij} \otimes 1 \otimes t_{ij}(u)\) and \(T_2(v) = \sum 1 \otimes
  e_{ij} \otimes t_{ij}(v)\). Furthermore, this relation is equivalent
  to the defining relations of the Yangian~\ref{eq:yangian-relns1}.
\end{thm}
\begin{proof}
  First, note that \(P \circ (e_{ij} \otimes e_{kl}) = e_{kj} \otimes
  e_{il} \in (\End \C^N)^{\otimes 2}\) by just observing where this
  endomorphism would send an \(i\)th coordinate vector and a \(k\)th
  coordinate vector. Similarly, \((e_{ij} \otimes e_{kl}) \circ P =
  e_{il} \otimes e_{kj}\).
  
  Now, we can directly compute
  \begin{align*}
    R(u-v)T_1(u)T_2(v)
    & = \left( 1-\frac{1}{u-v}P \right)\left( \sum_{i,j} e_{ij}
      \otimes 1 \otimes t_{ij}(u) \right) \left( \sum_{k,l} 1 \otimes
      e_{kl} \otimes t_{kl}(v) \right) \\
    & = \sum_{i,j,k,l} e_{ij} \otimes e_{kl} \otimes t_{ij}(u)t_{kl}(v) -
      \frac{1}{u-v} \sum_{i,j,k,l} e_{kj} \otimes e_{il} \otimes
      t_{ij}(u)t_{kl}(v)\\
    & = \sum_{i,j,k,l} e_{ij} \otimes e_{kl} \otimes t_{ij}(u)t_{kl}(v) -
      \frac{1}{u-v} \sum_{k,l,i,j} e_{il} \otimes e_{kj} \otimes
      t_{kl}(u)t_{ij}(v)\\
    & = \left( \sum_{k,l} 1 \otimes
      e_{kl} \otimes t_{kl}(u) \right) \left( \sum_{i,j} e_{ij}
      \otimes 1 \otimes t_{ij}(v) \right) \left( 1-\frac{1}{u-v}P
      \right) \\
    & = T_2(v)T_1(u)R(u-v)
  \end{align*}
\end{proof}
\begin{prop}\label{yangian-lie-bracket}
  \[
     [t_{ij}^{(r)}, t_{kl}^{(s)}] = \sum_{a=1}^{\min(r,s)}
     t_{kj}^{(a-1)}t_{il}^{(r+s-1)} - t_{kj}^{(r+s-a)} t_{il}^{(a-1)}
  \]
\end{prop}
\begin{proof}
  Multiply \(\sum_{p=0}^\infty u^{-p-1}v^p\) to~\ref{yangian-relns2}
  to get \[
    [t_{ij}(u), t_{kl}(v)] = (t_{kj}(u)t_{il}(v) -
    t_{kj}(v)t_{il}(u))\left( \sum_{p=0}^\infty u^{-p-1}v^p \right)
  \]
  Then, taking coefficients of \(u^{-r}v^{-s}\) on both sides gives \[
    [t^{(r)}_{ij}, t^{(s)}_{kl}] = \sum_{a=1}^r
    (t_{kj}^{(a-1)}t_{il}^{(r+s-a)}-t_{kj}^{(r+s-a)}t_{il}^{(a-1)}) 
  \]
  which gives our desired formula if \(r \leq s\). However, if \(r >
  s\), then \[
    \sum_{a=s+1}^r (t_{kj}^{(a-1)}t_{il}^{(r+s-a)} -
    t_{kj}^{(r+s-a)}t_{il}^{(a-1)}) = 0
  \]
  since, for instance, at \(a=s+1\), we get the term \[
    t_{kj}^{(s)}t_{il}^{(r-1)} - t_{kj}^{(r-1)}t_{il}^{(s)}
  \]
  and at \(a=r\) we get \[
    t_{kj}^{(r-1)}t_{il}^{(s)}-t_{kj}^{(s)}t_{il}^{(r-1)}
  \]
  the sum of which is zero.
\end{proof}
\begin{defn}
  Let us define the \de{natural filtration} on \(Y_N\) to be given by
  \(\deg t_{ij}^{(r)} = r\). 
\end{defn}
\begin{lem}
  \(\associatedGraded Y(\gl_N)\) is commutative. 
\end{lem}
\begin{proof}
  This is immediate from a degree argument based on the relations~\ref{yangian-relns1}.
\end{proof}
  In fact, we will later see that \(\associatedGraded Y(\gl_N) \isom
  \C[t_{ij}^{(r)}]\) as algebras.
\begin{thm}[PBW Theorem]
  Given an arbitrary linear order of the \(t_{ij}^{(r)}\), the order
  monomials form a basis of \(Y(\gl_N)\).
\end{thm}
\begin{proof}[Idea of Proof]
  If we let \(\ov{t}_{ij}^{(r)}\) be the image of \(t_{ij}^{(r)}\) in
  the \(r\)th component of \(\associatedGraded Y(\gl_N)\), then it
  suffised to show that the \(\ov{t}_{ij}^{(r)}\) are algebraically
  independent. Then, one constructs a homomorphism for any \(M \geq 0\), \[
    \ov{\zeta}_M \from \associatedGraded Y(\gl_N) \to \Sym(\gl_{N+M})
  \]
  such that \(\ov{\zeta}_M(\ov{t}_{ij}^{(r)}) = p_{ij}^{(r)}\) such
  that \(p_{ij}^{(r)}(X) = (X^r)_{ij}\) for any \(X \in
  \gl_{M+N}\). Then, if one considers a finite family of elements
  \(\ov{t}_{ij}^{(r)}\), there is some \(R\) such that \(1 \leq r \leq
  R\) and one needs to demonstrate that the parameter \(M\) can be
  chosen so that the polynomials \(p_{ij}^{(r)}\) are algebraically
  independent. 
\end{proof}
\begin{cor}
  \begin{enumerate}
  \item \(\associatedGraded Y(\gl_N) = \C[\ov{t}_{ij}^{(r)}]\).
  \item There exists an embedding \(\U(\gl_N) \into Y(\gl_N)\) such
    that \(E_{ij} \mapsto t_{ij}^{(1)}\).
  \item There exists an ``evaluation'' map \(Y(\gl_N) \onto
    \U(\gl_N)\) induced by sending the power series \(t_{ij}(u)
    \mapsto \delta_{ij}+E_{ij}u^{-1}\).
  \item 
  \end{enumerate}
\end{cor}
\begin{prop}\label{anti-automorphisms}
  There exists three families of automorphisms of \(Y(\gl_N)\) induced
  by sending the power series 
    \(T(u)\) to the following:
    \begin{enumerate}
    \item \(T(u) \mapsto f(u)T(u)\) for any \(f(u) :=
      1+f_1u^{-1}+\cdots \in \C[[u^{-1}]]\),
    \item \(T(u) \mapsto T(u-c)\) for any \(c \in \C\),
    \item \(T(u) \mapsto BT(u)B^{-1}\) for any \(B \in GL_N\C\).
    \end{enumerate}
\end{prop}
\begin{proof}
  One needs only show that these maps are actually homomorphisms since
  any map from a given family also contains its own inverse.

  To show that the maps are homomorphisms, one shows they satisfy the
  RTT relations~\ref{rtt-relations}. The first two are more or less
  immediate. The third is a little more work, but is still straightforward.
\end{proof}
\begin{prop}
  There exist antiautomorphisms of \(Y(\gl_N)\) induced by sending the power series
  \(T(u)\) to the following:
  \begin{enumerate}
  \item \(\sigma(T(u)) = T(-u)\),
  \item \(t(T(u)) = T^t(u) := \sum e_{ji} \otimes t_{ij}(u)\), the
    transpose of \(T(u)\),
  \item \(S(T(u)) = T^{-1}(u)\)
  \end{enumerate}
\end{prop}
\begin{thm}
  The Yangian \(Y(\gl_N)\) has a Hopf algebra structure with
  comultiplication induced by \[
    \Delta(t_{ij}(u)) = \sum_{k=1}^N t_{ik}(u) \otimes t_{kj}(u)
  \]
  antipode given by \(S\) from~\ref{anti-automorphisms}(c), and counit
  induced by \(\epsilon(T(u)) = 1\).
\end{thm}
\todo{Write down proof}
\begin{defn}
  We define the \de{loop filtration} on \(Y(\gl_N)\) to be given by 
  \(\deg' t_{ij}^{(r)} = r-1\). 
\end{defn}
\begin{prop}
  \(\U(\gl_N[z]) \isom \associatedGraded'(Y(\gl_N))\) as Hopf algebras
  via the map sending \[
    E_{ij} z^{r-1} \mapsto t_{ij}^{(r)}
  \]
  for \(r \geq 1\)
\end{prop}
\begin{proof}
  In \(\associatedGraded' Y_N\), we have the relation
  \[ [\ov{t}_{ij}^{(r)}, \ov{t}_{kl}^{(s)}] = \delta_{kj}
    \ov{t}_{il}^{(r+s-1)} - \delta_{il}\ov{t}_{kj}^{(r+s-1)}
  \]
  which follows from~\ref{yangian-lie-bracket}. Thus, the map gives a
  surjective homomorphism of algebras with trivial kernel by the PBW
  Theorem~\ref{pbw-theorem}. Then, one needs only compare the
  comultiplication structures to conclude this is a morphism of Hopf
  algebras. 
\end{proof}
\begin{rmk}
  Consider \(Y(\gl_N,h)\) for any \(h \in \C\) which is an associative
  algebra generated by \(t_{ij}^{(r)}\) subject to the relations \[
    [t_{ij}^{(r)}, t_{kl}^{(s)}] = \delta_{kj} t_{il}^{(r+s-1)} -
    \delta_{il}t_{kl}^{(r+s-1)} + h\left( \sum_{a=2}^{\min(r,s)}
      t_{kj}^{(a-1)} t_{il}^{(r+s-a)} - t_{kj}^{(r+s-a)}t_{il}^{(a-1)} \right)
  \]
  with comultiplication \[
    \Delta(t_{ij}^{(r)}) = t_{ij}^{(r)} \otimes 1 + 1 \otimes
    t_{ij}^{(r)} + h \sum_{k=1}^N \sum_{s=1}^{r-1} t_{ik}^{(s)}
    \otimes t_{kj}^{(r-s)}
  \]
  Then,
  \begin{itemize}
  \item \(Y(\gl_N,0) \isom \U(\gl_N[z])\),
  \item \(Y(\gl_N,1) = Y(\gl_N)\), 
  \item For all \(h \neq 0\), \(Y(\gl_N,h) \isom Y(\gl_N)\) as Hopf
    algebras via \(t_{ij}^{(r)} \mapsto t_{ij}^{(r)} h^{r-1}\).
  \end{itemize}
\end{rmk}
\subsection{Quantum Determinant}
The quantum determinant will help us get a concrete description of the
center of \(Y(\gl_N)\).
\begin{defn}
  Let us define the rational function in variables \(u_1, \ldots,
  u_m\) with values in \((\End \C^N)^{\otimes m}\) by \[
Q(u_1,\ldots,u_m) := (Q_{m-1,m})(Q_{m-2,m}Q_{m-2,m-1}) \cdots
  (\underbrace{Q_{m-a,m} \cdots Q_{m-a,m-a+1}}_{a \text{ terms}})
  \cdots (Q_{1,m}\cdots Q_{1,2})
  \]
  where \(Q_{i,j} = R_{ij}(u_i-u_j) = 1-\frac{P_{ij}}{u_i-u_j}\). 
\end{defn}
\begin{prop}\label{quantum-det-reln1}
  We have that \[
    Q(u_1, \ldots, u_m) T_1(u_1) \cdots T_m(u_m) = T_m(u_m) \cdots
    T_1(u_1) Q(u_1, \ldots, u_m)
  \]
\end{prop}
\begin{prop}\label{quantum-det-reln2}
  \begin{enumerate}
  \item If \(u_i-u_{i+1} = 1\) for all \(1 \leq i < m\), then
    \[
      Q(u_1, \ldots, u_m) = A_m
    \]
    where \(A_m\) is the natural image of the anti-symmetrizer under
    the natural action of \(\Sym_m\) on \((\C^N)^{\otimes m}\).
  \item If \(u_i-u_{i+1} = -1\) for all \(1 \leq i < m\), then \[
      Q(u_1, \ldots, u_m) = H_m
    \]
    where \(H_m\) is the natural image of the symmetrizer under the
    natural action of \(\Sym_m\) on \((\C^N)^{\otimes m}\).
  \end{enumerate}
\end{prop}
\begin{proof}
  The proof is done via induction.
\end{proof}
Now, consider that, by~\ref{quantum-det-reln1}
and~\ref{quantum-det-reln2}(a), we have the equality \[ 
  A_m T_1(u) \cdots T_m(u-m+1) = T(u-m+1) \cdots T_1(u) A_m
\]
where \(A_m\) is identified with \(A_m \otimes 1\). Then, if \(m=N\),
the space \(A_N (C^N)^{\otimes N} = \langle e_1 \otimes \cdots \otimes
e_N\rangle\) is one dimensional. Thus, \[
  A_N T_1(u) \cdots T_N(u-N+1) = C A_N
\]
for some \(C \in Y_N[[u^{-1}]]\) and so we are led to the following
definition. 
\begin{defn}
  We define \(\qdet T(u) \in Y_N[[u^{-1}]]\) to be the element
  such that \[
    A_N T_1(u) \cdots T_N(u-N+1) = A_N \qdet T(u)
  \]
  Furthermore, we define \(d_i \in Y_N\) by the formal power series
  \(\qdet T(u) = 1+d_1 u^{-1} + d_2 u^{-2} + \cdots\)
\end{defn}
\begin{prop}
  We have explicit formula \[
    \qdet T(u) = \sum_{\sigma \in \Sym_N} \sgn \sigma
    t_{\sigma(1),1}(u) \cdots t_{\sigma(N),N}(u-N+1) 
  \]
\end{prop}
\begin{defn}
  For \(m \leq N\), we define \de{quantum minors}
  \(t^{a_1,\ldots,a_m}_{b_1,\ldots,b_m}(u)\) to be the numbers given
  by \[
    A_m T_1(u) \cdots t_m(u-m+1) = \sum_{a_i,b_i=1}^N e_{a_1 b_1}
    \otimes \cdots \otimes e_{a_m, b_m} \otimes t^{a_1,\ldots,a_m}_{b_1,\ldots,b_m}(u)
  \]
\end{defn}
\begin{rmk}
  When \(m=N\), we recover the quantum determinant. In other words,
  \(\qdet T(u) = t^{1 \cdots N}_{1 \cdots N}(u)\).
\end{rmk}
\begin{prop}
  For \(\sigma \in \Sym_m\), we have \(t^{\sigma(a_1) \cdots
    \sigma(a_m)}_{b_1 \cdots b_m}(u) = \sgn \sigma t^{a_1 \cdots
    a_m}_{b_1 \cdots b_m}(u)\).
\end{prop}
\begin{lem}
  \cite{molev}*{Prop 1.7.1} \[
    (u-v)[t_{kl}(u),t^{a_1\cdots a_m}_{b_1 \cdots b_m}(v)] =
    \sum_{i=1}^m t_{a_i l}(u)t^{a_1 \cdots k \cdots a_m}_{b_1 \cdots
      b_m}(v) - \sum_{i=1}^m t^{a_1 \cdots a_m}_{b_1 \cdots l \cdots
      b_m}(v)t_{k b_i}(u)
  \]
  where the indices \(k\) and \(l\) in the quantum minors replace
  \(a_i\) and \(b_i\), respectively.
\end{lem}
\begin{cor}\label{quantum-det-is-central}
  \cite{molev}*{Corollary 1.7.2} For any indices \(i,j\) we have that
  \([t_{a_i b_j}(u), t^{a_1 
    \cdots a_m}_{b_1 \cdots b_m}(v)] = 0\). In particular,
  \([t_{kl}(u), t^{1 \cdots N}_{1 \cdots N}(v)] = 0\) for all \(1 \leq
  k,l \leq N\). 
\end{cor}
\begin{thm}
  \cite{molev}*{Theorem 1.7.5} The elements \(d_1, d_2, \ldots\) generate the center of
  \(Y(\gl_N)\) and they are algebraically independent. 
\end{thm}
\begin{proof}[Idea of Proof]
  The centrality of the \(d_i\) are given
  by~\ref{quantum-det-is-central}.
  
  For independence, we send \(d_i\) to an element of \(\U(\gl_N[z]) \isom
  \associatedGraded' Y_N\) and see that the images are algebraically
  independent, so the \(d_i\)'s themselves must be. We do not address
  the fact that the \(d_i\) generate the center here.
\end{proof}
\begin{bibdiv}
  \begin{biblist}
    \bib{molev}{book}{
      author={Molev, Alexander}
      title={Yangians and Classical Lie Algebras}
      year={2007}
    }
  \end{biblist}
\end{bibdiv}
\end{document}