\documentclass[springer-theory-notes.tex]{subfiles}

\begin{document}
\section{Applications of the Jacobson-Morozov Theorem}
\subsection{Statement and Immediate Applications}
\begin{defn}
  Let \((e,f,h)\) be a triple of elements in some semisimple Lie
  algebra \(\g\) over a field of characteristic \(0\). We say
  \((e,f,h)\) is a \de{\(\sl_2\)-triple} if \[
    [h,e] =  2e, \ [h,f] = -2f, \ [e,f] = h.
  \]
\end{defn}
We recall from a first course in semisimple Lie algebras the following
proposition. 
\begin{prop}
  A Lie algebra \(\g\) decomposes into a direct sum of
  finite-dimensional subspaces, each of which is isomorphic to
  \(V_j\), the \(j+1\)-dimensional simple \(\sl_2\)-module with
  highest weight \(j\). \todo{Elaborate}
\end{prop}
To each of these simple \(\sl_2\)-modules, we can associate an
\(\sl_2\)-triple in \(\g\). However, the Jacobson-Morozov Theorem can
provide us with a partial converse to this fact.
\begin{thm}[Jacboson-Morozov Theorem]
  \cite{cg}*{3.7.1} Let \(\g\) be a complex semisimple Lie
  algebra. For any nilpotent 
  element \(e \in \g\), there 
  exists \(h,f \in \g\) such that \(\{e,f,h\}\) is an
  \(\sl_2\)-triple. \\

  Thus, there exists a Lie algebra homomorphism \(\gamma \from \sl_2
  \to \g\) such that \[
    \left(
      \begin{array}{cc}
        0&1\\
        0&0
      \end{array}
\right) \mapsto e, \ \left(
  \begin{array}{cc}
    0&0\\
    1&0
  \end{array}
\right) \mapsto f, \ \left(
  \begin{array}{cc}
    1&0\\
    0&-1
  \end{array}
\right) \mapsto h
\]
  Moreover, \(h\) is a semisimple element and \(f\) is a nilpotent
  element of \(\g\).
\end{thm}
\begin{rmk}
  This triple \((e,f,h)\) associated with \(e\) is not necessarily
  unique. The following proposition, due to Kostant, tells us to what
  extent the associated triple is unique.
\end{rmk}
\begin{prop}[Kostant's Theorem]
  \cite{cg}*{3.7.3} Let \(Z_G(e)\) denote the centralizer in \(G\) of
  \(e\). Then, the 
  above homomorphism \(\gamma \from \sl_2(\C) \to \g\) is determined
  uniquely up to conjugation by an element in the unipotent radical of
  the group \(Z_G(e)\).
\end{prop}
\begin{proof}[Proof of Jacobson-Morozov Theorem when \(\g = \sl_n(\C)\)]
  \cite{cg}*{p 184} By the Jordan canonical form, any nilpotent
  element is conjugate to 
  a direct sum of Jordan blocks. Thus, we need only show the theorem
  is true when \(e\) is a single \(m\) by \(m\) Jordan block. Since
  \(e\) is nilpotent, all of its eigenvalues are \(0\), so \[
    e = \left(
      \begin{tikzcd}
        0 \ar[rrdd, dash]&1 \ar[rd, dash]& \\
        & & 1 \\
        & & 0
      \end{tikzcd}
\right)
\]
Then, we take
\begin{align*}
  h = \left(
    \begin{array}{cccc}
      m-1&0&\cdots&0 \\
      0&m-3&\cdots&0 \\
      \vdots &&\ddots& \\
      0&\cdots& &-m+1
    \end{array}
\right) \\  f = \left(
  \begin{array}{cccccc}
    0&0&&&& \\
    m-1&0&0&&&\\
    0&2(m-2)&0&&&\\
     &&\ddots&&&\\
     &&&-2(m-2)&0&0 \\
    &&&&-(m-1)&0
  \end{array}
\right)
\end{align*}
From this, it is straightforward to check that the set \((e,f,h)\)
actually forms an \(\sl_2\)-triple.
\end{proof}
\begin{lem}
 \cite{cg}*{3.2.16} \(e \in \g\) is nilpotent if and only if \[
    (e,x) = \tr(\ad_e \cdot \ad_x) = 0 \text{ for all } x \in Z_\g(e)
  \]
\end{lem}
\begin{proof}
  \((\implies)\). Assume \(e\) is nilpotent and \(x \in Z_\g(e)\). Since \(ad_e\) and
  \(\ad_x\) commute \todo{why?}, for large enough \(k\), \[
    (\ad_e \cdot \ad_x)^k = \ad^k_e \cdot \ad^k_x = 0
  \]
  by the nilpotency of \(x\). Thus, \(\ad_e \cdot \ad_x\) is also
  nilpotent and thus has trace \(0\). \\

  \((\impliedby)\). Now assume \((e,x) = 0\) for all \(x \in
  Z_\g(e)\). We note that \(\ad_e\) is 
  skew-symmetric with respect to the Killing form and so
  \(\im(\ad_e) = \ker (\ad_e)^\perp\), where \(\perp\) stands for
  the annihilator with respect to \((,)\). Thus, we have
  \begin{lem}
    \(e \in
  \im(\ad_e)\) is equivalent to \((e,Z_\g(e)) = 0\) is equivalent to
  there exists an 
  element \(h \in \g\) such that \([h,e]=e\).
  \end{lem}
  However, we can also show there exists a \emph{semisimple} element
  \(s \in \g\) such that \([s,e] = e\). To do this, take \(h\) as
  above and write its Jordan decomposition \(h = s + n\) where \(s\)
  semisimple and \(n\) nilpotent. Since \(e\) is an eigenvector for
  \(\ad_h\), then \(e\) is an eigenvector for \(\ad_s\) and
  \(\ad_n\). However, \(\ad_n\) is nilpotent and thus only has
  eigenvalues \(0\), so \(\ad_n e = 0\). Therefore, \(\ad_s e = \ad_h
  e = e\) and so we have found semisimple \(s \in \g\) such that
  \([s,e] = e\). \\

  So, take \(h \in \g\) to be semisimple such that \([h,e] =
  e\). Then, we can decompose \(\g\) into \(\ad_h\)-eigenspaces: \[
    \g = \bigoplus_{\alpha \in \C} \g_\alpha, \ \g_\alpha := \{x \in
    \g \st \ad_h(x) = \alpha \cdot x\}
  \]
  where \(h \in \g_0\) and \(e \in \g_1\). We also have relation
  \(\ad_h \circ \ad_e = \ad_e \circ (1+\ad_h)\) which tells us
  \(\ad_e\) takes \(\g_\alpha\) to \(\g_{\alpha+1}\). Thus,
  \(\ad^k_e\) takes \(\g_\alpha\) to \(\g_{\alpha+k}\) and since there
  are only finitely many non-zero spaces \(\g_\alpha\), we get \(\ad^k
  e = 0\) for sufficiently large \(k\). Thus, \(e\) is nilpotent.
\end{proof}
\begin{lem}
  If \(e \in \g\) is nilpotent, then there exists a semisimple \(h \in
  \g\) such that \([h,e] = e\).
\end{lem}
\begin{proof}
  If \(e \in \g\) is nilpotent, then by the lemma above, \((e,
  Z_\g(e)) = 0\). Thus, following the ``only if'' part of the proof
  above, we can find such an \(h\).
\end{proof}
\begin{proof}[Proof of Full Theorem]
  \cite{cg}*{pp191--2} Fix a nilpotent \(e \in \g\). Note that if
  \(\dim \g = 3\), then we 
  are done since \(\g = \sl_2\).
  \begin{enumerate}[label=(\arabic*)]
  \item Decomposing a non-nilpotent \(x \in Z_\g(e)\) (assuming such
    an \(x\) exists) into its Jordan
    decomposition \(x = s+n\) and using \([x,e] = 0 \implies
    [s,e]=0\) (see \cite{humph}*{p } \todo{Fill in page number}), we argue by induction on \(\dim
    \g\) to reduce to the case where 
    the subalgebra \(Z_\g(e)\) consists of nilpotent elements
    only. Since \([s,e] = 0\), we have \(s \in Z_\g(e)\), so
    \(Z_\g(e)\) contains a non-zero semisimple element. Furthermore,
    the centralizer of such an element is a \emph{proper} reductive
    Lie subalgebra of \(\mathfrak{t} \subset \g\). Since \(e\) commutes
    with the semisimple component of \(\mathfrak{t}\), we have \(e \in
    \mathfrak{t}\). Thus, this semisimple component of
    \(\mathfrak{t}\) is a Lie algebra of dimension less than \(\dim
    \g\) containing \(e\), so we are done by the inductive
    hypothesis. So, we must deal with the case of \(Z_\g(e)\)
    consisting only of nilpotent elements.
  \item Show there is a semisimple \(h \in \g\) such that \([h,e]=2e\). By the
    lemma above, for any nilpotent element \(e \in \g\), we have
    a semisimple \(h \in \g\) such that \([h,e] = e \implies [2h,e] = 2e\).
  \item Fix an \(h \in \g\) such that \([h,e]=2e\) as above and introduce
    weight space decomposition \[
      \g = \bigoplus_{\alpha \in \C} \g_\alpha, \ \g_\alpha := \{x \in
      \g \st \ad_h(x) = \alpha \cdot x\}.
    \]
    where \(h \in \g_0, e \in \g_2\) and \(\ad_h \circ \ad_e = \ad_e
    \circ (2+ \ad_h) \implies \ad_e\) takes \(\g_\alpha\) to
    \(\g_{\alpha+2}\). Thus, if we find \(f \in \g_{-2}\) such that
    \(\ad_e(f) = h\), we are done. Since \(\ad_e\) shifts the
    gradation by \(2\), this amounts to showing that \(h \in
    \im(\ad_e)\). However, from the
    lemma in the proof above, we see this holds if and only if
    \((h,Z_\g(e)) = 0\). To prove this, we use the Jacobi identity and
    the fact that \([h,e]=2e\) to get that, for \(x \in Z_\g(e)\),
    \begin{align*}
      [e,[h,x]] & = -[h,[x,e]]-[x,[e,h]] \\
                & = -[h,0] + [x,2e] \\
                & = 0
    \end{align*}
    and so \([h,Z_\g(e)] \subset Z_\g(e)\). So, we have that \(\C
    \cdot h + Z_\g(e)\) is a Lie subalgebra of \(\g\). Now, by the
    first part of the proof, we can assume \(x\) is nilpotent and
    using Engel's theorem, this tells us that \(Z_\g(e)\) is
    nilpotent and thus \(\C \cdot h + Z_\g(e)\) is a solvable Lie
    algebra. Thus, by Lie's theorem, for all \(x \in \C \cdot h +
    Z_\g(e)\), we can put \(\ad_x\) in the upper triangular form so
    that, for \(x \in Z_\g(e)\), \(\ad_x\) is strictly upper
    triangular. Therefore, for any \(x \in Z_\g(e)\), \(\ad_h \cdot
    \ad_x\) is strictly upper triangular and so \(\tr(\ad_h \cdot
    \ad_x) = 0 \implies (h,Z_\g(e)) = 0 \implies h \in \im(\ad_e)\)
    gives us the existence of an \(f \in \g_{-2}\) such that
    \(\ad_e(f) = h\). 
  \end{enumerate}
\end{proof}
\begin{rmk}
  In \cite{cm}, parts (1) and (2) of the proof are morally
  equivalent. However, for part (3) in \cite{cm}*{pp 38--39}, they proceed by
  contradiction, but follow a somewhat similar proof. In \cite{cg}*{p
    190}, they claim that the standard proof is quite elementary but
  involves several tricky lemmas. However, this proof made no use of
  anything terribly sophisticated in geometry, so I do not quite
  understand this statement.
\end{rmk}
\begin{cor}
  Given a nilpotent \(e \in \g\), there exists a rational homomorphism
  \(\gamma \from SL_2(\C) \to G\) such that its differential \(d\gamma
  \from \sl_2(\C) \to \g\) sends \(\left(
    \begin{array}{cc}
      0&1\\
      0&0
    \end{array}
\right)\) to \(e\).
\end{cor}
\begin{proof}
  \(SL_2(\C)\) is simply connected. Thus, any Lie algebra homomorphism
  can be extended to a (unique) Lie group homomorphism \footnote{This
    theorem is a highly non-trivial differential geometry proof, but
    proved in \cite{hall}*{section 3.6}}. 
\end{proof}
Recall that any simple finite dimensional \(\sl_2(\C)\)-module \(V\)
looks like \[
  \begin{tikzcd}
    \text{highest weight} & \cdot \ar[r, bend left, "f"] & \ar[l, bend
    left, "e"] \cdot \ar[r, bend left, "f"] & \ar[l, bend
    left, "e"] \cdots  \ar[r, bend left, "f"] & \ar[l, bend
    left, "e"] \cdot & \text{lowest weight}
  \end{tikzcd}
\]
where the dots correspond to \(h\)-eigenspaces, each of dimension
\(1\), and \(V = \ker f \oplus \im e\) where \(\ker f\) is the
rightmost vertex. This decomposition holds for any, not necessarily
simple, \(\sl_2\)-module.
\begin{cor}
  Let the \(\sl_2\)-triple \((e,f,h)\) act on a finite dimensional
  vector space \(V\). Assume that \(v \in V\) is such that \(f \cdot v
  = 0\) and \(h \cdot v = -m \cdot v\). Then, \(m\) is a non-negative
  integer and we have \(e^{m+1} \cdot v = 0\).
\end{cor}
\begin{proof}
  This follows since every \(\sl_2(\C)\)-module is symmetric about the
  \(0\)-eigenspace. 
\end{proof}
\begin{cor}
  Any nilpotent element of a semisimple Lie algebra \(\g\) is acting
  as a nilpotent operator on any finite dimensional \(\g\)-module.
\end{cor}
\begin{proof}
  By the Jacobson-Morosov Theorem, any nilpotent element \(e\) is part of an
  \(\sl_2\)-triple and thus, by the corollary above, for each \(v \in
  V\), there is an \(m
  \in \N\) such that \(e^{m+1} \cdot v
  = 0\). Thus, since \(V\) is finite-dimensional, \(e\) is a nilpotent
  operator on \(V\).
\end{proof}
\begin{cor}
  Fix a nilpotent \(e \in \g\) and corresponding \(\sl_2\)-triple
  \((e,f,h)\). Then,
  \begin{enumerate}
  \item All the eigenvalues of the operator \(\ad_h \from Z_\g(e) \to
    Z_\g(e)\) are non-negative integers.
  \item If all the eigenvalues of the operator \(\ad_h \from \g \to
    \g\) are even, then \(\dim Z_\g(e) = \dim Z_\g(h)\).
  \end{enumerate}
\end{cor}
\begin{proof}
  From the \(\sl_2\)-triple, we have an embedding \(\sl_2(\C) \into
  \g\). The adjoint action on \(\g\) of the image of the embedding
  makes \(\g\) an \(\sl_2(\C)\)-module. Then, by the decomposition of
  \(\sl_2(\C)\)-modules, we see that all the eigenvalues of \(\ad_h
  \from \g \to \g\) are integers and the weight space decomposition
  into \(\ad_h\)-eigenspaces yields a \(\Z\)-grading on the Lie
  algebra \(\g\): \[
    \g = \bigoplus_{i \in \Z} \g_i, \ [\g_i, \g_j] \subset \g_{i+j},
    \forall i, j \in \Z
  \]
  Furthermore, by the Jacobi identity, \(Z_\g(e)\) is
  \(\ad_h\)-stable (\([e,[h,x]] =
  -[h,[x,e]]-[x,[e,h]] = -[h,0]-[x,2e] = 0\)), so we have \[
    Z_\g(e) = \bigoplus_{i \in \Z} Z_{\g_i}(e), \ Z_{\g_i}(e) =
    Z_\g(e) \intersect \g_i
  \]
  Finally, since \(\ker(e) = Z_\g(e)\), then \(Z_\g(e) \intersect
  \im(f) = \{0\}\), so \(Z_\g\) contains only highest weight
  vectors. From the general theory, we know that if \(v_0\) is a
  highest weight vector, then \(h.v_0 = [\dim(\Span\{v_0, f.v_0,
  f^2.v_0, \ldots\})-1]v_0\), but such an eigenvalue must be a
  nonnegative integer. If all the eigenvalues are even, then repeated application
  of \(f\)
  to each element of \(Z_\g(e)\) will land it in \(\g_0 =
  Z_\g(h)\) since \(f \from \g_m \to \g_{m-2}\) for all \(m \in \Z\).
\end{proof}
\begin{thm}
  (\cite{cg} cor 3.2.13 and thm 3.7.13). Let \(e_1, \ldots, e_n\) be a
  set of root vectors in \(\n\) for \(\g = \h \oplus \n \oplus
  \n^-\), one for each simple root determined by \(\b\). Then, \(x =
  \sum e_i\) is a regular nilpotent element in \(\g\).
\end{thm}
\begin{rmk}
  This theorem can be proved using the fact that \(\n^{reg}\) is a
  single \(B\)-orbit consisting of regular nilpotent elements in
  \(\g\) since \(n \in \n^{reg}\). However, we will use the corollary
  above to prove it instead.
\end{rmk}
\begin{proof}[Proof of Theorem]
  Let \(\alpha_1, \ldots, \alpha_n\) be the set of simple roots
  determined by \(\b\) and let \(\n_{\alpha_1}, \ldots,
  \n_{\alpha_n}\) for the corresponding root spaces in \(\n\) (so
  that, in  particular, \(e_i \in \n_{\alpha_i}\)). Let
  \(\n_{-\alpha_1}, \ldots, \n_{-\alpha_n}\) denote the corresponding negative root
  spaces and \(e_{-1}, \ldots, e_{-n}\) the corresponding negative
  root vectors.

  From the general structure theory of semisimple Lie algebras, there
  is, for each \(1 \leq i \leq n\), a unique multiple \(h_i\) of
  \([e_i, e_{-i}]\) such that \(\alpha_i(h_i) = 2\). Furthermore, the
  \(h_i\)'s form a basis of \(\h\). Now, fix \(h \in \h\) so that
  \(\alpha_i(h) = 2\) for all \(i\) and define complex numbers
  \(\mu_1, \ldots, \mu_n\) by the equation \(h = \sum \mu_i h_i\). Let
  \(y := \sum \mu_i e_{-i}\). Now, for any simple roots \(\alpha,
  \beta\) where \(\alpha \neq \beta\), \(\alpha - \beta\) is not a
  root. Thus, \([e_i, e_{-j}] = 0\) whenever \(i \neq j\) because
  \todo{why?}. Now, we calculate \todo{Where did this come from?}
  \begin{align*}
    [h,x] & = [h,\sum e_i] = \sum [h,e_i] = \sum_i \alpha_i(h) e_i = 2x \\
    [h,y] & = \sum_i \mu_i(-\alpha_i)(h)e_{-i} = -2y \\
    [x,y] & = \sum_{i,j} \mu_j [e_i, e_{-j}] = \sum_i \mu_i [e_i,
            e_{-i}] = \sum_i \mu_i h_i = h
  \end{align*}
  Thus, \((x,y,h)\) is an \(\sl_2\)-triple and all the eigenvalues of
  \(h\) on \(\g\) are even since \(\alpha_i(h) = 2\). By the corollary
  above, we obtain \(\dim Z_\g(x) = \dim Z_\g(h) = \dim \h\) and
  therefore \(x\) is regular.
\end{proof}
\subsection{Consequences using Transversal Slices}
\todo{This whole section is completely  confusing and I do not
  understand anything except Kostant's theorem.}
\begin{defn}
  (\cite{cg} 3.2.19) A locally closed (in the ordinary Hausdorff topology) complex
  analytic subset \(S \subset X\) containing the point \(y\) will be
  called a \de{transverse slice} to \(Y\) at \(y\) if there is an open
  neighborhood of \(y\) (in the ordinary Hausdorff topology), \(U
  \subset X\), and an analytic isomorphism \(f \from (Y \intersect U)
  \times S \isomto U\) such that \(f\) restricts to the tautological
  maps of the factors \[
    f \from \{y\} \times S \isomto S \text{ and } (Y \intersect U)
    \times \{y\} \isomto Y \intersect U.
  \]
\end{defn}
\begin{lem}
  (\cite{cg} 3.2.20) Let \(G\) be an algebraic group, \(V\) a smooth
  algebraic \(G\)-variety, \(X\) a \(G\)-stable algebraic
  subvariety in \(V\), and \(S_V\) a locally-closed complex-analytic
  submanifold which is transverse to \(\O\) at \(y \in \O\). Then the
  intersection with \(X\) of a small enough open neighborhood of \(y\)
  in \(S_V\) is a transverse slice to \(\O\) in \(X\). 
\end{lem}
\begin{proof}
  See \cite{cg}.
\end{proof}
\begin{prop}
  (\cite{cg} 3.7.15)Fix a nilpotent \(e \in \g\) and corresponding \(\sl_2\)-triple
  \((e,f,h)\). Let \(\O\) be the \(G\)-conjugacy class of \(e\). Then,
  the affice space \(e + Z_\g(f)\) is transverse to the orbit \(\O\)
  in \(\g\). Moreover, we have \(\O \intersect (e + Z_\g(f)) = e\).
\end{prop}
\begin{rmk}
  The affine space \(e + Z_\g(f)\), or its intersection with \(\cN\),
  is often referred to as the \de{standard slice} to the orbit \(\O\)
  at the point \(e\). This is due to the fact that there exists a
  small enough neighborhood \(U\) containing \(e\) such that \(\cN
  \intersect (e + Z_\g(f)) \intersect U\) is a transverse slice to
  \(\O\) in \(\cN\).
\end{rmk}
\begin{cor}
  (\cite{cg} 3.7.19) The fiber \(\mu^{-1}(e) \subset \tilde{\cN}\) is a homotopy retract
  of the variety \(\tilde{S} = \mu^{-1}(e + Z_\g(f)) \subset \tilde{N}\).
\end{cor}
\begin{lem}\label{transversal-lem}
  (\cite{cg} 3.7.21) Let \(U \subset Z_G(e)\) be a unipotent normal subgroup
  corresponding to the Lie algebra \(\u = \bigoplus_{i > 0} Z_{\g_i}(e)\).
  \begin{enumerate}
  \item We have \(\u = \ker(e) \intersect \im(e)\).
  \item The affine space \(h + \u\) is stable under the adjoint
    \(U\)-action; moreover, \(h+\u = U \cdot h\) is a single \(U\)-orbit.
  \end{enumerate}
\end{lem}
\begin{proof}[Proof of Kostant's Theorem]
  Let \((e,f,h)\) and \((e,f',h')\) be two \(\sl_2\)-triples. If
  \(h=h'\), then \[
    [f',e] = h' = h = [f,e] \implies [f'-f,e] = 0 \text{ and } f'-f
    \in Z_\g(e)
  \]
  However, \(f,f' \in \g_{-2}\) in the grading \(\g = \bigoplus_{i \in
 \Z} \g_i\) into \(\ad_h\)-eigenspaces. Thus, \(f'-f \in
  Z_{\g_{-2}}(e) = 0\) and so \(f' = f\). \\

  Now, for any two \(\sl_2\)-triples \((e,f,h)\) and \((e,f',h')\), we
  get \[
    [h',e] = [h,e] = 2e \implies [h'-h,e] = 0 \implies h'-h \in
    Z_\g(e) = \ker(e)
  \]
  Furthermore, we have \[
    h'-h = [e,f'-f] \in \im(e) \implies h'-h \in \ker(e) \intersect \im(e)
  \]
  So, by Lemma \ref{transversal-lem}(a), we get \(h' \in h + \u\)
  and thus, by Lemma \ref{transversal-lem}(b), there is a \(u \in
  U\) such that \(h' = uhu^{-1}\). Therefore, \[
    e = u^{-1}eu, \ h = u^{-1}h'u, \ f'' := u^{-1}f' u
  \]
  and so \((e,h,f'')\) is an \(\sl_2\)-triple for \(e\). So, by above,
  \(f'' = f\) and we are done.
\end{proof}
\subsection{Correspondance between nilpotent orbits and
  \(G\)-conjugacy classes of \(\sl_2\)-triples}
This exposition is borrowed from chapter 3 of \cite{cm} and modified
given what we already have proven.
\begin{prop}
  For semisimple Lie algebra \(\g\), there is a bijection \[
    \Hom(\sl_2,\g) \setminus \{0\} \correspondsto \{\text{standard
      triples in }\g\}
  \]
\end{prop}
\begin{proof}
  Given a nonzero homomorphism \(\phi \from \sl_2 \to \g\), we get
  standard triple \(\{\phi(e), \phi(f), \phi(h)\}\). Conversely, an
  \(\sl_2\)-triple gives an injection of \(\sl_2 \into \g\).
\end{proof}
\begin{prop}
  There is a map defined by
  \begin{align*}
    \Omega \from \{G_{ad}\text{-conjugacy classes of standard triples
    in }\g\} & \to \{\text{nonzero nilpotent orbits in }\g\}\\
    \{e,f,h\} & \mapsto \O_e
  \end{align*}
\end{prop}
\begin{proof}
  This map is induced by the fact that \(\Hom(\sl_2,\g) \setminus
  \{0\}\) and standard triples in \(\g\) are invariant under the
  action of \(G_{ad}\).
\end{proof}
\begin{prop}
  The map \(\Omega\) is surjective.
\end{prop}
\begin{proof}
  By the Jacobson-Morozov theorem, every nilpotent \(e \in \g\)
  belongs to some \(\sl_2\)-triple. Thus, there is a conjugacy class
  of triples \(\{e,f,h\}\) mapping to \(\O_e\).
\end{proof}
\begin{prop}[Kostant's Theorem, rephrased]
  Let \(\g\) be a complex semisimple Lie algebra. Any two standard
  triples \((e,f,h)\) amd \((e,f',h')\) with the same nilpositive
  element are conjugate by an element of \(G_{ad}\), the adjoint
  group. Thus, the map \[
    \Omega \from \{G_{ad}\text{-conjugacy classes of standard triples
      in }\g\} \to \{\text{nonzero nilpotent orbits in }\g\}
  \]
  is injective.
\end{prop}
\begin{rmk}
  
\end{rmk}
\begin{thm}
  The map \(\Omega\) is a one-to-one correspondance between the set
  of \(G_{ad}\)-conjugacy classes of standard triples in \(\g\)
  and the set of nonzero nilpotent orbits in \(\g\).
\end{thm}
\begin{example}
  Consider \(\g = \sl_n(\C)\). Then, \(G_{ad} = PSL_n\) and we can
  pick orbit representatives \[
    e_{[d_1, \ldots, d_k]} = \left(
      \begin{array}{cccc}
        J_{d_1}(0)&&& \\
                  &J_{d_2}(0)&& \\
                  &&\ddots& \\
        &&&J_{d_k}(0)
      \end{array}
\right)
\]
  where \([d_1, \ldots, d_k]\) is a partition of \(n\). As discussed
  in the \(\sl_n\) proof of the Jacobson-Morozov theorem, if we
  let \[
    \rho_r(h) = \left(
      \begin{array}{ccccc}
        r&&&&\\
         &r-2&&&\\
         &&\ddots&&\\
         &&&-r+2&\\
        &&&&-r
      \end{array}
\right) \ \ \rho_r(f) = \left(
  \begin{array}{ccccc}
    0&0&&&\\
    \mu_1&0&&&\\
     &\ddots&\ddots&& \\
     &&&0&0\\
    &&&\mu_r&0
  \end{array}
\right)
\]
  where \(\mu_i = i(r+1-i)\) for \(1 \leq i \leq r\), we can associate
  \(\sl_2\)-triple to \(e_{[d_1, \ldots, d_k]}\) given by \[
    h_{[d_1, \ldots, d_k]} = \rho_{d_1}(h) \oplus \cdots \oplus
    \rho_{d_k}(h), \ f_{[d_1, \ldots, d_k]} = \rho_{d_1}(f) \oplus
    \cdots \oplus \rho_{d_k}(f)
  \]
  Conversely, given an \(\sl_2\)-triple in \(\sl_n\), \(\{e,f,h\}\),
  the corresponding orbit is simply \(\O_e\).
\end{example}

\end{document}
%%% Local Variables:
%%% TeX-master: "springer-theory-notes.tex"
%%% End: