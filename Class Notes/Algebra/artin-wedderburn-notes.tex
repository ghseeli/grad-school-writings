\documentclass[11pt,leqno,oneside]{amsbook}
\usepackage{tikz}
\usetikzlibrary{cd}
\usepackage{bbm}
\usepackage{todonotes}

\usepackage{../notes}
\usepackage{../../ReAdTeX/readtex-core}
\usepackage{../../ReAdTeX/readtex-abstract-algebra}

\numberwithin{thm}{section}

\title[Artin-Wedderburn Theory]{Artin-Wedderburn Theory \\ Notes
  inspired by a class taught by Brian Parshall in Fall 2017}
\author{George H. Seelinger}
\date{Fall 2017}
\begin{document}
\maketitle
\section{Krull-Schmidt}
We start with an underated lemma.
\begin{thm}[Fitting's Lemma]
  Let \(0 \neq M\) be a left \(R\)-module with a composition series
  and \(f \from M \to M\) a module homomorphism. Since \(M\) has a
  composition series, there exists an
  \(n\) such that \[
    \im f \supset \im f^2 \supset \cdots \supset \im f^n = \im f^{n+1}
    = \cdots
  \]
  and \[
    \ker f \subset \ker f^2 \subset \cdots \subset \ker f^n = \ker
    f^{n+1} = \cdots 
  \]
  Define \(f^\infty(M) := \im f^n\) and \(f^{-\infty}(M) := \ker
  f^n\). Then, \[
    M \isom f^\infty(M) \oplus f^{-\infty}(M)
  \]
\end{thm}
\begin{proof}
  Let \(x \in f^\infty(M) \intersect f^{-\infty}(M)\). Then, \(x =
  f^n(y) = f^{2n}(z)\) and \(f^n(x) = 0\). Then, \(0 = f^n(x) =
  f^{2n}(y)\). Thus, \(y \in \ker f^{2n} = \ker f^n\), so \(x = f^n(y)
  = 0\). \\

  Now, let \(x \in M\). Then, \(x = [x-f^n(y)]+f^n(y)\) where \(f^n(y)
  \in f^\infty(M)\) and \(y\) such that \(f^{2n}(y) =
  f^n(x)\). Then, \[
    f^n(x-f^n(y)) = f^n(x) - f^{2n}(y) = f^n(x) - f^n(x) = 0
  \]
  Thus, \(x - f^n(y) \in f^{-\infty}(M)\) and \(f^n(y) \in
  f^\infty(M)\). 
\end{proof}
Now, we seek to prove the Krull-Schmidt theorem.
\begin{thm}[Krull-Schmidt Theorem]
  Let \(0 \neq M\) be an \(R\)-module which is both Artinian and
  Noetherian. Suppose
  \begin{align*}
    M & = M_1 \oplus \cdots \oplus M_h \\
    & = N_1 \oplus \cdots \oplus N_k
  \end{align*}
  where \(M_i, N_j\) are indecomposable. Then, \(h=k\) and, up to
  rearranging terms, \(M_i \isom N_i\) for \(i = 1, \ldots, h\). 
\end{thm}
To prove this theorem, we need the following lemmas.
\begin{lem}
  Let \(M,N\) be \(R\)-modules with \(N\) decomposable and
  homomorphisms \(\nu \from M \to N\) and \(\mu \from N \to M\). If
  \(\mu \nu \from M \to M\) is an automorphism, then \(\mu\) and
  \(\nu\) are isomorphisms.
\end{lem}
\begin{proof}
  Let \(\tau = \mu \nu\) be an automorphism. Now, replace \(\mu\) with
  \(\tau^{-1} \mu\). Then, we have split short exact sequence \[
    \begin{tikzcd}
      0 \rar & M \ar[r,swap,"\nu"] & N \rar \ar[l,bend right = 30, swap,"\mu"] & N/M \rar & 0
    \end{tikzcd}
  \]
  and so \(N \isom M \oplus N/M\). However, \(N\) is indecomposable,
  so \(N = M\).
\end{proof}
\begin{lem}
  Let \(0 \neq M\) be indecomposable, Artinian, and Noetherian. Let \(\tau_1, \ldots, \tau_r \in \End_R M\) satisfy \(\tau_1 +
  \cdots + \tau_r \in \Aut_R (M)\). Then, at least one \(\tau_i\) is
  an automorphism of \(M\)
\end{lem}
\begin{proof}
  Since \(\tau = \tau_1 + (\tau_2 + \cdots + \tau_r)\), it suffices to
  prove for the sum of 2 endomorphisms. Let \(\psi = \tau_1 + \tau_2\)
  be an automorphism. Then, \(\phi_1 = \psi^{-1} \tau_1\) and \(\phi_2
  = \psi^{-1} \tau_2\). We then have \(\phi_1 + \phi_2 = 1_M\) and so,
  since \(\phi_1 = 1_M - \phi_2\), we get \(\phi_1 \phi_2 = \phi_2
  \phi_1\). Now, we can apply the binomial theorem to get \[
    0 \neq 1_M^m = (\phi_1 + \phi_2)^m = \sum_{k=1}^m \binom{m}{k} \phi_1^k \phi_2^{m-k}
  \]
  So, at least one of the \(\phi_i\)'s cannot be nilpotent. Without
  loss of generality, assume \(\phi_1\) is not nilpotent. Since
  \(\phi_1\) is not nilpotent and \(M\) has a composition series, we
  can apply Fitting's Lemma to get \[
    M \isom \phi_1^\infty(M) \oplus \phi_1^{-\infty}(M)
  \]
  However, \(M\) is indecomposable and \(\phi_1^\infty(M) \neq 0\), so
  \(M = \phi_1^\infty(M)\) and \(\ker \phi_1 \subset
  \phi_1^{-\infty}(M) = 0\), so \(\phi_1\) is an
  automorphism. Therefore, \(\tau_1\) is an automorphism.
\end{proof}
We are now ready to prove the Krull-Schmidt theorem.
\begin{proof}[Proof of Krull-Schmidt]
  Let \(0 \neq M\) be an Artinian and Noetherian \(R\)-module. Suppose
  \begin{align*}
    M & = M_1 \oplus \cdots \oplus M_h \\
    & = N_1 \oplus \cdots \oplus N_k
  \end{align*}
  where the \(M_i,N_j\) are indecomposable. We will proceed by
  induction on \(h\). If \(h=1\), then \(M\) is already indecomposable
  and we are done. Now, for the inductive step, let \(\mu \from M \to
  M_i\) be the projection onto the \(i\)th summand and, similarly, let
  \(\nu_j \from M \to N_j\) be a projection. Then, we have \(\mu_i
  \mu_j = \delta_{ij} \mu_i\) and \(\nu_i \nu_j = \delta_{ij}
  \nu_j\). Now, we have the following identity \[
    \mu_1 = \mu_1 \circ 1_M = \mu_1(\nu_1 + \cdots + \nu_k)
  \]
  Now, let \(\ov{\nu}_j = \nu_j |_{M_i} \from M_i \to M\). If we
  restrict the above identity to \(M_1\), we get \[
    1_{M_1} = \mu_1 \ov{\nu}_1 + \cdots + \mu_1 \ov{\nu}_k \in \Aut M_1
  \]
  Thus, we can use the lemma above to assume that \(\mu_1 \ov{nu}_1
  \in \Aut M_1\). Furthermore, by the other lemma, since \(\mu_1 \ov{\nu}_1\) is
  an automorphism, we get that \(\ov{nu}_1\) is an isomorphism, as is
  \(\mu_1 \from N_1 \to M_1\). Now,
  consider \[
    N_1 + M_2 + \cdots + M_n \submodule M
  \]
  We wish to show this sum is actually direct. Assume there is \(n_1
  \in N_1\) and \(m_2 \in M_2, \ldots, m_h \in M_h\) such that \[
    n_1 + m_2 + \cdots + m_h = 0.
  \]
  We can just apply \(\mu_1\) to this sum above to get \[
    0 = \mu_1(n_1 + m_2 + \cdots + m_h) = \mu_1(n_1)
  \]
  However, \(\mu_1\) is an isomorphism between \(N_1\) and \(M_1\), so
  \(\mu_1(n_1) = 0 \implies n_1 = 0\). So, furthermore, it must be
  that \(m_2 = \cdots = m_h = 0\) by the direct sum construction,
  so \[
    M' := N_1 + M_2 + \cdots + M_h = N_1 \oplus M_2 \oplus \cdots
    \oplus M_h \submodule M
  \]
  Thus, we wish to finally show that \(M' = M\). Let \[
    \rho \from M_1 \oplus \cdots \oplus M_h \to M'
  \]
  be given by \(\rho = \ov{\nu}_1 + \mu_2 + \cdots + \mu_h\). Note
  that \(\rho\) is an isomorphism since \(\ov{\nu}_1\) is an
  isomorphism from \(M_1 \to M\). Since
  \(M\) is Artinian, there is some \(a \gg 0\) such that \[
    \rho^{a+1}(M) = \rho^a(M)
  \]
  Hence, given \(m \in M\), there is some \(m' \in M\) such that
  \begin{align*}
    \rho^{a+1}(m') = \rho^a(m)
    & \implies \rho^a(m-\rho(m')) = 0 \\
    & \implies m-\rho(m') \in \ker \rho^a = \{0\}
  \end{align*}
  Thus, \(M' = M\) and \(\rho\) is an automorphism of
  \(M\). Furthermore, \(\rho(M_i) = M_i\) for \(i = 2,\ldots,h\),
  \(\rho(M_1) = N_1\), and \[
    N_1 \oplus M_2 \oplus \cdots \oplus M_h = \rho(M) = M = M_1 \oplus
    \cdots \oplus M_h
  \]
  Hence,
  \begin{align*}
    M_2 \oplus \cdots \oplus M_h
    & \isom M/M_1 \\
    & \isom \rho(M)/\rho(M_1) \\
    & \isom \rho(M)/N_1 \\
    & \isom (N_1 \oplus \cdots \oplus N_k) / N_1 \\
    & \isom N_2 \oplus \cdots \oplus N_k
  \end{align*}
  Thus, by induction, \(h-1 = k-1 \implies h = k\) and \(M_i \isom
  N_i\) for \(i=2,\ldots,h\), and we already know \(M_1 \isom N_1\) by
  above.
\end{proof}
\begin{cor}
  Let \(N\) be a direct summand of \(M\) as in the Krull-Schmidt
  theorem. Then, there exist a subset of \(\{M_i\}\) indecomposable submodules 
  \(M_{i_1}, \cdots , M_{i_t}\) such that \[
    N \isom M_{i_1} \oplus \cdots \oplus M_{i_t}
  \]
\end{cor}
\section{Nakayama's Lemma: A Hard to Remember Lemma}
While one statement of Nakayam's Lemma may be hard to remember, there
are actually many special cases of this lemma and so it has a
reputation of being hard to remember. To understand and prove
Nakayama's Lemma, we must first lay the groundwork of Jacobson
radicals.
\begin{defn}
  The \de{Jacobson radical} (often just called the radical) \(R\) of a
  ring \(A\) is \[
    R = \{x \in A \st xM = 0, \forall \text{ simple }A\text{-modules }M\}
  \]
\end{defn}
\begin{defn}
  A ring \(A\) is called \de{semisimple} when its radical \(R = 0\). 
\end{defn}
\begin{thm}\label{radical-is-2-sided-ideal}
  The radical \(R\) of \(A\) is a two-sided ideal of \(A\) and \(A/R\)
  is semisimple.
\end{thm}
\begin{proof}
  If \(M\) is a simple \(A\)-module, then \(M\) is also an
  \(A/R\)-module. Conversely, any \(A/R\)-modules are naturally
  \(A\)-modules. Thus, if there were an \(x + R \in A/R\) such that
  \((x+R)M = xM = 
  0\) for all simple \(A/R\)-modules \(M\), then \(x\) would also
  annihilate all simple \(A\)-modules and thus \(x \in R\), so \(x+R =
  0 + R\) and thus the radical of \(A/R\) is trivial.
\end{proof}
\begin{thm}\label{radical-is-intersection-of-maxl-ideals}
  Let \(R\) be the radical of a ring \(A\). Then, \[
    R = \Intersect_{M \text{ simple }} \Ann_A (M) = \Intersect_{\substack{I
      \ideal A \\ \text{Maximal left ideal}}} I
  \]
  That is, \(R\) is the intersection of all maximal left ideals of \(A\).
\end{thm}
\begin{proof}
  Let \(x \in R\) and let \(I\) be a maximal left ideal. We want to
  show that \(x \in I\). Since \(I\) is maximal, \(A/I\) is a simple
  left \(A\)-module. thus, \(x \in \Ann(A/I)\). Thus,
  \begin{align*}
    0 = x(1+I) & \implies x \in I \\
               & \implies x \in \Intersect I \\
               & \implies R \subset \Intersect I
  \end{align*}
  Conversely, let \(x \in \Intersect I\). Let \(M\) be a simple module
  \(0 \neq m \in M\). We show \(xm = 0\). We have
  \begin{align*}
    A & \overset{\theta}{\onto} M \\
    a & \mapsto am
  \end{align*}
  Then, \(M \isom A/I\) where \(I = \Ann_A(m)\). Thus, \(xm = 0\).
\end{proof}
\begin{thm}\label{one-minus-radical-is-unit}
  Given a ring \(A\), \[
    \rad A = \{x \in A \st 1-axb \in A^\times, \forall a,b \in A\}
  \]
  where \(A^\times\) is the group of units of \(A\).
\end{thm}
\begin{proof}
  Let \(x \in \rad A\) with \(a,b \in A\). Then, \(\rad A \ideal A\)
  by \ref{radical-is-2-sided-ideal}, so \(axb \in \rad A\). Now, write
  \(y := axb\). We wish to show \(1-y\) is a unit. \\

  If \(A(1-y) \neq A\), let \(I\) be a maximal left ideal containing
  \(A(1-y)\). Then, by \ref{radical-is-intersection-of-maxl-ideals},
  \(\rad A \subset I\), so \(y \in \rad A \subset I\) and \(1-y \in
  A\). This would give that \(1 = y + (1-y) \in I \implies I = A\),
  which is a contradiction of the maximality of \(I\). So, \(A(1-y) =
  A\) for all \(y \in \rad A\). \\

  Now, since \(A(1-y) = A\), there is a \(t\) such that \(t(1-y) =
  1\). So, we wish to show \((1-y)t = 1\). Rearranging our expression
  \(t(1-y) = 1\), we get
  \begin{align*}
    1-t = -ty \in \rad A
    & \implies A(1-(1-t)) = A
    & \text{ by above since }A(1-y) = A, \forall y \in \rad A \\
    & \implies At = A \\
    & \implies \exists u \in A \text{ such that } ut=1 \\
    & \implies u = ut(1-y) = 1-y
    & \text{since }t(1-y)=1 \text{ and }ut=1 \\
    & \implies 1-y \in A^\times
  \end{align*}
  Thus, we have shown one containment. \\

  Now, let \(x \in A\) be such that \(1-axb \in A^\times\) for all
  \(a,b \in A\). We want to prove \(x \in \rad A\). We show \(xM = 0\)
  if \(M\) is a simple \(A\)-module. Let \(0 \neq m \in M\). Assume
  \(xm \neq 0\). Then, \(M = Axm\) since \(M\) is simple so \((xm) =
  M\). Thus, \(m = axm\) for some \(a \in A\) and so \((1-ax)m =
  0\). However, \(1-ax \in A^\times\), so \(m = 0\), which is a
  contradiction. Thus, \(xM = 0\) and so \(x \in \rad A\).
\end{proof}
\begin{thm}[Nakayama's Lemma]
  \cite{jacobson}*{p 415} Let \(A\) be a ring and let \(M\) be a finitely-generated
  \(A\)-module. If \((\rad A)M = M\), then \(M = 0\).
\end{thm}
\begin{proof}
  Let \(m_1, m_2, \ldots, m_k\) be a minimum generating set of
  \(M\). Given \(m \in M\), there exists \(a_1, \ldots, a_k \in A\)
  such that \[
    m = a_1 m_1 + \cdots + a_k m_k
  \]
  Take \(m = m_1\). Then, \[
    (1-a_1)m_1 = a_2 m_2 + \cdots + a_k m_k
  \]
  If \((\rad A) M = M\), we can assume each \(a_i \in \rad A\), we can
  assume each \(a_i \in \rad A\) since \(\rad A\) is a 2-sided ideal
  by theorem \ref{radical-is-2-sided-ideal}. Thus, by
  \ref{one-minus-radical-is-unit}, \(1-a_1\) is a unit, so \[
    m_1 = (1-a_1)^{-1} a_2 m_2 + \cdots + (1-a_2)^{-1} a_k m_k \in
    \langle m_2, \ldots, m_k \rangle
  \]
  However, this contradicts the minimality of the the generating set.
\end{proof}
\begin{cor}
  Let \(M\) be a finitely generated left \(A\)-module and let \(N
  \submodule M\) be a submodule such that \[
    N+(\rad A)M = M
  \]
  Then, \(N = M\)
\end{cor}
\begin{proof}
  Given the above, \(\rad A(M/N) = M/N\) simply by quotienting both
  sides of the equality by \(N\). Thus, by Nakayama's lemma, \(M/N =
  0\) and so \(M=N\).
\end{proof}
\begin{cor}
  Let \(J\) be a maximal ideal in a ring \(A\). Then, \(\rad A \subset
  J\).
\end{cor}
\begin{proof}
  Assume not. Then \(\rad A + J = A \implies J = A\) by the corollary
  above, which is a contradiction to \(J\) being a maximal ideal.
\end{proof}
\section{Completely Reducible Modules}
Understanding all the modules of a ring, in general, is an incredibly
difficult problem. However, usually the first step in such a program
is understanding the simple (or irreducible) modules. Understanding
these modules would then give a complete understanding of completely
reducible modules.
\begin{defn}
  Let \(A\) be a ring. A left \(A\)-module \(M\) is called
  \de{completely reducible} if, given any submodule \(N \subset M\),
  there exists a submodule \(N'\) such that \[
    M = N \oplus N'
  \]
\end{defn}
However, there are equivalent notions of completely reducible.
\begin{thm}
  Assume \(M\) is a left \(A\)-module. Then, the following are
  equivalent.
  \begin{enumerate}
  \item \(M\) is completely reducible.
  \item \(M\) is a direct sum of irreducible submodules, that is \[
      M \isom \bigoplus_{i \in I} L_i
    \]
    where \(I\) is some indexing set and \(L_i\) is simple.
  \item \(M\) is a sum of irreducible submodules, that is \[
      M \isom \sum_{i \in I} L_i
    \]
    where \(I\) is some indexing set and \(L_i\) is simple.
  \end{enumerate}
\end{thm}
To prove this result, we use the following lemma.
\begin{lem}
  If an \(A\)-module \(M\) is completely reducible, so is any
  submodule \(N \submodule M\).
\end{lem}
\begin{proof}[Proof of Lemma]
  Consider submodule \(N\) of completely reducible a \(A\)-module
  \(M\). Then, by definition, there exists an \(N'\) such that \[
    M \isom N \oplus N'.
  \]
  Now, let \(S \submodule N\). Then, we also have \(S \submodule M\)
  and so there is a submodule \(T\) such that \[
    M \isom S \oplus T.
  \]
  Thus, we have
  \begin{align*}
    N & = (S \oplus T) \intersect N \\
      & = S \oplus (T \intersect N)
  \end{align*}
  since \(S \submodule N\). Since \(T \intersect N \submodule N\),
  then \(N\) is completely irreducible.
\end{proof}
\begin{lem}
  Let \(M\) be a completely reducible \(A\)-module. Then, every
  nontrivial submodule \(N\) of \(M\) contains an irreducible submodule.
\end{lem}
\begin{proof}[Proof of Lemma]
    Let \(N \submodule M\) and \(0 \neq n \in N\). Consider the
    collection \[
      S = \{N' \submodule N \st n \not \in N'\}
    \]
    We note that \(S\) is nonempty since \((0) \in S\). Thus, since
    \(S\) is nonempty, it must contain a maximal element, say
    \(N_0\). By the lemma above, we know that \(N\) is completely
    reducible and so \(N = N_0 \oplus N_1\) for some submodule \(N_1
    \submodule N\). \(N_1\) must be irreducible because, if not, then
    there would be a proper submodule \(N_2 \submodule N_1\) and \(N_1
    = N_2 \oplus N_3\) for some submodule \(N_3 \submodule
    N_1\). However, this would give us \[
      N = N_0 \oplus N_2 \oplus N_3 \implies \text{ either } n \not
      \in N_0 + N_2 \text{ or } n \not \in N_0 + N_3
    \]
    since \((N_0 + N_2) \intersect (N_0 + N_3) = N_0\). Such a result
    contradicts the maximality of \(N_0\) in \(S\), and so it must be
    that \(N_1\) is irreducible.
\end{proof}
\begin{proof}[Proof of Theorem]
  We first assume that \(M \neq (0)\), otherwise the theorem is
  immediately true.
  \begin{enumerate}
  \item[(\((a) \implies (b)\)).] Let \(\{M_i \st i \in I\}\) be the
    collection of all irreducible submodules of \(M\) and let \[
      T = \left\{J \subset I \ \st \ \sum_{j \in J} M_j \text{
          is direct} \right\}
    \]
    \(T\) is nonempty since it contains at least singleton sets are in
    \(T\) and any union
    of an ascending chain of elements in \(T\) is in \(T\). Thus, we
    can apply Zorn's lemma to get a maximal element of \(T\), say
    \(J_0\). Now, let \[
      M' := \bigoplus_{j \in J_0} M_j
    \]
    be properly contained in \(M\). Then, by complete reducibility of
    \(M\), we get \[
      M = M' \oplus M''
    \]
    for \((0) \neq M'' \submodule M\). However, 
    since \(M''\) must be completely reducible, by the lemma above,
    \(M''\) must contain an an irreducible submodule, say
    \(M_{i_0}\). Then, \[
      \left(\bigoplus_{j \in J_0} M_j \right) + M_{i_0} \text{ is direct}
    \]
    and thus we have violated the maximality of \(J_0\).
  \item[(\((b) \implies (c)\)).] This result is immediate by the
    definition of direct sum.
  \item[(\((c) \implies (a)\)).] Let \(N \submodule M\) and let \(N'\)
    be a maximal submodule with respect to the property that \(N
    \intersect N' = (0)\). Assume \(M \neq N \oplus N'\). Then, there
    is an \(m \in M, m \not \in N \oplus N'\). However, since \(M\) is
    a sum of irreducible submodules, \(m = m_1 + \cdots + m_k\) where
    each \(m_i\) belonds to an irreducible summand. Thus, at least one
    \(m_i \not \in N \oplus N'\) since \(m \not \in N \oplus N'\) and
    thus \(M_i \not \subset N \oplus N'\). However, \(M_i\) is
    irreducible and so \(M_i \intersect (N \oplus N') = (0)\). Thus,
    \(N' \propsubgroup N' + M_i\) and \((N'+M_i) \intersect N = (0)\),
    thus violating the maximality of \(N'\). Thus, \(M = N \oplus N'\).
  \end{enumerate}
\end{proof}
\section{Nilpotent and non-Nilpotent Ideals in Artinian Rings}
To get to the Artin-Wedderburn Theorem, we must have an understanding
of idempotents in Artinian rings, which are intimately (non)-related to
nilpotent ideals. We wish to culminate in a theorem that says any
non-nilpotent left ideal in a left-artinian ring must contain an
idempotent. First, we present a theorem similar to Schur's Lemma, but
for Noetherian rings.
\begin{thm}
  Let \(M\) be a (left) Noetherian \(A\)-module for ring \(A\). If \(f
  \in \End_A(M)\) is surjective, then \(f\) is an isomorphism
\end{thm}
\begin{proof}
  If \(f\) is surjective, then \(\im f = M\) and so \(f\) is not
  nilpotent. Thus, by Fitting's Lemma \[
    M = f^\infty(M) \oplus f^{-\infty}(M)
  \]
  we seek to show \(\ker f = 0\). For some integer \(n\), \(\ker f^n =
  \ker f^{n+1}\). Let \(x \in \ker f\). Then, since \(f\) is
  surjective, \[
    f^n(M) = f^{n+1}(M) = M
  \]
  Thus, \(x = f^n(y)\) for some \(y \in M\). We then see \[
    0 = f(x) = f^{n+1}(y) \implies y \in \ker f^{n+1} = \ker f^n
    \implies 0 = f^n(y) = x
  \]
  So, we have that \(\ker f = 0\) and thus \(f\) is injective.
\end{proof}
\begin{thm}
  Let \(A\) be a left Noetherian ring. If \(a,b \in A\) be such that \(ab =
  1\), then \(ba = 1\) and \(a,b \in A^\times\).
\end{thm}
\begin{proof}
  Let \(M = {}_A A\). Then, \(M\) is a (left) Noetherian
  \(A\)-module. Then, since \(ab = 1\), \[
    A = Aab \subset Ab \subset A
  \]
  and so \(Ab = A\). Thus, we have a map
  \begin{align*}
    f \from M & \onto M \\
    x & \mapsto xb
  \end{align*}
  and so, by the theorem above, \(f\) is an isomorphism. However,
  \begin{align*}
    f(1-ba) & = (1-ba)b \\
            & = b-bab\\
            & = b-b & \text{since }ab = 1\\
    & = 0
  \end{align*}
  Thus, \(1-ba = 0 \implies ba = 1 = ab\). 
\end{proof}
\begin{defn}
  An \de{idempotent} of a ring \(A\) is an element \(0 \neq e \in A\)
  such that \(e^2 = e\).
\end{defn}
One advantage of idempotents is that they allow us to ``project'' the
ring onto ``orthogonal'' compotents, that is, given a ring \(A\) with
idempotent \(e\), then, as a module over itself \[
  {}_A A \isom {}_A Ae \oplus {}_A A(1-e)
\]
This also tells us the following
\begin{rmk}
  \({}_A Ae\) is projective as an \(A\)-module since it is the direct
  summand of a free \(A\)-module.
\end{rmk}

\begin{defn}
  We call a left or right ideal, \(I\), \de{nilpotent} if there is an \(m
  \in \N\) such that \(I^m = \{0\}\).
\end{defn}
\begin{example}
  Consider \(R = \Z/p^n \Z\) where \(p\) is a prime number. Then,
  since \(R\) is a PID, every proper ideal is generated by some \(p^k
  + (p^n)\) and is nilpotent. The only idempotent of \(R\) is \(1 + (p^n)\).
\end{example}
\begin{thm}
  Let \(N\) be a nilpotent left ideal of a ring \(A\). Let \(x \in A\)
  be non-nilpotent such that \(x^2-x \in N\). Then, the left ideal
  \(Ax\) has an idempotent \(y\).
\end{thm}
The idea to proving this theorem is to take \(A\) and factor out
\(N\). Then, one can find an idempotent \(y\) such that, under the quotient
map \(q \from A \to A/N\), \(q(x) = q(y)\).
\begin{proof}
  Assume \(N^k = 0\) for some positive integer \(k\). Let \(m_1 :=
  x^2-x \in N\). If \(m_1 = 0\), then \(x^2-x = 0\) so \(x\) is an
  idempotent itself and we can take \(y = x\). \\

  Assume \(m_1 \neq 0\). Then, let \[
    x_1 := x + m_1 - 2xm_1 \in Ax \text{ since }m_1 \in Ax
  \]
  Note that \(x_1,x,m_1\) all commute. Then, note that \(x_1\) is not nilpotent
  as well and \(x^2 = x + m_1\). Consider
  \begin{align*}
    x_1^2 - x_1
    & = (x+m_1-2xm_1)^2 - (x+m_1-2xm_1) \\
    & = x^2 + xm_1 - 2x^2m_1 + m_1 x + m_1^2 -2xm_1^2 - 2x^2m_1 -
      2xm_1^2 + 4 x^2 m_1^2 - x - m_1 + 2x m_1 \\
    & = 4x^2 m_1^2 - 4x^2 m_1 - 4x m_1^2 + x^2 + 4xm_1 + m_1^2 - x - m_1 \\
    & = 4x^2 m_1^2 - 4x^2 m_1 - 4x m_1^2 + 4xm_1 + m_1^2 \\
    & = (4x^2-4x)m_1^2 + (-4x^2+4x)m_1 + m_1^2 \\
    & = (4x^2-4x-4m_1)m_1^2 + (-4x^2+4x+4m_1)m_1 + m_1^2 + 4m_1^3 - 4m_1^2 \\
    & = 4m_1^3 - 3m_1^2
  \end{align*}
  Then, take \[
    m_2 = 4m_1^3 - 3m_1^2 \ x_2 = x_1 + m_2 - 2x_1 m_2.
  \]
  and note that \(m_2\) contains \(m_1^2\) as a factor. Thus, we can
  successively construct non-nilpotent elements \(x_1, x_2, \ldots\)
  in \(Ax\) such that \(x_i^2 - x_i\) contains \(m_1^{2^i}\) as a
  factor and commutes with \(x\). Since \(m_1\) is nilpotent, then
  \(m_1^{2^i} = 0\) for sufficiently large \(i\) and so \(x_i^2 - x_i
  = 0\) for some sufficiently large \(i\). Therefore, for that
  sufficiently large \(i\), \(x_i\) is nilpotent.
  \end{proof}
\begin{rmk}
  Note that any nilpotent ideal cannot contain an idempotent
  element. However, the following theorem gives us a (useful!)
  converse to that fact.
\end{rmk}
\begin{thm}
  Let \(L\) be a non-nilpotent left ideal in a left-artinian ring
  \(A\). Then, \(L\) contains an idempotent \(e\).
\end{thm}
\begin{proof}
  We seek to use the theorem above to get such an idempotent by
  finding a non-nilpotent \(x \in A\) such that \(x^2-x\) is in a
  nilpotent ideal. \\

  Choose a minimal left ideal \(L_1 \subset L\) which is not
  nilpotent. Then, \(0 \neq L_1 L_1 \subset L_1\) is not nilpotent, so
  \(L_1 L_1 = L_1\) by the minimality of \(L_1\). \\

  Let \(I\) be a left ideal contained in \(L_1\) such that \(L_1 I
  \neq 0\) and minimal with respect to this property. Let \(a \in I\)
  be such that \(L_1 a \neq 0\). Then, \[
    L_1 L_1 a = L_1^2 a = L_1 a \neq 0 \text{ and } L_1 a \subset I
  \]
  Hence, \(I = L_1 a\) by the minimality of \(I\). Thus, there is an
  \(x \in L_1\) such that \(a=xa\). Hence, \[
    0 \neq a = xa = x^2 a = \cdots = x^k a = \cdots
  \]
  Therefore, \(x\) is not nilpotent. \\

  Let \(N = \{b \in L_1 \st ba = 0\}\). This is a left ideal contained
  in
  \(L_1\). Since \(xa = x^2 a\), \[
    (x-x^2)a = 0 \implies x-x^2 \in N
  \]
  Also, \(L_1 a \neq 0\) so \(N \propsubset L_1\). Hence, \(N\) is
  nilpotent by the minimality of \(L_1\) as a non-nilpotent
  ideal. Thus, we now have \(x,N\) as in the theorem above and so
  there is an idempotent \(e \in Ax \subset L_1 \subset L\).
\end{proof}
\begin{thm}
  Let \(A\) be a left artinian ring. Then \(\rad A\) is the largest
  nilpotent left ideal.
\end{thm}
\begin{prop}
  If \(f \from A \to B\) is a surjective homomorphism of rings,
  then \[
    f(\rad A) \subset \rad B
  \]
\end{prop}
\section{The Radical of an Artinian Ring}
\begin{thm}
  Let \(A\) be a left artinian ring. Then, the sum of all left nilpotent
  ideals of \(A\) is a nilpotent ideal, say \(N\). It contains every
  nilpotent right ideal of \(A\). Also, the quotient ring \(A/N\) has
  no nontrivial nilpotent ideal.
\end{thm}
\begin{thm}
  Let \(A\) be a left artinian ring. If is \(N\) the sum of all
  left nilpotent ideals, then \(N = \rad A\).
\end{thm}
\begin{thm}
  Let \(A\) be a left artinian ring. Then, \({}_A A\) is completely
  reducible if and only if \(A\) is semisimple.
\end{thm}
\begin{thm}
  Let \(A\) be a (left) artinian ring. Then, \(A\) is semisimple if
  and only if every \(A\)-module is completely reducible.
\end{thm}
\begin{thm}
  Let \(A\) be artinian and semisimple. Then, every irreducible
  \(A\)-module \(L\) is isomorphic to a non-zero minimum left ideal of
  \(A\). Thus, \(A\) has only a finite number of irreducible modules.
\end{thm}
\begin{thm}
  Let \(A\) be artinian and semisimple. Let \(Ae\) be a left ideal for
  idempotent \(e\). Then, \(Ae\) is irreducible if and only if \(eAe\)
  is a division ring.
\end{thm}
\section{The Structure of Semisimple (Left) Artinian Rings}
\begin{defn}
  A left artinian ring \(A\) is called \de{simple} if its only
  two-sided ideals
  are \(0\) and \(A\).
\end{defn}
We now seek to prove the Artin-Wedderburn theorem, that will tell us
that a semisimple left artinian ring is isomorphic to a direct sum of
matrix rings over division rings.
\begin{lem}
  Let \(L,L'\) be minimal left ideals. Then, \(L \isom L'\) as left
  \(A\)-modules if and only if there is an \(a' \in L'\) such that
  \(L' = La'\).
\end{lem}
\begin{lem}
  Let \(L,L'\) be minimal left ideals in \(A\). Then, \[
    L \isom L' \iff LL' = L'
  \]
\end{lem}
\begin{thm}[Wedderburn]
  Let \(A\) be a semisimple artinian ring. Let \(L\) be a minimal left
  ideal. The sum of all minimal left ideals \(\{L' \ideal A
  \st L' \isom L\}\), say \(B_L\), is a simple subring and a
  \(2\)-sided ideal in \(A\). Also, \[
    A = \bigoplus_{i=1}^r B_{L_i}
  \]
  where \(L_1, \ldots, L_r\) are representatives of distinct
  isomorphism classes of left ideals.
\end{thm}
\begin{thm}[Wedderburn]
  Let \(A\) be a simple artinian ring. Then, there exists a unique
  division ring and a unique positive integer \(n\) such that \[
    A \isom M_n(D)
  \]
  Conversely, if \(D\) is a division ring, then \(M_n(D)\) is a simple
  artinian ring. 
\end{thm}
Thus, combining our results, we arrive at the landmark theorem.
\begin{thm}[Artin-Wedderburn Theorem]
  Let \(A\) be a semisimple left artinian ring. Then, up to reordering,
  there is a unique decomposition \[
    A \isom M_{n_1}(D_1) \times M_{n_2}(D_2) \times \cdots \times M_{n_r}(D_r)
  \]
  where each \(D_i\) is a division ring and each \(n_i\) is a positive
  integer.
\end{thm}
\begin{bibdiv}
  \begin{biblist}
    \bib{cr}{book}{
      author={Curtis, Curtis W.}
      author={Reiner, Irving}
      title={Representation Theory of Finite Groups and Associative
        Algebras}
      year={1962}
    }
    \bib{jacobson}{book}{
      author={Jacobson, Nathan}
      title={Basic Algebra II}
      year={1989}
      note={2nd edition}
    }
  \end{biblist}
\end{bibdiv}

\end{document}