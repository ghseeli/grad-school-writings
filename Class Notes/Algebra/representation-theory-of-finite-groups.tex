\documentclass[11pt,leqno,oneside]{amsbook}
\usepackage{tikz}
\usetikzlibrary{cd}
\usepackage{bbm}
\usepackage{todonotes}

\usepackage{../notes}
\usepackage{../../ReAdTeX/readtex-core}
\usepackage{../../ReAdTeX/readtex-abstract-algebra}

\newcommand{\bbk}{\mathbbm{k}}

\numberwithin{thm}{section}

\title[Representation Theory of Finite Groups]{Representation Theory
  of Finite Groups \\ Notes
  inspired by a class taught by Brian Parshall in Fall 2017}
\author{George H. Seelinger}
\date{Fall 2017}
\begin{document}
\maketitle
\section{Introduction}
From one perspective, the representation theory of finite groups is merely a
special case of the representation theory of associative algebras by
considering the fact that a sufficiently nice finite group ring \(\bbk[G]\)
is semisimple, and thus the results from Artin-Wedderburn
theory apply. However, the extra structure of the group provides a
rich connection between \(\bbk[G]\)-modules and linear algebra.


A representation of a finite group is a way to induce an action of a
finite group on a vector space. Sometimes, such a perspective allows
mathematicians to see structure or symmetries in groups that may not
have been readily apparent from a purely group theoretic point of
view, much like modules can provide insight into the structure of
rings. In fact, a reprsentation of a finite group is fundamentally the
same as a module of a finite group algebra, but the representation
theoretic perspective allows us to leverage tools from linear algebra
to gain additional insights.

\section{Group Actions}
The following exposition is a summary of \cite{princeton-companion}. The reader is probably familiar with the notion of a group action.
\begin{defn}
  A \de{group action} on a set \(X\) is a homomorphism \(\phi \from G \to \Aut(X)\)
  such that \(\phi(e) = Id_X\).
\end{defn}
If the reader is unfamiliar with group actions, most introductory
texts on group theory will provide more than adequate treatment.
\begin{defn}
  A group action is called \de{faithful} if \(\phi\) is injective. A
  group action is called \de{transitive} if \(\phi(G)\) induces only
  one orbit on \(X\).
\end{defn}
When an action is not transitive, we have multiple orbits and we can
consider the group action on each orbit separately. Thus, in a sense,
we can ``decompose'' this group action (on the level of sets). Thus,
let us focus on transitive actions as our basic building blocks of
group actions.
\begin{thm}
  Let \(\phi\) be a transitive group action of \(G\) on a set \(X\). Then,
  consider \(H = \Stab_G(x)\). Then, \(G/H \isom X\) as sets with a
  (left) \(G\)-action.
\end{thm}
\begin{proof}
  Given \(H = \Stab_G(x)\), consider the correspondance \[
    gH \correspondsto g.x = \phi(g)x
  \]
  Then, the action of \(G\) preserves this correspondance, namely \[
    g'.gH = (g'g)H \correspondsto (g'g).x = \phi(g'g)x = \phi(g')\phi(g)x
    = g'.\phi(g)x = g'.(g.x)
  \]
\end{proof}
Thus, we have reducted the classification of transitive \(G\)-actions
on \(X\) to the study of conjugacy classes of subgroups of
\(G\). Thus, the structure of a group that controls a group action on
a set \(X\) is the subgroup structure of \(G\). However, understanding
the subgroup structure of a group is, in general, incredibly
difficult. For instance, recall Cayley's theorem.
\begin{thm}[Cayley's Theorem]
  Any finite group \(G\) with \(|G| = n < \infty\) can be embedded
  into \(\Sym_n\) via the action of \(G\) on itself. 
\end{thm}
Thus, to understand the subgroups of \(S_n\), one must understand all
finite groups \(G\) with order less than \(n\). Thus, while group
actions are incredibly useful and general, they lack structure that
can be otherwise useful.
\section{Group Algebras}
\begin{defn}
  The \de{group ring} or \de{group algebra} of a group \(G\) over a ring \(R\), denoted
  \(R[G]\) is a free \(R\)-module over the set \[
    \{e_g \st g \in G\}
  \]
  with addition being formal sums. Additionally, \(R[G]\) has the
  structure of a 
  ring where multiplication is given by \(e_g
  \cdot e_h = e_{gh}\) for \(g,h \in G\). Thus, \(1_{R[G]} = e_{1_G}\)
\end{defn}
Typically, we will take \(R\) to be a field, usually
\(\C\). Furthermore, we will often replace \(e_g\) by \(g\) when it is
understood that we are working in the group algebra. That is, for
\(\lambda \in R\), we may
rewrite \[
  \lambda e_g \to \lambda g 
\]
Since group algebras are \(R\)-algebras, we can ask many mathematical
questions about their modules.
\begin{thm}[Maschke's Theorem]\label{maschke}
  Let \(G\) be a finite group and \(\bbk\) be a field such that
  \(\Char \bbk \notdivides |G|\). Then, \(\bbk[G]\) is completely
  reducible as a module over itself. 
\end{thm}
\begin{proof}
  Let \(V\) be a proper \(\bbk[G]\)-submodule of \(\bbk[G]\). Consider the
  \(\bbk\)-linear map
  map \(\pi \from \bbk[G] \to B\) and let \(\phi \from \bbk[G] \to V\)
  be given by \[
    \phi(x) = \frac{1}{|G|} \sum_{s \in G} s.\pi(s^{-1}.x)
  \]
  Thus, \(\phi\) is also a projection since, for \(v \in V\), \[
    \phi(v) = \frac{1}{|G|} \sum_{s \in G} s.\pi(s^{-1}.v) =
    \frac{1}{|G|} \sum_{s \in G} s.s^{-1}.v = v
  \]
  and it is \(\bbk\)-linear since it is simply the (appropriately scaled) sum of \(\bbk\)
  linear maps. In fact,
  \begin{align*}
    \phi(t.x) & = \frac{1}{|G|} \sum_{s \in G} s.\pi(s^{-1}.t.x) \\
              & = \frac{1}{|G|} \sum_{s' \in G} (ts') \cdot \pi(s'^{-1}.x)
              & (s'^{-1} = s^{-1}t \implies s = ts') \\
              & = t.\phi(x)
  \end{align*}
  so, \(\phi\) is \(\bbk[G]\)-linear. Thus, \(\bbk[G] \isom V \oplus
  \ker \phi\), but \(V\) was an arbitrary proper submodule. Thus,
  \(\bbk[G]\) is completely reducible.
\end{proof}
\begin{rmk}
  The converse of Maschke's theorem is also true. 
\end{rmk}
Thus, since \(\bbk[G]\) is finite-dimensional and thus also artinian,
\(\bbk[G]\) is a semisimple ring when \(\Char \bbk \notdivides |G|\)
and, using artin-wedderburn theory, 
has decomposition as an module 
over itself \[ 
  \bbk[G] \isom \bigoplus_{i=1}^k M_{n_i}(D_i)
\]
for \(n_i \in \N\) and \(D_i\) a division ring, and these direct
summands represent all the irreducible \(\bbk[G]\)-modules. However,
we still wish to understand what these division rings and \(n_i\)'s
actually are (see \cite{aw}). Thus, we will shift our perspective to looking at
\emph{group representations} instead of group algebra modules.
\section{Representations}
In this section, we seek to establish some of the basic ideas of
representations of finite groups. Much of this exposition is borrowed
from \cite{etingof} where the results are presented more generally for
\(R\)-algebras. Also, many of these results can be rephrased using the
language in \cite{aw}.
\begin{defn}
  A \de{representation of a finite group \(G\) in a (complex) vector space \(V\)} is a
  homomorphism \(\rho \from G \to GL(V)\).
\end{defn}
\begin{rmk}
  Throughout this monograph, we will often simply say
  ``representation'' to mean a representation of a finite group. When
  we refer to a vector space \(V\), we will mean a vector space over \(\C\).
\end{rmk}
\begin{defn}
  We say two representations \(\rho, \rho' \from G \to GL(V)\) are
  \de{similar} or \de{isomorphic} if 
  there is a linear transformation \(\tau \from V \to V'\) such
  that \[
    \tau \circ \rho(g) = \rho'(g) \circ \tau
  \]
  for all \(g \in G\).
\end{defn}
\begin{defn}
  We say that a subspace \(W \subset V\) is \de{\(G\)-stable} if
  \(\rho(g)W \subset W\) for all \(g \in G\).
\end{defn}
\begin{prop}
  Let \(\rho \from G \to GL(V)\) be a linear representation of \(G\)
  in \(V\) and let \(W\) be a subspace of \(V\) such that \(\rho(g)W =
  W\) for all \(g \in G\). Then, there exists a complement \(W'\) of
  \(W\) in \(V\) such that \(\rho(g)W' \subset W'\) for all \(g \in G\)
\end{prop}
\begin{proof}
  Let \(W'\) be a vector space complement of \(W\) in \(V\) (not
  necessarily \(G\)-stable) and let \(p \from V \to W\) be the
  standard projection. Then, consider the map \[
    p^0 := \frac{1}{|G|} \sum_{t \in G} \rho(t) \circ p \circ \rho(t)^{-1}
  \]
  Such a map is a projection of \(V\) onto a subspace of \(W\). In
  fact, \(p^0|_W = Id_W\) since, for \(w \in W\), \[
    p \circ \rho(t)^{-1} w = \rho(t)^{-1} w \implies \rho(t) \circ p
    \circ \rho(t)^{-1} w = w \implies p^0 w = w
  \]
  Thus, we seek to show \(W^0 := \ker p^0\) is stable under the \(G\)
  action via \(\rho\). Indeed, it is easy to check \(\rho(s) p^0
  \rho(s)^{-1} = p^0\) and thus \(p^0 \circ \rho(s)x = \rho(s) \circ
  p^0 x = 0\), that is, \(\rho(s) x \in W^0\). Thus, \(V = W \oplus
  W^0\) is a decomposition into \(G\)-stable subspaces. 
\end{proof}
\begin{rmk}
  Note the similarity between this proof and the proof of Maschke's
  theorem above (\ref{maschke}). Indeed, these proofs are more or less
  equivalent and the proposition above is the same as Maschke's
  theorem for \(\bbk = \C\).
\end{rmk}
\begin{defn}
  Given representations \(\rho_1 \from G \to GL(V_1)\) and \(\rho_2 \from
  G \to GL(V_2)\), we define \(\rho := \rho_1 \oplus \rho_2 \from G \to GL(V_1
  \oplus V_2)\) by the map \[
    g \mapsto \left(
      \begin{array}{cc}
        \rho_1(g)&0\\
        0&\rho_2(g)
      \end{array}
    \right)
  \]
  That is, \(\rho(g)|_{V_i} = \rho_i(g)\).
\end{defn}
\begin{defn}
  We say that a representation \(\rho \from G \to GL(V)\) is
  \de{irreducible} or \de{simple} if \(V \neq 0\) and there is no
  subspace \(W
  \subset V\) such that \(W\) is stable under the action of \(G\) via
  \(\rho\). By the proposition above, this is equivalent to saying
  that \(\rho\) does not break into a direct sum of representations.
\end{defn}
\begin{prop}
  Every representation is a direct sum of irreducible representations.
\end{prop}
\begin{proof}
  The proof follows from induction on \(\dim V\) and application of
  the proposition above.
\end{proof}
\begin{defn}
  Let \(\rho_1 \from G \to GL(V_1)\) and \(\rho_2 \from G \to
  GL(V_2)\) be representations. Then, we define \(\rho := \rho_1
  \otimes \rho_2 \from G \to GL(V_1 \otimes V_2)\) by, for \(s \in G\), \[
    \rho(s)(v_1 \otimes v_2) = \rho_1(s)v_1 \otimes \rho_2(s) v_2
  \]
\end{defn}
\begin{rmk}
  The tensor product of two irreducible representations is not
  typically irreducible. 
\end{rmk}
\begin{thm}
  Let \(\rho \from G \to GL(V)\) be a representation. Then, given
  \(\rho \otimes \rho \to GL(V \otimes V)\) decomposes as \[
    V \otimes V \isom \frac{V \otimes V}{(x \otimes y - y \otimes x)} \oplus
    \frac{V \otimes V}{(x \otimes y + y \otimes x)} =
    \operatorname{Sym}^2(V) \oplus \operatorname{Alt}^2(V)
  \]
\end{thm}
\begin{proof}
  Consider the automorphism of \(V \otimes V\) given by \(\tau(e_i
  \otimes e_j) = e_j \otimes e_i\) for basis \(\{e_1, \ldots, e_n\}\)
  of \(V\). Then, \(\tau(v \otimes w) = w \otimes v\) for any \(v,w
  \in V\) and \(\tau^2 = Id_{V \otimes V}\). Thus, as vector spaces, \[
    V \otimes V \isom  \ker (\tau-Id) \oplus \im (\tau-Id) 
  \]
  However, \(\ker(\tau-Id) = \langle x \otimes y \st x \otimes y - y
  \otimes x = 0 \rangle \isom (V \otimes V)/(x \otimes y - y \otimes
  x)\). Now, consider that \[
    (\tau-Id)(x \otimes y) + \tau(\tau-Id)(x \otimes y) = (\tau-Id)(x
    \otimes y) + (Id - \tau)(x \otimes y) = 0
  \]
  and so, for any \(v \in \im(\tau-Id)\), we get \(v + \tau(v) =
  0\). Thus, \(\im(\tau-Id) \subset (V \otimes V)/(x \otimes y + y
  \otimes x)\). However, by dimension counting, we note that
  \begin{align*}
        \dim \ker(\tau-Id) = \dim \operatorname{Sym}^2(V) &= \frac{n(n+1)}{2} \implies \\ \dim
    \im(\tau-Id) = n - \frac{n(n+1)}{2} &= \frac{n(n-1)}{2} = \dim \operatorname{Alt}^2(V)
  \end{align*}
  Thus, \(\im(\tau-Id) = \operatorname{Alt}^2(V)\).

  However, we must also check that these spaces are stable under
  \(G\). However, for \(\rho' = \rho \otimes \rho\),
  \begin{align*}
    \rho'(g) \circ \tau \circ \rho'(g^{-1}) (v \otimes w)
    & = \rho'(g)
      \circ \tau (\rho^{-1}(g)v \otimes \rho^{-1}(g)w) \\
    & = \rho'(g)(\rho^{-1}(g)w \otimes \rho^{-1}(g)v) \\
    & = w \otimes v \\
    & = \tau(v \otimes w)
  \end{align*}
  and so the image and kernel of \(\tau-Id\) is stable under the
  action of \(G\).
\end{proof}
We now also seek to prove some other useful tools in representation
theory.
\begin{thm}[Schur's Lemma]
  Let \(V_1, V_2\) be representations of \(G\) and let \(\phi \from
  V_1 \to V_2\) be a nontrivial homomorphism of representations. Then,
  \begin{enumerate}
  \item If \(V_1\) is irreducible, then \(\phi\) is injective.
  \item If \(V_2\) is irreducible, then \(\phi\) is surjective.
  \end{enumerate}
\end{thm}
\begin{proof}
  Exercise for the reader.
\end{proof}
\begin{cor}[Schur's Lemma for algebraically closed fields]\label{schur-alg-closed}
  Let \(V\) be a finite dimensional irreducible representation of a
  group \(G\) over an algebraically closed field \(\bbk\), and \(\phi
  \from V \to V\) a commuting homomorphism (ie \(\rho(g) \circ \phi = \phi
  \circ \rho(g)\)). Then, \(\phi = \lambda Id_V\) for some \(\lambda
  \in \bbk\).
\end{cor}
\begin{rmk}
  \begin{enumerate}
  \item We sometimes call \(\phi\) a ``scalar operator'' or say that
    \(\phi\) ``acts as a scalar'' in this situation. In \cite{serre},
    Serre calls such an operator a ``homothey.''
  \item This proposition is false over \(\R\). Just consider \(G = \Z_2
    \times \Z_2\) with representation \(\rho \from G \to
    GL(\R^4)\). \todo{Check this!}
  \end{enumerate}
\end{rmk}
\begin{proof}
  Let \(\lambda\) be an eigenvalue of \(\phi\), which exists since
  \(\bbk\) is algebraically closed. Then, since \(\phi\) commutes with
  \(\rho(g)\), so does \(\phi - \lambda Id \from V \to V\). However, \(\phi -
  \lambda Id\) is not an isomorphism since \(\det(\phi - \lambda Id) =
  0\). Thus, \(\phi - \lambda Id = 0 \implies \phi = \lambda Id\).
\end{proof}
\begin{cor}
  Let \(G\) be an abelian group. Then, every irreducible finite
  dimensional representation of \(G\) is \(1\)-dimensional.
\end{cor}
\begin{proof}
  Let \(V\) be an irreducible finite dimensional representation of
  \(G\). Then, for \(g,h \in G\), \(v \in V\) \[
    \rho(g) \rho(h) v = \rho(gh)v = \rho(hg)v = \rho(h)\rho(g) v
  \]
  Thus, \(\rho(g)\) commutes with all \(\rho(h)\) and so, by the
  corollary above, \(\rho(g) = \lambda Id\) for some \(\lambda \in
  \C\). Thus, every vector subspace of \(V\) is a subrepresentation,
  but \(V\) is irreducible. Thus, \(\dim V = 1\).
\end{proof}
\begin{defn}
  For an irreducible representation \(\rho \from G \to GL(V)\) of
  \(G\), then for all \(g \in Z(G)\), \(\rho(g)\) commutes with all
  \(\rho(h), h \in H\). Thus, \(\rho(g) = \lambda_g Id_V\) by \ref{schur-alg-closed} and
  we call the map \(\chi_V \from Z(G) \to \C\) the \de{central
    character} of \(V\).
\end{defn}
\begin{prop}
  \(\chi_V \from Z(G) \to \C\) is a homomorphism.
\end{prop}
\section{Character Theory}
Throughout this section, let \(G\) be a finite group and \(F =
\C\). We wish to extend the notion of the central character above to a
general character \(\chi_V \from G \to \C\) that will still carry
useful information about the group representation \(V\).
\begin{prop}
  If \(V\) is a \(G\)-module, then every \(g \in G\), viewed as a
  linear operator \(g_V \from V \to V\), is semisimple, and thus \(V\)
  has a basis of eigenvectors for \(g_V\).
\end{prop}
\begin{proof}
  Since \(G\) is a finite group, \(g\) is of finite order, so \(g_V^n
  = Id_V\). Thus, \(g_V^n = (D+N)^n\) for diagonalizable \(D\) and
  nilpotent \(N\) (by Jordan Canonical Form since \(\C\) is
  algebraically closed) such that \(DN = ND\). Thus, \[
    Id_V = g_V^n = (D+N)^n = D^n + nND^{n-1} + \binom{n}{2} N^2 D^{n-2} +
    \cdots + nN^{n-1}D + N^n 
  \]
  which thus tells us that \(N = 0\) and \(D^n = Id_V\) since \(Id_V\)
  has no nilpotent part.
\end{proof}
\begin{defn}
  The \de{character} of group representation \(\rho \from G \to V\), denoted \(\chi_\rho = \chi^\rho\), is the function
  \(\chi_\rho \from G \to \bbk\) defined by \(\chi_\rho(g) =
  \tr(\rho(g))\).
\end{defn}
\begin{prop}
  For \(\bbk = \C\) and fixed group representation \(\rho\),
  \begin{enumerate}
  \item \(\chi(1) = \dim V\),
  \item \(\chi(g^{-1}) = \ov{\chi(g)}\) for all \(g \in G\),
  \item \(\chi(tst^{-1}) = \chi(s)\) for all \(s,t \in G\).
  \end{enumerate}
\end{prop}
\begin{proof}
  All these properties follow from standard properties of trace.
\end{proof}
\begin{defn}
  A function \(f \from G \to \C\) is called a \de{class function}
  because it is constant on conjugacy classes. Note that
  \(\operatorname{Class}(G)\) is the \(\C\)-algebra of all class
  functions on \(G\) and \(\dim \operatorname{Class}(G) =\) the number
  of conjugacy classes in \(G\).
\end{defn}
\begin{prop}
  Let \(\rho_i \from G \to GL(V_i)\), \(i = 1,2\) be representations
  of \(G\) with characters \(\chi_{\rho_i}\).
  \begin{enumerate}
  \item \(\chi_{\rho_1 \oplus \rho_2} = \chi_{\rho_1} + \chi_{\rho_2}\)
  \item \(\chi_{\rho_1 \otimes \rho_2} = \chi_{\rho_1} \cdot \chi_{\rho_2}\)
  \end{enumerate}
\end{prop}
\begin{proof}
  This follows from the definition of direct sum and tensor product of
  representations and the definition of the character.
\end{proof}
\begin{bibdiv}
  \begin{biblist}
    \bib{cr}{book}{
      author={Curtis, Charles W.}
      author={Reiner, Irving}
      title={Representation Theory of Finite Groups and Associative
        Algebras}
      year={1962}
    }
    \bib{etingof}{article}{
      author={Etingof, Pavel}
      author={Golberg, Oleg}
      author={Hensel, Sebastian}
      author={Liu, Tiankai}
      author={Schwendner, Alex}
      author={Vaintrob, Dmitry}
      author={Yudovina}
      title={Introduction to Representation Theory}
      year={2011}
    }
    \bib{princeton-companion}{article}{
      author={Gronjnowski, Ian}
      title={Representation Theory}
      journal={The Princeton Companion to Mathematics}
      pages={419--431}
    }
    \bib{aw}{article}{
      author={Seelinger, George H.}
      title={Artin-Wedderburn Theory}
      year={2017}
      note={See \url{https://github.com/ghseeli/grad-school-writings/releases/latest}}
    }
    \bib{serre}{book}{
      author={Serre, Jean-Pierre}
      title={Linear Representations of Finite Groups}
      year={1997}
      note={Translated from the French by Leonard L. Scott}
    }

  \end{biblist}
\end{bibdiv}

\end{document}