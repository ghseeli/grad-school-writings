\documentclass[11pt,leqno,oneside]{amsbook}
\usepackage{tikz}
\usetikzlibrary{cd}
\usepackage{bbm}
\usepackage{todonotes}

\usepackage{../notes}
\usepackage{../../ReAdTeX/readtex-core}
\usepackage{../../ReAdTeX/readtex-abstract-algebra}

\newcommand{\bbk}{\mathbbm{k}}
\newcommand{\Class}{\operatorname{Class}}

\numberwithin{thm}{section}

\title[Representation Theory of Finite Groups]{Representation Theory
  of Finite Groups \\ Notes
  inspired by a class taught by Brian Parshall in Fall 2017}
\author{George H. Seelinger}
\date{Fall 2017}
\begin{document}
\maketitle
\section{Introduction}
From one perspective, the representation theory of finite groups is merely a
special case of the representation theory of associative algebras by
considering the fact that a sufficiently nice finite group ring \(\bbk[G]\)
is semisimple, and thus the results from Artin-Wedderburn
theory apply. However, the extra structure of the group provides a
rich connection between \(\bbk[G]\)-modules and linear algebra.

A representation of a finite group is a way to induce an action of a
finite group on a vector space. Sometimes, such a perspective allows
mathematicians to see structure or symmetries in groups that may not
have been readily apparent from a purely group theoretic point of
view, much like modules can provide insight into the structure of
rings. In fact, a reprsentation of a finite group is fundamentally the
same as a module of a finite group algebra, but the representation
theoretic perspective allows us to leverage tools from linear algebra
to gain additional insights.

While the classic texts in representation theory continue to be cited
and used often, and contain many details ommitted here, I am eternally
grateful for \cite{etingof} and \cite{smith} for being approachable,
providing simplifying insights, and using more modern language to
rephrase classical results. Much of the writings here are shamelessly
borrowed and synthesized from these sources.
\section{Group Actions}
The following exposition is a summary of \cite{princeton-companion}. The reader is probably familiar with the notion of a group action.
\begin{defn}
  A \de{group action} on a set \(X\) is a homomorphism \(\phi \from G \to \Aut(X)\)
  such that \(\phi(e) = Id_X\).
\end{defn}
If the reader is unfamiliar with group actions, most introductory
texts on group theory will provide more than adequate treatment.
\begin{defn}
  A group action is called \de{faithful} if \(\phi\) is injective. A
  group action is called \de{transitive} if \(\phi(G)\) induces only
  one orbit on \(X\).
\end{defn}
When an action is not transitive, we have multiple orbits and we can
consider the group action on each orbit separately. Thus, in a sense,
we can ``decompose'' this group action (on the level of sets). Thus,
let us focus on transitive actions as our basic building blocks of
group actions.
\begin{thm}
  Let \(\phi\) be a transitive group action of \(G\) on a set \(X\). Then,
  consider \(H = \Stab_G(x)\). Then, \(G/H \isom X\) as sets with a
  (left) \(G\)-action.
\end{thm}
\begin{proof}
  Given \(H = \Stab_G(x)\), consider the correspondance \[
    gH \correspondsto g.x = \phi(g)x
  \]
  Then, the action of \(G\) preserves this correspondance, namely \[
    g'.gH = (g'g)H \correspondsto (g'g).x = \phi(g'g)x = \phi(g')\phi(g)x
    = g'.\phi(g)x = g'.(g.x)
  \]
\end{proof}
Thus, we have reducted the classification of transitive \(G\)-actions
on \(X\) to the study of conjugacy classes of subgroups of
\(G\). Thus, the structure of a group that controls a group action on
a set \(X\) is the subgroup structure of \(G\). However, understanding
the subgroup structure of a group is, in general, incredibly
difficult. For instance, recall Cayley's theorem.
\begin{thm}[Cayley's Theorem]
  Any finite group \(G\) with \(|G| = n < \infty\) can be embedded
  into \(\Sym_n\) via the action of \(G\) on itself. 
\end{thm}
Thus, to understand the subgroups of \(\Sym_n\), one must understand all
finite groups \(G\) with order less than \(n\). Thus, while group
actions are incredibly useful and general, they lack structure that
can be otherwise useful.
\section{Group Algebras}
\begin{defn}
  The \de{group ring} or \de{group algebra} of a group \(G\) over a ring \(R\), denoted
  \(R[G]\) is a free \(R\)-module over the set \[
    \{e_g \st g \in G\}
  \]
  with addition being formal sums. Additionally, \(R[G]\) has the
  structure of a 
  ring where multiplication is given by \(e_g
  \cdot e_h = e_{gh}\) for \(g,h \in G\). Thus, \(1_{R[G]} = e_{1_G}\)
\end{defn}
Typically, we will take \(R\) to be a field, usually
\(\C\). Furthermore, we will often replace \(e_g\) by \(g\) when it is
understood that we are working in the group algebra. That is, for
\(\lambda \in R\), we may
rewrite \[
  \lambda e_g \to \lambda g 
\]
Since group algebras are \(R\)-algebras, we can ask many mathematical
questions about their modules.
\begin{thm}[Maschke's Theorem]\label{maschke}
  Let \(G\) be a finite group and \(\bbk\) be a field such that
  \(\Char \bbk \notdivides |G|\). Then, \(\bbk[G]\) is completely
  reducible as a module over itself. 
\end{thm}
\begin{proof}
  Let \(V\) be a proper \(\bbk[G]\)-submodule of \(\bbk[G]\). Consider the
  \(\bbk\)-linear map
  map \(\pi \from \bbk[G] \to B\) and let \(\phi \from \bbk[G] \to V\)
  be given by \[
    \phi(x) = \frac{1}{|G|} \sum_{s \in G} s.\pi(s^{-1}.x)
  \]
  Thus, \(\phi\) is also a projection since, for \(v \in V\), \[
    \phi(v) = \frac{1}{|G|} \sum_{s \in G} s.\pi(s^{-1}.v) =
    \frac{1}{|G|} \sum_{s \in G} s.s^{-1}.v = v
  \]
  and it is \(\bbk\)-linear since it is simply the (appropriately scaled) sum of \(\bbk\)
  linear maps. In fact,
  \begin{align*}
    \phi(t.x) & = \frac{1}{|G|} \sum_{s \in G} s.\pi(s^{-1}.t.x) \\
              & = \frac{1}{|G|} \sum_{s' \in G} (ts') \cdot \pi(s'^{-1}.x)
              & (s'^{-1} = s^{-1}t \implies s = ts') \\
              & = t.\phi(x)
  \end{align*}
  so, \(\phi\) is \(\bbk[G]\)-linear. Thus, \(\bbk[G] \isom V \oplus
  \ker \phi\), but \(V\) was an arbitrary proper submodule. Thus,
  \(\bbk[G]\) is completely reducible.
\end{proof}
\begin{rmk}
  The converse of Maschke's theorem is also true. 
\end{rmk}
Thus, since \(\bbk[G]\) is finite-dimensional and thus also artinian,
\(\bbk[G]\) is a semisimple ring when \(\Char \bbk \notdivides |G|\)
and, using artin-wedderburn theory, 
has decomposition as an module 
over itself \[ 
  \bbk[G] \isom \bigoplus_{i=1}^k M_{n_i}(D_i)
\]
for \(n_i \in \N\) and \(D_i\) a division ring, and these direct
summands represent all the irreducible \(\bbk[G]\)-modules. However,
we still wish to understand what these division rings and \(n_i\)'s
actually are (see \cite{aw}). Thus, we will shift our perspective to looking at
\emph{group representations} instead of group algebra modules.
\section{Representations}
In this section, we seek to establish some of the basic ideas of
representations of finite groups. Much of this exposition is borrowed
from \cite{etingof} where the results are presented more generally for
\(R\)-algebras. Also, many of these results can be rephrased using the
language in \cite{aw}.
\begin{defn}
  A \de{representation of a finite group \(G\) in a (complex) vector space \(V\)} is a
  homomorphism \(\rho \from G \to GL(V)\).
\end{defn}
\begin{rmk}
  Throughout this monograph, we will often simply say
  ``representation'' to mean a representation of a finite group. When
  we refer to a vector space \(V\), we will mean a vector space over \(\C\).
\end{rmk}
\begin{thm}
  There is a natural bijection between \(\C[G]\)-modules and complex
  representations of \(G\).
\end{thm}
\begin{proof}
  Given a representation of \(G\) given by \(\rho \from G \to GL(V)\),
  we get a \(\C[G]\) action on \(v \in V\) via \[
    e_g.v = \rho(g)v
  \]
  and extend via linearity. Conversely, given \(M\) a
  \(\C[G]\)-module, we can consider \(M\) as 
  a vector space over \(\C e_1\) and then view the action of each
  \(e_g\) as an invertible linear transformation on this vector space.
\end{proof}

\begin{defn}
  We say two representations \(\rho, \rho' \from G \to GL(V)\) are
  \de{similar} or \de{isomorphic} if 
  there is a linear transformation \(\tau \from V \to V'\) such
  that \[
    \tau \circ \rho(g) = \rho'(g) \circ \tau
  \]
  for all \(g \in G\).
\end{defn}
\begin{defn}
  We say that a subspace \(W \subset V\) is \de{\(G\)-stable} if
  \(\rho(g)W \subset W\) for all \(g \in G\).
\end{defn}
\begin{prop}
  Let \(\rho \from G \to GL(V)\) be a linear representation of \(G\)
  in \(V\) and let \(W\) be a subspace of \(V\) such that \(\rho(g)W =
  W\) for all \(g \in G\). Then, there exists a complement \(W'\) of
  \(W\) in \(V\) such that \(\rho(g)W' \subset W'\) for all \(g \in G\)
\end{prop}
\begin{proof}
  Let \(W'\) be a vector space complement of \(W\) in \(V\) (not
  necessarily \(G\)-stable) and let \(p \from V \to W\) be the
  standard projection. Then, consider the map \[
    p^0 := \frac{1}{|G|} \sum_{t \in G} \rho(t) \circ p \circ \rho(t)^{-1}
  \]
  Such a map is a projection of \(V\) onto a subspace of \(W\). In
  fact, \(p^0|_W = Id_W\) since, for \(w \in W\), \[
    p \circ \rho(t)^{-1} w = \rho(t)^{-1} w \implies \rho(t) \circ p
    \circ \rho(t)^{-1} w = w \implies p^0 w = w
  \]
  Thus, we seek to show \(W^0 := \ker p^0\) is stable under the \(G\)
  action via \(\rho\). Indeed, it is easy to check \(\rho(s) p^0
  \rho(s)^{-1} = p^0\) and thus \(p^0 \circ \rho(s)x = \rho(s) \circ
  p^0 x = 0\), that is, \(\rho(s) x \in W^0\). Thus, \(V = W \oplus
  W^0\) is a decomposition into \(G\)-stable subspaces. 
\end{proof}
\begin{rmk}
  Note the similarity between this proof and the proof of Maschke's
  theorem above (\ref{maschke}). Indeed, these proofs are more or less
  equivalent and the proposition above is the same as Maschke's
  theorem for \(\bbk = \C\).
\end{rmk}
\begin{defn}
  Given representations \(\rho_1 \from G \to GL(V_1)\) and \(\rho_2 \from
  G \to GL(V_2)\), we define \(\rho := \rho_1 \oplus \rho_2 \from G \to GL(V_1
  \oplus V_2)\) by the map \[
    g \mapsto \left(
      \begin{array}{cc}
        \rho_1(g)&0\\
        0&\rho_2(g)
      \end{array}
    \right)
  \]
  That is, \(\rho(g)|_{V_i} = \rho_i(g)\).
\end{defn}
\begin{defn}
  We say that a representation \(\rho \from G \to GL(V)\) is
  \de{irreducible} or \de{simple} if \(V \neq 0\) and there is no
  subspace \(W
  \subset V\) such that \(W\) is stable under the action of \(G\) via
  \(\rho\). By the proposition above, this is equivalent to saying
  that \(\rho\) does not break into a direct sum of representations.
\end{defn}
\begin{prop}
  Every representation is a direct sum of irreducible representations.
\end{prop}
\begin{proof}
  The proof follows from induction on \(\dim V\) and application of
  the proposition above.
\end{proof}
\begin{defn}
  Let \(\rho_1 \from G \to GL(V_1)\) and \(\rho_2 \from G \to
  GL(V_2)\) be representations. Then, we define \(\rho := \rho_1
  \otimes \rho_2 \from G \to GL(V_1 \otimes V_2)\) by, for \(s \in G\), \[
    \rho(s)(v_1 \otimes v_2) = \rho_1(s)v_1 \otimes \rho_2(s) v_2
  \]
\end{defn}
\begin{rmk}
  The tensor product of two irreducible representations is not
  typically irreducible. 
\end{rmk}
\begin{thm}
  Let \(\rho \from G \to GL(V)\) be a representation. Then, given
  \(\rho \otimes \rho \to GL(V \otimes V)\) decomposes as \[
    V \otimes V \isom \frac{V \otimes V}{(x \otimes y - y \otimes x)} \oplus
    \frac{V \otimes V}{(x \otimes y + y \otimes x)} =
    \operatorname{Sym}^2(V) \oplus \operatorname{Alt}^2(V)
  \]
\end{thm}
\begin{proof}
  Consider the automorphism of \(V \otimes V\) given by \(\tau(e_i
  \otimes e_j) = e_j \otimes e_i\) for basis \(\{e_1, \ldots, e_n\}\)
  of \(V\). Then, \(\tau(v \otimes w) = w \otimes v\) for any \(v,w
  \in V\) and \(\tau^2 = Id_{V \otimes V}\). Thus, as vector spaces, \[
    V \otimes V \isom  \ker (\tau-Id) \oplus \im (\tau-Id) 
  \]
  However, \(\ker(\tau-Id) = \langle x \otimes y \st x \otimes y - y
  \otimes x = 0 \rangle \isom (V \otimes V)/(x \otimes y - y \otimes
  x)\). Now, consider that \[
    (\tau-Id)(x \otimes y) + \tau(\tau-Id)(x \otimes y) = (\tau-Id)(x
    \otimes y) + (Id - \tau)(x \otimes y) = 0
  \]
  and so, for any \(v \in \im(\tau-Id)\), we get \(v + \tau(v) =
  0\). Thus, \(\im(\tau-Id) \subset (V \otimes V)/(x \otimes y + y
  \otimes x)\). However, by dimension counting, we note that
  \begin{align*}
        \dim \ker(\tau-Id) = \dim \operatorname{Sym}^2(V) &= \frac{n(n+1)}{2} \implies \\ \dim
    \im(\tau-Id) = n - \frac{n(n+1)}{2} &= \frac{n(n-1)}{2} = \dim \operatorname{Alt}^2(V)
  \end{align*}
  Thus, \(\im(\tau-Id) = \operatorname{Alt}^2(V)\).

  However, we must also check that these spaces are stable under
  \(G\). However, for \(\rho' = \rho \otimes \rho\),
  \begin{align*}
    \rho'(g) \circ \tau \circ \rho'(g^{-1}) (v \otimes w)
    & = \rho'(g)
      \circ \tau (\rho^{-1}(g)v \otimes \rho^{-1}(g)w) \\
    & = \rho'(g)(\rho^{-1}(g)w \otimes \rho^{-1}(g)v) \\
    & = w \otimes v \\
    & = \tau(v \otimes w)
  \end{align*}
  and so the image and kernel of \(\tau-Id\) is stable under the
  action of \(G\).
\end{proof}
We now also seek to prove some other useful tools in representation
theory.
\begin{thm}[Schur's Lemma]
  Let \(V_1, V_2\) be representations of \(G\) and let \(\phi \from
  V_1 \to V_2\) be a nontrivial homomorphism of representations. Then,
  \begin{enumerate}
  \item If \(V_1\) is irreducible, then \(\phi\) is injective.
  \item If \(V_2\) is irreducible, then \(\phi\) is surjective.
  \end{enumerate}
\end{thm}
\begin{proof}
  Exercise for the reader.
\end{proof}
\begin{cor}[Schur's Lemma for algebraically closed fields]\label{schur-alg-closed}
  Let \(V\) be a finite dimensional irreducible representation of a
  group \(G\) over an algebraically closed field \(\bbk\), and \(\phi
  \from V \to V\) a commuting homomorphism (ie \(\rho(g) \circ \phi = \phi
  \circ \rho(g)\)). Then, \(\phi = \lambda Id_V\) for some \(\lambda
  \in \bbk\).
\end{cor}
\begin{rmk}
  \begin{enumerate}
  \item We sometimes call \(\phi\) a ``scalar operator'' or say that
    \(\phi\) ``acts as a scalar'' in this situation. In \cite{serre},
    Serre calls such an operator a ``homothey.''
  \item This proposition is false over \(\R\).
  \end{enumerate}
\end{rmk}
\begin{proof}
  Let \(\lambda\) be an eigenvalue of \(\phi\), which exists since
  \(\bbk\) is algebraically closed. Then, since \(\phi\) commutes with
  \(\rho(g)\), so does \(\phi - \lambda Id \from V \to V\). However, \(\phi -
  \lambda Id\) is not an isomorphism since \(\det(\phi - \lambda Id) =
  0\). Thus, \(\phi - \lambda Id = 0 \implies \phi = \lambda Id\).
\end{proof}
\begin{cor}
  Let \(G\) be an abelian group. Then, every irreducible finite
  dimensional representation of \(G\) is \(1\)-dimensional.
\end{cor}
\begin{proof}
  Let \(V\) be an irreducible finite dimensional representation of
  \(G\). Then, for \(g,h \in G\), \(v \in V\) \[
    \rho(g) \rho(h) v = \rho(gh)v = \rho(hg)v = \rho(h)\rho(g) v
  \]
  Thus, \(\rho(g)\) commutes with all \(\rho(h)\) and so, by the
  corollary above, \(\rho(g) = \lambda Id\) for some \(\lambda \in
  \C\). Thus, every vector subspace of \(V\) is a subrepresentation,
  but \(V\) is irreducible. Thus, \(\dim V = 1\).
\end{proof}
\begin{example}
  Let \(G=\Sym_3\), the smallest order non-abelian group, and let
  \(V\) be a complex representation of \(G\). Since \(\Sym_3\) is
  generated by \(\sigma = (123)\) and \(\tau = (12)\), we need only
  find subrepresentations of \(V\) that are stable under the actions
  of \(\sigma\) and \(\tau\). An important tool for this process will
  be to find eigenvectors. Let \(v\) be an eigenvector for
  \(\sigma\) with eigenvalue \(\lambda\). Then, consider \(\tau
  v\). Such a vector must be an 
  eigenvector for \(\sigma\) as well since \[
    \sigma \tau v = \tau \sigma^2 v = \tau \lambda^2 v = \lambda^2
    \tau v
  \]
  Thus, the space \(\Span \{v,\tau v\}\) is stable under the actions of
  \(\sigma,\tau\) and must be a subrepresentation of \(V\). \\

  Now, if we wish to find all irreducible representations of \(G\), we
  know that the dimension must be less than or equal to
  \(2\). However, we can figure out more. We know that \(\sigma^3 =
  Id\), and so it must be that \(v = \sigma^3 v = \lambda^3 v\) and so
  \(\lambda = 1, \zeta_3,\) or \(\zeta_3^2\), where \(\zeta_3\) is a
  primitive \(3\)rd root of unity.
  \begin{itemize}
  \item Let \(\lambda \neq 1\). Then, \(\lambda \neq \lambda^2\) and
    so \(v\) and \(\tau v\) have distinct eigenvalues and thus they
    are linearly independent. Thus, \(\Span\{v,\tau v\}\) defines an
    irreducible, \(2\)-dimensional representation. This is called the
    \de{standard representation}.
  \item Let \(\lambda = 1\) and \(\tau v = -v\). Then, \(\Span\{v\}\)
    is an irreducible \(1\)-dimensional representation, but it is
    distinct from the trivial representation. It is called the
    \de{alternating representation}.
  \item Let \(\lambda = 1\) and \(\tau v = v\). This is the trivial
    representation. 
  \end{itemize}
  Thus, we have found \(3\) irreducible representations of
  \(\Sym_3\).
\end{example}
\section{The Regular Representation}
One of the most important representations is the regular
representation. Later, we will see that every irreducible
representation appears in the regular representation. Furthermore, the
regular representation is a somewhat natural representation to define.
\begin{defn}
  Let \(G\) be a finite group. Then, the \de{regular representation}
  of \(G\) is a homomorphism \(\rho_{reg} \from G \to GL(V)\) where \(V =
  \C^{|G|}\) given
  by \[
    g \mapsto (g \from V \to V, e_h \mapsto e_{gh}) 
  \]
\end{defn}
Note that the regular representation is equivalent to viewing
\(\bbk[G]\) as a module over itself.
\begin{example}
  Consider \(G = \Sym_3\). Then, the regular representation of
  \(\Sym_3\) is given by \[
    \sigma.e_{\sigma'} = e_{\sigma \sigma'}
  \]
  Now, consider that the element \[
    e_{(1)} + e_{(12)} + e_{(13)} + e_{(23)} + e_{(123)} + e_{(321)}
  \]
  is a \(G\)-invariant subspace of \(\C^6\) under the regular
  representation. This subrepresentation is isomorphic to the trivial
  represenation. Similarly, the space spanned by \[
    e_{(1)} - e_{(12)} - e_{(13)} - e_{(23)} + e_{(123)} + e_{(321)}
  \]
  is \(G\)-stable and, as a subrepresentation of the regular
  representation, is isomorphic to the alternating
  representation. Thus, we have \[
    \rho_{reg} = \rho_{trivial} \oplus \rho_{alt} \oplus \rho'
  \]
  where \(\rho' \from G \to GL(\C^6/(V_{trivial} \oplus
  V_{alt}))\). We know from our analysis above that \(\rho'\) must
  break up further, and we will see later that it breaks up as the
  direct sum of 2 standard representations.
\end{example}
\section{\(\Hom\) and dual representation}
We take a quick detour to view some other methods for constructing new
representations from old ones. These results will be useful later, so
the reader may choose to skip them for now and come back later when
they are needed.
\begin{prop}\label{def-hom-rep}
  Let \(V,W\) be representations of a finite group \(G\). Then, the
  action of \(G\) on 
  the vector space \(\Hom_\C(V,W)\) given by, for \(\phi \in \Hom_\C(V,W)\),
  \begin{align*}
    g.\phi \from V & \to W \\
    v & \mapsto g.\phi(g^{-1}.v)
  \end{align*}
  makes \(\Hom_\C(V,W)\) into a representation of \(G\). 
\end{prop}
\begin{cor}\label{def-dual-rep}
  Let \(V\) be a representation of a finite group \(G\). Then, \(V^* =
  \Hom_\C(V,\C)\) has a natural representation
  structure. Explicitly, for basis \(f \in V^*\), \[
    \rho_{V^*}(g).f = f \circ \rho_V(g^{-1}) 
  \]
\end{cor}
\begin{prop}\label{tensor-hom-relation}
  Given finite-dimensional representations \(V,W\) of \(G\), then
  there is a natural 
  isomorphism \[
    V^* \otimes W \isomto \Hom_\C(V,W)
  \]
\end{prop}
\begin{proof}
  Consider the \(\C\)-bilinear map
  \begin{align*}
    V^* \times W & \to \Hom_\C(V,W) \\
    (f(\cdot),w) & \mapsto f(\cdot) w
  \end{align*}
  Then, by the universal property of tensor products, there is a
  unique homomorphism \(\phi \from V^* \otimes W \to \Hom_\C(V,W)\)
  such that \(f(\cdot) \otimes w \mapsto f(\cdot) w\). Such a map is
  certainly injective since \(\ker \phi = \{0\}\). It is surjective
  since \(\Hom_\C(V,W)\) is isomorphic to \(\dim W \times \dim V\)
  matrices, so \(\dim \Hom_\C(V,W) = \dim V \cdot \dim W = \dim V
  \otimes W = \dim V^* \otimes W\). 
\end{proof}
% \begin{defn}
%   For an irreducible representation \(\rho \from G \to GL(V)\) of
%   \(G\), then for all \(g \in Z(G)\), \(\rho(g)\) commutes with all
%   \(\rho(h), h \in H\). Thus, \(\rho(g) = \lambda_g Id_V\) by \ref{schur-alg-closed} and
%   we call the map \(\chi_V \from Z(G) \to \C\) the \de{central
%     character} of \(V\).
% \end{defn}
% \begin{prop}
%   \(\chi_V \from Z(G) \to \C\) is a homomorphism.
% \end{prop}
\section{Character Theory}
Throughout this section, let \(G\) be a finite group and \(F =
\C\). We wish to extend the notion of the central character above to a
general character \(\chi_V \from G \to \C\) that will still carry
useful information about the group representation \(V\).
\begin{prop}\label{g-is-ss}
  If \(V\) is a \(G\)-module, then every \(g \in G\), viewed as a
  linear operator \(g_V \from V \to V\), is semisimple, and thus \(V\)
  has a basis of eigenvectors for \(g_V\).
\end{prop}
\begin{proof}
  Since \(G\) is a finite group, \(g\) is of finite order, so \(g_V^n
  = Id_V\). Thus, \(g_V^n = (D+N)^n\) for diagonalizable \(D\) and
  nilpotent \(N\) (by Jordan Canonical Form since \(\C\) is
  algebraically closed) such that \(DN = ND\). Thus, \[
    Id_V = g_V^n = (D+N)^n = D^n + nND^{n-1} + \binom{n}{2} N^2 D^{n-2} +
    \cdots + nN^{n-1}D + N^n 
  \]
  which thus tells us that \(N = 0\) and \(D^n = Id_V\) since \(Id_V\)
  has no nilpotent part.
\end{proof}
In general, it is quite annoying to compute the eigenvalues for each
\(g \in G\). However, we will see that it suffices to merely compute
their sum using the trace. 
\begin{defn}
  The \de{character} of group representation \(\rho \from G \to V\), denoted \(\chi_\rho = \chi^\rho\), is the function
  \(\chi_\rho \from G \to \bbk\) defined by \(\chi_\rho(g) =
  \tr(\rho(g))\).
\end{defn}
\begin{prop}
  For \(\bbk = \C\) and fixed group representation \(\rho\),
  \begin{enumerate}
  \item \(\chi(1) = \dim V\),
  \item \(\chi(g^{-1}) = \ov{\chi(g)}\) for all \(g \in G\),
  \item \(\chi(tst^{-1}) = \chi(s)\) for all \(s,t \in G\).
  \end{enumerate}
\end{prop}
\begin{proof}
  All these properties follow from standard properties of trace.
\end{proof}
\begin{prop}\label{dual-character}
  Let \(V\) be a group representation and let \(\chi_V\) be its
  character. Then, \[
    \ov{\chi_V(g)} = \chi_{V^*}(g)
  \]
\end{prop}
\begin{proof}
  Consider that, by \ref{def-dual-rep}, \(\chi_{V^*}(g) =
  \chi_V(g^{-1})\). However, since \(\chi_V(g)\) is just the sum of
  eigenvalues of \(g\) (which must be roots of unity by the proof of \ref{g-is-ss}), we have that \[
    \chi_{V^*}(g) = \chi_V(g^{-1}) = \sum \lambda_i^{-1} = \sum
    \ov{\lambda_i} = \ov{\sum \lambda_i} = \ov{\chi_V(g)}
  \]
\end{proof}
\begin{defn}
  A function \(f \from G \to \C\) is called a \de{class function}
  if it is constant on conjugacy classes. Note that
  \(\operatorname{Class}(G)\) is the \(\C\)-algebra of all class
  functions on \(G\) and \(\dim \operatorname{Class}(G) =\) the number
  of conjugacy classes in \(G\).
\end{defn}
\begin{prop}
  Given a representation of a finite group \(G\) and a representation
  \(\rho\), \(\chi_\rho \from G \to \C\)  is a class function. 
\end{prop}
\begin{proof}
  This follows from part (c) of the previous proposition.
\end{proof}
\begin{prop}
  Let \(\rho_i \from G \to GL(V_i)\), \(i = 1,2\) be representations
  of \(G\) with characters \(\chi_{\rho_i}\).
  \begin{enumerate}
  \item \(\chi_{\rho_1 \oplus \rho_2} = \chi_{\rho_1} + \chi_{\rho_2}\)
  \item \(\chi_{\rho_1 \otimes \rho_2} = \chi_{\rho_1} \cdot \chi_{\rho_2}\)
  \end{enumerate}
\end{prop}
\begin{proof}
  This follows from the definition of direct sum and tensor product of
  representations and the definition of the character.
\end{proof}
Based on the identity (a) above, we see that the characters of
irreducible representations will be of primary interest.
\section{Orthogonality Relations of Irreducible Characters}
Inspired by the proof of Maschke's theorem, we present the following
proposition.
\begin{prop}
  Let \(h \from V_1 \to V_2\) be a linear transformation and let 
  \(\rho^1 \from G \to GL(V_1), \rho^2 \from G \to GL(V_2)\) be
  irreducible representations of finite group \(G\). Consider \[
    h^0 := \frac{1}{|G|} \sum_{t \in G} \rho^2(t^{-1}) h \rho^1(t)
  \]
  \(h^0\) is a homomorphism of representations. Furthermore
  \begin{enumerate}
  \item If \(\rho^1 \not \isom \rho^2\), then \(h^0 = 0\).
  \item If \(\rho^1 \isom \rho^2\) (and thus \(V_1 \isom V_2\)), then \(h^0 = \frac{1}{\dim V_1}
    \tr(h) Id_{V_1}\).
  \end{enumerate}
\end{prop}
\begin{proof}
  \todo{Fill this in}
\end{proof}
\begin{cor}
  Consider that \(\rho(t) \in GL(V)\) and thus we can represent
  \(\rho(t)\) as a \(n \times n\) matrix, say \(\rho^1(t) =
  [r_{i_1,j_1}(t)]\) and \(\rho^2(t) = [r_{i_2,j_2}(t)]\). Then,
  \begin{enumerate}
  \item If
  \(\rho^1 \isom \rho^2\), \[
    \frac{1}{|G|} \sum_{t \in G} r_{i_2,j_2}(t^{-1}) r_{j_1,i_1}(t) =
    \frac{1}{n} \delta_{j_1,j_2} \delta_{i_1,i_2}
  \]
\item If \(\rho^1 \not \isom \rho^2\), \[
  \frac{1}{|G|} \sum_{t \in G} r_{i_2,j_2}(t^{-1}) r_{j_1,i_1}(t) = 0
  \]
  \end{enumerate}
\end{cor}
This provides us motivation for definiting a positive definite
Hermitian inner product on the set of class functions on \(G\).
\begin{defn}
  For class functions \(\phi, \psi \from G \to \C\), we define inner
  product \[ 
    \langle \phi, \psi \rangle = \frac{1}{|G|} \sum_{g \in G}
    \phi(g^{-1}) \psi(g)
  \]
  and also \[
    (\phi,\psi) = \frac{1}{|G|} \sum_{g \in G} \phi(g) \ov{\psi(g)}
  \]
\end{defn}
\begin{prop}
  Let \(\langle \cdot, \cdot \rangle\) and \((\cdot, \cdot)\) be as
  above.
  \begin{enumerate}
  \item \(\langle \cdot, \cdot \rangle\) is a bilinear form.
  \item \(\langle \phi, \psi \rangle = \langle \psi, \phi \rangle\),
    that is \(\langle \cdot, \cdot \rangle\) is symmetric.
  \item \((\phi, \psi) = \langle \phi, \ov{\psi \circ i} \rangle\)
    where \(i \from G \to G\) takes \(i(g) = g^{-1}\).
  \item \((\cdot, \cdot)\) is a positive definite Hermitian inner form.
  \end{enumerate}
\end{prop}
\begin{prop}
  If \(\rho\) is a representation of \(G\) in \(V\) and if \(\psi(t) =
  \tr \rho(t)\), then \[
    \langle \phi, \psi \rangle = (\phi, \psi)
  \]
\end{prop}
Thus, it is with the form \((\cdot, \cdot)\) that we can arise at the
following
\begin{thm}\label{orthonormal}
  For any representations \(V,W\), \[
    (\chi_V, \chi_W) = \dim \Hom_G(W,V) 
  \]
  and, if \(V,W\) are irreducible, \[
    (\chi_V, \chi_W) =
    \begin{cases}
      1 & V \isom W \\
      0 & V \not \isom W
    \end{cases}
  \]
\end{thm}
\begin{proof}
  We first compute for arbitrary representations \(V,W\),
  \begin{align*}
    (\chi_V, \chi_W) & = \frac{1}{|G|} \sum_{g \in G} \chi_V(g)
                       \ov{\chi_W(g)} \\
                     & = \frac{1}{|G|} \sum_{g \in G} \chi_V(g)
                       \chi_{W^*}(g) & \text{by \ref{dual-character}}\\
                     & = \frac{1}{|G|} \sum_{g \in G} \chi_{V \otimes
                       W^*}(g) \\
                     & = \frac{1}{|G|} \sum_{g \in G}
                       \chi_{\Hom_\C(W,V)}(g) & \text{by \ref{tensor-hom-relation}}\\
                     & = \tr_{\Hom_\C(W,V)}\left(\frac{1}{|G|}\sum_{g \in G} g\right)
  \end{align*}
  However, \(P := \frac{1}{|G|} \sum_{g \in G} g\) is a homomorphism of
  \(G\) representations and the image of \(P\) is
  stable under any action of \(G\). Thus, for arbitrary representation
  \(U\), \(\tr_U(P)\) simply counts
  the number of times the trivial representation occurs as a
  subrepresentation of \(U\). However, considering \(\Hom_\C(W,V)\) as
  a \(G\)-representation, the \(G\)-invariant subrepresentation is
  exactly \(\Hom_G(W,V)\). Thus, \[
   (\chi_V, \chi_W) = \tr_{\Hom_\C(W,V)}\left(\frac{1}{|G|}\sum_{g \in G} g\right) = \dim \Hom_G(W,V)
  \]
  Furthermore, if \(V,W\) are irreducible, Schur's lemma tells us
  that \[
    \dim \Hom_G(W,V) =
    \begin{cases}
      1 & V \isom W \\
      0 & \text{ else }
    \end{cases}
  \]
\end{proof}
Thus, we have shown that the irreducible characters of a finite group
are orthogonal to each other under \((\cdot, \cdot)\). In fact, we
have the following theorem
\begin{thm}
  The irreducible characters of \(G\), \(\{\chi_1, \ldots, \chi_k\}\) form an orthonormal basis for
  the set of class functions on \(G\). 
\end{thm}
\begin{lem}
  Let \(\rho \from G \to GL(V)\) with \(\dim V = n\) be an irreducible
  representation of \(G\) and let \(f\) 
  be a class function on \(G\). Then, given \[
    \rho_f \from V \to V, \rho_f := \sum_{g \in G} f(g)\rho(g)
  \]
  we have \(\rho_f = \frac{1}{n}|G| (f, \ov{\chi_\rho}) Id_V\).
\end{lem}
\begin{proof}[Proof of Lemma]
  Let \(h \in G\). Then,
  \begin{align*}
    \rho_f \rho(h) & = \sum_{g \in G} f(g) \rho(g) \rho(h) \\
                   & = \sum_{g \in G} f(g) \rho(gh) \\
                   & = \sum_{g \in G} \rho(h) \rho(h^{-1}) f(g) \rho(gh) \\
                   & = \rho(h) \sum_{g \in G} f(g) \rho(h^{-1}gh) \\
                   & = \rho(h) \sum_{g \in G} f(h^{-1}gh) \rho(h^{-1}gh) \\
    & = \rho(h) \rho_f
  \end{align*}
  Since \(V\) is irreducible, it must be that \(\rho_f = \lambda 1_V\)
  by Schur's lemma. Thus, taking the trace of both sides, we get \[
    n \lambda = \tr(\rho_f) = \sum_{g \in G} f(g) \chi_\rho(g)
  \]
  and thus \[
    \lambda = \frac{1}{n} |G| (f, \ov{\chi})
  \]
\end{proof}
\begin{proof}[Proof of Theorem]
  By the theorem above, we know that the irreducible characters form
  an orthonormal system. Thus, we need only show that they generate
  \(\Class(G)\). It will suffice to show that if \(f \in \Class(G)\)
  has \((f,\ov{\chi_i}) = 0\) for all irreducible characters
  \(\chi_i\), then \(f \identically 0\). Assume \((f,\ov{\chi_i}) =
  0\). Then, by the lemma above, \(\rho_f = 0\) for any irreducible
  representation \(\rho\). Thus, from the direct sum decomposition, we
  can conlude that \(\rho_f = 0\) for any \(\rho\). So, applying this
  to the regular representation, \(\rho_{reg}\), and computing the
  image of a basis vector \(e_1\), we get \[
    0 = (\rho_{reg})_f e_1 = \sum_{g \in G} f(g) \rho_{reg}(g) e_1 = \sum_{g
      \in G} f(g) e_g \implies f(g) = 0, \forall g \in G. 
  \]
  Thus, \(f = 0\).
\end{proof}
\begin{cor}
  The number of irreducible
  representations of \(G\) is equal to the number of
  conjugacy classes of \(G\).
\end{cor}
\begin{rmk}
  We also know this from Wedderburn's theorem + Maschke's theorem.
\end{rmk}
\begin{proof}
  Let \(g \in G\) and \(f_g \in \Class(G)\) be defined by \[
    f_g(h) :=
    \begin{cases}
      1 & \text{ if }g \text{ and }h\text{ are conjugate} \\
      0 & \text{ otherwise}
    \end{cases}
  \]
  These \(f_g\) functions form a natural basis for \(\Class(G)\) when
  one is chosen for each conjugacy class in \(G\). Thus, \(\dim
  \Class(G)\) is the number of conjugacy classes of \(G\). On the
  other hand, using the theorem above, \(\dim \Class(G)\) is equal to
  the number of irreducible representations of \(G\). Thus, these two
  quantities are equal.
\end{proof}
\begin{thm}
  Let \(g \in G\) and \(C(g)\) be the number of elements in the
  conjugacy class of \(g\). Then,
  \begin{enumerate}
  \item \(\sum_{i=1}^d \ov{\chi_i}(g) \chi_i(g) = \frac{|G|}{C(g)}\)
  \item \(\sum_{i=1}^d \ov{\chi_i}(g) \chi_i(h) = 0\) if \(g\) is not
    conjugate to \(h\).
  \end{enumerate}
\end{thm}
\begin{proof}
  Using \(f_g \in \Class(G)\) as above, write \[
    f_g = \sum_{i=1}^k \lambda_i \chi_i, \lambda_i \in \C
  \]
  Then, \(\lambda_i = (f_g, \chi_i) = \frac{C(g)}{|G|}
  \ov{\chi_i}(g)\) where \(C(g)\) is the number of elements in the
  conjugacy class of \(g\). Therefore, \[
    f_g(h) = \frac{C(g)}{|G|} \sum_{i=1}^k \ov{\chi_i(g)} \chi_i(h)
  \]
  Thus, by definition of \(f_g(h)\), the identities in the theorem are proved.
\end{proof}
\begin{example}
  Let \(G = \Sym_3\). We know that \(\Sym_3\) has \(3\) conjugacy
  classes given by cycle type. Thus, \(\Sym_3\) has \(3\) distinct
  irreducible characters, denoted \(\chi_1, \chi_2\), and \(\theta\)
  for trivial, alternating, and standard representations. We then get
  the following ``character table'' \\
  \begin{center}
    \begin{tabular}{c|ccc}
      &(1)&(12)&(123) \\
      \hline
      \(\chi_1\) &1&1&1 \\
      \(\chi_2\) &1&-1&1\\
      \(\theta\) &2&0&-1
    \end{tabular}
  \end{center}
  where the first \(2\) rows follow from just computing the trace of
  the trivial and alternating representations. The third row then
  follows from our orthogonality relations, since
  \begin{align*}
    \theta(1) & = 2 \\
    0 = (\chi_1,\theta) & = \frac{1}{6} \sum_{g \in \Sym_3}
    \ov{\theta(g)} \\
    0 = (\chi_2, \theta) & = \frac{1}{6} \sum_{g \in \Sym_3} \sgn(g)
    \ov{\theta(g)} \\
    \implies 0 & = \frac{1}{3}\left( 2 + \theta((123)) + \theta((321))
    \right) \\
    \implies \theta((123)) & = -1
  \end{align*}
  and \[
    0 = \chi_1((1)) \chi_1((12)) + \chi_2((1)) \chi_2((12)) +
    \theta((1)) \theta((12)) = 1 - 1 + 2 \theta((12)) \implies
    \theta((12)) = 0
  \]
  Finally, returning to the regular representation of \(\Sym_3\), we
  know that \(\chi_{reg}(1) = 6\) and \(\chi_{reg} = a \chi_1 + b
  \chi_2 + c \theta\). However, we also know that \(\chi_{reg}(g) =
  0\) for all non-identity \(g \in \Sym_3\) since no such elements fix
  any basis elements. In particular,
  \(\chi_{reg}((123)) = 0\). Thus, the unique
  decomposition of \(\chi_{reg}\) is \[
    \chi_{reg} = \chi_1 + \chi_2 + 2 \theta
  \]
  From this, we will see that \[
    \rho_{reg} = \rho_{trivial} \oplus \rho_{alt} \oplus \rho_{std}
    \oplus \rho_{std} \isom M_1(\C) \oplus M_1(\C) \oplus M_2(\C)
  \]
\end{example}
\section{Results of Character Theory}
Perhaps the most important reason to study character theory is the
following theorem.
\begin{cor}
  The multiplicity of an irreducible representation \(W\) in a
  representation \(V\) is \(\chi_W, \chi_V\). 
\end{cor}
\begin{proof}
  Let \(V \isom W_1^{m_1} \oplus \cdots \oplus W_k^{m_k}\) be a
  decomposition of \(V\) into irreducibles. Then, we have \[
    \chi_V = \chi_{W_1^{m_1} \oplus \cdots \oplus W_k^{m_k}} = m_1
    \chi_{W_1} + \cdots + m_k \chi_{W_k}.
  \]
  Moreover, since the irreducible characters form an \emph{orthonormal
  basis} of class functions, we can apply \((\cdot, \chi_{W_i})\) to
  our equality to get \[
    (\chi_V, \chi_{W_i}) = m_i
  \]
\end{proof}
\begin{cor}
  Two finite dimensional complex representations of a finite group
  \(G\), say \(V,W\), are isomorphic if and only if they have the same
  character. In other words \[
    V \isom W \iff \chi_V = \chi_W
  \]
\end{cor}
\begin{proof}
  The forward direction follows by definition of character. The
  backwards direction follows from the fact that the irreducible characters form a
  basis of \(\Class(G)\). Namely, \(\chi_V = \chi_W\) implies that
  \(\chi_V\) and \(\chi_W\) have the same unique decomposition into
  irreducible characters, and thus from the corollary above, \(V,W\) have the
  same multiplicity of each irreducible representation.
\end{proof}
\begin{cor}
  A representation \(V\) is irreducible if and only if \((\chi_V,
  \chi_V) = 1\). 
\end{cor}
\begin{proof}
  The forward direction was proved earlier in \ref{orthonormal}. For the
  backwards direction, if \(\chi_V = m_1 \chi_{V_1} + \cdots + m_k
  \chi_{V_k}\) for irreducibles \(V_1, \ldots, V_k\), then using the
  fact that the \(\chi_{V_i}\)'s form an orthonormal basis, we get \[
    1 = (\chi_V, \chi_V) = m_1^2 + \cdots + m_k^2
  \]
  Thus, since each \(m_i^2 \geq 0\) and are in \(\Z\), it must be that
  exactly one \(m_i = 1\) and all others are \(0\). Thus, \(V\) must
  be irreducible by the corollary above.
\end{proof}
\section{The Regular Representation Revisited}
\begin{cor}
  The multiplicity of any irreducible representation \(V\) in the regular
  representation equals its dimension.
\end{cor}
\begin{proof}
  Recall that the dimension of the regular representation is \(|G|\). Consider \[
    (\chi_V, \chi_{reg}) = \frac{1}{|G|}\chi_V(1) \chi_{reg}(1) =
    \chi_V(1) = \dim V
  \]
\end{proof}
\begin{cor}
  Let \(\{V_1, \ldots, V_k\}\) be the set of all irreducible
  representations of finite group \(G\). Then, \[
    |G| = \sum_{i=1}^k (\dim V_i)^2
  \]
\end{cor}
\begin{proof}
  From above, we have \(\chi_{reg} = \sum_{i=1}^k \dim V_i \cdot
  \chi_{V_i}\), and so we simply take \((\chi_{reg}, \chi_{reg})\) to
  get \[
    (\chi_{reg}, \chi_{reg}) = \sum_{1\leq i \leq k, 1 \leq j \leq k}
    \dim V_i \dim V_j (\chi_{V_i}, \chi_{V_j}) = \sum_{i=1}^k (\dim
    V_i)^2
  \]
  using the orthogonality of the irreducible characters. However,
  since \[
    \chi_{reg}(g) =
    \begin{cases}
      |G| & g = 1 \\
      0 & \text{otherwise}
    \end{cases}
  \]
  we also have \((\chi_{reg}, \chi_{reg}) = \frac{|G|^2}{|G|} = |G|\)
  by definition of the inner product \((\cdot, \cdot)\).
\end{proof}
\begin{rmk}
  This result should not come as a surprise to a reader familiar with
  Artin-Wedderburn theory. However, the question of what the
  analogous decomposition of the regular representation looks like
  still remains. 
\end{rmk}
\begin{lem}
  If \(\rho_1\) and \(\rho_2\) are irreducible representations of
  \(G_1\) and \(G_2\) respectively, then
  \(\rho_1 \otimes \rho_2\) is an irreducible representation of \(G_1
  \times G_2\)
\end{lem}
\begin{proof}
  We have, from above, that for characters \(\chi_1\) of \(\rho_1\)
  and \(\chi_2\) of \(\rho_2\), that \[
    (\chi_1, \chi_1) = 1 = (\chi_2,\chi_2)
  \]
  However, \(\chi := \chi_{\rho_1 \otimes \rho_2}\) has
  \begin{align*}
    (\chi,\chi) & = 
    \frac{1}{|G_1||G_2|} \sum_{g \in G_1, g' \in G_2} \chi_1(g)
                  \chi_2(g') \ov{\chi_1(g) \chi_2(g')} \\
    & = \frac{1}{|G_1|} \left(
      \sum_{g \in G_1} \chi_1(g) \ov{\chi_1(g)} \right)
      \frac{1}{|G_2|} \left( 
      \sum_{g' \in G_2} \chi_2(g') \ov{\chi_2(g')} \right) \\
    & = (\chi_1, \chi_1) \cdot (\chi_2, \chi_2) \\
    & = 1
  \end{align*}
  Thus, it must be that \(\rho_1 \otimes \rho_2\) is irreducible.
\end{proof}
\begin{rmk}
  The converse, that each irreducible \(G_1 \times
  G_2\)-representation is isomorphic to a tensor product of
  \(G_i\)-representations is also true. The argument is to show that
  each class function of \(G_1 \times G_2\) which is orthogonal to
  characters of the form \(\chi_1 \cdot \chi_2\) is zero.
\end{rmk}
\begin{thm}
  The regular representation decomposes as a direct sum of \(G \times
  G\) representations. More specifically, \[
    \C[G] \isom \bigoplus_{i=1}^k V_i \otimes V_i^*
  \]
  for irreducible \(G\)-representations \(\{V_1, \ldots, V_k\}\).
\end{thm}
\begin{rmk}
  Thus, in a sense, we have arrived at the Artin-Wedderburn theorem in
  different language. 
\end{rmk}
\section{Some Applications of Representation Theory: Frobenius
  Divisibility and Burnside}
We now seek to discuss some useful applications of representation
theory that would be much more difficult to prove using only group
theory. First, we will have to introduce the terminology of algebraic
numbers.
\begin{defn}
  We say \(z \in R\), a commutative ring, is an \de{algebraic integer}
  or \de{integral over \(\Z\)} if \(z\) is a root of a monic
  polynomial with coefficients in \(\Z\).
\end{defn}
\begin{rmk}
  With this definition, ``integer'' is a somewhat misleading word. For
  instance, if \(R = \C\), then \(i \in \C\) is an algebraic integer
  since it is a solution to the polynomial \(x^2+1\).
\end{rmk}
\begin{lem}
  The set of all algebraic integers is a ring.
\end{lem}
\begin{proof}
  Let \(\alpha\) be an eigenvalue of some \(A \in M_n(R)\) with
  eigenvector \(v\) and
  \(\beta\) be an eigenvalue of some \(B \in M_m(R)\) with eigenvector
  \(w\). Then, \(\alpha \pm \beta\) is an eigenvalue of \[
    A \otimes Id_m \pm Id_n \otimes B
  \]
  and \(\alpha \beta\) is an eigenvalue of \(A \otimes B\). In both
  cases, the appropriate eigenvector is \(v \otimes w\). Thus, the
  lemma is proven.
\end{proof}
\begin{prop}
  Let \(z\) be an element of a commutative ring \(R\). The following
  properties are equivalent.
  \begin{enumerate}
  \item \(z\) is integral over \(\Z\).
  \item The subring \(\Z[z] \subset R\) generated by \(z\) is finitely
    generated as a \(\Z\)-module.
  \item There exists a finitely generated sub-module of \(R\)
    considered as a \(\Z\)-module containing \(\Z[z]\)
  \end{enumerate}
\end{prop}
\begin{prop}\label{characters-are-algebraic-integers}
  Let \(\chi\) be the character of a representation \(\rho\) of a finite group
  \(G\). Then, \(\chi(g)\) is an algebraic integer for each \(g \in G\).
\end{prop}
\begin{proof}
  \(\chi(g) = \tr(\rho(g))\), so it is a sum of eigenvalues of
  \(\rho(g)\). However, all the eigenvalues are roots of unity, which
  are clearly algebraic integers. Thus, since the algebraic integers
  form a ring, this sum is an algebraic integer.
\end{proof}
\begin{thm}
  Given a complex irreducible representation \(V\) of \(G\), \(\dim V
  \divides |G|\). 
\end{thm}
\begin{proof}
  Let \(C_1, \ldots, C_k\) be the conjugacy classes of \(G\) and set \[
    \lambda_i := \chi_V(C_i) \frac{|C_i|}{\dim V}
  \]
  where \(\chi_V(C_i) = \chi_V(g)\) for some \(g \in C_i\) (recall
  that \(\chi_V\) is a class function). Then, all of these
  \(\lambda_i\) are algebraic integers. To see this, consider \[
    P_i = \sum_{h \in C_i} h
  \]
  which is central in \(\Z[G]\) by a quick computation. Thus, by
  Schur's lemma, \(P_i = \mu_i Id_V\) since \(V\) is irreducible. Furthermore,
  \(\Z[G]\) is finitely generated, so \(\mu\) is an algebraic
  integer. Thus, we compute \[
    \tr(P_i) = |C_i| \chi_V(C_i) = \mu_i \dim V \implies \mu_i =
    \frac{|C_i| \chi_V(C_i)}{\dim V} = \lambda_i
  \]
  Now, to finish our proof, we must show that \[
    \sum_i \lambda_i \ov{\chi_V(C_i)}
  \]
  is an algebraic integer. However, we already know that the algebraic
  integers form a ring and, by above, each \(\lambda_i\) is an
  algebraic integer. Moreover, \(\chi_V(C_i)\) is an algebraic integer
  by \ref{characters-are-algebraic-integers}. Thus, the sum is an
  algebraic integer by the ring structure on the algebraic integers.

  Now, we note, by definition of \(\lambda_i\), so we get the
  following
  \begin{align*}
    \sum_{i} \lambda_i \ov{\chi_V(C_i)}
    & = \frac{1}{\dim V}\sum_i |C_i| \chi_V(C_i) \ov{\chi_V(C_i)} \\
    & = \frac{1}{\dim V}\sum_{g \in G} \chi_V(g) \ov{\chi_V(g)}
    & \text{since characters are class functions} \\
    & = \frac{1}{\dim V} |G| (\chi_V,\chi_V) & \text{by definition of
                                               }(\cdot, \cdot)\\
    & = \frac{|G|}{\dim V} & \text{since }V\text{ is irreducible, so
                             }(\chi_V, \chi_V) = 1    
  \end{align*}
  Thus, it must be that \(\frac{|G|}{\dim V}\) is an algebraic
  integer. However, \(\frac{|G|}{\dim V} \in \Q\), so by Gauss's
  lemma, \(\frac{|G|}{\dim V} \in \Z\).
\end{proof}
While this result is useful in and of itself, we have something
better, namely 
\begin{thm}[Frobenius Divisibility]
  Given an irreducible representation \(V\) of \(G\), \[
    \dim V  \divides \frac{|G|}{|Z(G)|}.
  \]
\end{thm}
\begin{proof}
  
\end{proof}
\section{Induced Representations}

\begin{bibdiv}
  \begin{biblist}
    \bib{cr}{book}{
      author={Curtis, Charles W.}
      author={Reiner, Irving}
      title={Representation Theory of Finite Groups and Associative
        Algebras}
      year={1962}
    }
    \bib{etingof}{article}{
      author={Etingof, Pavel}
      author={Golberg, Oleg}
      author={Hensel, Sebastian}
      author={Liu, Tiankai}
      author={Schwendner, Alex}
      author={Vaintrob, Dmitry}
      author={Yudovina, Elena}
      title={Introduction to Representation Theory}
      year={2011}
      note={\url{http://math.mit.edu/~etingof/replect.pdf}}
    }
    \bib{princeton-companion}{article}{
      author={Gronjnowski, Ian}
      title={Representation Theory}
      journal={The Princeton Companion to Mathematics}
      pages={419--431}
    }
    \bib{aw}{article}{
      author={Seelinger, George H.}
      title={Artin-Wedderburn Theory}
      year={2017}
      note={See \url{https://github.com/ghseeli/grad-school-writings/releases/latest}}
    }
    \bib{serre}{book}{
      author={Serre, Jean-Pierre}
      title={Linear Representations of Finite Groups}
      year={1997}
      note={Translated from the French by Leonard L. Scott}
    }
    \bib{smith}{article}{
      author={Smith, Karen}
      title={Groups and their Representations}
      year={2010}
      note={\url{http://www.math.lsa.umich.edu/~kesmith/rep.pdf}}
    }
  \end{biblist}
\end{bibdiv}

\end{document}