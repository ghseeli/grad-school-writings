\documentclass[11pt,leqno,oneside]{amsbook}
\usepackage{tikz}
\usetikzlibrary{cd}
\usepackage{bbm}
\usepackage{ytableau}
\usepackage{todonotes}

\usepackage{../notes}
\usepackage{../../ReAdTeX/readtex-core}
\usepackage{../../ReAdTeX/readtex-abstract-algebra}

\newcommand{\bbk}{\mathbbm{k}}
\newcommand{\Class}{\operatorname{Class}}
\newcommand{\Res}{\operatorname{Res}}
\newcommand{\Ind}{\operatorname{Ind}}
\newcommand{\bs}{\textbackslash}
\newcommand{\partitionof}{\vdash}
\newcommand{\T}{\mathsf{T}}

\numberwithin{thm}{section}

\title[Representation Theory of Finite Groups]{Representation Theory
  of Finite Groups \\ Notes
  inspired by a class taught by Brian Parshall in Fall 2017}
\author{George H. Seelinger}
\date{Fall 2017}
\begin{document}
\maketitle
\section{Introduction}
From one perspective, the representation theory of finite groups is merely a
special case of the representation theory of associative algebras by
considering the fact that a sufficiently nice finite group ring \(\bbk[G]\)
is semisimple, and thus the results from Artin-Wedderburn
theory apply. However, the extra structure of the group provides a
rich connection between \(\bbk[G]\)-modules and linear algebra.

A representation of a finite group is a way to induce an action of a
finite group on a vector space. Sometimes, such a perspective allows
mathematicians to see structure or symmetries in groups that may not
have been readily apparent from a purely group theoretic point of
view, much like modules can provide insight into the structure of
rings. In fact, a reprsentation of a finite group is fundamentally the
same as a module of a finite group algebra, but the representation
theoretic perspective allows us to leverage tools from linear algebra
to gain additional insights.

While the classic texts in representation theory continue to be cited
and used often, and contain many details ommitted here, I am eternally
grateful for \cite{etingof} and \cite{smith} for being approachable,
providing simplifying insights, and using more modern language to
rephrase classical results. I am also grateful for \cite{ct}, out of
which I first learned the representation theory of finite groups, even
if I did not understand a lot of it at the time. Much of the writings
here are shamelessly  borrowed and synthesized from these sources.
\section{Group Actions}
The following exposition is a summary of \cite{princeton-companion}. The reader is probably familiar with the notion of a group action.
\begin{defn}
  A \de{group action} on a set \(X\) is a homomorphism \(\phi \from G \to \Aut(X)\)
  such that \(\phi(e) = Id_X\).
\end{defn}
If the reader is unfamiliar with group actions, most introductory
texts on group theory will provide more than adequate treatment.
\begin{defn}
  A group action is called \de{faithful} if \(\phi\) is injective. A
  group action is called \de{transitive} if \(\phi(G)\) induces only
  one orbit on \(X\).
\end{defn}
When an action is not transitive, we have multiple orbits and we can
consider the group action on each orbit separately. Thus, in a sense,
we can ``decompose'' this group action (on the level of sets). Thus,
let us focus on transitive actions as our basic building blocks of
group actions.
\begin{thm}
  Let \(\phi\) be a transitive group action of \(G\) on a set \(X\). Then,
  consider \(H = \Stab_G(x)\). Then, \(G/H \isom X\) as sets with a
  (left) \(G\)-action.
\end{thm}
\begin{proof}
  Given \(H = \Stab_G(x)\), consider the correspondance \[
    gH \correspondsto g.x = \phi(g)x
  \]
  Then, the action of \(G\) preserves this correspondance, namely \[
    g'.gH = (g'g)H \correspondsto (g'g).x = \phi(g'g)x = \phi(g')\phi(g)x
    = g'.\phi(g)x = g'.(g.x)
  \]
\end{proof}
Thus, we have reducted the classification of transitive \(G\)-actions
on \(X\) to the study of conjugacy classes of subgroups of
\(G\). Thus, the structure of a group that controls a group action on
a set \(X\) is the subgroup structure of \(G\). However, understanding
the subgroup structure of a group is, in general, incredibly
difficult. For instance, recall Cayley's theorem.
\begin{thm}[Cayley's Theorem]
  Any finite group \(G\) with \(|G| = n < \infty\) can be embedded
  into \(\Sym_n\) via the action of \(G\) on itself. 
\end{thm}
Thus, to understand the subgroups of \(\Sym_n\), one must understand all
finite groups \(G\) with order less than \(n\). Thus, while group
actions are incredibly useful and general, they lack structure that
can be otherwise useful.
\section{Group Algebras}
\begin{defn}
  The \de{group ring} or \de{group algebra} of a group \(G\) over a ring \(R\), denoted
  \(R[G]\) is a free \(R\)-module over the set \[
    \{e_g \st g \in G\}
  \]
  with addition being formal sums. Additionally, \(R[G]\) has the
  structure of a 
  ring where multiplication is given by \(e_g
  \cdot e_h = e_{gh}\) for \(g,h \in G\). Thus, \(1_{R[G]} = e_{1_G}\)
\end{defn}
Typically, we will take \(R\) to be a field, usually
\(\C\). Furthermore, we will often replace \(e_g\) by \(g\) when it is
understood that we are working in the group algebra. That is, for
\(\lambda \in R\), we may
rewrite \[
  \lambda e_g \to \lambda g 
\]
Since group algebras are \(R\)-algebras, we can ask many mathematical
questions about their modules.
\begin{thm}[Maschke's Theorem]\label{maschke}
  Let \(G\) be a finite group and \(\bbk\) be a field such that
  \(\Char \bbk \notdivides |G|\). Then, \(\bbk[G]\) is completely
  reducible as a module over itself. 
\end{thm}
\begin{proof}
  Let \(V\) be a proper \(\bbk[G]\)-submodule of \(\bbk[G]\). Consider the
  \(\bbk\)-linear map
  map \(\pi \from \bbk[G] \to B\) and let \(\phi \from \bbk[G] \to V\)
  be given by \[
    \phi(x) = \frac{1}{|G|} \sum_{s \in G} s.\pi(s^{-1}.x)
  \]
  Thus, \(\phi\) is also a projection since, for \(v \in V\), \[
    \phi(v) = \frac{1}{|G|} \sum_{s \in G} s.\pi(s^{-1}.v) =
    \frac{1}{|G|} \sum_{s \in G} s.s^{-1}.v = v
  \]
  and it is \(\bbk\)-linear since it is simply the (appropriately scaled) sum of \(\bbk\)
  linear maps. In fact,
  \begin{align*}
    \phi(t.x) & = \frac{1}{|G|} \sum_{s \in G} s.\pi(s^{-1}.t.x) \\
              & = \frac{1}{|G|} \sum_{s' \in G} (ts') \cdot \pi(s'^{-1}.x)
              & (s'^{-1} = s^{-1}t \implies s = ts') \\
              & = t.\phi(x)
  \end{align*}
  so, \(\phi\) is \(\bbk[G]\)-linear. Thus, \(\bbk[G] \isom V \oplus
  \ker \phi\), but \(V\) was an arbitrary proper submodule. Thus,
  \(\bbk[G]\) is completely reducible.
\end{proof}
\begin{rmk}
  The converse of Maschke's theorem is also true. 
\end{rmk}
Thus, since \(\bbk[G]\) is finite-dimensional and thus also artinian,
\(\bbk[G]\) is a semisimple ring when \(\Char \bbk \notdivides |G|\)
and, using artin-wedderburn theory, 
has decomposition as an module 
over itself \[ 
  \bbk[G] \isom \bigoplus_{i=1}^k M_{n_i}(D_i)
\]
for \(n_i \in \N\) and \(D_i\) a division ring, and these direct
summands represent all the irreducible \(\bbk[G]\)-modules. However,
we still wish to understand what these division rings and \(n_i\)'s
actually are (see \cite{aw}). Thus, we will shift our perspective to looking at
\emph{group representations} instead of group algebra modules.
\section{Representations}
In this section, we seek to establish some of the basic ideas of
representations of finite groups. Much of this exposition is borrowed
from \cite{etingof} where the results are presented more generally for
\(R\)-algebras. Also, many of these results can be rephrased using the
language in \cite{aw}.
\begin{defn}
  A \de{representation of a finite group \(G\) in a (complex) vector space \(V\)} is a
  homomorphism \(\rho \from G \to GL(V)\).
\end{defn}
\begin{rmk}
  Throughout this monograph, we will often simply say
  ``representation'' to mean a representation of a finite group. When
  we refer to a vector space \(V\), we will mean a vector space over \(\C\).
\end{rmk}
\begin{thm}
  There is a natural bijection between \(\C[G]\)-modules and complex
  representations of \(G\).
\end{thm}
\begin{proof}
  Given a representation of \(G\) given by \(\rho \from G \to GL(V)\),
  we get a \(\C[G]\) action on \(v \in V\) via \[
    e_g.v = \rho(g)v
  \]
  and extend via linearity. Conversely, given \(M\) a
  \(\C[G]\)-module, we can consider \(M\) as 
  a vector space over \(\C e_1\) and then view the action of each
  \(e_g\) as an invertible linear transformation on this vector space.
\end{proof}
\begin{defn}
  We say two representations \(\rho, \rho' \from G \to GL(V)\) are
  \de{similar} or \de{isomorphic} if 
  there is a linear transformation \(\tau \from V \to V'\) such
  that \[
    \tau \circ \rho(g) = \rho'(g) \circ \tau
  \]
  for all \(g \in G\).
\end{defn}
\begin{defn}
  We say that a subspace \(W \subset V\) is \de{\(G\)-stable} if
  \(\rho(g)W \subset W\) for all \(g \in G\).
\end{defn}
\begin{prop}
  Let \(\rho \from G \to GL(V)\) be a linear representation of \(G\)
  in \(V\) and let \(W\) be a subspace of \(V\) such that \(\rho(g)W =
  W\) for all \(g \in G\). Then, there exists a complement \(W'\) of
  \(W\) in \(V\) such that \(\rho(g)W' \subset W'\) for all \(g \in G\)
\end{prop}
\begin{proof}
  Let \(W'\) be a vector space complement of \(W\) in \(V\) (not
  necessarily \(G\)-stable) and let \(p \from V \to W\) be the
  standard projection. Then, consider the map \[
    p^0 := \frac{1}{|G|} \sum_{t \in G} \rho(t) \circ p \circ \rho(t)^{-1}
  \]
  Such a map is a projection of \(V\) onto a subspace of \(W\). In
  fact, \(p^0|_W = Id_W\) since, for \(w \in W\), \[
    p \circ \rho(t)^{-1} w = \rho(t)^{-1} w \implies \rho(t) \circ p
    \circ \rho(t)^{-1} w = w \implies p^0 w = w
  \]
  Thus, we seek to show \(W^0 := \ker p^0\) is stable under the \(G\)
  action via \(\rho\). Indeed, it is easy to check \(\rho(s) p^0
  \rho(s)^{-1} = p^0\) and thus \(p^0 \circ \rho(s)x = \rho(s) \circ
  p^0 x = 0\), that is, \(\rho(s) x \in W^0\). Thus, \(V = W \oplus
  W^0\) is a decomposition into \(G\)-stable subspaces. 
\end{proof}
\begin{rmk}
  Note the similarity between this proof and the proof of Maschke's
  theorem above (\ref{maschke}). Indeed, these proofs are more or less
  equivalent and the proposition above is the same as Maschke's
  theorem for \(\bbk = \C\).
\end{rmk}
\begin{defn}
  Given representations \(\rho_1 \from G \to GL(V_1)\) and \(\rho_2 \from
  G \to GL(V_2)\), we define \(\rho := \rho_1 \oplus \rho_2 \from G \to GL(V_1
  \oplus V_2)\) by the map \[
    g \mapsto \left(
      \begin{array}{cc}
        \rho_1(g)&0\\
        0&\rho_2(g)
      \end{array}
    \right)
  \]
  That is, \(\rho(g)|_{V_i} = \rho_i(g)\).
\end{defn}
\begin{defn}
  We say that a representation \(\rho \from G \to GL(V)\) is
  \de{irreducible} or \de{simple} if \(V \neq 0\) and there is no
  subspace \(W
  \subset V\) such that \(W\) is stable under the action of \(G\) via
  \(\rho\). By the proposition above, this is equivalent to saying
  that \(\rho\) does not break into a direct sum of representations.
\end{defn}
\begin{prop}
  Every representation is a direct sum of irreducible representations.
\end{prop}
\begin{proof}
  The proof follows from induction on \(\dim V\) and application of
  the proposition above.
\end{proof}
\begin{defn}
  Let \(\rho_1 \from G \to GL(V_1)\) and \(\rho_2 \from G \to
  GL(V_2)\) be representations. Then, we define \(\rho := \rho_1
  \otimes \rho_2 \from G \to GL(V_1 \otimes V_2)\) by, for \(s \in G\), \[
    \rho(s)(v_1 \otimes v_2) = \rho_1(s)v_1 \otimes \rho_2(s) v_2
  \]
\end{defn}
\begin{rmk}
  The tensor product of two irreducible representations is not
  typically irreducible. 
\end{rmk}
\begin{thm}
  Let \(\rho \from G \to GL(V)\) be a representation. Then, given
  \(\rho \otimes \rho \to GL(V \otimes V)\) decomposes as \[
    V \otimes V \isom \frac{V \otimes V}{(x \otimes y - y \otimes x)} \oplus
    \frac{V \otimes V}{(x \otimes y + y \otimes x)} =
    \operatorname{Sym}^2(V) \oplus \operatorname{Alt}^2(V)
  \]
\end{thm}
\begin{proof}
  Consider the automorphism of \(V \otimes V\) given by \(\tau(e_i
  \otimes e_j) = e_j \otimes e_i\) for basis \(\{e_1, \ldots, e_n\}\)
  of \(V\). Then, \(\tau(v \otimes w) = w \otimes v\) for any \(v,w
  \in V\) and \(\tau^2 = Id_{V \otimes V}\). Thus, as vector spaces, \[
    V \otimes V \isom  \ker (\tau-Id) \oplus \im (\tau-Id) 
  \]
  However, \(\ker(\tau-Id) = \langle x \otimes y \st x \otimes y - y
  \otimes x = 0 \rangle \isom (V \otimes V)/(x \otimes y - y \otimes
  x)\). Now, consider that \[
    (\tau-Id)(x \otimes y) + \tau(\tau-Id)(x \otimes y) = (\tau-Id)(x
    \otimes y) + (Id - \tau)(x \otimes y) = 0
  \]
  and so, for any \(v \in \im(\tau-Id)\), we get \(v + \tau(v) =
  0\). Thus, \(\im(\tau-Id) \subset (V \otimes V)/(x \otimes y + y
  \otimes x)\). However, by dimension counting, we note that
  \begin{align*}
        \dim \ker(\tau-Id) = \dim \operatorname{Sym}^2(V) &= \frac{n(n+1)}{2} \implies \\ \dim
    \im(\tau-Id) = n - \frac{n(n+1)}{2} &= \frac{n(n-1)}{2} = \dim \operatorname{Alt}^2(V)
  \end{align*}
  Thus, \(\im(\tau-Id) = \operatorname{Alt}^2(V)\).

  However, we must also check that these spaces are stable under
  \(G\). However, for \(\rho' = \rho \otimes \rho\),
  \begin{align*}
    \rho'(g) \circ \tau \circ \rho'(g^{-1}) (v \otimes w)
    & = \rho'(g)
      \circ \tau (\rho^{-1}(g)v \otimes \rho^{-1}(g)w) \\
    & = \rho'(g)(\rho^{-1}(g)w \otimes \rho^{-1}(g)v) \\
    & = w \otimes v \\
    & = \tau(v \otimes w)
  \end{align*}
  and so the image and kernel of \(\tau-Id\) is stable under the
  action of \(G\).
\end{proof}
We now also seek to prove some other useful tools in representation
theory.
\begin{thm}[Schur's Lemma]
  Let \(V_1, V_2\) be representations of \(G\) and let \(\phi \from
  V_1 \to V_2\) be a nontrivial homomorphism of representations. Then,
  \begin{enumerate}
  \item If \(V_1\) is irreducible, then \(\phi\) is injective.
  \item If \(V_2\) is irreducible, then \(\phi\) is surjective.
  \end{enumerate}
\end{thm}
\begin{proof}
  Exercise for the reader.
\end{proof}
\begin{cor}[Schur's Lemma for algebraically closed fields]\label{schur-alg-closed}
  Let \(V\) be a finite dimensional irreducible representation of a
  group \(G\) over an algebraically closed field \(\bbk\), and \(\phi
  \from V \to V\) a commuting homomorphism (ie \(\rho(g) \circ \phi = \phi
  \circ \rho(g)\)). Then, \(\phi = \lambda Id_V\) for some \(\lambda
  \in \bbk\).
\end{cor}
\begin{rmk}
  \begin{enumerate}
  \item We sometimes call \(\phi\) a ``scalar operator'' or say that
    \(\phi\) ``acts as a scalar'' in this situation. In \cite{serre},
    Serre calls such an operator a ``homothey.''
  \item This proposition is false over \(\R\).
  \end{enumerate}
\end{rmk}
\begin{proof}
  Let \(\lambda\) be an eigenvalue of \(\phi\), which exists since
  \(\bbk\) is algebraically closed. Then, since \(\phi\) commutes with
  \(\rho(g)\), so does \(\phi - \lambda Id \from V \to V\). However, \(\phi -
  \lambda Id\) is not an isomorphism since \(\det(\phi - \lambda Id) =
  0\). Thus, \(\phi - \lambda Id = 0 \implies \phi = \lambda Id\).
\end{proof}
\begin{cor}
  Let \(G\) be an abelian group. Then, every irreducible finite
  dimensional representation of \(G\) is \(1\)-dimensional.
\end{cor}
\begin{proof}
  Let \(V\) be an irreducible finite dimensional representation of
  \(G\). Then, for \(g,h \in G\), \(v \in V\) \[
    \rho(g) \rho(h) v = \rho(gh)v = \rho(hg)v = \rho(h)\rho(g) v
  \]
  Thus, \(\rho(g)\) commutes with all \(\rho(h)\) and so, by the
  corollary above, \(\rho(g) = \lambda Id\) for some \(\lambda \in
  \C\). Thus, every vector subspace of \(V\) is a subrepresentation,
  but \(V\) is irreducible. Thus, \(\dim V = 1\).
\end{proof}
\begin{example}
  Let \(G=\Sym_3\), the smallest order non-abelian group, and let
  \(V\) be a complex representation of \(G\). Since \(\Sym_3\) is
  generated by \(\sigma = (123)\) and \(\tau = (12)\), we need only
  find subrepresentations of \(V\) that are stable under the actions
  of \(\sigma\) and \(\tau\). An important tool for this process will
  be to find eigenvectors. Let \(v\) be an eigenvector for
  \(\sigma\) with eigenvalue \(\lambda\). Then, consider \(\tau
  v\). Such a vector must be an 
  eigenvector for \(\sigma\) as well since \[
    \sigma \tau v = \tau \sigma^2 v = \tau \lambda^2 v = \lambda^2
    \tau v
  \]
  Thus, the space \(\Span \{v,\tau v\}\) is stable under the actions of
  \(\sigma,\tau\) and must be a subrepresentation of \(V\). \\

  Now, if we wish to find all irreducible representations of \(G\), we
  know that the dimension must be less than or equal to
  \(2\). However, we can figure out more. We know that \(\sigma^3 =
  Id\), and so it must be that \(v = \sigma^3 v = \lambda^3 v\) and so
  \(\lambda = 1, \zeta_3,\) or \(\zeta_3^2\), where \(\zeta_3\) is a
  primitive \(3\)rd root of unity.
  \begin{itemize}
  \item Let \(\lambda \neq 1\). Then, \(\lambda \neq \lambda^2\) and
    so \(v\) and \(\tau v\) have distinct eigenvalues and thus they
    are linearly independent. Thus, \(\Span\{v,\tau v\}\) defines an
    irreducible, \(2\)-dimensional representation. This is called the
    \de{standard representation}. Explicitly, it is given by \[
      \tau \mapsto \left(
        \begin{array}{cc}
          0&1\\
          1&0
        \end{array}
\right), \ \sigma \mapsto \left(
  \begin{array}{cc}
    \zeta_3&0\\
    0&\zeta_3^2
  \end{array}
\right)
    \]
  \item Let \(\lambda = 1\) and \(\tau v = -v\). Then, \(\Span\{v\}\)
    is an irreducible \(1\)-dimensional representation, but it is
    distinct from the trivial representation. It is called the
    \de{alternating representation}.
  \item Let \(\lambda = 1\) and \(\tau v = v\). This is the trivial
    representation. 
  \end{itemize}
  Thus, we have found \(3\) irreducible representations of
  \(\Sym_3\).
\end{example}
\begin{defn}
  We say a representation \(\rho \from G \to GL(V)\) is \de{faithful}
  if \(\ker \rho = \{1\}\).
\end{defn}
\begin{prop}\label{quotient-rep}
  Given a representation \(\rho \from G \to GL(V)\) and a normal
  subgroup \(H \normsubgroup G\) such that \(\rho\) acts trivially on
  \(H\), (that is, \(\rho(H) = \{Id_V\}\)), \(\rho\) can descend to a
  representation of \(G/H\), say \(\ov{\rho}\), via \[
    \ov{\rho}(gH) = \rho(g)
  \]
\end{prop}
\begin{proof}
  We must check that such an action is well defined. Let \(g,g' \in
  gH\). Then, \(g = hg'\) so \[
    \ov{\rho}(gH) = \rho(g) = \rho(hg') = \rho(h)\rho(g') = \rho(g') = \ov{\rho}(g'H)
  \]
\end{proof}
\begin{prop}
  Given representation \(\rho \from G \to GL(V)\), we have that
  \(\ov{\rho} \from G/\ker \rho \to GL(V)\) is faithful.
\end{prop}
\begin{prop}\label{lifted-rep}
  Let \(\rho \from G/H \to GL(V)\) be a representation of \(G/H\) for \(H
  \normsubgroup G\). Then, we can lift \(\rho\) to a representation of
  \(G\), say \(\tilde{\rho}\), via \[
    \tilde{\rho}(g) = \rho(gH)
  \]
  Furthermore, if \(\rho\) is a faitful representation of \(G/H\), then
  \(\ker \tilde{\rho} = H\).
\end{prop}
\begin{proof}
  This construction is immediately well defined. If \(\rho\) is
  faithful, then we note that \[
    \tilde{\rho}(g) = Id \iff \rho(gH) = Id \iff gH = H
  \]
\end{proof}
\section{The Regular Representation}
One of the most important representations is the regular
representation. Later, we will see that every irreducible
representation appears in the regular representation. Furthermore, the
regular representation is a somewhat natural representation to define.
\begin{defn}
  Let \(G\) be a finite group. Then, the \de{regular representation}
  of \(G\) is a homomorphism \(\rho_{reg} \from G \to GL(V)\) where \(V =
  \C^{|G|}\) given
  by \[
    g \mapsto (g \from V \to V, e_h \mapsto e_{gh}) 
  \]
\end{defn}
Note that the regular representation is equivalent to viewing
\(\bbk[G]\) as a module over itself.
\begin{example}
  Consider \(G = \Sym_3\). Then, the regular representation of
  \(\Sym_3\) is given by \[
    \sigma.e_{\sigma'} = e_{\sigma \sigma'}
  \]
  Now, consider that the element \[
    e_{(1)} + e_{(12)} + e_{(13)} + e_{(23)} + e_{(123)} + e_{(321)}
  \]
  is a \(G\)-invariant subspace of \(\C^6\) under the regular
  representation. This subrepresentation is isomorphic to the trivial
  represenation. Similarly, the space spanned by \[
    e_{(1)} - e_{(12)} - e_{(13)} - e_{(23)} + e_{(123)} + e_{(321)}
  \]
  is \(G\)-stable and, as a subrepresentation of the regular
  representation, is isomorphic to the alternating
  representation. Thus, we have \[
    \rho_{reg} = \rho_{trivial} \oplus \rho_{alt} \oplus \rho'
  \]
  where \(\rho' \from G \to GL(\C^6/(V_{trivial} \oplus
  V_{alt}))\). We know from our analysis above that \(\rho'\) must
  break up further, and we will see later that it breaks up as the
  direct sum of 2 standard representations.
\end{example}
\section{\(\Hom\) and dual representation}
We take a quick detour to view some other methods for constructing new
representations from old ones. These results will be useful later, so
the reader may choose to skip them for now and come back later when
they are needed.
\begin{prop}\label{def-hom-rep}
  Let \(V,W\) be representations of a finite group \(G\). Then, the
  action of \(G\) on 
  the vector space \(\Hom_\C(V,W)\) given by, for \(\phi \in \Hom_\C(V,W)\),
  \begin{align*}
    g.\phi \from V & \to W \\
    v & \mapsto g.\phi(g^{-1}.v)
  \end{align*}
  makes \(\Hom_\C(V,W)\) into a representation of \(G\). 
\end{prop}
\begin{cor}\label{def-dual-rep}
  Let \(V\) be a representation of a finite group \(G\). Then, \(V^* =
  \Hom_\C(V,\C)\) has a natural representation
  structure. Explicitly, for basis \(f \in V^*\), \[
    \rho_{V^*}(g).f = f \circ \rho_V(g^{-1}) 
  \]
\end{cor}
\begin{prop}\label{tensor-hom-relation}
  Given finite-dimensional representations \(V,W\) of \(G\), then
  there is a natural 
  isomorphism \[
    V^* \otimes W \isomto \Hom_\C(V,W)
  \]
\end{prop}
\begin{proof}
  Consider the \(\C\)-bilinear map
  \begin{align*}
    V^* \times W & \to \Hom_\C(V,W) \\
    (f(\cdot),w) & \mapsto f(\cdot) w
  \end{align*}
  Then, by the universal property of tensor products, there is a
  unique homomorphism \(\phi \from V^* \otimes W \to \Hom_\C(V,W)\)
  such that \(f(\cdot) \otimes w \mapsto f(\cdot) w\). Such a map is
  certainly injective since \(\ker \phi = \{0\}\). It is surjective
  since \(\Hom_\C(V,W)\) is isomorphic to \(\dim W \times \dim V\)
  matrices, so \(\dim \Hom_\C(V,W) = \dim V \cdot \dim W = \dim V
  \otimes W = \dim V^* \otimes W\). 
\end{proof}
% \begin{defn}
%   For an irreducible representation \(\rho \from G \to GL(V)\) of
%   \(G\), then for all \(g \in Z(G)\), \(\rho(g)\) commutes with all
%   \(\rho(h), h \in H\). Thus, \(\rho(g) = \lambda_g Id_V\) by \ref{schur-alg-closed} and
%   we call the map \(\chi_V \from Z(G) \to \C\) the \de{central
%     character} of \(V\).
% \end{defn}
% \begin{prop}
%   \(\chi_V \from Z(G) \to \C\) is a homomorphism.
% \end{prop}
\begin{rmk}
  The construction of the \(\Hom\) representation more generally comes
  from the \emph{Hopf algebra} structure on \(\C[G]\). 
\end{rmk}
\section{Character Theory}
Throughout this section, let \(G\) be a finite group and \(F =
\C\). We wish to extend the notion of the central character above to a
general character \(\chi_V \from G \to \C\) that will still carry
useful information about the group representation \(V\).
\begin{prop}\label{g-is-ss}
  If \(V\) is a \(G\)-module, then every \(g \in G\), viewed as a
  linear operator \(g_V \from V \to V\), is semisimple, and thus \(V\)
  has a basis of eigenvectors for \(g_V\).
\end{prop}
\begin{proof}
  Since \(G\) is a finite group, \(g\) is of finite order, so \(g_V^n
  = Id_V\). Thus, \(g_V^n = (D+N)^n\) for diagonalizable \(D\) and
  nilpotent \(N\) (by Jordan Canonical Form since \(\C\) is
  algebraically closed) such that \(DN = ND\). Thus, \[
    Id_V = g_V^n = (D+N)^n = D^n + nND^{n-1} + \binom{n}{2} N^2 D^{n-2} +
    \cdots + nN^{n-1}D + N^n 
  \]
  which thus tells us that \(N = 0\) and \(D^n = Id_V\) since \(Id_V\)
  has no nilpotent part.
\end{proof}
In general, it is quite annoying to compute the eigenvalues for each
\(g \in G\). However, we will see that it suffices to merely compute
their sum using the trace. 
\begin{defn}
  The \de{character} of group representation \(\rho \from G \to V\), denoted \(\chi_\rho = \chi^\rho\), is the function
  \(\chi_\rho \from G \to \bbk\) defined by \(\chi_\rho(g) =
  \tr(\rho(g))\).
\end{defn}
\begin{prop}
  For \(\bbk = \C\) and fixed group representation \(\rho\),
  \begin{enumerate}
  \item \(\chi(1) = \dim V\),
  \item \(\chi(g^{-1}) = \ov{\chi(g)}\) for all \(g \in G\),
  \item \(\chi(tst^{-1}) = \chi(s)\) for all \(s,t \in G\).
  \end{enumerate}
\end{prop}
\begin{proof}
  All these properties follow from standard properties of trace.
\end{proof}
\begin{prop}\label{dual-character}
  Let \(V\) be a group representation and let \(\chi_V\) be its
  character. Then, \[
    \ov{\chi_V(g)} = \chi_{V^*}(g)
  \]
\end{prop}
\begin{proof}
  Consider that, by \ref{def-dual-rep}, \(\chi_{V^*}(g) =
  \chi_V(g^{-1})\). However, since \(\chi_V(g)\) is just the sum of
  eigenvalues of \(g\) (which must be roots of unity by the proof of \ref{g-is-ss}), we have that \[
    \chi_{V^*}(g) = \chi_V(g^{-1}) = \sum \lambda_i^{-1} = \sum
    \ov{\lambda_i} = \ov{\sum \lambda_i} = \ov{\chi_V(g)}
  \]
\end{proof}
\begin{defn}
  A function \(f \from G \to \C\) is called a \de{class function}
  if it is constant on conjugacy classes. Note that
  \(\operatorname{Class}(G)\) is the \(\C\)-algebra of all class
  functions on \(G\) and \(\dim \operatorname{Class}(G) =\) the number
  of conjugacy classes in \(G\).
\end{defn}
\begin{prop}
  Given a representation of a finite group \(G\) and a representation
  \(\rho\), \(\chi_\rho \from G \to \C\)  is a class function. 
\end{prop}
\begin{proof}
  This follows from part (c) of the previous proposition.
\end{proof}
\begin{prop}
  Let \(\rho_i \from G \to GL(V_i)\), \(i = 1,2\) be representations
  of \(G\) with characters \(\chi_{\rho_i}\).
  \begin{enumerate}
  \item \(\chi_{\rho_1 \oplus \rho_2} = \chi_{\rho_1} + \chi_{\rho_2}\)
  \item \(\chi_{\rho_1 \otimes \rho_2} = \chi_{\rho_1} \cdot \chi_{\rho_2}\)
  \end{enumerate}
\end{prop}
\begin{proof}
  This follows from the definition of direct sum and tensor product of
  representations and the definition of the character.
\end{proof}
Based on the identity (a) above, we see that the characters of
irreducible representations will be of primary interest.
\section{Orthogonality Relations of Irreducible Characters}
Inspired by the proof of Maschke's theorem, we present the following
proposition.
\begin{prop}
  Let \(h \from V_1 \to V_2\) be a linear transformation and let 
  \(\rho^1 \from G \to GL(V_1), \rho^2 \from G \to GL(V_2)\) be
  irreducible representations of finite group \(G\). Consider \[
    h^0 := \frac{1}{|G|} \sum_{t \in G} \rho^2(t^{-1}) h \rho^1(t)
  \]
  \(h^0\) is a homomorphism of representations. Furthermore
  \begin{enumerate}
  \item If \(\rho^1 \not \isom \rho^2\), then \(h^0 = 0\).
  \item If \(\rho^1 \isom \rho^2\) (and thus \(V_1 \isom V_2\)), then \(h^0 = \frac{1}{\dim V_1}
    \tr(h) Id_{V_1}\).
  \end{enumerate}
\end{prop}
\begin{proof}
  \todo{Fill this in}
\end{proof}
\begin{cor}
  Consider that \(\rho(t) \in GL(V)\) and thus we can represent
  \(\rho(t)\) as a \(n \times n\) matrix, say \(\rho^1(t) =
  [r_{i_1,j_1}(t)]\) and \(\rho^2(t) = [r_{i_2,j_2}(t)]\). Then,
  \begin{enumerate}
  \item If
  \(\rho^1 \isom \rho^2\), \[
    \frac{1}{|G|} \sum_{t \in G} r_{i_2,j_2}(t^{-1}) r_{j_1,i_1}(t) =
    \frac{1}{n} \delta_{j_1,j_2} \delta_{i_1,i_2}
  \]
\item If \(\rho^1 \not \isom \rho^2\), \[
  \frac{1}{|G|} \sum_{t \in G} r_{i_2,j_2}(t^{-1}) r_{j_1,i_1}(t) = 0
  \]
  \end{enumerate}
\end{cor}
This provides us motivation for definiting a positive definite
Hermitian inner product on the set of class functions on \(G\).
\begin{defn}
  For class functions \(\phi, \psi \from G \to \C\), we define inner
  product \[ 
    \langle \phi, \psi \rangle = \frac{1}{|G|} \sum_{g \in G}
    \phi(g^{-1}) \psi(g)
  \]
  and also \[
    (\phi,\psi) = \frac{1}{|G|} \sum_{g \in G} \phi(g) \ov{\psi(g)}
  \]
\end{defn}
\begin{prop}
  Let \(\langle \cdot, \cdot \rangle\) and \((\cdot, \cdot)\) be as
  above.
  \begin{enumerate}
  \item \(\langle \cdot, \cdot \rangle\) is a bilinear form.
  \item \(\langle \phi, \psi \rangle = \langle \psi, \phi \rangle\),
    that is \(\langle \cdot, \cdot \rangle\) is symmetric.
  \item \((\phi, \psi) = \langle \phi, \ov{\psi \circ i} \rangle\)
    where \(i \from G \to G\) takes \(i(g) = g^{-1}\).
  \item \((\cdot, \cdot)\) is a positive definite Hermitian inner form.
  \end{enumerate}
\end{prop}
\begin{prop}
  If \(\rho\) is a representation of \(G\) in \(V\) and if \(\psi(t) =
  \tr \rho(t)\), then \[
    \langle \phi, \psi \rangle = (\phi, \psi)
  \]
\end{prop}
Thus, it is with the form \((\cdot, \cdot)\) that we can arise at the
following
\begin{thm}\label{orthonormal}
  For any representations \(V,W\), \[
    (\chi_V, \chi_W) = \dim \Hom_G(W,V) 
  \]
  and, if \(V,W\) are irreducible, \[
    (\chi_V, \chi_W) =
    \begin{cases}
      1 & V \isom W \\
      0 & V \not \isom W
    \end{cases}
  \]
\end{thm}
\begin{proof}
  We first compute for arbitrary representations \(V,W\),
  \begin{align*}
    (\chi_V, \chi_W) & = \frac{1}{|G|} \sum_{g \in G} \chi_V(g)
                       \ov{\chi_W(g)} \\
                     & = \frac{1}{|G|} \sum_{g \in G} \chi_V(g)
                       \chi_{W^*}(g) & \text{by \ref{dual-character}}\\
                     & = \frac{1}{|G|} \sum_{g \in G} \chi_{V \otimes
                       W^*}(g) \\
                     & = \frac{1}{|G|} \sum_{g \in G}
                       \chi_{\Hom_\C(W,V)}(g) & \text{by \ref{tensor-hom-relation}}\\
                     & = \tr_{\Hom_\C(W,V)}\left(\frac{1}{|G|}\sum_{g \in G} g\right)
  \end{align*}
  However, \(P := \frac{1}{|G|} \sum_{g \in G} g\) is a homomorphism of
  \(G\) representations (that is, \(P.v\) is a representation
  of \(G\)) and the image of \(P\) is
  stable under any action of \(G\) since \(g.Pv = Pv\). However, the only
  such irreducible representation is the trivial representation or the
  zero representation. Thus, for
  arbitrary representation 
  \(U\), \(\tr_U(P)\) simply counts
  the number of times the trivial representation occurs as a
  subrepresentation of \(U\). However, considering \(\Hom_\C(W,V)\) as
  a \(G\)-representation, the \(G\)-invariant subrepresentation is
  exactly \(\Hom_G(W,V)\). Thus, \[
   (\chi_V, \chi_W) = \tr_{\Hom_\C(W,V)}\left(\frac{1}{|G|}\sum_{g \in G} g\right) = \dim \Hom_G(W,V)
  \]
  Furthermore, if \(V,W\) are irreducible, Schur's lemma tells us
  that \[
    \dim \Hom_G(W,V) =
    \begin{cases}
      1 & V \isom W \\
      0 & \text{ else }
    \end{cases}
  \]
\end{proof}
Thus, we have shown that the irreducible characters of a finite group
are orthogonal to each other under \((\cdot, \cdot)\). In fact, we
have the following theorem
\begin{thm}
  The irreducible characters of \(G\), \(\{\chi_1, \ldots, \chi_k\}\) form an orthonormal basis for
  the set of class functions on \(G\). 
\end{thm}
\begin{lem}
  Let \(\rho \from G \to GL(V)\) with \(\dim V = n\) be an irreducible
  representation of \(G\) and let \(f\) 
  be a class function on \(G\). Then, given \[
    \rho_f \from V \to V, \rho_f := \sum_{g \in G} f(g)\rho(g)
  \]
  we have \(\rho_f = \frac{1}{n}|G| (f, \ov{\chi_\rho}) Id_V\).
\end{lem}
\begin{proof}[Proof of Lemma]
  Let \(h \in G\). Then,
  \begin{align*}
    \rho_f \rho(h) & = \sum_{g \in G} f(g) \rho(g) \rho(h) \\
                   & = \sum_{g \in G} f(g) \rho(gh) \\
                   & = \sum_{g \in G} \rho(h) \rho(h^{-1}) f(g) \rho(gh) \\
                   & = \rho(h) \sum_{g \in G} f(g) \rho(h^{-1}gh) \\
                   & = \rho(h) \sum_{g \in G} f(h^{-1}gh) \rho(h^{-1}gh) \\
    & = \rho(h) \rho_f
  \end{align*}
  Since \(V\) is irreducible, it must be that \(\rho_f = \lambda 1_V\)
  by Schur's lemma. Thus, taking the trace of both sides, we get \[
    n \lambda = \tr(\rho_f) = \sum_{g \in G} f(g) \chi_\rho(g)
  \]
  and thus \[
    \lambda = \frac{1}{n} |G| (f, \ov{\chi})
  \]
\end{proof}
\begin{proof}[Proof of Theorem]
  By the theorem above, we know that the irreducible characters form
  an orthonormal system. Thus, we need only show that they generate
  \(\Class(G)\). It will suffice to show that if \(f \in \Class(G)\)
  has \((f,\ov{\chi_i}) = 0\) for all irreducible characters
  \(\chi_i\), then \(f \identically 0\). Assume \((f,\ov{\chi_i}) =
  0\). Then, by the lemma above, \(\rho_f = 0\) for any irreducible
  representation \(\rho\). Thus, from the direct sum decomposition, we
  can conlude that \(\rho_f = 0\) for any \(\rho\). So, applying this
  to the regular representation, \(\rho_{reg}\), and computing the
  image of a basis vector \(e_1\), we get \[
    0 = (\rho_{reg})_f e_1 = \sum_{g \in G} f(g) \rho_{reg}(g) e_1 = \sum_{g
      \in G} f(g) e_g \implies f(g) = 0, \forall g \in G. 
  \]
  Thus, \(f = 0\).
\end{proof}
\begin{cor}
  The number of irreducible
  representations of \(G\) is equal to the number of
  conjugacy classes of \(G\).
\end{cor}
\begin{rmk}
  We also know this from Wedderburn's theorem + Maschke's theorem.
\end{rmk}
\begin{proof}
  Let \(g \in G\) and \(f_g \in \Class(G)\) be defined by \[
    f_g(h) :=
    \begin{cases}
      1 & \text{ if }g \text{ and }h\text{ are conjugate} \\
      0 & \text{ otherwise}
    \end{cases}
  \]
  These \(f_g\) functions form a natural basis for \(\Class(G)\) when
  one is chosen for each conjugacy class in \(G\). Thus, \(\dim
  \Class(G)\) is the number of conjugacy classes of \(G\). On the
  other hand, using the theorem above, \(\dim \Class(G)\) is equal to
  the number of irreducible representations of \(G\). Thus, these two
  quantities are equal.
\end{proof}
\begin{thm}\label{2nd-ortho-reln}
  Let \(g \in G\) and \(C(g)\) be the number of elements in the
  conjugacy class of \(g\). Then,
  \begin{enumerate}
  \item \(\sum_{i=1}^d \ov{\chi_i}(g) \chi_i(g) = \frac{|G|}{C(g)}\)
  \item \(\sum_{i=1}^d \ov{\chi_i}(g) \chi_i(h) = 0\) if \(g\) is not
    conjugate to \(h\).
  \end{enumerate}
\end{thm}
\begin{rmk}\label{sum-of-squares-is-ord-G}
  In particular, \(\sum_{i=1}^d \ov{\chi_i}(1)\chi_i(1) = |G|\).
\end{rmk}
\begin{proof}
  Using \(f_g \in \Class(G)\) as above, write \[
    f_g = \sum_{i=1}^k \lambda_i \chi_i, \lambda_i \in \C
  \]
  Then, \(\lambda_i = (f_g, \chi_i) = \frac{C(g)}{|G|}
  \ov{\chi_i}(g)\) where \(C(g)\) is the number of elements in the
  conjugacy class of \(g\). Therefore, \[
    f_g(h) = \frac{C(g)}{|G|} \sum_{i=1}^k \ov{\chi_i(g)} \chi_i(h)
  \]
  Thus, by definition of \(f_g(h)\), the identities in the theorem are proved.
\end{proof}
\begin{example}\label{char-table-for-S3}
  Let \(G = \Sym_3\). We know that \(\Sym_3\) has \(3\) conjugacy
  classes given by cycle type. Thus, \(\Sym_3\) has \(3\) distinct
  irreducible characters, denoted \(\chi_1, \chi_2\), and \(\theta\)
  for trivial, alternating, and standard representations. We then get
  the following ``character table'' \\
  \begin{center}
    \begin{tabular}{c|ccc}
      &(1)&(12)&(123) \\
      \hline
      \(\chi_1\) &1&1&1 \\
      \(\chi_2\) &1&-1&1\\
      \(\theta\) &2&0&-1
    \end{tabular}
  \end{center}
  where the first \(2\) rows follow from just computing the trace of
  the trivial and alternating representations. The third row then
  follows from our orthogonality relations, since
  \begin{align*}
    \theta(1) & = 2 \\
    0 = (\chi_1,\theta) & = \frac{1}{6} \sum_{g \in \Sym_3}
    \ov{\theta(g)} \\
    0 = (\chi_2, \theta) & = \frac{1}{6} \sum_{g \in \Sym_3} \sgn(g)
    \ov{\theta(g)} \\
    \implies 0 & = \frac{1}{3}\left( 2 + \theta((123)) + \theta((321))
    \right) \\
    \implies \theta((123)) & = -1
  \end{align*}
  and \[
    0 = \chi_1((1)) \chi_1((12)) + \chi_2((1)) \chi_2((12)) +
    \theta((1)) \theta((12)) = 1 - 1 + 2 \theta((12)) \implies
    \theta((12)) = 0
  \]
  Finally, returning to the regular representation of \(\Sym_3\), we
  know that \(\chi_{reg}(1) = 6\) and \(\chi_{reg} = a \chi_1 + b
  \chi_2 + c \theta\). However, we also know that \(\chi_{reg}(g) =
  0\) for all non-identity \(g \in \Sym_3\) since no such elements fix
  any basis elements. In particular,
  \(\chi_{reg}((123)) = 0\). Thus, the unique
  decomposition of \(\chi_{reg}\) is \[
    \chi_{reg} = \chi_1 + \chi_2 + 2 \theta
  \]
  From this, we will see that \[
    \rho_{reg} = \rho_{trivial} \oplus \rho_{alt} \oplus \rho_{std}
    \oplus \rho_{std} \isom M_1(\C) \oplus M_1(\C) \oplus M_2(\C)
  \]
\end{example}
\section{Results of Character Theory}
Now, we wish to use character theory to tell us more about
representations of a group. 
\begin{prop}
  Let \(H \normsubgroup G\) and let \(\rho \from G/H \to GL(V)\) be a
  representation. Then, \(\rho\) is irreducible if and only if its
  lifted representation to \(G\), \(\tilde{\rho}\), is
  irreducible. (See \ref{lifted-rep} for a reminder of how a lifted
  representation is constructed.)
\end{prop}
\begin{proof}
  We first note that \[
    \tr(\tilde{\rho}(g)) = \tr(\rho(gN))
  \]
  by definition of \(\tilde{\rho}\). Thus, if \(\chi\) is the
  character of \(\tilde{\rho}\) and \(\chi'\) is the character of
  \(\rho\), then \(\chi(g) = \chi'(gH)\). Now, we compute
  \begin{align*}
    (\chi', \chi') & = \frac{1}{|G/H|} \sum_{gH \in G/H} \chi'(gH)
                     \ov{\chi'}(gH) \\
                   & = \frac{|H|}{|G|} \sum_{gH \in G/H} \chi(g) \ov{\chi(g)} \\
    & = \frac{1}{|G|} \sum_{gH \in G/H} \sum_{h \in H} \chi(gh)
      \ov{\chi(gh)} \\
                   & = \frac{1}{|G|} \sum_{g \in G} \chi(g) \ov{\chi(g)} \\
    & = (\chi, \chi)
  \end{align*}
\end{proof}
\begin{example}
  Let \(G = A_4\). Then, \(|A_4| = 12\) and \(A_4\) has \(4\)
  conjugacy classes given by \[
    C_1 = \{(1)\}, C_2 = \{(123), \ldots\}, C_3 = \{(132), \ldots\},
    C_4 = \{(12)(34), \ldots\}
  \]
  and thus must have \(4\) irreducible characters, say \(\chi_1,
  \chi_2, \chi_3, \chi_4\), and let \(\chi_1\) be the trivial
  representation.
  
  Now, consider that \(C_4\) is also a normal subgroup of \(A_4\)
  isomorphic to \(\Z/2\Z \times \Z/2\Z\), so
  we can find irreducible representations of \(A_4/C_4 \isom \Z/3\Z\), we could
  lift them to irreducible representations of \(A_4\) by the
  proposition above. Since \(\Z/3\Z\) is
  abelian, it immediately gives character table \[
    \begin{array}{c|ccc}
      & (1) & (123) & (132) \\
      \hline
      \chi_1 & 1 & 1 & 1 \\
      \chi_{\zeta_3} & 1 & \zeta_3 & \zeta_3^2 \\
      \chi_{\zeta_3^2} & 1 & \zeta_3^2 & \zeta_3
    \end{array}
  \]
  Of course, the trivial representation with character \(\chi_1\) lifts to a trivial
  representation of \(A_4\). However, the lifted representation \(\tilde{\rho_{\zeta_3}}\) and \(\tilde{\rho_{\zeta_3^2}}\) will
  be irreducible by the proposition above and thus give us irreducible
  characters of \(A_4\). It is immediate that, up to reordering,
  \(\chi_2 = \tilde{\rho_{\zeta_3}}\) and \(\chi_3 =
  \tilde{\rho_{\zeta_3^2}}\), so we now fill in the rows of our
  character table to get \[
    \begin{array}{c|cccc}
      &C_1&C_2&C_3&C_4 \\
      \hline
      \chi_1 & 1 & 1 & 1 & 1\\
      \chi_2 & 1 & \zeta_3 & \zeta_3^2 & 1\\
      \chi_3 & 1 & \zeta_3^2 & \zeta_3 & 1\\
      \chi_4 & a & b & c & d 
    \end{array}
  \]
  However, we know that \[
    1^2 + 1^2 + 1^2 +a^2 = 12 \implies a = \pm 3
  \]
    by \ref{sum-of-squares-is-ord-G} and \(a = \chi_4(1) = \dim V\) for \(V\) the corresponding vector
  space of \(\rho_4\). Thus \(a > 0 \implies a = 3\). Furthermore,
 \[
    1^2 + 1^2 + 1^2 + ad = 0 \implies ad = -3 \implies d = -1
  \]
  by \ref{2nd-ortho-reln}.
  Now, we can use our orthogonality relations to determine
  \(\chi_4\). Using the orthongality of rows (\ref{orthonormal}), we get \[
    \begin{cases}
      (\chi_4,\chi_1) = 3 + 4b+4c-3\cdot1 = 0\\
      (\chi_4,\chi_2) = 3+4b\zeta_3^2+4c\zeta_3-3\cdot1 = 0\\
      (\chi_4,\chi_3) = 3+4b\zeta_3+4c\zeta_3^2-3\cdot1 = 0\\
    \end{cases}
    \implies
    \begin{cases}
      b = 0 \\
      c = 0\\
    \end{cases}
  \]
  Thus, our completed character table is given by \[
    \begin{array}{c|cccc}
      &C_1&C_2&C_3&C_4 \\
      \hline
      \chi_1 & 1 & 1 & 1 & 1\\
      \chi_2 & 1 & \zeta_3 & \zeta_3^2 & 1\\
      \chi_3 & 1 & \zeta_3^2 & \zeta_3 & 1\\
      \chi_4 & 3 & 0 & 0 & -1
    \end{array}
  \]
\end{example}
Perhaps the most important reason to study character theory is the
following theorem.
\begin{cor}
  The multiplicity of an irreducible representation \(W\) in a
  representation \(V\) is \((\chi_W, \chi_V)\). 
\end{cor}
\begin{proof}
  Let \(V \isom W_1^{m_1} \oplus \cdots \oplus W_k^{m_k}\) be a
  decomposition of \(V\) into irreducibles. Then, we have \[
    \chi_V = \chi_{W_1^{m_1} \oplus \cdots \oplus W_k^{m_k}} = m_1
    \chi_{W_1} + \cdots + m_k \chi_{W_k}.
  \]
  Moreover, since the irreducible characters form an \emph{orthonormal
  basis} of class functions, we can apply \((\cdot, \chi_{W_i})\) to
  our equality to get \[
    (\chi_V, \chi_{W_i}) = m_i
  \]
\end{proof}
\begin{cor}
  Two finite dimensional complex representations of a finite group
  \(G\), say \(V,W\), are isomorphic if and only if they have the same
  character. In other words \[
    V \isom W \iff \chi_V = \chi_W
  \]
\end{cor}
\begin{proof}
  The forward direction follows by definition of character. The
  backwards direction follows from the fact that the irreducible characters form a
  basis of \(\Class(G)\). Namely, \(\chi_V = \chi_W\) implies that
  \(\chi_V\) and \(\chi_W\) have the same unique decomposition into
  irreducible characters, and thus from the corollary above, \(V,W\) have the
  same multiplicity of each irreducible representation.
\end{proof}
\begin{cor}
  A representation \(V\) is irreducible if and only if \((\chi_V,
  \chi_V) = 1\). 
\end{cor}
\begin{proof}
  The forward direction was proved earlier in \ref{orthonormal}. For the
  backwards direction, if \(\chi_V = m_1 \chi_{V_1} + \cdots + m_k
  \chi_{V_k}\) for irreducibles \(V_1, \ldots, V_k\), then using the
  fact that the \(\chi_{V_i}\)'s form an orthonormal basis, we get \[
    1 = (\chi_V, \chi_V) = m_1^2 + \cdots + m_k^2
  \]
  Thus, since each \(m_i^2 \geq 0\) and are in \(\Z\), it must be that
  exactly one \(m_i = 1\) and all others are \(0\). Thus, \(V\) must
  be irreducible by the corollary above.
\end{proof}
\section{The Regular Representation Revisited}
\begin{cor}
  The multiplicity of any irreducible representation \(V\) in the regular
  representation equals its dimension.
\end{cor}
\begin{proof}
  Recall that the dimension of the regular representation is \(|G|\). Consider \[
    (\chi_V, \chi_{reg}) = \frac{1}{|G|}\chi_V(1) \chi_{reg}(1) =
    \chi_V(1) = \dim V
  \]
\end{proof}
\begin{cor}
  Let \(\{V_1, \ldots, V_k\}\) be the set of all irreducible
  representations of finite group \(G\). Then, \[
    |G| = \sum_{i=1}^k (\dim V_i)^2
  \]
\end{cor}
\begin{proof}
  From above, we have \(\chi_{reg} = \sum_{i=1}^k \dim V_i \cdot
  \chi_{V_i}\), and so we simply take \((\chi_{reg}, \chi_{reg})\) to
  get \[
    (\chi_{reg}, \chi_{reg}) = \sum_{1\leq i \leq k, 1 \leq j \leq k}
    \dim V_i \dim V_j (\chi_{V_i}, \chi_{V_j}) = \sum_{i=1}^k (\dim
    V_i)^2
  \]
  using the orthogonality of the irreducible characters. However,
  since \[
    \chi_{reg}(g) =
    \begin{cases}
      |G| & g = 1 \\
      0 & \text{otherwise}
    \end{cases}
  \]
  we also have \((\chi_{reg}, \chi_{reg}) = \frac{|G|^2}{|G|} = |G|\)
  by definition of the inner product \((\cdot, \cdot)\).
\end{proof}
\begin{rmk}
  This result should not come as a surprise to a reader familiar with
  Artin-Wedderburn theory. However, the question of what the
  analogous decomposition of the regular representation looks like
  still remains. 
\end{rmk}
\begin{lem}\label{tensor-of-irred-is-irred}
  If \(\rho_1\) and \(\rho_2\) are irreducible representations of
  \(G_1\) and \(G_2\) respectively, then
  \(\rho_1 \otimes \rho_2\) is an irreducible representation of \(G_1
  \times G_2\)
\end{lem}
\begin{proof}
  We have, from above, that for characters \(\chi_1\) of \(\rho_1\)
  and \(\chi_2\) of \(\rho_2\), that \[
    (\chi_1, \chi_1) = 1 = (\chi_2,\chi_2)
  \]
  However, \(\chi := \chi_{\rho_1 \otimes \rho_2}\) has
  \begin{align*}
    (\chi,\chi) & = 
    \frac{1}{|G_1||G_2|} \sum_{g \in G_1, g' \in G_2} \chi_1(g)
                  \chi_2(g') \ov{\chi_1(g) \chi_2(g')} \\
    & = \frac{1}{|G_1|} \left(
      \sum_{g \in G_1} \chi_1(g) \ov{\chi_1(g)} \right)
      \frac{1}{|G_2|} \left( 
      \sum_{g' \in G_2} \chi_2(g') \ov{\chi_2(g')} \right) \\
    & = (\chi_1, \chi_1) \cdot (\chi_2, \chi_2) \\
    & = 1
  \end{align*}
  Thus, it must be that \(\rho_1 \otimes \rho_2\) is irreducible.
\end{proof}
\begin{rmk}
  The converse, that each irreducible \(G_1 \times
  G_2\)-representation is isomorphic to a tensor product of
  \(G_i\)-representations is also true. The argument is to show that
  each class function of \(G_1 \times G_2\) which is orthogonal to
  characters of the form \(\chi_1 \cdot \chi_2\) is zero.
\end{rmk}
\begin{prop}
  One can realize \(\C[G]\) as a \(G \times G\)-module via the
  action \[
    (h,k).g = hgk^{-1}
  \]
  extended linearly. 
\end{prop}
\begin{proof}
  Indeed, \[
    (h',k').(h,k).g = (h',k').hgk^{-1} = h'hgk^{-1}k'^{-1} = (h'h,k'k).g
  \]
\end{proof}
\begin{thm}
  The regular representation decomposes as a direct sum of \(G \times
  G\) representations. More specifically, \[
    \C[G] \isom \bigoplus_{i=1}^k V_i \otimes V_i^*
  \]
  for irreducible \(G\)-representations \(\{V_1, \ldots, V_k\}\).
\end{thm}
\begin{rmk}
  Thus, in a sense, we have arrived at the Artin-Wedderburn theorem in
  different language. 
\end{rmk}
\section{Some Applications of Representation Theory: Frobenius
  Divisibility and Burnside}
We now seek to discuss some useful applications of representation
theory that would be much more difficult to prove using only group
theory. First, we will have to introduce the terminology of algebraic
numbers.
\begin{defn}
  We say \(z \in R\), a commutative ring, is an \de{algebraic integer}
  or \de{integral over \(\Z\)} if \(z\) is a root of a monic
  polynomial with coefficients in \(\Z\).
\end{defn}
\begin{rmk}
  With this definition, ``integer'' is a somewhat misleading word. For
  instance, if \(R = \C\), then \(i \in \C\) is an algebraic integer
  since it is a solution to the polynomial \(x^2+1\).
\end{rmk}
\begin{lem}
  The set of all algebraic integers is a ring.
\end{lem}
\begin{proof}
  Let \(\alpha\) be an eigenvalue of some \(A \in M_n(R)\) with
  eigenvector \(v\) and
  \(\beta\) be an eigenvalue of some \(B \in M_m(R)\) with eigenvector
  \(w\). Then, \(\alpha \pm \beta\) is an eigenvalue of \[
    A \otimes Id_m \pm Id_n \otimes B
  \]
  and \(\alpha \beta\) is an eigenvalue of \(A \otimes B\). In both
  cases, the appropriate eigenvector is \(v \otimes w\). Thus, the
  lemma is proven.
\end{proof}
\begin{prop}
  Let \(z\) be an element of a commutative ring \(R\). The following
  properties are equivalent.
  \begin{enumerate}
  \item \(z\) is integral over \(\Z\).
  \item The subring \(\Z[z] \subset R\) generated by \(z\) is finitely
    generated as a \(\Z\)-module.
  \item There exists a finitely generated sub-module of \(R\)
    considered as a \(\Z\)-module containing \(\Z[z]\)
  \end{enumerate}
\end{prop}
\begin{prop}\label{characters-are-algebraic-integers}
  Let \(\chi\) be the character of a representation \(\rho\) of a finite group
  \(G\). Then, \(\chi(g)\) is an algebraic integer for each \(g \in G\).
\end{prop}
\begin{proof}
  \(\chi(g) = \tr(\rho(g))\), so it is a sum of eigenvalues of
  \(\rho(g)\). However, all the eigenvalues are roots of unity, which
  are clearly algebraic integers. Thus, since the algebraic integers
  form a ring, this sum is an algebraic integer.
\end{proof}
\begin{thm}
  Given a complex irreducible representation \(V\) of \(G\), \(\dim V
  \divides |G|\). 
\end{thm}
\begin{proof}
  Let \(C_1, \ldots, C_k\) be the conjugacy classes of \(G\) and set \[
    \lambda_i := \chi_V(C_i) \frac{|C_i|}{\dim V}
  \]
  where \(\chi_V(C_i) = \chi_V(g)\) for some \(g \in C_i\) (recall
  that \(\chi_V\) is a class function). Then, all of these
  \(\lambda_i\) are algebraic integers. To see this, consider \[
    P_i = \sum_{h \in C_i} h
  \]
  which is central in \(\Z[G]\) by a quick computation. Thus, by
  Schur's lemma, \(P_i = \mu_i Id_V\) since \(V\) is irreducible. Furthermore,
  \(\Z[G]\) is finitely generated, so \(\mu\) is an algebraic
  integer. Thus, we compute \[
    \tr(P_i) = |C_i| \chi_V(C_i) = \mu_i \dim V \implies \mu_i =
    \frac{|C_i| \chi_V(C_i)}{\dim V} = \lambda_i
  \]
  Now, to finish our proof, we must show that \[
    \sum_i \lambda_i \ov{\chi_V(C_i)}
  \]
  is an algebraic integer. However, we already know that the algebraic
  integers form a ring and, by above, each \(\lambda_i\) is an
  algebraic integer. Moreover, \(\chi_V(C_i)\) is an algebraic integer
  by \ref{characters-are-algebraic-integers}. Thus, the sum is an
  algebraic integer by the ring structure on the algebraic integers.

  Now, we note, by definition of \(\lambda_i\), so we get the
  following
  \begin{align*}
    \sum_{i} \lambda_i \ov{\chi_V(C_i)}
    & = \frac{1}{\dim V}\sum_i |C_i| \chi_V(C_i) \ov{\chi_V(C_i)} \\
    & = \frac{1}{\dim V}\sum_{g \in G} \chi_V(g) \ov{\chi_V(g)}
    & \text{since characters are class functions} \\
    & = \frac{1}{\dim V} |G| (\chi_V,\chi_V) & \text{by definition of
                                               }(\cdot, \cdot)\\
    & = \frac{|G|}{\dim V} & \text{since }V\text{ is irreducible, so
                             }(\chi_V, \chi_V) = 1    
  \end{align*}
  Thus, it must be that \(\frac{|G|}{\dim V}\) is an algebraic
  integer. However, \(\frac{|G|}{\dim V} \in \Q\), so by Gauss's
  lemma, \(\frac{|G|}{\dim V} \in \Z\).
\end{proof}
While this result is useful in and of itself, we have something
better, namely 
\begin{thm}[Frobenius Divisibility]
  Given an irreducible representation \(V\) of \(G\), \[
    \dim V  \divides \frac{|G|}{|Z(G)|}.
  \]
\end{thm}
\begin{proof}
  For \(g \in Z(G)\), we note that \(\rho(g)\) commutes with
  \(\rho(h)\) for all \(h \in G\), so \(\rho(g) = \lambda_g Id_V\) for
  some \(\lambda \in \C\). Consider the map from \(Z(G)\) to \(\C\)
  given by \(g \mapsto \lambda_g\). Such a map is, in fact, a
  homomorphism from \(Z(G)\) to \(\C^\times\). Now, for some \(m \geq
  0\), consider \[
    \rho^{\otimes m} \from G^{\times m} \to GL(V^{\otimes m})
  \]
  By \ref{tensor-of-irred-is-irred}, this is an irreducible representation of \(G^{\times
    m}\). So, given \((g_1, \ldots, g_m) \in Z(G) \times Z(G)
  \times \cdots \times Z(G) \subgroup Z(G^{\times m})\), it must be
  that \(\rho(g_1, \ldots, g_m) = \lambda Id_{V^{\otimes m}}\). Now,
  consider \(H = \{(g_1, \ldots, g_m) \in (Z(G))^{\times m} \st g_1
  \ldots g_m = 1_G\} \subgroup G\). Then, for \((h_1, \ldots, h_m) \in
  H\),
  \begin{align*}
    (h_1, \ldots, h_m).(v_1 \otimes \cdots \otimes v_m)
    & = \rho(h_1)v_1 \otimes \cdots \otimes \rho(h_m)v_m \\
    & = \lambda_{h_1} v_1 \otimes \cdots \otimes \lambda_{h_m} v_m \\
    & = (\lambda_{h_1} \cdots \lambda_{h_m})v_1 \otimes \cdots \otimes
      v_m \\
    & = (h_1 \cdots h_m)v_1 \otimes \cdots \otimes v_m \\
    & = v_1 \otimes \cdots \otimes v_m
  \end{align*}
  so \(H\) acts trivially on \(V^{\otimes m}\). Thus, consider that
  \(\rho^{\otimes m}\) induces an irreducible representation on \(H\)
  \(G^{\times m}/H\). By the theorem above, \((\dim V)^n \divides
  |G^{\times m}/H| = \frac{|G|^m}{|Z(G)|^{m-1}}\), where \(|H| =
  |Z(G)|^{m-1}\) since \(g_1 \cdots g_m = 1\) constrains a degree of
  freedom. Thus, \(\left(\frac{|G|}{|Z(G)| \dim V}\right)^m \in
  |Z(G)|^{-1} \Z\) for all \(m\) and thus \(\frac{|G|}{|Z(G)| \dim
    V}\) is an integer.
\end{proof}
Another famous result is Burnside's theorem, which can be proved using
representation theory.
\begin{thm}
  Any group \(G\) of order \(p^a q^b\) for primes \(p,q\) and \(a,b
  \geq 0\) is solvable.
\end{thm}
To prove Burnside's theorem, we will use the following results.
\begin{thm}
  Let \(V\) be an irreducible representation of finite group \(G\) and
  let \(C\) be a conjugacy class of \(G\) such that \(\gcd(|C|,\dim V)
  = 1\). Then, for any \(g \in C\), either \(\chi_V(g) = 0\) or \(g\)
  acts as a scalar on \(V\).
\end{thm}
\begin{lem}
  If \(\lambda_1, \lambda_2, \ldots, \lambda_n\) are roots of unity
  such that \(\frac{1}{n}(\lambda_1 + \lambda_2 + \cdots +
  \lambda_n)\) is an algebraic integer, then either \(\lambda_1 =
  \lambda_2 = \cdots = \lambda_n\) or \(\lambda_1 + \cdots + \lambda_n
  = 0\).
\end{lem}
\begin{thm}
  Let \(G\) be a finite group and \(C\) be a conjugacy class in \(G\)
  with \(|C| = p^k\) for \(p\) prime and \(k \in \Z_+\). Then, \(G\)
  is not simple.
\end{thm}
\begin{lem}
  There exists a non-trivial irreducible representation of \(G\), say
  \(V\), with \(p \notdivides \dim V\) such that \(\chi_V(g) \neq 0\).
\end{lem}
\todo{State and prove Burnside's theorem.}
\section{Induced Representations}
Given a representation of a group \(G\), there is a natural way to
restrict the representation to a subgroup \(H \subgroup G\).
\begin{defn}
  Let \(\rho \from G \to GL(V)\) be a representation of \(G\) and \(H \subgroup G\). Then,
  we define the \de{restricted representation} \(\Res_H^G V\) to be \[
    \rho_{\Res_H^G V} = \rho|_H
  \]
\end{defn}
However, we may ask if we can go the other way. That is, given a
representation \(V\) of a subgroup \(H \subgroup G\), can we construct
a representation of \(G\)? We present the following
\begin{defn}
  Let \(\rho \from H \to GL(V)\) be a representation of \(H \subgroup
  G\). Then, we define the \de{induced representation} \(\Ind_H^G V\)
  to be
  \begin{align*}
    \Ind_H^G V & = \{f \from G \to V \st f(hx) = \rho(h)f(x), \forall x
    \in G, h \in H\} \\
    (g.f)(x) & = f(xg), \forall g \in G
  \end{align*}
  Less abstractly, this says that, \(\Ind_H^G V = \bigoplus_{g \in G/H}
  g.V\) where each \(g.V \isom V\) and the action is given by,
  where \(gg_i = g_{\sigma(i)}h_i\) for \(\{g_i\}\) a full set of
  coset representatives, \[
    \rho_{\Ind_H^G V}(g) \cdot \left( \sum_{i = 1}^k g_i v_i \right)
    = \sum_{i=1}^k g_{\sigma(i)} \rho(h_i) v_i
  \]
\end{defn}
\begin{prop}\label{ind-is-tensor}
  Given a representation \(\rho \from H \to GL(V)\) and \(H \subgroup
  G\), we have that \[
    \Ind_H^G V \isom \C[G] \otimes_{\C[H]} V
  \]
  as \(\C[G]\)-modules with \(G\)-action \[
    g'.(e_g \otimes v) = e_{g'g} \otimes v = e_{g''} \otimes \rho(h) v
  \]
  where \(g'g = g''h\) for \(g'' \in G, h \in H\).
\end{prop}
\begin{proof}
  This follows from the fact that \(\C[G] \isom \bigoplus_{g \in G/H}
  g.\C[H]\), so \[
    \C[G] \otimes_{\C[H]} V \isom \left( \bigoplus_{g \in G/H} g.\C[H]
    \right) \otimes_{\C[H]} V \isom \bigoplus g.(\C[H] \otimes_{\C[H]}
    V) \isom \bigoplus_{g \in G/H} g.V
  \]
\end{proof}
\begin{rmk}
  Under this isomorphism, we get \(\Ind_H^G V = \bigoplus_{g \in G/H}
  g.V\) where \(V \isom g.V\) via \(v \mapsto e_g \otimes v\)
\end{rmk}
\todo{Check that all these actions are actually correct}
\begin{example}
  \begin{enumerate}
  \item Let \(H = \{1\} \subgroup G\) and let \(W\) be the trivial
    representation. Then, since \(W = \C\) \[
      \Ind_H^G W \isom \C[G] \otimes_{\C} W \isom \C[G] \otimes_{\C}
      \C \isom \C[G]
    \]
    is the regular representation.
  \item Let \(H \subgroup G\) and \(W\) be the trivial
    representation. Then, \[
      \Ind_H^G W \isom \C[G] \otimes_{\C[H]} \C \isom \C[G/H]
    \]
    since \(g \otimes \lambda = g'h \otimes \lambda = g' \otimes
    h.\lambda = g' \otimes \lambda\) for \(g' \in G, h \in H\).
  \end{enumerate}
\end{example}
\begin{prop}
  Let \(V, V_1, V_2\) be representations of \(H \subgroup G\). Then, 
  \begin{enumerate}
  \item \(\Ind_H^G(V_1 \oplus V_2) \isom \Ind_H^G V_1 \oplus \Ind_H^G
    V_2\).
  \item \(\dim \Ind_H^G V = |G/H| \dim V\).
  \item For \(H \subgroup K \subgroup G\), \(\Ind_K^G \Ind_H^K V \isom
    \Ind_H^G V\).
  \end{enumerate}
\end{prop}
\begin{proof}
  (b) has already been shown in the example above. For (a), \[
    \Ind_H^G(V_1 \oplus V_2) \isom \C[G] \otimes_{\C[H]}(V_1 \oplus
    V_2)  = (\C[G] \otimes_{\C[H]} V_1) \oplus (\C[H] \otimes_{\C[H]}
    V_2) \isom \Ind_H^G V_1 \oplus \Ind_H^G V_2
  \]
  and for (c), \[
    \Ind_K^G \Ind_H^K V \isom \C[G] \otimes_{\C[K]} (\C[K] \otimes_{\C[H]} V) \isom C[G]
    \otimes_{\C[H]} V \isom \Ind_H^G V
  \]
\end{proof}
\begin{defn}
  Let \(\phi \in \Class(H)\) for \(H \subgroup G\). Then, we define
  the \de{induced class function} on \(G\) by \[
    \Ind_H^G(\phi)(g) := \sum_{s \in G/H, s^{-1}gs \in H} \tilde{\phi}(s^{-1}gs) = \frac{1}{|H|}\sum_{t \in
    G} \tilde{\phi}(t^{-1}gt)
  \]
  where \[
    \tilde{\phi}(g) =
    \begin{cases}
      \phi(g) & g \in H \\
      0 & \text{else}
    \end{cases}
  \]
\end{defn}
\begin{example}
  Let \(G = \Sym_3\) and consider \(H = \Z/3\Z \isom \{(1),(123),(132)\} \normsubgroup
  \Sym_3\). Then, since \(\Z/3\Z\) is abelian, it has character
  table \[
    \begin{array}{c|ccc}
      & (1) & (123) & (132) \\
      \hline
      \chi_1 & 1 & 1 & 1 \\
      \chi_{\zeta} & 1 & \zeta & \zeta^2 \\
      \chi_{\zeta^2} & 1 & \zeta^2 & \zeta
    \end{array}
  \]
  where \(\zeta = \zeta_3\), a third root of unity, since \(g^3 = 1\)
  for all \(g \in \Z/3\Z\). We also have, from earlier, the
  irreducible characters of \(\Sym_3\) as \(\chi_1,\chi_2,\theta\)
  (see \ref{char-table-for-S3}). 
  Then, using the fact that \(H\) is normal and \(G/H = \{H, (12)H\}\),
  we see that
  \begin{align*}
    \Ind_H^G(\chi)(h) & = \chi(h) + \chi((12)h(12)) & \forall h \in H
    \\
    & = \chi(h) + \ov{\chi(h)} & \text{ since }(12)(123)(12) = (132)\\
    \Ind_H^G(\chi)((12)) & = \tilde{\chi}((12)) +
                             \tilde{\chi}((12)(12)(12)) \\
    & = 0
  \end{align*}
  And thus
  \begin{align*}
    \Ind_H^G(\chi_1) & = \chi_1 + \chi_2 \\
    \Ind_H^G(\chi_\zeta) & = \theta \\
    \Ind_H^G(\chi_{\zeta^2}) & = \theta
  \end{align*}
  since \(\zeta + \zeta^2 = -1\).
\end{example}
\begin{prop}
  Given \(\phi \in \Class(H)\) for \(H \subgroup G\), we get
  \(\Ind_H^G(\phi) \in \Class(G)\).
\end{prop}
\begin{proof}
  Let \(g,k \in G\), then \[
    \Ind_H^G(\phi)(k^{-1}gk) = \frac{1}{|H|} \sum_{t \in G}
    \tilde{\phi}(t^{-1}k^{-1}gkt) = \frac{1}{|H|} \sum_{t' \in G}
    \phi(t'^{-1}gt') = \Ind_H^G(\phi)(g)
  \]
\end{proof}
\begin{thm}[The Mackey formula]
  The character of the induced representation \(\Ind_H^G V\) is \(\Ind_H^G(\chi_V)\).
\end{thm}
\begin{proof}
  We seek to directly compute the trace of the action of \(s \in
  G\) on \(\Ind_H^G V \isom \bigoplus_{g \in G/H} g.V\). Consider that
  \(sgH = gH \iff g^{-1}sg = t \in H\). We only 
  consider such cosets since all others will not contribute to the
  trace.  On \(g.V\), we
  get \[
    s. (e_g \otimes v) = e_{sg}
    \otimes v = e_{gt} \otimes v = e_g \otimes \rho(t)w
  \]
  so the action of \(s\) on \(g.V\) corresponds to the action of \(t =
  g^{-1}sg\) on \(V\). Thus, the block corresponding to \(g.V\)
  contributes \(\chi_V(g^{-1}sg)\) to the trace of \(\rho_{\Ind_H^G
    V}(s)\). Thus, summing over all such \(g\), we get \[
    \chi_{\Ind_H^G \rho}(s) = \sum_{g \in G/H, g^{-1}sg \in H}
      \chi_V(g^{-1}sg) = \sum_{g \in G/H} \tilde{\chi_V}(g^{-1}sg) = \Ind_H^G(\chi_V)(s)
  \]
\end{proof}
\begin{example}
  Thus, from our example above, this actually tells us the
  decomposition of the induced representations of \(\Sym_3\) from \(\Z/3\Z\).
\end{example}
\section{Frobenius Reciprocity}
We seek to find relationships between \(\Ind\) and \(\Res\) since
\(\Res\) is much easier to compute than \(\Ind\). By far the most
important one is Frobenius reciprocity.
\begin{thm}[Frobenius Reciprocity]
  Let \(H \subgroup G\), a finite group. Then, for \(\phi \in
  \Class(G)\) and \(\psi \in \Class(H)\), we have \[
    (\Ind_H^G \psi, \phi) = (\psi, \Res_H^G \phi)
  \]
  which is to say, \(\Ind_H^G\) and \(\Res_H^G\) are Hermetian adjoint.
\end{thm}
\begin{lem}
  Given \(M\) an \(R\)-module and \(N\) an \(S\)-module and ring
  homomorphism \(f \from R \to S\) for \(R,S\) (unital) rings, then \[
    \Hom_{R}(M,N) \isom \Hom_{S}(S \otimes_{R} M, N)
  \]
  where \(N\) is considered as an \(R\) module via \(r.n = f(r)n\)
  (that is, the restriction of scalars).
\end{lem}
\begin{proof}[Proof of Lemma]
  Given \(u \in \Hom_R(M,N)\), let \(F \from \Hom_R(M,N) \to \Hom_S(S
  \otimes_R M,N)\) be defined by sending \(u\) to the composition
  \begin{align*}
    S \otimes_R M \to[id_S \otimes u] S \otimes_R N & \to N \\
    s \otimes n &\mapsto sn
  \end{align*}
  Then, \(F(u)\) is an \(S\)-module homomorphism and so \(F\) is
  well-defined and a homomorphism of abelian groups. We now construct
  and inverse to \(F\). Let \(G \from \Hom_S(S \otimes_R M, N) \to
  \Hom_R(M,N)\) be defined by sending \(v \in \Hom_S(S \otimes_R M,
  N)\) to the composition
  \begin{align*}
    M & \to R \otimes_R M \to[f \otimes id_M] S \otimes_R M \to[v] N
    \\
    m & \mapsto 1 \otimes m
  \end{align*}
  One can check that \(F,G\) are inverse to each other.
\end{proof}
\begin{rmk}
  One can also check that this isomorphism depends only on the ring homomorphism
  \(f\). Thus, such an isomorphism is functorial and so the extensions
  of scalars, \(S \otimes -\), is left adjoint to the restriction of
  scalars functor.
\end{rmk}
\begin{proof}[Proof of Frobenius Reciprocity]
  Frobenius reciprocity ends up being a special case of the
  lemma. That is,
  \begin{align*}
    (\Ind_H^G \chi_V, \chi_W)
    & = \dim \Hom_G(\C[G] \otimes_{\C[H]} V,
      W) & \text{by }\ref{orthonormal}\\
    & = \dim \Hom_H(V,W) & \text{by lemma} \\
    & = (\chi_V, \Res_H^G \chi_W)
  \end{align*}
\end{proof}
Frobenius reciprocity is useful for computing various representations
of groups from representations of their subgroups.
\begin{example}
  Consider \(\Z/2\Z \isom \langle (12) \rangle \subgroup
  \Sym_3\). Then, \(\Z/2\Z\) has two irreducible 1-dimensional
  characters, let us say \(\chi_1\) and \(\chi_{-1}\). Then,
  \((\Ind_{\Z/2\Z}^{\Sym_3} \chi_1, \chi_V) = (\chi_1,
  \Res_{\Z/2\Z}^{\Sym_3} \chi_V)\) for any irreducible representation
  \(V\) of \(\Sym_3\). Thus,
  \begin{align*}
    (\Ind_{\Z/2\Z}^{\Sym_3} \chi_1, \chi_{trivial}) & = (\chi_1,
    \Res_{\Z/2\Z}^{\Sym_3} \chi_{trivial}) = (\chi_1, \chi_1) = 1 \\
    (\Ind_{\Z/2\Z}^{\Sym_3} \chi_1, \chi_{alt}) & = (\chi_1,
    \Res_{\Z/2\Z}^{\Sym_3} \chi_{alt}) = (\chi_1, \chi_{-1}) = 0 \\
    (\Ind_{\Z/2\Z}^{\Sym_3} \chi_1, \chi_{std}) & = (\chi_1,
  \Res_{\Z/2\Z}^{\Sym_3} \chi_{std})  = (\chi_1, \chi_1 + \chi_{-1}) =
    1 \\
    (\Ind_{\Z/2\Z}^{\Sym_3} \chi_{-1}, \chi_{trivial}) & = (\chi_{-1},
    \Res_{\Z/2\Z}^{\Sym_3} \chi_{trivial}) = (\chi_{-1}, \chi_1) = 0 \\
    (\Ind_{\Z/2\Z}^{\Sym_3} \chi_{-1}, \chi_{alt}) & = (\chi_{-1},
    \Res_{\Z/2\Z}^{\Sym_3} \chi_{alt}) = (\chi_{-1}, \chi_{-1}) = 1 \\
    (\Ind_{\Z/2\Z}^{\Sym_3} \chi_{-1}, \chi_{std}) & = (\chi_{-1},
  \Res_{\Z/2\Z}^{\Sym_3} \chi_{std}) = (\chi_{-1}, \chi_1 + \chi_{-1}) =
    1
  \end{align*}
  Thus, \(\Ind_{\Z/2\Z}^{\Sym_3} \chi_1 = \chi_{trivial} +
  \chi_{std}\) and \(\Ind_{\Z/2\Z}^{\Sym_3} \chi_{-1} = \chi_{alt} + \chi_{std}\).
\end{example}
\section{Mackey Theory}
To gain more insight into induced representations, we wish to
understand \(\Res_K^G \Ind_H^G V\) for \(H,K \subgroup G\) and \(V\) a
representation of \(G\). To describe our results, we will need the
language of double cosets.
\begin{defn}
  Let \(H,K \subgroup G\), a group. The set of \de{double cosets}
  \(K\bs G /H\) is the set of \(K \times H\)-orbits on
  \(G\) where \((k,h).g = kgh\). Often, elements of such a set are
  denoted \(KgH\) for \(g \in G\), a representative.
\end{defn}
\begin{rmk}
  If \(K=H \normsubgroup G\), then \(K \bs G / H = G/H\). Also note
  that there are other definitions of double cosets.
\end{rmk}
\begin{defn}
  Following \cite{serre}, for \(H,K \subgroup G\) and \(s \in S\), a set
  of double-coset representatives, let \[
    H_s := sHs^{-1} \intersect K \subgroup K
  \]
  and, given representation \(\rho \from H \to GL(V)\), let \(\rho_s
  \from H_s \to GL(V_s)\) given by \[ 
    \rho_s(x) := \rho(s^{-1}xs)
  \]
  be a representation of \(H_s\). In other words, \(V_s \isom s
  \otimes V\) with action of \(H_s\) given by \[
    (shs^{-1})(s \otimes v) = s \otimes hv
  \]

\end{defn}
\begin{rmk}
  Note that \(H_s\) is the stabilizer of \(sH\) under the action of
  \(K\) on \(G/H\). Furthermore, if one embeds \(H_s \into H\) via \(x
  \mapsto s^{-1}xs\), then \(\im H_s\) is the stabiliser of \(Ks\)
  under the action of \(H\) on \(H \bs G\).
\end{rmk}
\begin{prop}
  Given \(S\), a set of double coset representatives for \(K \bs G /
  H\), then \[
    \Res_K^G \Ind_H^G V \isom \bigoplus_{s \in S} \Ind_{H_s}^K V_s
  \]
\end{prop}
\begin{proof}
  Let us consider \(\C[G]\) as a \((\C[K], \C[H])\)-bimodule. Then, \[
    \C[G] \isom \bigoplus_{s \in S} \C[KsH]
  \]
  Note, too, this viewpoint gives the isomorphism as \((\C[K],\C[H])\)-bimodules
  \begin{align*}
    \C[KsH] & \isom \C[K] \otimes_{\C[H_s]} (s \otimes \C[H]) \\
    ksh & \mapsto k \otimes s \otimes h
  \end{align*}
  Thus, using this decomposition, we get the following isomorphisms of \(\C[K]\)-modules
  \begin{align*}
    \Res_K^G \Ind_H^G V & \isom \Res_K^G \left( \C[G] \otimes_{\C[H]}
                          V \right) \\
                        & \isom \bigoplus_{s \in S} \C[KsH] \otimes_{\C[H]} V \\
    & \isom \bigoplus_{s \in S} \C[K] \otimes_{\C[H_s]} (s
      \otimes \C[H] ) \otimes_{\C[H]} V \\
    & \isom \bigoplus_{s \in S} \C[K] \otimes_{\C[H_s]}
      V_s \\
    & \isom \bigoplus_{s \in S} \Ind_{H_s}^K V_s 
  \end{align*}  
\end{proof}
\begin{thm}[Mackey's Irreducibility Criterion]
  \(\Ind_H^G V\) is irreducible if and only if
  \begin{enumerate}
  \item \(V\) is irreducible and
  \item \(V_s\) and \(\Res_{H_s}^H V\) share no irreducibles for all \(s \in
    S\), a set of double coset representatives of \(K \bs G / H\).
  \end{enumerate}
\end{thm}
\begin{proof}
  We wish to show that \((\Ind_H^G V, \Ind_H^G V) = 1\).  We compute
  \begin{align*}
    (\Ind_H^G V, \Ind_H^G V)
    & = (V, \Res_H^G \Ind_H^G V)
    & \text{ by Frobenius Reciprocity}\\
    & = (V, \bigoplus_{s \in H \bs G / H} \Ind_{H_s}^H(\rho_s))
    & \text{ by proposition above}\\
    & = \sum_{s \in H \bs G / H} (\Res_{H_s}^H(\rho), \rho_s)
    & \text{ by Frobenius reciprocity}
  \end{align*}
  Note that the last equality also makes use of the fact \[
    \Hom(A,\bigoplus_{i \in I} B_i) =
    \bigoplus_{i \in I} \Hom(A,B_i) \text{ for finite index set } I
  \]
  Now, when \(s = 1\), \((\Res_H^H \rho,\rho_1) = (\rho, \rho) \geq
  1\). Thus, for our sum to be exactly \(1\), it must be that \[
    (\rho, \rho) = 1 \text{ and } (\Res_{H_s}^H \rho, \rho_s) = 0 \text{ for } s \neq 1
  \]
  This will happen precisely when \(\rho\) is irreducible and
  \(\Res_{H_s}^H \rho\) and \(\rho_s\) share no irreducibles.
\end{proof}
\begin{cor}
  If \(H \normsubgroup G\), then \(\Ind_H^G V\) is irreducible if and
  only if \(V\) is irreducible and \(V \not \isom V_s\) for all \(s \in G/H\)
\end{cor}
\begin{proof}
  If \(H \normsubgroup G\), then \(H_s = H\) and \(\Res_{H_s}^H \rho =
  \rho\), so it must be that \(\rho\) is irreducible and not
  isomorphic to any \(\rho_s\), otherwise \(\rho_s\) and
  \(\Res_{H_s}^H \rho = \rho\) will share an irreducible.
\end{proof}
\begin{example}
  Consider \(\langle r,s \st r^4 = s^2 = 1, sr = r^{-1}s \rangle \isom
  D_8 \isom \Z_4 \rtimes \Z_2\). Then, we know \(\langle r \rangle
  \isom \Z_4
  \normsubgroup D_8\) with character table \[
    \begin{array}{c|cccc}
      &1&r&r^2&r^3 \\
      \hline
      \chi_1 & 1 & 1 & 1 & 1 \\
      \chi_2 & 1 & i & -1 & -i\\
      \chi_3 & 1 & -1 & 1 & -1 \\
      \chi_4 & 1 & -i & -1 & i
    \end{array}
  \]
  If \(\rho^i\) is the irreducible representation of \(\Z_4\)
  corresponding to \(\chi_i\), then we notice that \(\tr \rho^3_s(r) =
  \tr \rho^3(srs) = \tr \rho^3(r^{-1}) = -1 = \tr \rho^3(r)\), so
  \(\Ind_{\Z_4}^{D_8} 
  \rho^3\) will \emph{not} be irreducible. On the other hand, one can
  check \(\tr \rho^2_s(r) = \tr \rho^2(srs) = \tr \rho^2(r^{-1}) = -i
  = \tr \rho^4(r)\), so
  \(\rho^2_s\) and \(\rho^2\) are orthogonal, thus sharing no
  irreducibles. Thus, it must be that \(\Ind_{\Z_4}^{D_8} \rho^2\) is
  irreducible. In fact, one can check \(\Ind_{\Z_4}^{D_8} \rho^2 =
  \Ind_{\Z_4}^{D_8} \rho^4\) will give the only irreducible
  \(2\)-dimensional representation of \(D_8\).
\end{example}
\section{Representations of Nilpotent Groups}
\begin{thm}
  Given an irreducible representation of a nilpotent group \(G\), it is induced
  from a one-dimensional representation of some subgroup \(H \subgroup G\).
\end{thm}
\begin{rmk}
  This is a specific case of Brauer's theorem (see
  \ref{brauers-thm}). We will probe this theorem using the program
  given in \cite{ct}.
\end{rmk}
\begin{lem}
  Given finite group \(G\) with \(N \normsubgroup G\) and irreducible
  representation \(V\), then \(V = \Ind_H^G W\) for some \(N \subgroup H
  \subgroup G\) and \(\Res_N^H W \isom U^{\oplus r}\) for \(U\) an irreducible
  representation of \(N\), \(r \in \N\).
\end{lem}
\begin{proof}
  Let \(\Res_N^G V = \bigoplus_i V_i\) for \(V_i\) irreducible \(N\)
  representations. If there is a single summand, then \(H=G\) and we
  are done. Otherwise, given \(\rho \from N \to GL(V)\), note that
  \(\rho_s(n) := \rho(s^{-1}ns)\) for \(n \in N\), \(s \in G\) will
  permute the summands of \(\Res_N^G V\) since conjugation defines an
  automorphism of the normal subgroup \(N\), and so must permute the
  irreducible characters of \(N\). Thus, this conjugation action gives
  us an isomorphism of \(N\)-representations. Furthermore, this
  permutation action by \(G\) must be transitive since we assumed that \(V\)
  has no \(G\)-invariant subspaces and so each non-zero \(V_i\) has
  the same dimension. \\

  Now choose a non-zero block \(V_k\) and let \(H =
  \Stab_G(V_k)\). Then, it must be that \(\dim V = \frac{|G|}{|H|}
  \dim V_k = \dim \Ind_H^G V_k\) by the orbit-stabilizer theorem. Furthermore, we get \[
    \dim \Hom_G(\Ind_H^G V_k, V) = \dim \Hom_H(V_k, \Res_H^G V) \geq 1
  \]
  Thus, since \(V\) is irreducible as a \(G\)-representation and
  \(\dim V = \dim \Ind_H^G V_k\), it must be that \[
    (\Ind_H^G V_k, V) = \dim \Hom_G(\Ind_H^G V_k, V) = 1
  \]
  and so \(V = \Ind_H^G V_k\).
\end{proof}
\begin{cor}
  If \(N \normsubgroup G\) is abelian, then either its action on \(V\)
  is a scalar or \(V = \Ind_H^G W\) for some proper subgroup \(H
  \propsubgroup G\).
\end{cor}
We will also use the following (standard) group theory fact.
\begin{lem}
  If \(G\) is nilpotent, then it contains a characteristic, abelian, non-central
  subgroup.
\end{lem}
\begin{proof}[Proof of Theorem]
Let \(\rho \from G \to GL(V)\) be an irreducible representation of a
nilpotent group \(G\). Then, \(G/\ker \rho\) is nilpotent. If \(G/\ker
\rho\) is abelian, then \(\ov{\rho} \from G/\ker \rho \to GL(V)\) is
one-dimensional and so \(\rho \from G \to GL(V)\) must be as
well. Otherwise, take \(A \normsubgroup G/\ker \rho\) to be a
characteristic, abelian, and non-central subgroup given by the lemma
above. Then, for non-central element \(a \in A\), \(\ov{\rho}(a)\)
cannot be a scalar.

Thus, from the corollary above, there is a proper subgroup \(H
\propsubgroup G\) with a representation \(\rho' \from H/\ker \rho \to
GL(W)\) such that \(\ov{\rho} = \Ind_{H/\ker \rho}^{G/\ker \rho} \rho'
= \Ind_H^G \tilde{\rho'}\) where \(\tilde{\rho'} \from H \to GL(W)\)
is a lift of \(\rho'\). Since \(H\) is also nilpotent, we can repeat
this argument until we find a \(1\)-dimensional representation \(\rho'\).
\end{proof}
\section{Artin's Theorem}
Artin's theorem allows us to gain more insight into how induced
characters or representations may be related to the irreducible
representations of a finite group.
\begin{defn}
  Let \(G\) be a finite group. A \de{virtual representation} of \(G\)
  is an integer linear combination of irreducible representations of
  \(G\). 
\end{defn}
\begin{prop}
  Let \(V\) be an virtual representation with character \(\chi_V\). If
  \((\chi_V,\chi_V)=1\) and \(\chi_V(1) > 0\), then \(\chi_V\) is a
  character of an irreducible representation of \(G\).
\end{prop}
\begin{proof}
  Let \(V = \sum n_i V_i\) for irreducible representations \(V_i\) and
  \(n_i \in \Z\). Then, using the fact that the irreducible characters
  form an orthonomal basis, we get \(1 = (\chi_V,\chi_V) = \sum
  n_i^2\). Thus, \(n_i = \pm 1\) for one \(i\) and \(0\)
  otherwise. However, \(\chi_V(1) > 0\) by assumption, so \(n_i = 1\).
\end{proof}
\begin{thm}[Artin's theorem]
  Let \(X\) be a conjugation-invariant family of subgroups of a finite
  group \(G\). Then, the following are equivalent.
  \begin{enumerate}
  \item Any element of \(G\) belongs to a subgroup \(H \in X\), that
    is, \(G = \Union_{H \in X} H\).
  \item The character of any irreducible representation of \(G\)
    belongs to the \(\Q\)-span of characters of induced
    representations \(\Ind_H^G V\) where \(H \in X\) and \(V\) is an
    irreducible representation of \(H\). 
  \end{enumerate}
\end{thm}
\begin{rmk}
  Part (b) is equivalent to the statement that \(\coker \Ind\) is a
  finite group where \(\Ind\) is defined by
  \begin{align*}
    \Ind \from \bigoplus_{H \in X} R(H) & \to R(G) \\
    (\theta_H)_{H \in X} & \mapsto \sum_{H \in X} \Ind_H^G \theta_H
  \end{align*}
  For \(R(H)\) the ring of virtual characters of \(H\) and \(R(G)\)
  similarly defined. In fact, this map can be represented by a
  matrix \[
    \Ind = \left[ (W_i, \Ind_H^G V_{H,j}) \right]
  \]
  where \(W_i\) ranges over all irreducible representations of \(G\)
  and \(V_{H,j}\) ranges over all irreducible representations of \(H\)
  for al \(H \in X\). Thus, the cokernel is a finite group if and only
  if the rows of this matrix are linearly independent.
\end{rmk}
\begin{proof}
  For \((b) \implies (a)\), let \(S = \Union_{H \in X} H\). Then, take
  \(g \in G \setminus S\). Since \(S\) is conjugation invariant, no
  conjugate of \(g\) is in \(S\). So, by the Mackey formula,
  \(\chi_{\Ind_H^G V}(g) = 0\) for all \(H \in X\) and \(V\). Thus,
  by assumption, for any irreducible representation \(W\) of \(G\) is
  a \(\Q\)-span of characters of induced representations, so
  \(\chi_W(g) = 0\). However, irreducible characters also span the
  class functions, so this would mean that any class function would
  vanish of \(g\), a contradiction. \\

  For \((a) \implies (b)\), let \(U\) be a virtual representation of
  \(G\) such that \((\chi_U, \chi_{\Ind_H^G V}) = 0\) for all \(H\)
  and \(V\). Then, by Frobenius reciprocity, we get \((\chi_{\Res_H^G
    U}, \chi_V) = 0\). Thus, \(\chi_U\) vanishes on \(H\) for any \(H
  \in X\) and so \(\chi_U \identically 0\). \todo{Expand on this}
\end{proof}
\begin{cor}
  Any irreducible character of a finite group is a rational linear
  combination of induced characters from its cyclic subgroups.
\end{cor}
\begin{example}
  Let \(G = A_4\). Then, \(|A_4| = 12\). Let \(H = \langle 1,
  (123),(321) \rangle\), \(K = \langle 1, (12)(34) \rangle\). Then,
  given characters of \(A_4\), \(\chi_1, \chi_{\zeta},
  \chi_{\zeta^2}\), and \(\theta\), characters of \(H\), \(\chi_1,
  \chi_{\zeta}, \chi_{\zeta^2}\), and characters of \(K\), \(\chi_1,
  \chi_{-1}\), one can use Frobenius reciprocity 
  to check
  \begin{align*}
    \Ind_H^G \chi_1 & = \chi_1 + \theta \\
    \Ind_H^G \chi_{\zeta} & = \chi_{\zeta} + \theta \\
    \Ind_H^G \chi_{\zeta^2} & = \chi_{\zeta^2} + \theta \\
    \Ind_K^G \chi_1 & = \chi_1 + \chi_{\zeta} + \chi_{\zeta^2} + \theta \\
    \Ind_K^G \chi_{-1} & = 2 \theta
  \end{align*}
  and so
  \begin{align*}
    \chi_1 & = \Ind_H^G \chi_1 - \frac{1}{2} \Ind_K^G \chi_{-1} \\
    \chi_\zeta & = \Ind_H^G \chi_\zeta - \frac{1}{2} \Ind_K^G
                 \chi_{-1} \\
    \chi_{\zeta^2} & = \Ind_H^G \chi_{\zeta^2} - \frac{1}{2} \Ind_K^G
                     \chi_{-1} \\
    \theta & = \frac{1}{2} \Ind_K^G \chi_{-1}
  \end{align*}
  Thus, \(A_4 = \Union_{g \in A_4} gHg^{-1} \union \Union_{g \in A_4} gKg^{-1}\).
\end{example}
\section{Brauer's Theorem}
While Artin's theorem gives us a way to write characters as a rational
linear combination of characters induced from cyclic subgroups, the
use of the rational numbers leaves something to be desired. We would
rather have integral linear combinations. In fact, if we are willing
to look a little more broadly than cyclic groups, we can arrive at
such a decomposition.
\begin{defn}
  Let \(p\) be a prime. A group \(G\) is called \de{\(p\)-elementary}
  if \[
    G \isom Z \times P
  \]
  where \(Z\) is a cyclic group of order prime to \(p\), (that is
  \(\gcd(p,|Z|)=1\)), and \(P\) is a \(p\)-group (i.e. \(|P| =
  p^r\)). \(G\) is \de{elementary} if it is \(p\)-elementary for some
  prime \(p\).
\end{defn}
\begin{thm}[Brauer's Theorem]\label{brauers-thm}
  Every complex character is a \(\Z\)-linear combination of linear
  characters of elementary subgroups \(P\) induced to \(G\), that
  is \[
    \chi = \sum_P c_P \Ind_P^G \xi_p, \ c_P \in \Z
  \]
  where \(\xi_P\) is a linear character of \(P\).
\end{thm}
However, the proof of Brauer's theorem is not so straight-forward. It
requires use of the following theorem.
\begin{thm}
  Let \(G\) be a finite group and let \(V_p\) be the subgroup of
  \(R(G)\) generated by characters induced from those of
  \(p\)-elementary subgroups of \(G\). Then, \([R(G):V_p] < \infty\) 
  and \(\gcd([R(G):V_p],p) = 1\).
\end{thm}
\todo{Actually prove Brauer's theorem.}
\section{Example: Representations of the Symmetric Group}
Elements of the symmetric group \(\Sym_n\) can be represented in
\(1\)-line notation as products of disjoint cycles, and to each
element, we can assign a \de{cycle type} in the form of a partition of
\(n\).
\begin{example}
  Consider \((12345)(876)(9,10) \in \Sym_{10}\). This cycle has cycle
  type \((5,3,2) \partitionof 10\). 
\end{example}
Partitions are useful combinatorial tools with many applications
beyond what is described here. We can also define a partial order on
tableau.
\begin{defn}
  Given partitions \(\lambda = (\lambda_1, \lambda_2,
  \ldots) \partitionof n\) and \(\mu = (\mu_1, \mu_2, \ldots)\), we
  say that \(\lambda \leq \mu\) if \[
    \begin{cases}
      \lambda_1 \leq \mu_1 \\
      \lambda_1 + \lambda_2 \leq \mu_1 + \mu_2 \\
      \vdots\\
      \lambda_1 + \cdots + \lambda_k \leq \mu_1 + \cdots + \mu_k \\
      \vdots
    \end{cases}
  \]
\end{defn}
\begin{example}
  Using the notation that \((2^2,1^2) = (2,2,1,1)\), the following
  partial order is induced on the partitions of \(6\). 
\[  \begin{tikzcd}[row sep=tiny, column sep=tiny]
    & (6) \ar[d] & \\
    & (5,1) \ar[d] & \\
    & (4,2) \ar[ld] \ar[rd] & \\
    (3,3) \ar[rd] & & (4,1^2) \ar[ld] \\
    & (3,2,1) \ar[ld] \ar[rd] & \\
    (3,1^3) \ar[rd]& & (2^3) \ar[ld] \\
    & (2^2, 1^2) \ar[d] & \\
    & (2,1^4) \ar[d] & \\
    & (1^6) &
  \end{tikzcd}
\]
\end{example}


It is also useful to realize partitions of \(n\) as Young
diagrams. Once again, we shall define the correspondance via an
example.
\begin{example}
  We can represent a partition \((\lambda_1, \lambda_2, \ldots)\) as a
  \de{Young diagram} with \(\lambda_i\) boxes on the \(i\)th row. So,
  consider \((5,3,2) \partitionof 10\). The corresponding diagram
  would be \[
    \ydiagram{5,3,2}
  \]
  Furthermore, we can fill in the boxes of the diagram to get a \de{Young
    tableau}. For example \[ 
    \begin{ytableau}
      1 & 2 & 3 & 4 & 5 \\
      6 & 7 & 8 \\
      9 & 10
    \end{ytableau}
  \]
  Furthermore, \(\Sym_n\) can act on a Young tableau by permuting the
  numbers. So, \((12345)(876)(9,10) \in \Sym_{10}\) would act on the
  above diagram to yield \[
    \begin{ytableau}
      2 & 3 & 4 & 5 & 1\\
      8 & 6 & 7 \\
      10 & 9
    \end{ytableau}
  \]
\end{example}
Now, using this combinatorial device, we can define the following.
\begin{defn}
  Let \(\T\) be a tableau of shape \(\lambda \partitionof n\) for \(n
  \in \N\). Then, we define the \de{row stabalizer subgroup}
  \(R(\T) \subgroup \Sym_n\) to be the subgroup of
  permutations such that
  every permutation in \(R(\T)\) preserves the elements in
  the rows of \(\T\). One analogously defines \(C(\T)\) to be the
  \de{column stabalizer subgroup}.
\end{defn}
\begin{example}
  In the example above, \((12345)(876)(9,10)\) is in the row
  stabalizer of the tableau. In fact, in that setup \(R(\T) \isom \Sym_5 \times
  \Sym_3 \times \Sym_2\). 
\end{example}
Now, given \(\Sym_n\), we automatically know of \(2\) irreducible,
\(1\)-dimensional, complex representations, namely the trivial
representation 
and the sign/alternating representation. Furthermore, we know that
conjugacy classes of \(\Sym_n\) are encoded by cycle type, which are
encoded by partitions of \(n\). So, since character tables are square,
there must be as many irreducible representations of \(\Sym_n\) as
there are partitions of \(n\). Thus, we have \[
  \{\text{Partitions of }n\} \onetoonecorrespondance
  \{\text{Irreducible representations of }\Sym_n\}
\]
Our task, then, is to figure out what these irreducible
representations are. A systematic treatment will not be given here,
but some of the tools in the previous chapters will be used to get an
idea of some results about the representation theory of symmetric
groups. For more, see \cite{james}, \cite{fulton}.
\begin{defn}
  We define the \de{row symmetrizer} of a tableau \(\T\) of shape
  \(\lambda\) to be \[
    P(\T) := \sum_{\alpha \in R(\T)} \alpha
  \]
  and the \de{column symmetrizer} to be \[
    Q(\T) := \sum_{\alpha \in C(\T)} \sgn(\alpha) \alpha
  \]
\end{defn}
\todo{Finish this to give some idea of irreducibles.}
\begin{bibdiv}
  \begin{biblist}
    \bib{benson}{book}{
      author={Benson, D.J.}
      title={Representations and Cohomology I}
      year={1995}
    }
    \bib{cr}{book}{
      author={Curtis, Charles W.}
      author={Reiner, Irving}
      title={Representation Theory of Finite Groups and Associative
        Algebras}
      year={1962}
    }
    \bib{etingof}{article}{
      author={Etingof, Pavel}
      author={Golberg, Oleg}
      author={Hensel, Sebastian}
      author={Liu, Tiankai}
      author={Schwendner, Alex}
      author={Vaintrob, Dmitry}
      author={Yudovina, Elena}
      title={Introduction to Representation Theory}
      year={2011}
      note={\url{http://math.mit.edu/~etingof/replect.pdf}}
    }
    \bib{fulton}{book}{
      author={Fulton, Wiliam}
      title={Young Tableaux}
      year={1997}
    }
    \bib{princeton-companion}{article}{
      author={Gronjnowski, Ian}
      title={Representation Theory}
      journal={The Princeton Companion to Mathematics}
      pages={419--431}
    }
    \bib{james}{book}{
      author={James, G. D.}
      title={The Representation Theory of the Symmetric Groups}
      year={1978}
    }
    \bib{aw}{article}{
      author={Seelinger, George H.}
      title={Artin-Wedderburn Theory}
      year={2017}
      note={See \url{https://github.com/ghseeli/grad-school-writings/releases/latest}}
    }
    \bib{serre}{book}{
      author={Serre, Jean-Pierre}
      title={Linear Representations of Finite Groups}
      year={1997}
      note={Translated from the French by Leonard L. Scott}
    }
    \bib{smith}{article}{
      author={Smith, Karen}
      title={Groups and their Representations}
      year={2010}
      note={\url{http://www.math.lsa.umich.edu/~kesmith/rep.pdf}}
    }
    \bib{ct}{article}{
      author={Teleman, Constantin}
      title={Representation Theory}
      year={2005}
      note={\url{https://math.berkeley.edu/~teleman/math/RepThry.pdf}}
    }
  \end{biblist}
\end{bibdiv}

\end{document}