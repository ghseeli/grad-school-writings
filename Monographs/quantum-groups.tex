\documentclass[11pt,leqno,oneside]{amsart}
\usepackage[alphabetic,abbrev]{amsrefs} % use AMS ref scheme
\usepackage{../ReAdTeX/readtex-core}
\usepackage{../ReAdTeX/readtex-dangerous}
\usepackage{../ReAdTeX/readtex-abstract-algebra}
\usepackage{../ReAdTeX/readtex-lie-algebras}
\usepackage{./monographs}
\usepackage{caption}
\usepackage{subcaption}
\usepackage{todonotes}
\usepackage{tikz-cd}
\usepackage{tikz}
\numberwithin{thm}{section}

\newcommand{\rootlattice}{Q}
\newcommand{\weightlattice}{P}
\renewcommand{\simpleroots}{\Pi}
\newcommand{\T}{T}
\newcommand{\U}{\mathcal{U}}
\renewcommand{\b}{\mathfrak{b}}
\newcommand{\ch}{\operatorname{ch}}
\newcommand{\halfsum}{\rho}
\newcommand{\numposrootcombos}{\mathfrak{P}}
\newcommand{\qfactorial}[1]{[#1]_q!}
\newcommand{\qbinom}[3][q]{\binom{#2}{#3}_{#1}}
\newcommand{\A}{\mathbb{A}}
\DeclareMathOperator{\diag}{diag}
\DeclareMathOperator{\wt}{wt}

\title[Notes on Quantum Groups]{Notes on Quantum Groups}
\author{George H. Seelinger}
\date{November 2018}
\begin{document}
\maketitle
\section{Basics}
\subsection{Definitions}
\begin{defn}
  For an indeterminate \(q\) and \(n \in \Z\), we define
  \begin{enumerate}
  \item \([n]_q := \frac{q^n-q^{-n}}{q-q^{-1}}\)
  \item \(\qfactorial{0} := 1\) and \(\qfactorial{n} := [n]_q [n-1]_q
    \cdots [1]_q\) for \(n \in \Z_{\geq 0}\)
  \item If \(m \geq n \geq 0\), then \[
      \qbinom{m}{n} = \frac{\qfactorial{m}}{\qfactorial{n}\qfactorial{m-n}}
    \]
  \end{enumerate}
\end{defn}
\begin{prop}
  We have the identity \[
    \qbinom{m+1}{n} = q^n \qbinom{m}{n} + q^{-m+n-1} \qbinom{m}{n-1}
  \]
  and that both \([n]_q\) and \(\qbinom{m}{n}\) are elements of
  \(\Z[q,q^{-1}]\) 
\end{prop}
\begin{proof}
  Observe that \[
    [n]_q = \frac{q^n-q^{-n}}{q-q^{-1}} = \frac{q}{q^n}\cdot
    \frac{q^{2n}-1}{q^2-1} =
    \frac{1}{q^{n-1}}(q^{2n-2}+q^{2n-4}+\cdots+q^2+1) \in \Z[q,q^{-1}]
  \]
  This immediately gives that \(\qfactorial{n} \in \Z[q,q^{-1}]\). In \
\end{proof}
The idea of these definitions is that they are a ``\(q\)-deformation''
of the integer \(n\) and the binomial expression. To see this idea,
we remark that \[
  [n]_q \to n \text{ and }\qbinom{m}{n} \to \binom{m}{n} \text{ as }q
  \to 1
\]
\begin{defn}
  Throughout, let \((A, \simpleroots, \simpleroots^\vee, \weightlattice,
  \weightlattice^\vee)\) be Cartan associated to \(A\) where \(A\) is
  a symmetrizable generalized Cartan matrix with a symmetrizing matrix
  \(D = \diag(s_i \in \Z_{\geq 0} \st i \in I)\).
\end{defn}
\begin{defn}
  The \de{quantum group} \(\U_q(\g)\) associated with Cartan datum
  \((A, \simpleroots, \simpleroots^\vee, \weightlattice,
  \weightlattice^\vee)\) is the associative algebra with \(1\) over
  \(F(q)\) generated by elements \(e_i, f_i\) for \(i \in I\) and
  \(K_\mu\) for \(\mu \in \weightlattice^\vee\) with the following
  relations
  \begin{enumerate}
  \item \(K_\mu = 1, K_{\mu} K_{\mu'} = K_{\mu+\mu'}\) for \(\mu,\mu' \in
    \weightlattice^\vee\)
  \item \(K_\mu e_i K_{-\mu} = q^{\alpha_i(\mu)}e_i\) for \(\mu \in
    \weightlattice^\vee\) 
  \item \(K_\mu f_i K_{-\mu} = q^{-\alpha_i(\mu)}f_i\) for \(\mu \in
    \weightlattice^\vee\)
  \item \(e_i f_j - f_j e_i = \delta_{ij} \frac{K_{s_i h_i} - K_{-s_i
        h_i}}{q^{s_i} - q^{-s_i}}\) for \(i,j \in I\)
  \item \(\sum_{k=0}^{1-a_{ij}} (-1)^k \qbinom[K_{s_i}]{1-a_{ij}}{k}
    e^{1-a_{ij}-k}e_j e_i^k = 0\) for \(i \neq j\)
  \item \(\sum_{k=0}^{1-a_{ij}} (-1)^k \qbinom[K_{s_i}]{1-a_{ij}}{k}
    f^{1-a_{ij}-k}f_j f_i^k = 0\) for \(i \neq j\)
  \end{enumerate}
\end{defn}
\begin{example}
  Let \(\g = \sl_2\). Then, \(\U_q(\sl_2) = \langle e,f,K_h,K_h^{-1}
  \rangle\). Since the Cartan matrix for \(\sl_2\) is just \(A =
  (2)\), which is already symmetric, we will take \(D = Id\). Then, many relations simplify. For instance, \[
    K_h e K_h^{-1} = q e, K_h f K_h^{-1}= q^{-1} f, ef-fe = \frac{K_{h}-K_{h}^{-1}}{q-q^{-1}}
  \]
  Since \(\weightlattice^\vee\) is spanned only by \(h\), we will
  usually just write \(K := K_h\) when working in \(\U_q(\sl_2)\).
\end{example}
\begin{example}
  Now, consider \(\U_q(\hat{\sl}_2)\), which has Cartan matrix \[
    A = \left(
      \begin{array}{cc}
        2&-2\\
        -2&2
      \end{array}
    \right)
  \]
  which will have generators \[
    \U_q(\hat{\sl}_2) = \langle e_0, e_1, f_0, f_1, K_0^{\pm}, K_1^{\pm} \rangle
  \]
  where we lazily encode \(K_i := K_{h_i}\) and relations
  \begin{enumerate}
  \item \[
      \begin{cases}
        K_i^\pm e_i K_i^\mp = q^2 e_i & i = 0,1\\
        K_i^\pm e_j K_i^\mp = q^{-2} e_i & i \neq j
      \end{cases}
    \]
  \item \[
      \begin{cases}
        K_i^\pm f_i K_i^\mp = q^{-2} f_i & i = 0,1\\
        K_i^\pm f_j K_i^\mp = q^{2} f_i & i \neq j
      \end{cases}
    \]
  \item \[
      \begin{cases}
        e_i f_j = f_j e_i & i \neq j\\
        e_i f_i - f_i e_i = \frac{K_i - K_i^{-1}}{q-q^{-1}} & i = 0,1
      \end{cases}
    \]
  \item \[
      \begin{cases}
        e_0^3 e_1 - (q^2+1+q^{-2}) e_0^2 e_1 e_0 + (q^2+1+q^{-2}) e_0
        e_1 e_0^2 - e_1 e_0^3 = 0\\
        e_1^3 e_0 - (q^2+1+q^{-2}) e_1^2 e_0 e_1 + (q^2+1+q^{-2}) e_1
        e_0 e_1^2 - e_0 e_1^3 = 0
      \end{cases}
    \]\[
      \implies
      \begin{cases}
        e_0^3 e_1 = e_1 e_0^3  + (q^2+1+q^{-2}) e_0^2 e_1 e_0 -
        (q^2+1+q^{-2}) e_0 e_1 e_0^2\\
        e_1^3 e_0 = e_0 e_1^3 + (q^2+1+q^{-2}) e_1^2 e_0 e_1 -
        (q^2+1+q^{-2}) e_1 e_0 e_1^2
      \end{cases}
    \]
  \end{enumerate}
\end{example}
\begin{defn}
  In order to simplify notation, we will write
  \begin{enumerate}
  \item \(K_i := K_{s_i}\)
  \item \(\tilde{K}_i := K_{s_i h_i}\)
  \item For \(\alpha = \sum_i \eta_i \alpha_i\) a root, we will write
    \(\tilde{K}_\alpha := \prod_i K_i^{\eta_i}\).
  \end{enumerate}
\end{defn}
\begin{prop}\label{root-space-decomp}
  If we set \(\deg f_i = -\alpha_i, \deg K_h = 0\), and \(\deg e_i =
  \alpha_i\), then we get the \de{root space decomposition} \[
    \U_q(\g) = \bigoplus_{\alpha \in \roots} (\U_q)_{\alpha}
  \]
  where \((\U_q)_\alpha := \{u \in \U_q(\g) \st K_h u K_{-h} =
  q^{\alpha(h)}u \text{ for all } h \in \weightlattice^\vee\}\).
\end{prop}
\begin{proof}
  The defining relations for the quantum group are all homogeneous
  with respect to our choice of degree, so we can write \(\U_q(\g)\)
  as a direct sum by degree.
\end{proof}
\begin{defn}
  Define the \de{quantum adjoint operators} by \[
    (\ad_q x)(y) := xy - q^{(\alpha | \beta)}yx
  \]
  for \(x \in (\U_q)_\alpha, y \in (\U_q)_\beta\), and \(\alpha,\beta
  \in \roots\) where \((\cdot | \cdot) \from \h \times \h \to F\) is
  the nondegenerate bilinear form
  defined by \[
    \begin{cases}
      (h_i | h) := \alpha_i(h)/s_i & \text{ for }h\in \h\\
      (d_s | d_t) := 0 & \text{ for }s,t = 1,\ldots,|I|-\rank A
    \end{cases}
  \]
  and then extend the definition by linearity to the entire quantum group. 
\end{defn}
\begin{lem}
  We get \[
    (\ad_q e_i)^N(e_j) = \sum_{k=0}^N (-1)^k K_i^{k(N+a_{ij}-1)}
    \qbinom[K_i]{N}{k} e_i^{N-k} e_j e_i^k
  \]
\end{lem}
\begin{rmk}
  The last two defining relation of the quantum group are called the
  \de{quantum Serre relations}. Using the lemma above, for \(i \neq
  j\), we can rewrite
  them as
  \begin{enumerate}
  \item \((\ad_q e_i)^{1-a_{ij}}(e_j) = 0\)
  \item \((\ad_q f_i)^{1-a_{ij}}(f_j) = 0\)
  \end{enumerate}
\end{rmk}
\begin{prop}
  \cite{hong-kang}*{Prop 3.1.2} The quantum group \(\U_q(\g)\) has a
  Hopf algebra structure given by
  \begin{enumerate}
  \item \(\Delta(K_h) = K_h \otimes K_h\)
  \item \(\Delta(e_i) = e_i \otimes K_{s_i h_i}^{-1} + 1 \otimes e_i\)
  \item \(\Delta(f_i) = f_i \otimes 1 + K_{s_i h_i} \otimes f_i\)
  \item \(\epsilon(K_h) = 1, \epsilon(e_i) = \epsilon(f_i) = 0\)
  \item \(S(K_h) = K_{-h}, S(e_i) = - e_i K_{s_i h_i}, S(f_i) = -
    K_{s_i h_i}^{-1} f_i\)
  \end{enumerate}
  for \(h \in \weightlattice^\vee\) and \(i \in I\).
\end{prop}
\begin{defn}
  Given quantum group \(\U_q(\g)\), we define \(\U_q^+\) (respectively
  \(\U_q^-\)) to be the
  subalgebra of \(U_q(\g)\) generated by the elements \(e_i\)
  (respectively \(\U_q^-\)) for \(i \in I\). We also define \(\U_q^0\)
  to be the subalgebra of \(\U_q(\g)\) generated by \(K_h\) for \(h
  \in \weightlattice^\vee\). 
\end{defn}
\begin{thm}\label{triangular-decomp}
  We have the \de{triangular decomposition} \[
    \U_q(\g) \isom \U_q^- \otimes \U_q^0 \otimes \U_q^+
  \]
\end{thm}
\begin{proof}
  We postpone the proof until we can establish the machinery to prove
  it below.
\end{proof}
\begin{defn}
  Let \(T \from \U_q(\g) \to \U_q(\g)\) be a linear map defined by \[
    T(K_h) = K_{-h}, T(e_i) = f_i, T(f_i) = e_i
  \]
  for \(h \in \weightlattice^\vee, i \in I\). Furthermore, let \[
    \sigma(a \otimes b) := b \otimes a
  \]
  for \(a,b \in \U_q(\g)\).
\end{defn}
\begin{prop}
  \(T\) is an algebra endomorphism on \(\U_q(\g)\). Furthermore,
  \begin{enumerate}
  \item \(T^2 = id\)
  \item \(\Delta \circ T = \sigma \circ (T \otimes T) \circ \Delta\)
  \item \(T(\U_q^+) = U_q^-\) inducing an algebra isomorphism between
    \(\U_q^+\) and \(\U_q^-\).
  \end{enumerate}
\end{prop}
\begin{proof}
  To prove this, we check the results on each of the generators.
  \begin{enumerate}
  \item \[
      T^2(K_h) = T(K_{-h}) = K_h, T^2(e_i) = T(f_i) = e_i, T^2(f_i) =
      T(e_i) = f_i
    \]
  \item We check
    \begin{align*}
      (\Delta \circ T)(K_h)
      & = K_{-h} \otimes K_{-h} \\
      & = \sigma(K_{-h} \otimes K_{-h}) \\
      & = \sigma(T(K_h) \otimes T(K_h)) \\
      & = (\sigma \circ (T \otimes T) \circ \Delta)(K_h) \\
      (\Delta \circ T)(e_i)
      & = f_i \otimes 1 + K_{s_i h_i} \otimes f_i \\
      & = \sigma(1 \otimes f_i + f_i \otimes K_{s_i h_i}) \\
      & = \sigma(T(1) \otimes T(e_i) + T(e_i) \otimes T(K_{-s_i h_i}))
      \\
      & = (\sigma \circ (T \otimes T)) (e_i \otimes K^{-1}_{s_i h_i} +
        1 \otimes e_i) \\
      & = \sigma \circ (T \otimes T) \circ \Delta       
    \end{align*}
    And the computation \((\Delta \circ T)(f_i)\) works similarly to
    the above.
  \item This part is immediate since \(T(e_i) = f_i \in \U_q^-\) and
    \(T^2 = id\), so \(T\) is, in particular, bijective. Thus, since
    \(T\) is an algebra homomorphism by construction, it gives an
    algebra isomorphism.
  \end{enumerate}
\end{proof}
\begin{lem}
  \cite{hong-kang}*{Lemma 3.1.4}
  \begin{enumerate}
  \item \(\U_q^{\geq 0} \isom \U_q^0 \otimes \U_q^+\)
  \item \(\U_q^{\leq 0} \isom \U_q^- \otimes \U_q^0\)
  \end{enumerate}
\end{lem}
\begin{proof}
  The proof is quite technical but for item (b) consists of
  \begin{enumerate}[label=(\roman*)]
  \item Let \(\{f_\xi\}_{\zeta \in \Omega}\) be a basis of \(\U_q^-\)
    consisting of monomials in \(f_i\)'s.
  \item Show the map \(\phi \from \U_q^- \otimes \U_q^0 \to \U_q^{\geq
    0}\) given by \(\phi(f_\zeta \otimes K_h) = f_\zeta K_h\) is
  surjective.
  \item Show the set \(\{f_\zeta K_h \st \zeta \in \Omega, h \in
    \weightlattice^\vee\}\) is linearly independent over \(F(q)\) by
    directly using the definition of linear independence and the
    comultiplication map.
  \end{enumerate}
  Part (a) proceeds analogously (presumably). 
\end{proof}
\begin{proof}[Proof of \ref{triangular-decomp}]
  \cite{hong-kang}*{p 42} Let \(\{f_\zeta\}_{\zeta \in \Omega}\) and
  \(\{e_\zeta\}_{\zeta \in \Omega}\) be monomial bases of \(\U_q^-\)
  and \(\U_q^+\), respectively. It suffices to show \(\{f_\zeta K_h
  e_\eta \st \zeta, \eta \in \Omega, h \in \weightlattice^\vee\}\) is
  linearly independent over \(F(q)\).

  Let \(C_{\zeta, h, \eta} \in F(q)\) be scalar such that  \[
    \sum_{\zeta, h, \eta}C_{\zeta, h, \eta} f_\zeta K_h e_\eta = 0
  \]
  Using the root space decomposition of \(\U_q(\g)\)
  \ref{root-space-decomp}, we
  have \[
    \sum_{\substack{h \in \weightlattice^\vee \\ \deg f_\zeta + \deg
        e_\eta = \gamma}} C_{\zeta, h, \eta} f_\zeta K_h e_\eta = 0,
    \forall \gamma \in \rootlattice
  \]
  However, if we apply the comultiplication map to this expression, we
  get \[
    0 = \sum_{\substack{h \in \weightlattice^\vee \\ \deg f_\zeta + \deg
        e_\eta = \gamma}} C_{\zeta, h, \eta} (f_\zeta \otimes 1 +
    \cdots + K_{- \deg f_\zeta} \otimes f_\zeta) (K_h \otimes K_h)
    (e_\eta \otimes K^{-1}_{\deg e_\eta} + \cdots + 1 \otimes e_\eta)
  \]
  Then, using the partial ordering on \(\weightlattice^\vee\) given by
  \(\lambda \leq \mu \iff \mu-\lambda \in \rootlattice_+\), we can
  pick \(\alpha = \deg f_\zeta\) minimal and \(\beta = \deg e_\eta\)
  maximal among those for which \(C_{\zeta,h,\eta}\) is
  nonzero. Then, \[
    \sum_{\substack{h \in \weightlattice^\vee \\ \deg f_\zeta =
        \alpha,  \deg
        e_\eta = \beta}} C_{\alpha, h, \beta} C_{\zeta, h, \eta}
    (f_\zeta K_h \otimes K_h e_\eta) = 0
  \]
  However, the vectors \(\{f_\zeta K_h\}_{\zeta,h}\) are linearly
  independent by the 
  lemma above, as are \(\{e_\eta K_h\}_{\eta,h}\). Thus, we get
  \(C_{\zeta,h,\eta} = 0\).
\end{proof}
\subsection{Representation theory of quantum groups}
The representation theory of quantum groups has similar results to
that of the representation theory of Kac-Moody algebras.
\begin{defn}
  \begin{enumerate}
  \item   We say a \(\U_q(\g)\)-module \(V^q\) is a \de{weight module} if it
  admits a weight space decomposition \[
    V^q = \bigoplus_{\mu \in \weightlattice} V_\mu^q \text{ where }
    V_\mu^1 = \{v \in V^q \st K_h v = q^{\mu(h)} v \text{ for all }h
    \in \weightlattice^\vee\}
  \]
  \item We say a vector \(v \in V^q\) is a \de{weight vector}
    of weight \(\mu \in \weightlattice\) if \(K_h v = q^{\mu(h)} v\)
    for all \(h \in \weightlattice^\vee\).
  \item We say a weight vector \(v \in V^q\) of weight \(\mu\) is a
    \de{maximal vector} if \(e_i v = 0\) for
    all \(i \in I\).
  \item If \(V_\mu^q \neq 0\), \(\mu\) is called a \de{weight} of
    \(V^q\) and \(V_\mu^q\) is called the \de{weight space} of weight
    \(\mu \in \weightlattice\). The set of all weights of \(V^q\) will
    be denoted \(\wt(V^q)\).
  \item The \de{weight multiplicity} of \(\mu\) in \(V^q\) is \(\dim
    V_\mu^q\).
  \item If \(\dim V^q_\mu < \infty\) for all \(\mu \in \wt(V^q)\),
    then the \de{character} of \(V^q\) is given by \[
      \ch V^q = \sum_\mu \dim V^q_\mu e^\mu
    \]
    where \(e^\mu\) are formal basis elements of the group algebra
    \(F[\weightlattice]\) with multiplication \(e^\lambda e^\mu =
    e^{\lambda + \mu}\).
  \end{enumerate}
\end{defn}
\begin{prop}
  Every submodule of a weight module over \(\U_q(\g)\) is also a
  weight module.
\end{prop}
\subsection{\(\A_1\)-forms}
\begin{defn}
  We define the following terminology.
  \begin{enumerate}
  \item  Let \(\A_1\) be the localization of \(F[q]\) at the ideal \((q-1)\)
  so that \[
    \A_1 = \{f(q) \in F(q) \st f \text{ is regular at }q=1\} = \{g/h
    \st g,h \in F[q], h(1) \neq 0\}
  \]
  \item Define, for \(n \in \Z\), \[
    [y;n]_x := \frac{yx^n-y^{-1}x^{-n}}{x-x^{-1}}
  \]
  and \[
    (y;n)_x := \frac{yx^n-1}{x-1}
  \]
  \item Define the \de{\(\A_1\)-form}, denoted \(\U_{\A_1}\), of the
    quantum group \(\U_q(\g)\) to be the \(\A_1\)-subalgebra of
    \(\U_q(\g)\) generated by the elements \(e_i, f_i, K_h\), and
    \((K_h;0)_q\) for \(i \in I, h \in \weightlattice^\vee\).
  \item Let \(\U_{\A_1}^+\) (resp \(\U_{\A_1}^-\)) be the
    \(\A_1\)-subalgebra of \(\U_{\A_1}\) generated by the elements
    \(e_i\) (resp \(f_i\)) for \(i \in I\).
  \item Let \(\U_{\A_1}^0\) be that \(\A_1\)-subalgebra of
    \(\U_{\A_1}\) generated by \(K_h\) and \((K_h;0)_q\) for \(h \in
    \weightlattice^\vee\). 
  \end{enumerate}
\end{defn}

\begin{bibdiv}
  \begin{biblist}
    \bib{carter}{book}{
      author={Carter, Roger}
      title={Lie Algebras of Finite and Affine Type}
      year={2005}
    }
    \bib{hong-kang}{book}{
      author={Hong, Jin}
      author={Kang, Seok-Jin}
      title={Introduction to Quantum Groups and Crystal Bases}
      year={2002}
    }
    \bib{humph}{book}{
      author={Humphreys, James E.}
      title={Introduction to Lie Algebras and Representaton Theory}
      year={1972}
      note={Third printing, revised}
    }
    \bib{cat-o}{book}{
      author={Humphreys, James E.}
      title={Representations of Semisimple Lie Algebras in the BGG
        Category $\mathcal{O}$}
      year={2008}
    }
    % \bib{wiki}{misc}{
    %   author={Wikipedia}
    %   title={Root system --- Wikipedia{,} The Free Encyclopedia}
    %   year={2017}
    %   url={https://en.wikipedia.org/w/index.php?title=Root_system&oldid=759729605}
    %   note={[Online; accessed 21-February-2017]}
    % }
  \end{biblist}
\end{bibdiv}

\end{document}