\documentclass[11pt,leqno,oneside]{amsart}
\usepackage[alphabetic,abbrev]{amsrefs} % use AMS ref scheme
\usepackage{../ReAdTeX/readtex-core}
\usepackage{../ReAdTeX/readtex-dangerous}
\usepackage{../ReAdTeX/readtex-abstract-algebra}
\usepackage{./monographs}
\usepackage{caption}
\usepackage{subcaption}
\usepackage{todonotes}
\usepackage{ytableau}
\usepackage{xcolor}
\usepackage{mathtools}
\usepackage{pgf}
\usepackage{tikz}
\usetikzlibrary{shapes.geometric}
\usepackage{tikz-cd}
\usepackage{graphicx}
\numberwithin{thm}{section}

\newcommand{\T}{\mathsf{T}} % use for tableaux
\renewcommand{\S}{\mathsf{S}}
\newcommand{\TP}{\mathsf{P}}
\newcommand{\TQ}{\mathsf{Q}}
\newcommand{\inv}{\operatorname{inv}}
\newcommand{\charge}{\operatorname{charge}}
\newcommand{\cocharge}{\operatorname{cocharge}}
\newcommand{\des}{\operatorname{des}}
\newcommand{\row}{\operatorname{row}}
\newcommand{\col}{\operatorname{col}}

\newcommand{\Vdet}{\Delta}
\newcommand{\rowshift}{\rho}
\newcommand{\SSYT}{\operatorname{SSYT}} % Semistandard Young Tableaux
\newcommand{\defeq}{\coloneqq}
\newcommand{\partitionof}{\vdash}
\newcommand{\dominates}{\mathrel{\unrhd}}
\newcommand{\strictlydominates}{\mathrel{\rhd}}
\newcommand{\dominatedby}{\mathrel{\unlhd}}
\newcommand{\strictlydominatedby}{\mathrel{\lhd}}
\newcommand{\covers}{\mathrel{\gtrdot}}
\newcommand{\coveredby}{\mathrel{\lessdot}}

\newcommand{\sym}{\Lambda}
\renewcommand{\k}{\Bbbk}

\newcommand*\circled[1]{\tikz[baseline=(char.base)]{
   \node[shape=circle,draw,inner sep=1pt] (char) {#1};}}
\newcommand{\mymk}[1]{\tikz \node[draw,circle, inner sep=0pt, minimum size=7mm]{#1};}

\makeatletter
\newcommand\mathcircled[1]{%
  \mathpalette\@mathcircled{#1}%
}
\newcommand\@mathcircled[2]{%
  \tikz[baseline=(math.base)] \node[draw,circle,inner sep=1pt] (math) {$\m@th#1#2$};%
}
\makeatother

\newcommand{\quantumBinom}[3][q]{\left[ \begin{array}{c} #2 \\
                                          #3 \end{array} \right]_#1}
\newcommand{\qbinom}{\quantumBinom}
\newcommand{\stat}{\operatorname{stat}}
\newcommand{\Des}{\operatorname{Des}}
\newcommand{\maj}{\operatorname{maj}}
\newcommand{\comaj}{\operatorname{comaj}}
\newcommand{\height}{\operatorname{ht}}
\newcommand{\size}{\operatorname{size}}
\newcommand{\sh}{\operatorname{sh}}
\newcommand{\spin}{\operatorname{spin}}
\newcommand{\cospin}{\operatorname{cospin}}
\newcommand{\wt}{\operatorname{wt}}
\newcommand{\length}{\operatorname{len}}
\newcommand{\LLT}{\operatorname{LLT}}

\title[Special Classes of Permutations]{Some Special Classes of Permutations}
\author{George H. Seelinger}
\date{August 2018}
\begin{document}
\maketitle
\section{Introduction}
Permutations are a fundamental object of study in combinatorics and
are also studied in abstract algebra to define concrete examples of
groups. In this monograph, we seek to explore some special classes of
permutations that are interesting to combinatorialists. In particular,
we will discuss
\begin{enumerate}
\item \(132\)-avoiding permutations,
\item dominant partitions,
\item Grassmannian partitions,
\item and vexillary permutations.
\end{enumerate}
\todo{Introduce some basic concepts like the Rothe diagram.}
This monograph is currently very incomplete and simply a rough outline
of details I would like to fully flush out at a later time. Currently,
almost all results are stated as in \cite{manivel}.
\section{\(132\)-avoiding permutations}
We first start with the general definition that
\begin{defn}
  \(\sigma \in \Sym_n\) avoids pattern \(\tau \in \Sym_r\) (for \(r
  \leq n\)) if there is no subword \(\sigma_{i_1}, \sigma_{i_2},
  \ldots, \sigma_{i_r}\) which ``reduces'' to \(\tau\).
\end{defn}
\begin{example}
  The permutation \(\sigma = 24513\) (in one line notation) does
  not avoid \(231\) since \(251 \mapsto 231\).
\end{example}
Thus, we immediately get the definition of a \(132\)-avoiding
permutation.
\begin{example}
  In \(\Sym_3\), every permutation except \(132\) is
  \(132\)-avoiding. In \(\Sym_4\), \(1234\) and \(4123\) are
  \(132\)-avoiding. 
\end{example}
Now, if we recall that the Catalan numbers \(C_n\) are given by
either \[
  C_n = \frac{1}{n+1} \binom{2n}{n} \text{ or }
  \begin{cases}
    C_0 = 1 \\
    C_{n+1} = \sum_{i=0}^n C_i C_{n-i}
  \end{cases}
\]
then we have
\begin{thm}
  The number of \(132\)-avoiding permutations in \(\Sym_n\) is given
  by \(C_{n-1}\).
\end{thm}
\begin{proof}
  This is a classic combinatorics homework exercise and so will be
  left to the reader. Perhaps some hints will be given by the
  propositions that follow.
\end{proof}
\begin{lem}
  For  a \(132\)-avoiding
  permutation \(\sigma\), there is no triple of integers \(i < j < k\)
  such that \(\sigma_i < \sigma_k < \sigma_j\).
\end{lem}
\begin{prop}
  There is a bijection mapping \(132\)-avoiding permutations in
  \(\Sym_n\) to Dyck paths of length \(2n\).
\end{prop}
\begin{proof}
  Given a Dyck path \(\pi\) with \((1,1)\) the top left square,
  we will construct a Rothe diagram for \(\sigma\) which is
  \(132\)-avoiding.
  \begin{enumerate}
  \item Place an \(X\) in each ``removable corner'' below the path
    \(\pi\) and put in the appropriate \(\cdot\)'s.
  \item Moving top to bottom, place \(X\) in westernmost empty
    sequence below \(\pi\) (and fill as appropriate). Note that since
    \(\pi\) never falls below the diagonal, there will be \(n\) rows
    and columns below \(\pi\).
  \end{enumerate}
  Note first that
  \[
    \pi = UUUU \cdots UR \cdots RRRR \mapsto \begin{array}{|c|c|c|c|} \hline
                             \times&\cdot&\cdot&\cdot\\ 
            \hline \cdot&\times&\cdot&\cdot\\ \hline
            \cdot&\cdot&\times&\cdot\\ \hline
            \cdot&\cdot&\cdot&\times\\ \hline \end{array} \mapsto
          \sigma = id
   \]
   which is \(132\)-avoiding. Otherwise, if \(\pi\) is not of this
   form, then the cell \((1,1)\) is above \(\pi\). Next, if \(i < j <
   k\) and \(\sigma_i < \sigma_j < \sigma_k\), we have a configuration
   with disconnected cells, which cannot be a Dyck path since no pair
   of cells can be disconnected. \[
     \begin{array}{|c|c|c|c|c|c|} \hline &&&&\times&\cdot\\ \hline &&&&\cdot&\times\\ \hline \times&\cdot&\cdot&\cdot&\cdot&\cdot\\ \hline \cdot&&\times&\cdot&\cdot&\cdot\\ \hline \cdot&\times&\cdot&\cdot&\cdot&\cdot\\ \hline \cdot&\cdot&\cdot&\times&\cdot&\cdot\\ \hline \end{array}
   \]
 \end{proof}
 \begin{prop}
   There is a bijection between \(132\)-avoiding permutations and
   \(321\)-avoiding permutations.
 \end{prop}
 \section{Dominant Permutations}
 \begin{defn}
   A permutations \(\sigma \in \Sym_n\) is a \de{dominant permutation}
   if it has partition Lehmer code, that is \[
     c(\sigma) = (\ell_1, \ell_2, \ldots, \ell_{n-1}) \text{ with }
     \ell_1 \geq \ell_2 \geq \cdots \geq \ell_{n-1}
   \]
 \end{defn}
 \begin{example}
   An example of a dominant permutation is \(\sigma = 6324571 \in \Sym_7\)
   since \[
     c(\sigma) = (5,2,1,1,1,1)
   \]
   Interestingly, the Rothe diagram of \(\sigma\) given below exhibits
   the Ferrers diagram of the partition \(5,2,1,1,1,1\). \[
     \begin{array}{|c|c|c|c|c|c|c|} \hline &&&&&\times&\cdot\\ \hline &&\times&\cdot&\cdot&\cdot&\cdot\\ \hline &\times&\cdot&\cdot&\cdot&\cdot&\cdot\\ \hline &\cdot&\cdot&\times&\cdot&\cdot&\cdot\\ \hline &\cdot&\cdot&\cdot&\times&\cdot&\cdot\\ \hline &\cdot&\cdot&\cdot&\cdot&\cdot&\times\\ \hline \times&\cdot&\cdot&\cdot&\cdot&\cdot&\cdot\\ \hline \end{array}
   \]
 \end{example}
 \begin{prop}
   A permutation is dominant if and only if the cells above its Rothe
   diagram are the Ferrers diagram of a partition (in English
   convention). 
 \end{prop}
\begin{proof}
  A common fact about Rothe diagrams is that the number of empty boxes
  in column \(i\) is the \(i\)th entry of the Lehmer code. Thus, since
  the transpose of the Rothe diagram is the Rothe diagram of
  \(\sigma^{-1}\) and such a transposition conjugates the embedded Ferrers
  diagram, we get that such a Rothe diagram gives a dominant
  permutation.

  \todo{Prove the converse.}
\end{proof}
\section{Grassmannian Permutations}
\begin{defn}
  A permutation \(\sigma \in \Sym_n\) is called \de{Grassmannian} if
  \(\sigma\) has at most one descent. In other words, there is at most
  one integer \(r\) such that \[
    \sigma_1 < \sigma_2 < \cdots < \sigma_r \text{ and } \sigma_{r+1}
    < \sigma_{r+2} < \cdots < \sigma_n
  \]
\end{defn}
\begin{example}
  The permutation \(\sigma = 245813679\) (in one line notation) has a
  descent at \(8\). Furthermore, it has code \[
    c(\sigma) = (1,2,2,5,0,0,0,0)
  \]
\end{example}
\begin{lem}
  A Grassmannian permutation \(\sigma\) with Lehmer code \(c(\sigma) =
  (\ell_1, \ldots, \ell_{n-1})\) has \[
    \ell_1 \leq \ell_2 \leq \cdots \leq \ell_r \leq n-r \text { and }
    \ell_{r+1} = \cdots = \ell_{n-1} = 0
  \]
\end{lem}
\begin{proof}
  If either of these conditions fails, there must be more than one
  descent since a Grassmannian permutation must have entries (in one
  line notation) increasing up to \(\sigma_r\) and then increasing
  again from \(\sigma_{r+1}\) to \(\sigma_n\).
\end{proof}
\begin{prop}
  Grassmannian permutations are in bijection with partitions inside
  an \(r \times n-r\) rectangle.
\end{prop}
\begin{defn}
  A permutation is called \de{bigrassmannian} if it is Grassmannian
  and its inverse is Grassmannian.
\end{defn}
\section{Vexillary Permutations}
\begin{defn}
  A permutations \(\sigma \in \Sym_n\) is called \de{vexillary} if it
  is \(2143\)-avoiding. In other words, \(\sigma\) is vexillary if and
  only if there does not exist a sequence \(i < j < k < l\) such that
  \(\sigma_j < \sigma_i < \sigma_l < \sigma_k\).
\end{defn}
\begin{lem}
  A permutation \(\sigma\) is vexillary if and only if, up to a
  permutation of its rows and columns, its Rothe diagram has a diagram
  of a partition above it.
\end{lem}
\todo{Prove this}
% \begin{proof}
%   The Rothe diagram of a permutation can be transformed so that a 
%   partition diagram lies above it if and only if the empty boxes in
%   the rows (or columns) can be ordered by inclusion. Such an ordering
%   is not possible if and only if there are points \((i,\sigma_j)\) and
%   \((k,\sigma_l)\) of the diagram with any of 
%   \begin{enumerate}
%   \item \(i < k\) such that \((i,\sigma_l)\) and \((k,\sigma_j)\) are
%     not in the same connected component.
%   \item \(\sigma_j < \sigma_l\) such that \((k,\sigma_j)\) and
%     \((i,\sigma_l)\) are not in the same connected component.
%   \end{enumerate}
% \end{proof}
\begin{cor}
  Dominant permutations and bigrassmannian permutations are vexillary.
\end{cor}
\begin{prop}
  A permutation is vexillary if and only if \(\lambda(\sigma) =
  \lambda(\sigma^{-1})'\), that is, the shape of \(\sigma\) is equal
  to the conjugate of the shape of \(\sigma^{-1}\).
\end{prop}
\begin{defn}
  The \de{flag} of a vexillary permutation \(\sigma\) is defined as
  follows.
  \begin{enumerate}
  \item If \(c(\sigma)_i \neq 0\), let \(e_i\) be the greatest integer
    \(j \geq i\) such that \(c(\sigma)_j \geq c(\sigma)_i\).
  \item The flag \(\phi(\sigma)\) is the sequence of integers \(e_i\)
    ordered to be increasing. 
  \end{enumerate}
\end{defn}
\begin{example}
  Let \(\sigma = 126354\). Then, we have Rothe diagram \[
\begin{array}{|c|c|c|c|c|c|} \hline \times&\cdot&\cdot&\cdot&\cdot&\cdot\\ \hline \cdot&\times&\cdot&\cdot&\cdot&\cdot\\ \hline \cdot&\cdot&&&&\times\\ \hline \cdot&\cdot&\times&\cdot&\cdot&\cdot\\ \hline \cdot&\cdot&\cdot&&\times&\cdot\\ \hline \cdot&\cdot&\cdot&\times&\cdot&\cdot\\ \hline \end{array}
\]
We have \(c(\sigma) = (0,0,3,0,1)\). Thus, \(e_3 = 3\) and \(e_5 = 5\)
and so \(\phi(\sigma) = (3,5)\).
\end{example}
\begin{prop}
  A vexillary permutation is completely determined by its shape and
  its flag.
\end{prop}
\begin{bibdiv}
  \begin{biblist}
    \bib{fulton}{book}{
      author={Fulton, William}
      title={Young Tableaux}
      year={1997}
    }
    \bib{macdonald}{book}{
      author={Macdonald, I.G.}
      title={Symmetric Functions and Hall Polynomials}
      year={1979}
      note={2nd Edition, 1995}
    }
    \bib{manivel}{book}{
      author={Manivel, Laurent}
      title={Symmetric Functions, Schubert Polynomials, and Degeneracy
        Loci}
      year={1998}
      note={Translated by John R. Swallow; 2001}
    }
  \end{biblist}
\end{bibdiv}

\end{document}