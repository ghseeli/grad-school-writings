\documentclass[11pt,leqno,oneside]{amsart}
\usepackage[alphabetic,abbrev]{amsrefs} % use AMS ref scheme
\usepackage{../ReAdTeX/readtex-core}
\usepackage{./monographs}
\usepackage{../ReAdTeX/readtex-abstract-algebra}
\usepackage{../ReAdTeX/readtex-lie-algebras}
\usepackage{tikz-cd}

\usepackage{caption}
\usepackage{subcaption}
\usepackage{todonotes}

\numberwithin{thm}{section}

\newcommand{\T}{\mathsf{T}}
\newcommand{\SymP}{\operatorname{Sym}}
\newcommand{\ExtP}{\wedge}
\newcommand{\Vdet}{\Delta}
\newcommand{\partitionof}{\vdash}
\newcommand{\rowshift}{\rho}
\newcommand{\sym}{\Lambda}

\newcommand{\ch}{\operatorname{ch}}
\newcommand{\sign}{\operatorname{sign}}
\newcommand{\Ind}{\operatorname{Ind}}
\newcommand{\Res}{\operatorname{Res}}

\newcommand{\sh}{\operatorname{sh}}
\newcommand{\height}{\operatorname{ht}}

\title[Representation Theory of Symmetric Groups]{Representation
  Theory of Symmetric Groups}
\author{George H. Seelinger}
\date{May 2018}
\begin{document}
\maketitle
\section{Introduction}
The representation theory of symmetric groups is a well-studied and
rich subject with connections to the representation theory of Lie
groups and Lie algebras, as well as to symmetric function theory and
combinatorics.

This monograph will assume the reader is already familiar with
material in \cite{rep-thry-of-finite-gps}*{Sections 1--14} and
\cite{alg-comb}*{Section 
2}, 
although not all of it is 
strictly speaking necessary. In this monograph, we will follow the
program in
\cite{fulton-harris}.

Our results are all stated over \(\C\) unless otherwise
noted. \(\Sym_d\) is a symmetric group on \(d\) letters.
\section{Small Examples}
For small symmetric groups, one can use the theory of the
representation theory of finite groups to directly compute the
character tables of \(\Sym_n\). For all symmetric groups, we have the
trivial representation and the sign representation given by \(w.v =
\sgn(w)v\) for \(w \in \Sym_n\). 
\begin{example}
  For \(G = \Sym_3\), since there are \(3\) conjugacy classes, there
  is only one missing representation of dimension \(2\). Thus, giving
  the character table
  \begin{center}
    \begin{tabular}{c|ccc}
      &(1)&(12)&(123) \\
      \hline
      \(\chi_1\) &1&1&1 \\
      \(\chi_2\) &1&-1&1\\
      \(\theta\) &2&0&-1
    \end{tabular}
  \end{center}
\end{example}
\section{Characters of Symmetric Groups Representations}
In this section, we follow the program of \cite{manivel}*{Section
  1.6} to develop some general character theory for \(\Sym_n\). Let
\(R^{(n)}\) be the free \(\Z\)-module generated by the 
irreducible 
characters of \(\Sym_n\) with \(R^{(0)} = \Z\).
\begin{prop}
  The direct sum \[
    R = \bigoplus_{n \geq 0} R^{(n)}
  \]
  has the structure of an associative and commutative graded ring
  under the product, for \(\phi \in R^{(m)}\) and \(\psi \in R^{(n)}\),
  \[
    \phi \cdot \psi = \Ind_{\Sym_m \times \Sym_n}^{\Sym_{m+n}}(\phi
    \times \psi)
  \]
\end{prop}
\begin{defn}
  For \(w \in \Sym_n\), let \(\lambda(w)\) be the partition of size
  \(n\) encoding the cycle type of \(w\). Then, the \de{characteristic
    map} \(\ch \from R \to \sym \otimes_\Z 
  \C\) is defined by, for \(\phi \in R^{(n)}\), \[
    \ch(\phi) := \frac{1}{n!} \sum_{w \in \Sym_n} \phi(w) p_{\lambda(w)}
  \]
  where \(p_{\lambda(w)}\) is the power sum symmetric function.
\end{defn}
\begin{thm}
  \cite{manivel}*{Proposition 1.6.3} The characteristic map defines a
  graded ring isomorphism from the 
  ring \(R\) of the characters of the symmetric group to the ring
  \(\sym\) of symmetric functions.
\end{thm}
\begin{lem}
  \[
    \ch(\phi) = \sum_{|\lambda|=n} z_\lambda^{-1} \phi_\lambda p_\lambda
  \]
  where \(\phi_\lambda\) is the value of \(\phi\) on the conjugacy
  class of cycle type \(\lambda\) and \(z_{\lambda}\) is the
  cardinality of the centralizer of an 
  element associated to the conjugacy class associated to
  \(\lambda\), that is, \(z_{\lambda} = \prod_{i} i^{m_i} m_i!\) where
  \(m_i\) is the multiplicity of \(i\) in \(\lambda\).
\end{lem}
\begin{proof}[Proof of Lemma]
  First we break up the sum \[
    \sum_{w \in \Sym_n} \phi(w) p_{\lambda(w)} = \sum_{|\lambda|=n}
    \sum_{w \text{ of cycle type }\lambda}  \phi(w) p_\lambda
  \]
  and, since characters are class functions, we may define
  \(\phi_\lambda\) as \(\phi(w)\) for any \(w\) with cycle type
  \(\lambda\). Finally, the size of the conjugacy class must be
  \(\frac{n!}{z_\lambda}\) by the orbit-stabilizer theorem, so we
  get \[
    \sum_{|\lambda|=n}
    \sum_{w \text{ of cycle type }\lambda}  \phi(w) p_\lambda =
    \sum_{|\lambda| = n} \frac{n!}{z_\lambda} \phi_\lambda p_\lambda
  \]
  giving us the desired formula after multiplying both sides by \(n!\).
\end{proof}
\begin{lem}
  \(\ch\) is an isometry, that is \[
    (\phi,\psi) = \langle \ch(\phi),\ch(\psi) \rangle
  \]
  where \((\cdot,\cdot)\) is the inner product on characters and
  \(\langle \cdot,\cdot \rangle\) is the Hall-inner product on
  symmetric functions. In particular, this means \(\ch\) is injective.
\end{lem}
\begin{proof}[Proof of Lemma]
  We check, for \(\phi, \psi \in R^{(n)}\),
  \begin{align*}
    \langle \ch(\phi), \ch(\psi) \rangle
    & = \langle
    \sum_{\lambda \partitionof n} z_\lambda^{-1} \phi_\lambda
    p_\lambda, \sum_{\mu \partitionof n} z_\mu^{-1} \psi_\mu p_\mu
      \rangle \\
    & = \sum_{\lambda \partitionof n} \sum_{\mu \partitionof n}
    \phi_\lambda \psi_\mu z_{\lambda}^{-1} z_\mu^{-1} \langle
      p_\lambda, p_\mu \rangle \\
    & = \sum_{\lambda \partitionof n} \sum_{\mu \partitionof n}
    \phi_\lambda \psi_\mu z_{\lambda}^{-1} z_\mu^{-1} z_\lambda
      \delta_{\lambda,\mu} \\
    & = \sum_{\lambda \partitionof n} \phi_\lambda
      \psi_\lambda z_{\lambda}^{-1} \\
    & = \frac{1}{n!}\sum_{w \in \Sym_n} \phi(w)\psi(w) \\
    & = (\phi,\psi)
  \end{align*}
\end{proof}
\begin{proof}[Proof of Theorem]
  First, we must define the class function \(p \from \Sym_n \to
  \sym^n\) via \[ 
    p(w) = p_{\lambda(w)}
  \]
  Then, we can rephrase \[
    \ch(\phi) =  (\phi, p)
  \]
  We check that
  \begin{align*}
    \ch(\phi \cdot \psi)
    & = ( \phi \cdot \psi, p)\\
    & = (\Ind_{\Sym_m \times \Sym_n}^{\Sym_{m+n}}(\phi \times
      \psi), p) \\
    & = (\phi \times \psi, \Res_{\Sym_m \times
      \Sym_n}^{\Sym_{m+n}}p)
    & \text{by Frobenius Reciprocity}\\
    & = \frac{1}{m!n!} \sum_{(w,w') \in \Sym_m \times \Sym_n} (\phi \times
      \psi)(ww') \ov{p(ww')}
    & \text{by definition of }(\cdot,\cdot)\\
    & = \frac{1}{m!n!} \sum_{w \in \Sym_m, w' \times \Sym_n} \phi(w)
      \psi(w') p_{w} p_{w'} \\
    & = \left( \frac{1}{m!} \sum_{w \in \Sym_m} \phi(w) p_w
      \right)\left( \frac{1}{n!} \sum_{w \in \Sym_n} \psi(w) p_w
      \right) \\
    & = \ch(\phi) \ch(\psi)
  \end{align*}
  Now, consider the trivial character \(1_n \in R^{(n)}\) of
  \(\Sym_n\). We compute \[
    \ch(1_n) = \sum_{\lambda \partitionof n} z_\lambda^{-1}
    p_\lambda = h_n
  \]
  where the \(h_n\) is the homogeneous symmetric polynomial and the
  equality comes from an argument on generating functions (see
  \cite{alg-comb}*{Section 2}). Furthermore, since \(\sym\) is
  algebraically generated by \(\{h_n\}_{n \in \N}\), it must be that
  \(\sym\) is in the image of \(\ch\). Furthermore, since \(\ch\) is
  also injective, it must be that \(\ch\) is an isomorphism.
\end{proof}
It bears repeating from the proof above.
\begin{cor}[Corollary of proof]
   \(\ch(1_n) = h_n\) for \(1_n\) the irreducible
   character of the trivial representation of \(\Sym_n\).
\end{cor}
\begin{prop}
  We have that, under the characteristic map, the elementary functions
  \(e_n\) correspond to the character of the sign representation of
  \(\Sym_n\), say \(\epsilon\). 
\end{prop}
\begin{proof}
  By our alternate characterization of the characteristic map, \[
    \ch(\epsilon) = \sum_{\lambda \partitionof n} z_\lambda^{-1}
    \epsilon(\lambda) p_\lambda = e_n
  \]
  where the last equality follows from an argument on generating
  functions for \(p_n\) and \(e_n\) (see \cite{alg-comb}).
\end{proof}
\begin{prop}
  The irreducible characters of \(\Sym_n\) are given by
  \(\{\ch^{-1}(s_\lambda) \st \lambda \partitionof d\}\). 
\end{prop}
\begin{proof}
  Recall that the irreducible characters of a group \(G\) form an
  orthonormal basis for the set of class functions of \(G\) under the
  inner product \((\cdot, \cdot)\), and since the set of class
  functions is a \(\Z\)-module, this basis is unique. Since \(\ch\) is
  an isometry and 
  the Schur functions \(s_\lambda\) form an orthonormal basis of
  \(\sym\) under the Hall-inner product, it must be that
  \(\{\ch^{-1}(s_\lambda) \st \lambda \partitionof n\}\) is the set of
  all irreducible characters of \(\Sym_n\) up to sign. We will later
  show they are all
  positive when evaluated on \(1 \in \Sym_n\).
\end{proof}
\begin{defn}
  We will denote the irreducible character \(\chi_\lambda :=
  \ch^{-1}(s_\lambda)\). 
\end{defn}
\begin{prop}
  \(\chi_\lambda = \det(1_{\lambda_i-i+j})_{1 \leq i,j \leq
    n}\) where \(1_{\lambda_i-i+j}\) is the trivial character
  for \(\Sym_{\lambda_i-i+j}\) (and \(0\) if \(\lambda_i-i+j \leq 0\)).
\end{prop}
\begin{proof}
  The Jacobi-Trudi identity tells us that, for \(\lambda \partitionof
  n\),
  \[
    s_\lambda = \det(h_{\lambda_i-i+j})_{1 \leq i,j \leq n}
  \]
  From above, we have \(\ch(1_n) = h_n\) and so, apply
  \(\ch^{-1}\) to both sides, we get our result.
\end{proof}
\begin{thm}[Frobenius Character Formula]
  \cite{manivel}*{1.6.6} Given a partition \(\mu \partitionof n\), \[
    p_\mu = \sum_{\lambda \partitionof n} \chi_\lambda(\mu) s_\lambda
  \]
  where \(\chi_\lambda(\mu) = \chi_\lambda(w)\) for \(w \in \Sym_n\)
  of cycle type \(\mu\).
\end{thm}
\begin{proof}
  First, we observe that \(\ch^{-1}(p_\mu) = z_\mu f_\mu\) where \[
    f_\mu(w) =
    \begin{cases}
      1 & \text{ if }w\text{ has cycle type }\mu\\
      0 & \text{ else}
    \end{cases}
  \]
  Next, by the fact that \(\ch\) is an isometry, \[
    \langle s_\lambda, p_\mu \rangle = (\chi_\lambda, z_\mu f_\mu) =
    \frac{1}{n!}\sum_{w \in \Sym_n} \chi_\lambda(w) z_\mu f_\mu(w) =
    \frac{z_\mu}{n!} \sum_{w \text{ with cycle type }\mu}
    \chi_\lambda(w) = \chi_\lambda(\mu)
  \]
  since the size of the conjugacy class is
  \(\frac{n!}{z_\mu}\). Therefore, for \(1 \in \Sym_n\), \[
    \chi_\lambda(1) = \langle s_\lambda, p_{1^n} \rangle = \langle
    s_\lambda, h_{1^n} \rangle = K_{\lambda,1^n} > 0
  \]
  since \(K_{\lambda,1^n}\) is the number of standard tableaux of
  shape \(\lambda\).
\end{proof}
\begin{cor}[Corollary of proof]
  \cite{manivel}*{Corollary 1.6.8} The dimension of the irreducible representation of \(\Sym_n\) with
  character \(\chi_\lambda\) is equal to the number of standard
  tableaux of shape \(\lambda\).
\end{cor}
\begin{cor}
  We can invert the Frobenius character formula to get \[
    s_\lambda = \sum_{\mu \partitionof n} z_\mu^{-1} \chi_\lambda(\mu) p_\mu
  \]
\end{cor}
\begin{proof}
  We know from our arguments proving Frobenius reciprocity that \[
    s_\lambda = \ch(\chi_\lambda) = \sum_{\mu \partitionof n}
    z_\mu^{-1} \chi_\lambda(\mu) p_\mu
  \]
  where the second equality follows from our alternate
  characterization of the characteristic map.
\end{proof}
\section{Explicitly Constructing Representations}
Given our knowledge of character theory above, let us systematically
construct some representations.
\begin{defn}
  Given a vector space \(V\) let \(\Sym_d\) act on \(V^{\otimes d} = V
  \otimes \cdots \otimes V\) by permuting the terms of the tensor
  product. In other words, for \(v_1, v_2, \ldots, v_n \in V\) (not
  necessarily distinct), let \[
    w.(v_1 \otimes \cdots \otimes v_n) = v_{w(1)} \otimes \cdots
    \otimes v_{w(n)}
  \]
\end{defn}
Given the symmetric group action defined above, we can also induce the
action on \(\SymP^r V\) and \(\ExtP^r V\).
\begin{prop}
  Given the action of \(\Sym_r\) on \(V^{\otimes r}\), we get that
  \begin{enumerate}
  \item \(\SymP^r V\) is the trivial representation with character
    \(h_r\) under the characteristic map.
  \item \(\ExtP^r V\) is the sign representation with character
    \(e_r\) under the characteristic map.
  \end{enumerate}
\end{prop}
\begin{proof}
   First, consider \(\SymP^{r} V\) as a representation of
  \(\Sym_r\). Then, any \(w \in \Sym_r\) permutes the terms of \(v_1
  \otimes \cdots \otimes v_d\), but this yields the same elementby
  definition of the symmetric power. Thus, this must be the trivial
  representation of \(\Sym_r\) with character \(h_{r}\).

  Similarly, if we consider \(\ExtP^r V\) as a representation of
  \(\Sym_r\), \(w \in \Sym_r\) permutes the terms of \(v_1 \wedge
  \cdots \wedge v_r\), but then \[
    v_{w(1)} \wedge \cdots \wedge v_{w(r)} = \sign(w) (v_1 \wedge
    \cdots \wedge v_r)
  \]
  by definition of the exterior power. Thus, we get that \(\ExtP^r V\)
  is the sign representation of \(\Sym_r\) with character \(e_r\).
\end{proof}
\begin{cor}
  Given \(\Sym_{r_1} \times \cdots \times \Sym_{r_\ell}\),
  then \[
    \SymP^{r_1}(V) \otimes \cdots \otimes \SymP^{r_\ell}(V)
  \]
  is the trivial representation and \[
    \ExtP^{r_1}(V) \otimes \cdots \otimes \ExtP^{r_\ell}(V)
  \]
  is the sign representation.
\end{cor}
\begin{proof}
  The first assertion follows immediately from the action of the group
  on this symmetric power; the action must be trivial. Similarly, from
  the above, it is almost immediate that \begin{align*}
    (w_1, \ldots, w_r).(u_1 \wedge \cdots \wedge u_{r_1} \otimes v_1
    \wedge \cdots \wedge v_{r_2} \otimes \cdots \otimes w_1 \wedge
    \cdots \wedge w_{r_\ell} ) \\ = (\sign(w_1) \sign(w_2) \cdots \sign(w_r))(u_1 \wedge \cdots \wedge u_{r_1} \otimes v_1
    \wedge \cdots \wedge v_{r_2} \otimes \cdots \otimes w_1 \wedge
    \cdots \wedge w_{r_\ell} )
  \end{align*}
\end{proof}
\begin{defn}
  For a partition \(\lambda = (\lambda_1, \ldots, \lambda_\ell)\), let
  \(\Sym_\lambda := \Sym_{\lambda_1} \times \cdots \times
  \Sym_{\lambda_\ell}\). Then, we define induced modules \[
    H_\lambda := \Ind_{\Sym_\lambda}^{\Sym_r}(\rho_1) \ \ E_{\lambda'}
    := \Ind_{\Sym_{\lambda'}}^{\Sym_r}(\rho_{sign})
  \]
  where \(\rho_1\) is the trivial representation and \(\rho_{sign}\)
  is the sign representation.
\end{defn}
\begin{prop}
  Given a partition \(\lambda\), the characterstic map applied to the
  character of 
  \(H_\lambda\) gives \(h_\lambda\) and the characteristic map applied
  to the character of \(E_{\lambda'}\) gives \(e_{\lambda'}\).
\end{prop}
\begin{proof}
  Let \(\chi_{\lambda_i}\) be the character of the trivial
  representation for \(\Sym_{\lambda_i}\). Then,
  consider that the character of \(H_\lambda\) is
  \(\Ind_{\Sym_{\lambda}}^{\Sym_r}(\chi_{\lambda_1} \times \cdots
  \times \chi_{\lambda_\ell}) = \chi_{\lambda_1} \cdots
  \chi_{\lambda_\ell}\). Thus, since the characteristic map is a ring
  isomorphism, \[
    \ch(\chi_{\lambda_1} \cdots
  \chi_{\lambda_\ell}) = \ch(\chi_{\lambda_1}) \cdots
  \ch(\chi_{\lambda_\ell}) = h_{\lambda_1} \cdots h_{\lambda_\ell} =
  h_\lambda 
  \]
  A nearly identical argument gives the result for \(E_{\lambda'}\).
\end{proof}
\section{Young Symmetrizers and Specht Modules}
The Frobenius character formula suggests that we will need to have the
symmetric group act on polynomials associated to standard
tableaux in order to explicitly realize the irreducible
representations of \(\Sym)_n\). There are a few ways to do this, one
of which we expand on below, following \cite{fulton-harris}.
\begin{defn}
  Given a tableau \(\T\) of shape \(\lambda\) labelled with integers
  \(1, \ldots, d\), we define subgroups of \(\Sym_d\) \[
    R_\T := \{w \in \Sym_d \st w \text{ preserves each row of
    }\T\}
  \]
  and \[
    C_\T := \{w \in \Sym_d \st w \text{ preserves each column of
    }\T\}
  \]
  Furthermore, we define elements of \(\C\Sym_d\), the \de{row
    stabalizer} \[
    a_\T := \sum_{w \in R_\T} e_w
  \]
  and the \de{column stabalizer} \[
    b_\T := \sum_{w \in C_\T} \sgn(w) e_w
  \]
  If \(\T^*\) is the canonical standard tableau of shape \(\lambda\), we
  define \[
    R_\lambda := R_{\T^*}, C_\lambda := C_{\T^*}, a_\lambda :=
    a_{\T^*}, b_{\lambda} := b_{\T^*}
  \]
\end{defn}
\begin{prop}
  Given that action of \(\Sym_d\) on \(V^{\otimes d}\) via \[
    w.(v_1 \otimes \cdots \otimes v_d) = v_{w(1)} \otimes \cdots
    \otimes v_{w(d)}
  \]
  that is, \(w\) permutes the terms in \(v_1 \otimes \cdots \otimes
  v_d\), we observe
  \begin{enumerate}
  \item \[
      \im(a_\lambda) = \SymP^{\lambda_1} V \otimes \SymP^{\lambda_2} V
      \otimes \cdots \otimes \SymP^{\lambda_\ell} V
    \]
  \item \[
      \im(b_\lambda) = \ExtP^{\lambda_1'} V \otimes \ExtP^{\lambda'_2} V
      \otimes \cdots \otimes \ExtP^{\lambda'_k} V
    \]
    where \(\lambda' = (\lambda_1', \ldots, \lambda_k')\) is the
    conjugate partition to \(\lambda\).
  \end{enumerate}
\end{prop}
\begin{proof}
  Now, we check \[
    a
  \]
\end{proof}
\begin{defn}
  We define the \de{Young symmetrizer} to be the element \[
    c_\lambda := a_\lambda b_\lambda
  \]
\end{defn}
\begin{example}
  If \(\lambda = (d)\), then \[
    c_{(d)} = a_{(d)} = \sum_{w \in \Sym_d} e_w
  \]
  and when \(\lambda = (1,\ldots,1)\), then \[
    c_{(1,\ldots,1)} = b_{(1, \ldots, 1)} = \sum_{w \in \Sym_d}
    \sgn(w)e_w
  \]
  Finally, for \(\lambda = (2,1)\), we have \[
    c_{(2,1)} = (e_1+e_{(12)})(e_1-e_{(13)}) = 1 +
    e_{(12)}-e_{(13)}-e_{(132)} 
  \]
  We will compute many other examples as needed.
\end{example}
\begin{thm}
  Given a partition \(\lambda\),
  \begin{enumerate}
  \item \(c_\lambda^2 = n_\lambda c_\lambda\), that is, \(c_\lambda\)
    is a scalar multiple of an idempotent.
  \item \(\C \Sym_d \cdot c_\lambda\) is an irreducible representation
    of \(\Sym_d\), say \(V_\lambda\).
  \item Every irreducible representation of \(\Sym_d\) can be obtained
    in this way.
  \item Since conjugacy classes in \(\Sym_d\) are given by cycle type,
    which is encoded in a partition, this sets up a one-to-one
    correspondence between conjugacy classes of \(\Sym_d\) and
    irreducible representations of \(\Sym_d\).
  \end{enumerate}
\end{thm}
\begin{prop}
  For any \(d\), we have \[
    V_{(d)} = \C \Sym_d \cdot \sum_{w \in \Sym_d} e_w = \C\cdot\left(
      \sum_{w \in \Sym_d} e_w \right)
  \]
  which is the trivial representation of \(\Sym_d\) since the sum is
  fixed by any \(w \in \Sym_d\). Similarly, \[
    V_{(1^d)} = \C \Sym_d \cdot \sum_{w \in \Sym_d} \sgn(w) e_w = \C
    \cdot \left( \sum_{w \in \Sym_d} \sgn(w)e_w \right)
  \]
  since the alternating sum has \(w'.\sum_{w \in \Sym_d} \sgn(w)e_w =
  \sgn(w') \sum_{w \in \Sym_d} \sgn(w) e_w\).
\end{prop}
\begin{prop}
  For \(\lambda\) a partition, \[
    V_{\lambda'} = V_\lambda \otimes U'
  \]
  where \(U'\) is the alternating representation.
\end{prop}
Recall from \cite{alg-comb} the Vandermonde determinant \[
  \Vdet = \prod_{i < j}(x_i-x_j)
\]
and the power sum symmetric functions \[
  p_j(x) = x_1^j + \cdots + x_\ell^j
\]
We then have, for \(\chi_\lambda\) the character of \(V_\lambda\),
\begin{thm}[Frobenius Formula]
  Let \(C_\mu\) be the conjugacy class of \(\Sym_d\) of elements of
  cycle type \(\mu = (\mu_1, \ldots, \mu_d)\) where \(\mu_i\) is the
  number of \(i\) cycles in the permutations of \(C_\mu\). Then, \[
    \chi_\lambda(C_\mu) = \left( \Vdet \cdot \prod_j p_j(x)^{\mu_j}
    \right)_{\lambda+\rowshift}
  \]
  where \(\rowshift = (\ell-1, \ell-2,\ldots,1,0)\) and so the right
  hand side is the coefficient of \[
    x^{\lambda+\rowshift} = x^{\lambda_1+\ell-1}x^{\lambda_2+\ell-2}
    \cdots x^{\lambda_\ell}
  \]
  Rewriting the formula, we get \[
    \prod_j p_j(x)^{\mu_j} = \sum_{\lambda \partitionof d,
      \ell(\lambda) \leq \ell} \chi_\lambda(C_\mu) s_\lambda(x)
  \]
  where \(s_\lambda(x)\) is a Schur function.
\end{thm}
\begin{cor}
  If \(\nu = \lambda+\rowshift\)
  \[
    \dim V_\lambda = \frac{d!}{\nu_1!\nu_2!\cdots\nu_\ell!} \prod_{i <
    j} (\nu_i - \nu_j)
  \]
\end{cor}
\begin{proof}[Proof of Corollary]
 The identity element \(id \in \Sym_d\) is the only element in the
 conjugacy class indexed by \(\mu = (d)\). So, the Frobenius formula
 gives \[
   \dim V_\lambda = \chi_{\lambda}(C_{(d)}) = (\Vdet \cdot
   (x_1+\cdots+x_\ell)^d)_{\lambda+\rowshift}
 \]
 However, using the Leibniz determinant formula, \[
   \Vdet = \sum_{\sigma \in \Sym_\ell} (\sgn(\sigma)) \prod_{j=1}^\ell
   x_j^{\sigma(\ell-j+1)-1} 
 \]
 and using the multinomial formula \[
   (x_1+\cdots+x_\ell)^d = \sum_{r_1+\cdots+r_\ell=d}
   \frac{d!}{r_1!\cdots r_\ell!} x_1^{r_1} \cdots x_{\ell}^{r_\ell}
 \]
 % So, the coefficient of \(x^{\lambda+\rowshift}\) is given by \[
 %   \sum_{\sigma \in \Sym_\ell,} \sgn(\sigma)
 %   \frac{d!}{\prod_{j=1}^\(\lambda_j+\ell-1-((j)))!} =
 %   \sum_{\sigma \in \Sym_\ell,} \sgn(\sigma) \frac{d!}{\lambda_j-j-1+\sigma(j)}
 % \]
 \todo{Finish this mess some other time}
\end{proof}
\begin{prop}
  We also have \[
    \dim V_\lambda = \frac{d!}{\prod (\text{Hook lengths})}
  \]
\end{prop}
\section{Two Sides of the Same Coin}
Using symmetric function theory, we prove some results about the
characters on \(\Sym_n\).
\begin{thm}[Branching Rule]
  Let \(\mu \partitionof n\). Then, \[
    \Ind_{\Sym_{n-1}}^{\Sym_n} \chi_\mu = \sum_{\lambda = \mu+\text{an
      addable cell} } \chi_\lambda
  \]
  Similarly, \(\lambda \partitionof n\). Then, \[
    \Res_{\Sym_{n-1}}^{\Sym_n} \chi_\lambda = \sum_{\mu = \lambda -
      \text{a removable cell}} \chi_\mu 
  \]
\end{thm}
\begin{proof}
  The first statement follows from the Pieri rule. Namely, \[
    \ch(\Ind_{\Sym_{n-1} \times 1}^{\Sym_n} (\chi_\mu \times
    \chi_{(1)})) = \ch(\chi_\mu) \ch(\chi_{(1)}) = s_\mu s_1 = h_1
    s_\mu = \sum_{\lambda = \mu + \text{horizontal }1\text{-strip}}
    s_\lambda 
  \]
  Thus giving us the result after taking \(\ch^{-1}\). The second
  result follows from Frobenius reciprocity. Namely, \[
    \langle \Ind_{\Sym_{n-1}}^{\Sym_n} \chi_\mu, \chi_\lambda
    \rangle = \langle \chi_\mu, \Res_{\Sym_{n-1}}^{\Sym_n}
    \chi_\lambda \rangle  
  \]
\end{proof}
\begin{thm}[Young's Rule]
  If \(\lambda \partitionof n\), then the multiplicity of \(S^\mu\) in
  \(H_\lambda\) is equal to \(K_{\mu \lambda}\) 
\end{thm}
\begin{proof}
  We know \[
    h_\lambda = \sum_\mu K_{\mu \lambda} s_\mu \implies 
    K_{\mu,\lambda} = \langle h_\lambda, s_\mu \rangle = (H_\lambda ,
    S^\mu)
  \]
\end{proof}
\begin{thm}[Murnaghan-Nakayama Rule]
  Given partitions \(\lambda, \mu \partitionof n\), the irreducible
  character \(\chi_\lambda\) of \(\Sym_n\) has value on the conjugacy
  class of cycle type \(\mu\), \[
    \chi_\lambda(\mu) = \sum_{\T} (-1)^{\height(\T)}
  \]
  where the sum is over all multi-ribbon tableaux with shape
  \(\lambda\) and weight \(\mu\).
\end{thm}
\begin{proof}
  If we take \[
    \sum_{\lambda \partitionof n} \chi_\lambda(\mu) s_\lambda =
    p_{\mu_1} \cdots p_{\mu_\ell} = \sum_{\T \text{ of weight }\mu}
    (-1)^{\height(\T)} s_{\sh(\T)} 
  \]
  where \(\T\) is a multi-ribbon tableau. (A ribon of length \(\mu_1\)
  labeled \(1\), a ribbon of length \(\mu_2\) labeled \(2\), and so
  on). Since the Schurs are a basis, gives \[ 
    \chi_\lambda(\mu) = \sum_{\T \text{ of weight }\mu\text{ and shape
      }\lambda} (-1)^{\height(\T)}
  \]
\end{proof}
\begin{bibdiv}
  \begin{biblist}
    \bib{fulton-harris}{book}{
      author={Fulton, William}
      author={Harris, Joe}
      title={Representation Theory: A First Course}
      year={1991}
    }
    \bib{manivel}{book}{
      author={Manivel, Laurent}
      title={Symmetric Functions, Schubert Polynomials, and Degeneracy
        Loci}
      year={1998}
      note={Translated by John R. Swallow; 2001}
    }
    \bib{alg-comb}{misc}{
      author={Seelinger, George H.}
      title={Algebraic Combinatorics}
      year={2018}
      note={[Online] \url{https://ghseeli.github.io/grad-school-writings/class-notes/algebraic-combinatorics.pdf}}
    }
    \bib{rep-thry-of-finite-gps}{misc}{
      author={Seelinger, George H.}
      title={Representation Theory of Finite Groups}
      year={2017}
      note={[Online] \url{https://ghseeli.github.io/grad-school-writings/class-notes/representation-theory-of-finite-groups.pdf}}
    }
  \end{biblist}
\end{bibdiv}

\end{document}