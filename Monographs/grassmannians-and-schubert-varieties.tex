\documentclass[11pt,leqno,oneside]{amsart}
\usepackage[alphabetic,abbrev]{amsrefs} % use AMS ref scheme
\usepackage{../ReAdTeX/readtex-core}
\usepackage{../ReAdTeX/readtex-dangerous}
\usepackage{../ReAdTeX/readtex-abstract-algebra}
\usepackage{../ReAdTeX/readtex-topology}
\usepackage{./monographs}
\usepackage{caption}
\usepackage{subcaption}
\usepackage{todonotes}
\usepackage{ytableau}
\usepackage{xcolor}
\usepackage{mathtools}
\usepackage{pgf}
\usepackage{tikz}
\usetikzlibrary{shapes.geometric}
\usepackage{tikz-cd}
\usepackage{graphicx}
\numberwithin{thm}{section}

\newcommand{\T}{\mathsf{T}} % use for tableaux
\renewcommand{\S}{\mathsf{S}}
\newcommand{\TP}{\mathsf{P}}
\newcommand{\TQ}{\mathsf{Q}}
\newcommand{\inv}{\operatorname{inv}}
\newcommand{\charge}{\operatorname{charge}}
\newcommand{\cocharge}{\operatorname{cocharge}}
\newcommand{\des}{\operatorname{des}}
\newcommand{\row}{\operatorname{row}}
\newcommand{\col}{\operatorname{col}}

\newcommand{\Vdet}{\Delta}
\newcommand{\rowshift}{\rho}
\newcommand{\SSYT}{\operatorname{SSYT}} % Semistandard Young Tableaux
\newcommand{\defeq}{\coloneqq}
\newcommand{\partitionof}{\vdash}
\newcommand{\dominates}{\mathrel{\unrhd}}
\newcommand{\strictlydominates}{\mathrel{\rhd}}
\newcommand{\dominatedby}{\mathrel{\unlhd}}
\newcommand{\strictlydominatedby}{\mathrel{\lhd}}
\newcommand{\covers}{\mathrel{\gtrdot}}
\newcommand{\coveredby}{\mathrel{\lessdot}}

\newcommand{\sym}{\Lambda}
\renewcommand{\k}{\Bbbk}

\newcommand*\circled[1]{\tikz[baseline=(char.base)]{
   \node[shape=circle,draw,inner sep=1pt] (char) {#1};}}
\newcommand{\mymk}[1]{\tikz \node[draw,circle, inner sep=0pt, minimum size=7mm]{#1};}

\makeatletter
\newcommand\mathcircled[1]{%
  \mathpalette\@mathcircled{#1}%
}
\newcommand\@mathcircled[2]{%
  \tikz[baseline=(math.base)] \node[draw,circle,inner sep=1pt] (math) {$\m@th#1#2$};%
}
\makeatother

\newcommand{\quantumBinom}[3][q]{\left[ \begin{array}{c} #2 \\
                                          #3 \end{array} \right]_#1}
\newcommand{\qbinom}{\quantumBinom}
\newcommand{\stat}{\operatorname{stat}}
\newcommand{\Des}{\operatorname{Des}}
\newcommand{\maj}{\operatorname{maj}}
\newcommand{\comaj}{\operatorname{comaj}}
\newcommand{\height}{\operatorname{ht}}
\newcommand{\size}{\operatorname{size}}
\newcommand{\sh}{\operatorname{sh}}
\newcommand{\spin}{\operatorname{spin}}
\newcommand{\cospin}{\operatorname{cospin}}
\newcommand{\wt}{\operatorname{wt}}
\newcommand{\length}{\operatorname{len}}
\newcommand{\LLT}{\operatorname{LLT}}

\renewcommand{\P}{\mathbb{P}}
\newcommand{\A}{\mathbb{A}}
\newcommand{\ExtP}{\bigwedge}

\title[Grassmannians and Schubert Varieties]{Grassmannians and
  Schubert Varieties}
\author{George H. Seelinger}
\date{August 2018}
\begin{document}
\maketitle
\section{Introduction}
\section{Projective Varieties}
The language of projective varieties is inherently useful for
describing Grassmannians. Therefore, this section contains a terse
reiteration of the relevant notions about projective varieties. For a
more complete treatment, consult any introductory text on algebraic
geometry. 
\begin{defn}
  Given \(n \in \N\), we define \de{projective \(n\)-space} over a
  field \(K\), denoted \(\P_K^n\) or \(\P^n\) when \(K\) is
  understood, to be the set of all 
  \(1\)-dimensional linear subpsaces of the vector space \(K^{n+1}\). 
\end{defn}
\begin{example}
  The easiest example of a projective variety is \(\P^1\). If we take
  two copies of the affine line \(\A^1\), denoted \(X_1, X_2\), and
  ``glue'' the open sets \(U,U' = \A^1 \setminus \{0\}\) together via
  the map \(f(x) = \frac{1}{x}\), we get \(\P^1\). In such a space, we
  have the identity ``\(\frac{1}{0} = \infty\)'' in the persective of
  \(X_1\) and so we have a compactification of the affine line. When
  \(K=\C\), this 
  resulting space is the same as the Riemann sphere \(\C \union
  \{\infty\} = \C\P^1\).
\end{example}
\begin{rmk}
  One often describes projective space using ``homogeneous coordinates'',
  which are invariant under scalar multiple, arising from the
  description \[
    \P^n = (K^{n+1} \setminus \{0\})/\sim
  \]
  where \(\sim\) is given by \[
    (x_0, \ldots, x_n) \sim (y_0, \ldots, y_n) \iff \exists \lambda
    \in K^\times \text{ such that } x_i = \lambda y_i \ \forall i
  \]
  We denote such an equivalence class by \([x_0:\cdots:x_n] \in \P^n\).
\end{rmk}
A projective variety can also be described by a set of homogeneous
polynomials \(f_1, \ldots, f_r \in K[x_0,\ldots,x_n]\) since, if
\(f_i\) is homogeneous of degree \(d\), then \[
  f_i(\lambda x_0, \ldots, \lambda x_n) = \lambda^d f(x_0, \ldots,
  x_n), \ \ \lambda \in K
\]
and so all scalar multiples of a solution to \(f_i(x_0, \ldots, x_n) =
0\) are solutions. 
\begin{defn}
  Let \(\pi \from \A^{n+1} \setminus \{0\} \to \P^n\) send \((x_0,
  \ldots, x_n) \mapsto [x_0 : \cdots : x_n]\). Then,
  \begin{enumerate}
  \item We say an affine variety \(X \subset \A^{n+1}\) is a \de{cone}
    if \(0 \in X\) and \(\lambda x \in X\) for all \(\lambda \in K\)
    and \(x \in X\).
  \item Given a cone \(X \subset \A^{n+1}\), we say \[
      \P(X) := \pi(X \setminus \{0\}) = \{[x_0 : \cdots : x_n] \in
      \P^n \st (x_0, \ldots, x_n) \in X\} \subset \P^n
    \]
    is the \de{projectivization} of \(X\).
  \item Given a projective variety \(X \subset \P^n\), we say \[
      C(X) := \{0\} \union \pi^{-1}(X) = \{0\} \union \{(x_0, \ldots,
      x_n) \st [x_0 : \cdots : x_n] \in X\} \subset \A^{n+1}
    \]
    is the \de{cone} over \(X\).
  \end{enumerate}
\end{defn}
\begin{rmk}
  From above, with some additional work, one can show there is a
  one-to-one correspondence
  \begin{align*}
    \{\text{cones in }\A^{n+1}\} & \correspondsto \{\text{projective
    varieties in }\P^n\}\\
    X & \mapsto \P(X)\\
    C(X) & \leftarrow X
  \end{align*}
\end{rmk}
\begin{lem}
  For non-empty projective variety \(X \subset \P^n\), the dimension
  of the cone \(C(X) \subset \A^{n+1}\)  is \(\dim X + 1\).
\end{lem}
\begin{proof}
  Let \[
    \emptyset \neq X_0 \propsubset \cdots \propsubset X_n \subset X
  \]
  be a chain of irreducible closed subsets in \(X\). Then, \[
    \{0\} \propsubset C(X_0) \propsubset \cdots \propsubset C(X_n)
    \subset C(X)
  \]
  is a chain of irreducible closed subsets in \(C(X)\). So, we have
  that \[
    \dim C(X) \geq \dim X + 1
  \]
  Similarly, if we show \todo{Do this explicitly} \[
    \codim(C(X)) \geq \codim(X) = n - \dim X
  \]
  If we assume \(X\) is irreducible (if not, take an irreducible
  decomposition), then \[
    \dim(C(X))+ \codim(C(X)) = \dim(\A^{n+1}) = n+1
  \]
  and so \[
    \dim(C(X)) = n+1 - \codim(C(X)) \leq n+1-n+\dim(X) = \dim(X)+1
  \]
\end{proof}
\section{Grassmannians}
\subsection{Defining and Understanding Grassmannians}
\begin{defn}
  A \de{Grassmannian}, denoted \(G(m,n)\), is the set of
  linear subpsaces of dimension \(m\) 
  (and therefore of codimension \(n\)) in \(K^{m+n}\). 
\end{defn}
In this way, one can represent the Grassmannian \(G(m,n)\) as
equivalence classes of full rank \(m \times (m+n)\) matrices whose 
row space encodes a linear subspace in \(G(m,n)\),
where the equivalence class is given by row operations, or more
formally \[ 
  A \sim B \iff A = E B,\ E \in GL_m(K)
\]
Therefore, we can always pick represenetatives of the form \[
  \left(
    \begin{array}{ccccccc}
      1&0&\cdots&0&x_{11}&\cdots&x_{1n}\\
      0&1&\cdots&0&x_{21}&\cdots&x_{2n}\\
      \vdots&&\ddots&&\vdots&&\vdots\\
      0&0&\cdots&1&x_{m1}&\cdots&x_{mn}
    \end{array}
\right)
\]
\begin{prop}
  The \(m\)-dimensional linear subspaces of \(K^{m+n}\) are in natural
  one-to-one correspondence with \((m-1)\)-dimensional linear
  subspaces of \(\P^{m+n-1}\). 
\end{prop}
\begin{proof}
  Given such a subspace of \(K^{m+n}\), an \(m\)-dimensional linear
  subspace \(W\) (which is defined by a set of homogeneous
  polynomials and thus a cone) can be sent to \(\P(W)\), which has
  dimension \(m-1\) in \(\P^{m+n-1}\).
\end{proof}
Therefore, we may also think of \(G(m,n)\) as the set of such
projective linear subspaces.
\subsection{The Pl\"{u}cker Embedding}
We wish to give a concrete realization of \(G(m,n)\) as a subset of
projective space via the ``Pl\"{u}cker Embedding.''
\begin{lem}
  If \(W \in G(m,n)\), then \(\ExtP^m W\) is a line (a cone of dimension
  \(1\)) in \(\ExtP^m K^{m+n}\).
\end{lem}
\begin{proof}
  Given \(W \isom K^m = \Span \{e_1, \ldots, e_n\}\), we have that
  \(\ExtP^m K^m \isom K\) since it 
  only has a single basis vector, namely \(e_1 \wedge \cdots \wedge
  e_n\).
\end{proof}
\begin{lem}
  Let \(v_1, \ldots, v_k \in K^n\) and \(w_1, \ldots, w_k \in K^n\)
  both be linearly independent. Then, \(v_1 \wedge \cdots \wedge v_k\)
  and \(w_1 \wedge \cdots \wedge w_k\) are linearly dependent in
  \(\ExtP^k K^n\) if and only if \(\Span \{v_1, \ldots, v_k\} =
  \Span\{w_1, \ldots, w_k\}\).
\end{lem}
\begin{proof}
  This is an exercise using only multi-linear algebra skills.
\end{proof}
\begin{thm}
  The map
  \begin{align*}
    \phi \from G(m,n) & \to \P(\ExtP^m K^{m+n}) \\
    W = \Span \{w_1, \ldots, w_m\} & \mapsto [w_1 \wedge \cdots
                                     \wedge w_m]
  \end{align*}
  is an embedding, called the \de{Pl\"{u}cker Embedding}.
\end{thm}
\begin{proof}
  The map is well-defined by the \(\impliedby\) direction of the lemma
  above and is injective by the \(\implies\) direction of the lemma
  above. 
\end{proof}
However, the description given by the proposition is not very
concrete. Using multi-linear algebra, we can remedy this problem.
\begin{lem}
  Given \(0 \leq k \leq n\) and \(v_1, \ldots, v_k \in K^n\) with
  \(v_j = \sum_i a_{j,i} 
  e_i\), we have \[
    v_1 \wedge \cdots \wedge v_k = \sum_{i_1, \ldots, i_k} a_{1,i_1}
    \cdots a_{k,i_k} \cdot e_{i_1} \wedge \cdots \wedge e_{i_k}
  \]
  and, for stricly increasing index sequence \((j_1, \ldots, j_k)\),
  we get the coefficient of \(e_{j_1} \wedge \cdots \wedge e_{j_k}\)
  in \(v_1 \wedge \cdots \wedge v_k\) is given by \[
    \sum \sgn(\sigma) a_{1,j_{\sigma(1)}} \cdots a_{k,j_{\sigma(k)}} =
    \det (a_{i,j})_{1 \leq i \leq k, j \in \{j_1, \ldots, j_k\}}
  \]
\end{lem}
Thus, we have
\begin{thm}[Pl\"{u}cker Coordinates]
  The homogeneous coordinates of \(\phi(W)\) in \(\P(\ExtP^m
  K^{m+n})\) are the minors of order \(m\) of the matrix whose row
  space is \(W\), say \((x_{i,j})_{1 \leq i \leq m, 1 \leq j \leq
    m+n}\), which we denote \[
    P_{i_1, \ldots, i_m} := \det(x_{p,i_q})_{1
      \leq p,q \leq m}, \ i_1 
    < \cdots < i_m
  \]
\end{thm}
\begin{example}
  Consider \(G(1,n-1)\) will have \[
    \Span\{a_1 e_1 + \cdots + a_n e_n\} \mapsto [a_1: \cdots :a_n] \in
    \P^{n-1} 
  \]
  Furthermore, since this map is surjective, we get \(G(1,n-1) \isom
  \P^{n-1}\), as discussed above.

  For a more complicated example, consider \(G(2,1)\). If we take the
  space spanned by \(e_1+e_2,e_1+e_3\), then we have \[
    \left(
      \begin{array}{ccc}
        1&1&0\\
        1&0&1
      \end{array}
\right) \mapsto [P_{1,2}:P_{1,3}:P_{2,3}] = \left[ \det \left(
    \begin{array}{cc}
      1&1\\
      1&0
    \end{array}
\right), \det \left(
  \begin{array}{cc}
    1&0\\
    1&1
  \end{array}
\right), \det \left(
  \begin{array}{cc}
    1&0\\
    0&1
  \end{array}
\right) \right] = [-1:1:1]
  \]
  
\end{example}
Thus, we have realized the Grassmannian in projective
space. However, we wish to show it is a variety. To do that, we must
show that it is a vanishing set of polynomials.
\begin{defn}
  \begin{enumerate}
  \item Let \(J_{m,n}\) be the set of strictly increasing\(m\)-tuples
  consisting of integers between \(1\) and \(m+n\).
  \item Let \(K[P_J \st J \in J_{m,n}]\) be the ring of polynomials in
    the Pl\"{u}cker coordinates.
  \item Let \(I(G(m,n))\) be the ideal consisting of homogeneous
    polynomials vanishing identically on \(G(m,n)\). 
  \end{enumerate}
\end{defn}
\begin{thm}
  Let \(i_1, \ldots, i_m\) and \(j_1, \ldots, j_m\) be two sets of
  integers between \(1\) and \(m+n\) and let \(l\) be an integer
  between \(1\) and \(m\). Then, identically on \(G(m,n)\), we have
  the relation \[
    \sum_{w \in
      \Sym_{i_l,\ldots,i_m,j_1,\ldots,j_l}/\Sym_{i_l,\ldots,i_m}
      \times \Sym_{j_1, \ldots, j_l}} \sgn(w) P_{i_1, \ldots,
      i_{l-1},w(i_l), \ldots,w(i_m)} P_{w(j_1),\ldots,w(j_l), j_{l+1},
    \ldots, j_m} = 0
  \]
\end{thm}
\begin{example}
  \cite{manivel}*{Exercise 3.1.5} If \(m=n=2\) and \(l=2\), then
  \(\Sym_{2,3,4}/\Sym_{2} \times 
  \Sym_{3,4} 
  \isom \{id, (23), (243)\}\) and we have the single equation \[
    P_{1,2}P_{3,4} - P_{1,3}P_{2,4} + P_{1,4} P_{2,3} = 0
  \]
  One can use this relation to show that the Pl\"{u}cker embedding
  gives \(G(2,2)\) as a quadratic hypersurface of \(\P^5\).
\end{example}
\begin{proof}
  \todo{Fill in proof}
\end{proof}
% However, the work
% involved is non-trivial, so we refer the reader to the lecture notes \cite{gathmann}
\section{Schubert Cells and Schubert Varieties}
\begin{defn}
  \cite{manivel}*{p 105} Fix a flag \[
    0 = V_0 \subset V_1 \subset \cdots \subset V_i \subset \cdots
    \subset V_{n+m} = \C^{m+n}
  \]
  Then, if \(\lambda\) is a partition contained in an \(n \times m\)
  rectangle, that is \(n \geq \lambda_1 \geq \cdots \geq \lambda_m
  \geq 0\), we associate to it the \de{Schubert cell} \[
    \Omega_\lambda := \{W \in G(m,n) \st \dim(W \intersect V_j) = i
    \text{ if } n+i-\lambda_i \leq j \leq n+i-\lambda_{i+1}\}
  \]
  and the \de{Schubert variety} \[
    X_\lambda = \{W \in G(m,n) \st \dim(W \intersect
    V_{n+i-\lambda_i}) \geq i, 1 \leq i \leq m\}
  \]
\end{defn}
\begin{example}
  \begin{enumerate}
  \item We have that \(X_\emptyset = G(m,n)\) since \[
      W \in G(m,n) \implies \dim W = m \implies \dim W + \dim V_{n+i}
      = m+n+i \implies \dim(W \intersect V_{n+i}) \geq i
    \]
  \item On the other extreme, \(X_{(n^m)} = \{V_m\}\) since \[
      \dim(W \intersect V_{n+i-n}) = \dim(W \intersect V_i) \geq i, \forall 1 \leq i \leq m
      \implies V_i \subset W, \forall 1 \leq i \leq m \implies W = V_m
    \]
  \item If \(\lambda = (k)\), that is, \(\lambda\) has only one part,
    then \[
      X_k = \{W \in G(m,n) \st W \intersect V_{n+1-k} \neq 0\}
    \]
    since \[
      \dim(W \intersect V_{n+1-k}) \geq 1 \iff W \intersect V_{n+1-k} \neq 0
    \]
    and we always have \(\dim(W \intersect V_{n+i}) \geq i\) for \(2
    \leq i \leq m\) 
    as discussed in (a).
  \item If \(\lambda\) is a partition such that its diagram is the
    complement of a \(q \times p\) rectangle inside an \(n \times m\)
    rectangle, then \[
      X_\lambda = \{W \in G(m,n) \st V_{m-p} \subset W \subset
      V_{m+q}\} \isom G(p,q)
    \]
    To see this, consider that \(\lambda\) is of the form \[
      \lambda = (\underbrace{n,n, \ldots, n}_{m-p}, \underbrace{n-q,
        \ldots, n-q}_{p}) 
    \]
    and so, as discussed in (b), we know that the first \(m-p\)
    entries of \(n\)'s will force \(V_{m-p} \subset
    W\). Furthermore, \[
      \dim(W \intersect V_{n+i-(n-q)}) \geq i, \forall m-p+1 \leq i
      \leq m \iff \dim(W \intersect V_{m+q}) \geq m \iff W \subset V_{m+q}
    \]
  \end{enumerate}
\end{example}
One important use of Schubert cells is that they form a cellular
decomposition of the Grassmannian.
\begin{lem}
  We have that \[
    G(m,n) = \Dunion_{\mu \text{ in an }n \times m \text{ rectangle}}
    \Omega_\mu 
  \]
\end{lem}
\begin{proof}
  Given \(W \in G(m,n)\), the sequence \(\{\dim(W \intersect V_i)\}_{1
  \leq i \leq n+m}\) runs from \(0\) to \(m\), increasing at most by
\(1\) for each step. Thus, since there are \(m\) jumps in the
sequence, say \(j_1, \ldots, j_m\), we can define a partition \(\mu\)
via \(\mu_i := n+i-j_i \iff j_i = n+i-\mu_i\) and, since \(j_i \geq
i\), get that \(\mu\) is contained in an \(n \times m\)
rectangle. Thus, \(W \in \Omega_\mu\) and, since we can recover all
such partitions this way, we get the cellular decomposition result.
\end{proof}
\begin{prop}
  \cite{manivel}*{Proposition 3.2.3} For all partitions \(\lambda\)
  inside an \(n 
  \times m\) rectangle, we have
  \begin{enumerate}
  \item the Schubert variety \(X_\lambda\) is an algebraic subvariety
    of \(G(m,n)\).
  \item \(X_\lambda = \Dunion_{\mu \supset
      \lambda} \Omega_\mu\)
  \item \(\Omega_\lambda \isom \C^{mn-|\lambda|}\)
  \item \(X_\lambda \supset X_\mu\) if and only if \(\lambda \subset
    \mu\).
  \item \(\Omega_\lambda\) is a dense open set of \(X_\lambda\)
    contained in the set of nonsingular points.
  \end{enumerate}
\end{prop}
\begin{proof}
  (a) follows from the fact that \[
    \dim(W \intersect V_i) \geq j \iff \rank(W \subset \C^{m+n} \to
    \C^{m+n}/V_i) \leq m-j
  \]
  which is equivalent to the vanishing of minors of order \(m-j+1\) of
  the matrix representing the map.

  For (b), we first note that, as in the proof of the lemma above,
  given a \(W \in 
  X_\lambda\), we have a sequence of dimensions \(\{\dim(W \intersect
  V_j)\}_{1 \leq j \leq n+m}\), but moreover we also
  have that \(\dim(W \intersect V_{n-i+\lambda_j}) \geq i\) for \(1
  \leq i \leq m\) and so, the first \(i\) jumps in our sequence must
  occur before \(n+i-\lambda_i\), which means \(n+i-\lambda_i \geq j_i
  = n+i-\mu_i \implies \lambda_i \leq \mu_i\). Thus, this gives \[
    X_\lambda = \Dunion_{\mu \supset \lambda} \Omega_\mu
  \]

  For (c), we choose a basis of \(\C^{m+n}\) that is compatible with
  our flag, that is \(v_1, \ldots, v_{m+n}\) such that \(V_i =
  \Span\{v_1, \ldots, v_i\}\). Then, for \(W \in \Omega_\lambda\), we
  have a unique basis consisting of vectors of the form \[
    w_i = v_{n+i-\lambda_i} + \sum_{\substack{1 \leq j \leq
        n+i-\lambda_i \\ j \neq n+k-\lambda_k \\ k \leq i}} x_{ij} v_j
  \]
  and thus the matrix defined by \((x_{ij})\) defines a linear
  transformation from \(\Omega_\lambda\) to \(\C^{mn-|\lambda|}\).

  To finish the proof, one argues that \(\Omega_\mu \subset
  \ov{\Omega_\lambda}\) by looking at elements of \(\Omega_\lambda\)
  and considering the limiting values. Thus, we get \(\Omega_\lambda
  \subset X_\lambda = \Dunion_{\mu \supset \lambda} \Omega_\mu \subset
  \ov{\Omega_\lambda}\), but since \(X_\lambda\) is closed, it must be
  that \(X_\lambda = \ov{\Omega_\lambda}\).
\end{proof}
Now, we are particularly interested in the consequences of this result
on the cohomology of the Grassmannian.
\begin{cor}\label{schubert-classes-form-basis}
  \begin{enumerate}
  \item   Thus, the Poincar\'{e} duals of the cohomology classes of
    \(\ov{\Omega_\lambda} = X_\lambda\), say \(\sigma_\lambda :=
    [X_\lambda]^*\), form a basis for the 
    integral cohomology of \(G(m,n)\). We call such classes
    \de{Schubert classes}.
  \item For
    all partitions \(\lambda\) contained in an \(n \times m\)
    rectangle, \(\sigma_\lambda\) is an element of
    \(H^{2|\lambda|(G(m,n))}\)
  \item In fact, we have the decomposition \[
      H^*(G(m,n)) = \bigoplus_{\lambda \text{ in an }n \times m\text{
          rectangle}} \Z \sigma_\lambda
    \]
  \end{enumerate}
\end{cor}
\begin{proof}
  For part (a), since each of the Schubert cells are isomorphic to
  \(\C^{mn-|\lambda|}\), it must be that all the cells are
  concentrated in even (real) dimension, thus giving that there are no
  relations for for the fundamental classes in the cohomology groups,
  and so they form a basis.

  For (b), this follows from the fact that \(\Omega_\lambda\) has real
  dimension \(2(mn-|\lambda|)\), and so when we take the Poincar\'{e}
  dual, we will end up in \(H^{2|\lambda|}\).

  (c) is an immediate consequence of (a).
\end{proof}
To analyze the cohomology further, we define
\begin{defn}
  Given a flag \[
    0 = V_0 \subset \cdots \subset C_{m+n} = \C^{m+n}
  \]
  with compatible basis \(v_1, \ldots, v_{m+n}\), we define the
  \de{dual flag} to be given by \[
    V_i' = \Span\{v_{m+n-i+1}, \ldots, v_{m+n}\}
  \]
  To the dual flag correspond Schubert cells and Schubert varieties,
  which we will denote by \(\Omega_\lambda'\) and \(X_\lambda'\).
\end{defn}
\begin{prop}
  Two Schubert varieties associated to the same partition \(\lambda\)
  (thus differing only by their defining flags)
  have the 
  same Schubert class \(\sigma_\lambda\).
\end{prop}
\begin{proof}
  The connected group \(GL_{m+n}(\C)\) acts transitively on the set of
  complete flags, so there is an element \(g \in GL_{m+n}(\C)\) such
  that \(g\) sends flag \(F\) to flag \(F'\). Since \(GL_{m+n}(\C)\)
  is connected,  there is a path \(H \from [0,1] \to GL_{m+n}(\C)\)
  such that \(H(0) = g\) and \(H(1) = I_{m+n}\).  Furthermore,
  \(GL_{m+n}(\C)\) acts continuously on 
  \(G(m,n)\), so this path induces a homotopy sending \(g \from G(m,n)
  \to G(m,n)\) to the identity map on \(G(m,n)\). Thus, since the two
  cells are homotopic, they are represented by the same (co)homology
  class. 
\end{proof} 
In particular, \(X_\lambda\) and \(X_\lambda'\) have the same
Schubert class \(\sigma_\lambda\).
\begin{lem}
  Given partitions \(\eta, \tau\) inside an \(n \times m\) rectangle,
  we have that \[
    \Omega_\eta \intersect \Omega_\tau' \neq \emptyset \implies \eta
    \subset \hat{\tau}
  \]
\end{lem}
\begin{proof}
  See exponsition in \cite{manivel}*{p 108}. Essentially, this result
  follows from analyzing what an element of \(\Omega_\eta \intersect
  \Omega_\nu'\) must look like, and it forces the desired containment.
\end{proof}
\begin{lem}
  Given irreducible algebraic subvarieties \(Y\) and
  \(Y'\) of a compact and connected complex variety \(X\) with
  codimensions \(c\) and \(c'\), respectively, we have \[
    Y \intersect Y' = \Intersect_{i \in I} Z_i
  \]
  for \(Z_i\) some irreducible subvarieties of \(X\) and, furthermore, 
  if this intersection is ``transverse'' (which implies \(Z_i\) has
  codimension \(c+c'\)), \[ 
    [Y] \cupprod [Y'] = \sum_{i \in I} [Z_i] \in H^{2c+2c'}(X)
  \]
\end{lem}
\begin{proof}
  This is a proof from algebraic geometry. See, for instance,
  \cite{fulton} appendix B. 
\end{proof}
Of course, since cohomology forms a
ring, we may ask about the ring 
structure on \(H^*(G(m,n))\). We get
\begin{prop}
  \cite{manivel}*{Proposition 3.2.7} Given two partitions \(\lambda,\mu\) contained in an \(n \times m\)
  rectangle such that \(|\lambda|+|\mu|=mn\), we get that \[
    \sigma_\lambda \cupprod \sigma_\mu = \delta_{\mu,\hat{\lambda}}
  \]
  where \(\hat{\lambda}\) is the complementary partition partition of
  \(\lambda\) in an \(n \times m\) rectangle, that is, the partition
  such that \(\lambda_i+\hat{\lambda}_{m-i} \leq n\). 
\end{prop}
\begin{proof}
  By the first lemma above, since \(|\lambda|+|\mu| = mn\), it must be
  that \[
    \mu \neq \hat{\lambda} \implies X_\lambda \intersect X_\mu' =
    \emptyset \implies \sigma_\lambda \cupprod \sigma_\mu = 0
  \]
  since \[
    \sigma_\lambda \cupprod \sigma_\mu = [X_\lambda \intersect X_\mu]^*
  \]
  by the second lemma above. Similarly, if \(\mu = \hat{\lambda}\),
  then \(\mu_i + \lambda_{m+1-i} = n\) for all \(i\) and \(X_\lambda
  \intersect X_\mu' \neq \emptyset\). So, take \(W \in X_\lambda
  \intersect X_\mu'\). This requires that  \[
    \begin{cases}
      \dim(W \intersect \Span\{v_1, \ldots, v_{n+i-\lambda_i}\}) \geq i \\
      \dim(W \intersect \Span\{v_{m+n-(n+i-\mu_i)+1}, \ldots,
      v_{m+n}\}) \geq i \\
      \iff \dim(W \intersect
      \Span\{v_{n+(m+1-i)-\lambda_{m+1-i}}, \ldots, v_{m+n}\}) \geq i
    \end{cases}
  \]
  The only \(W\) that can meet these conditions is \(W^\lambda :=
  \Span\{v_{n+1-\lambda_1}, \ldots, v_{n+m-\lambda_m}\}\). Thus, \[
    \sigma_\lambda \cupprod \sigma_\mu = [X_\lambda \intersect
    X_\mu]^* = [\{W^\lambda\}]^* = 1
  \]
  granting that \(X_\lambda\) and \(X_\mu\) intersect
  transversely. \todo{Apparently this is ``obvious'' because
    \(\Omega_\lambda\) and \(\Omega_{\mu}'\) correspond to coordinate
    subspaces.} 
  % which implies that 
  % \begin{align*}
  %   \dim W \intersect V_i \intersect V_{m+1-i}' + \dim W
  %   & =
  %   \dim (W \intersect V_i) \oplus (W \intersect V_{m+1-i}') \geq  m+1
  %   \\ & \implies \dim W \intersect V_i \intersect V_{m+1-i}' \geq 1
  % \end{align*}
  % However, since \(V_i \intersect V_{m+1-i}' =
  % \Span\{v_{n+i-\lambda_i}\}\), it must be that 
  % Furthermore, this point must be of
  % the form \[
  %   W^\lambda := \Span\{v_{n+1-\lambda_1}, \ldots, v_{n+m-\lambda_m}\}
  % \]
\end{proof}
\begin{defn}
  Given the above, we say that the classes \(\sigma_{\lambda}\) and
  \(\sigma_{\hat{\lambda}}\) are \de{dual} to each other.
\end{defn}
\begin{cor}
  Given \(x \in H^*(G(m,n))\), we have \[
    x = \sum_{\lambda \text{ in an }n \times m\text{ rectangle}} (x
    \cupprod \sigma_{\hat{\lambda}}) \sigma_\lambda
  \]
\end{cor}
\begin{proof}
  This follows immediately from the cup product structure outlined in
  the above proposition and the fact that the Schubert classes form a
  \(\Z\)-basis of the cohomology (\ref{schubert-classes-form-basis}).
\end{proof}
\begin{thm}[Pieri Rule]
  \cite{manivel}*{3.2.8} If \(\lambda\) is a partition contained in an
  \(n \times m\) rectangle and \(1 \leq k \leq n\), then \[
    \sigma_\lambda \cupprod \sigma_k = \sum_{\substack{\nu \text{
          contained in an }n\times m\text{ rectangle} \\ \nu = \lambda
        + \text{a horizontal }k\text{-strip}}} \sigma_\nu
  \]
\end{thm}
\begin{proof}
  By the previous corollary, we need only show that \[
    (\sigma_\lambda \cupprod \sigma_k) \cupprod \sigma_{\hat{\nu}} =
    \begin{cases}
      1 & \text{ if }\nu \text{ is a summand of the sum above}\\
      0 & \text{ otherwise}
    \end{cases}
  \]
  For \(\mu\) a partiton contained in an \(n \times m\) rectangle and
  \(|\lambda|+|\mu| = nm-k\), then
  \begin{align*}
    \hspace{-0.25in} \mu = \text{ complement of }(\lambda + \text{a horizontal }k\text{
    strip}) & \iff \hat{\lambda} = \mu + \text{a horizontal }k\text{ strip}\\
    & \iff n-\lambda_m \geq \mu_1 \geq n-\lambda_{m-1} \geq \mu_2 \geq \cdots
    \geq n-\lambda_1 \geq \mu_m \\
    & \implies \sigma_\lambda \cupprod
    \sigma_\mu \cupprod \sigma_k = 1
  \end{align*}
  and, if the inequalities are not satisfied, then the product is
  zero. If \(\lambda_i + \mu_{n+1-i} > n\), then \(\sigma_\lambda
  \cupprod \sigma_\mu = 0\), so we will assume that \(\lambda_i +
  \mu_{n+1-i} \leq n\). Now, we set
  \begin{align*}
    A_i & = \Span\{v_1, \ldots, v_{n+i-\lambda_i}\}
    & = V_{n+i-\lambda_i} \\
    B_i & = \Span\{v_{\mu_{m+1-i}+i}, \ldots, v_{m+n}\}
    & = V'_{n+m+1-i-\mu_{m+1-i}} \\
    C_i & = \Span \{v_{\mu_{m+1-i}+i}, \ldots, v_{n+i-\lambda_i}\}
    & = A_i \intersect B_i
  \end{align*}
  We note that \(\dim C_i = n-\lambda_i-\mu_{m+1-i}\), so if
  \(\mu_{m+1-i} \geq n-\lambda_i\), then \(A_i \intersect B_i = C_i =
  0\). \todo{Finish this proof; maybe find a different one.}
\end{proof}
The important fact about the Pieri rule is that it totally determines
the structure constants of the algebra. Furthermore, the ring of
symmetric function has its structure constants completely determined
by the Pieri rule.
\begin{cor}
  Let \(\sym_m\) be the symmetric functions in \(m\) variables. Then,
  the map \\
  \begin{minipage}{1.0\linewidth}
    \begin{align*}
      \phi_{m,n} \from \sym_m & \to H^*(G(m,n))\\
      s_\lambda & \to
                  \begin{cases}
                    \sigma_\lambda & \lambda \text{ in an }n \times
                    m\text{ rectangle}\\
                    0 & \text{ else}
                  \end{cases}
    \end{align*}
  \end{minipage}
  is a surjective ring homomorphism.
\end{cor}
\begin{bibdiv}
  \begin{biblist}
    \bib{fulton}{book}{
      author={Fulton, William}
      title={Young Tableaux}
      year={1997}
    }
    \bib{gathmann}{article}{
      author={Gathmann, Andreas}
      title={Algebraic Geometry}
      year={2014}
      
    }
    \bib{macdonald}{book}{
      author={Macdonald, I.G.}
      title={Symmetric Functions and Hall Polynomials}
      year={1979}
      note={2nd Edition, 1995}
    }
    \bib{manivel}{book}{
      author={Manivel, Laurent}
      title={Symmetric Functions, Schubert Polynomials, and Degeneracy
        Loci}
      year={1998}
      note={Translated by John R. Swallow; 2001}
    }
  \end{biblist}
\end{bibdiv}

\end{document}