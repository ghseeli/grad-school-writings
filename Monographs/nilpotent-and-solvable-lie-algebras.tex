\documentclass[11pt,leqno,oneside]{amsart}
\usepackage[alphabetic,abbrev]{amsrefs} % use AMS ref scheme
\usepackage{../ReAdTeX/readtex-core}
\usepackage{../ReAdTeX/readtex-dangerous}
\usepackage{./monographs}
\usepackage{../ReAdTeX/readtex-abstract-algebra}
\usepackage{../ReAdTeX/readtex-lie-algebras}
\usepackage{tikz-cd}

\usepackage{caption}
\usepackage{subcaption}
\usepackage{todonotes}

\numberwithin{thm}{section}

\title[Nilpotent, Solvable, and Semisimple Lie Algebras]{Nilpotent,
  Solvable, and Semisimple Lie
Algebras}
\author{George H. Seelinger}
\date{May 2018}
\begin{document}
\maketitle
\section{Introduction}
Just as in groups, Lie algebras can have additional structure which
make them ``nice.'' Due to the resemblence these structures have to
their group theoretic counter-parts, we talk about solvable,
nilpotent, and simple Lie algebras. The goal of this monograph is to
primarily understand the most important theorems surrounding these
structures, namely Engel's theorem, Lie's theorem, and Cartan's
theorem. \textbf{This monograph will borrow content freely from \cite{humph}
and makes no claim to originality of theorems, their
statements, or their proofs.} The reader should think
of this monograph as companion notes to \cite{humph}*{Sections
  1--5}. Unless otherwise stated, we are always 
working over 
algebraically closed ground
field \(F\) of characteristic \(0\), \(\g\) is an arbitrary Lie
algebra, and \(\gl(V) 
= \End(V)\) with the Lie bracket multiplication. 
\section{Basic Definitions}
\begin{defn}
  Consider a Lie algebra \(\g\). We say \(\g\) is \de{linear} if
  \([\g,\g] = 0\).
\end{defn}
While rather boring since the Lie bracket does not provide much
additional structure, linear Lie algebras are important to keep in
mind as easy examples. Linear Lie algebras play a similar role in Lie
theory to abelian groups in group theory.
\begin{thm}
  Any \(1\)-dimensional Lie algebra is linear.
\end{thm}
\begin{proof}
  By definition, a Lie algebra \(\g\) must have \([g,g] = 0\) for all
  \(g \in \g\). Now, let \(\g\) have basis element \(\{x\}\). Then,
  \([x,x] = 0 \implies [\g,\g] = 0\).
\end{proof}
\begin{defn}
  Consider a Lie algebra \(\g\). We say \(\g\) is \de{simple} if
  \(\g\) has no proper non-trivial ideals.
\end{defn}
Note that simple algebras are, in some sense, opposite to linear Lie
algebras, since simple non-linear Lie algebras have the following
property.
\begin{prop}
  Let \(\g\) be a simple non-linear Lie algebra. Then, \([\g,\g] =
  \g\). 
\end{prop}
\begin{proof}
  Since \([\g,\g] \ideal \g\) and \([\g,\g] \neq 0\) because \(\g\) is
  non-linear, \([\g,\g] = \g\) by the simplicity of \(\g\). 
\end{proof}
\begin{example}
  Consider \(\g = \sl_2(\C)\), the Lie algebra of \(2\) by \(2\)
  complex traceless matrices. Such a Lie algebra is simple and
  non-linear. This follows from the fact that the basis \[
    e = \left(
      \begin{array}{cc}
        0&1\\
        0&0
      \end{array}
  \right), \ f = \left(
  \begin{array}{cc}
    0&0\\
    1&0
  \end{array}
  \right), \ h = \left(
  \begin{array}{cc}
    1&0\\
    0&-1
  \end{array}
  \right)
  \]
  has the relations \[
    [h,e] = 2e, [h,f] = -2f, [e,f] = h 
  \]
  So, \([\g,\g] = \g\). Now, suppose \(I\) is some nonempty ideal of
  \(\sl_2(\C)\). Then, it contains some element \(ae+bf+ch\) for
  \(a,b,c \in \C\). Then, \[
    [h,[h,ae+bf+ch]] = [h,a[h,e] + b[h,f]] = [h,2ae - 2bf] = 4ae+4bf \implies c = 0 \text{
      or } h \in I
  \]
  since \(4(ae+bf+ch)-(4ae+4bf) \in I\). However, if \(h \in I\), then
  \([e,h] = -2e \in I\) and \([f,h] = 2f \in I\) and so \(I =
  \g\). So, it must be that \(c = 0\). Now, consider since \(ae-bf \in
  I\) by above, \[
    [e,ae-bf] = -bh \in I \implies b = 0
  \]
  However, if \(ae \in I\), then \([f,ae] = -ah \in I\) and so we are
  done. It must be that \(I = \sl_2(\C)\). \\

  Once more theory is developed, there are easier and more elegant
  ways to show \(\sl_2(\C)\) is simple.
\end{example}
\begin{defn}
  Let \(\g\) be a Lie algebra and \(\g^{(0)} = \g, \g^{(1)} =
  [\g,\g]\), and \(\g^{(i)} = 
  [\g^{(i-1)}, \g^{(i-1)}]\). We call \(\{\g^{(i)}\}_{i \geq 0}\) the
  \de{derived series} of \(\g\). 
\end{defn}
\begin{defn}
  Let \(\g\) be a Lie algebra. We say \(\g\) is \de{solvable} if there
  exists an \(n \in \N\) such that \(\g^{(n)} = 0\). 
\end{defn}
Solvable Lie algebras have many of the same properties as solvable
groups in regards to their behavior with ideals, homomorphisms, and
short exact sequences. See \cite{humph}*{p 11} for more information.
\begin{example}
  The canonical example of a solvable Lie algebra is the Lie algebra
  of all upper triangular matrices over a field \(F\). Let us denote
  such a Lie algebra as \(\b_n(F)\). Let \(\g = \b_n(F)\). Then, we
  first seek to compute \([\g, \g]\). Let \(x,y \in \g\). Then, \(x =
  d_x + n_x\), where \(d_x\) is the diagonal part of \(x\) and \(n_x\)
  is the nilpotent part of \(x\). We then check:
  \begin{align*}
    [x,y] & = xy - yx \\
          & = (d_x+n_x)(d_y+n_y) - (d_y+n_y)(d_x+n_x) \\
    & = d_x d_y + n_x d_y + d_x n_y + n_x n_y - d_y d_x - n_y d_x -
      d_y n_x - n_y n_x \\
    & = n_x d_y + d_x n_y + n_x n_y - n_y d_x -
      d_y n_x - n_y n_x & \text{ since } d_x d_y \text{ commute.}
  \end{align*}
  However, this final result must be \emph{strictly}
  upper-triangular. Such matrices actually form another Lie algebra,
  denoted \(\n_n(F)\). Thus, \([\g,\g] \subset \n_n(F)\). \\

  Let \(e_{i,j}\) be the matrix with a \(1\) in the \((i,j)\)th entry
  and \(0\) at all others. Then, \(\b_n(F)\) is spanned by all
  \(e_{i,j}\) such that \(j-i \geq 0\) and \(\n_n(F)\) is spanned by
  all \(e_{i,j}\) such that \(j-i \geq 1 = 2^0\). Furthermore, \[
    [e_{i,j}, e_{k,\ell}] = \delta_{j,k} e_{i,\ell} - \delta_{\ell,i} e_{k,j}
  \]
  Let us assume \(j-i \geq 2^{k-1}\) and \(\ell-k \geq 2^{k-1}\) for
  some \(k \in \N\) and \(\g^{(k)}\) is spanned by such \(e_{i,j}\). Then,
  \begin{align*}
    [e_{i,j}, e_{k,\ell}]
    & = \delta_{j,k} e_{i,\ell} - \delta_{\ell,i} e_{k,j} \\
    & =
      \begin{cases}
        e_{i,\ell} & j=k, \ell \neq i
        \\
        e_{k,j} & j \neq k, \ell = i \\
        (e_{i,i} - e_{j,j}) & j = k, \ell = i \\
        0 & \text{ else }
      \end{cases}
  \end{align*}
  However, if \(j = k\) and \(\ell = i\), then \(i-j \geq 2^{k-1}\),
  which is a contradiction since \(j-i \geq 2^{k-1}\). Thus, we
  actually have \[
  [e_{i,j}, e_{k,\ell}] =
      \begin{cases}
        e_{i,\ell} & j=k, \ell \neq i
        \\
        e_{k,j} & j \neq k, \ell = i \\
        0 & \text{ else }
      \end{cases}
  \]
  Moreover, if \(j = k\), then \(\ell - i \geq 2^{k-1} + k - i =
  2^{k-1} + j - i \geq 2^{k-1} + 2^{k-1} = 2^k\), and similarly if
  \(\ell = i\), then \(j-k \geq 2^k\). Thus, \(\g^{(k+1)}\) has a basis
  consisting of all elements \(e_{i,j}\) with \(j-i \geq 2^k\). \\

  Using this fact, we can see that \(\g^{\log_2 n + 1} = 0\) and thus
  \(\g\) is solvable.
  % Thus, if \(z \in \n_n(F)\), then \[
  %   z = \sum_{j > i} z_{i,j} e_{i,j}
  % \]
  % and so, it \(t \in \n_n(F)\), too, we get
  % \begin{align*}
  %   [z,t] & = \sum_{j > i} \sum_{k > \ell} z_{i,j}
  %           t_{k,\ell}[e_{i,j},e_{k,\ell}] \\
  %   & = \sum_{j > i} \sum_{k > \ell} z_{i,j}
  %           t_{k,\ell} (\delta_{j,k} e_{i,\ell}-\delta_{\ell,i}
  %     e_{k,j}) \\
  %   & = \sum_{j > i} \sum_{k > \ell} 
  %     \begin{cases}
  %       z_{i,j} t_{j,\ell} e_{i,\ell} & j=k, \ell \neq i
  %       \\
  %       -z_{i,j} t_{k,i} e_{k,j} & j \neq k, \ell = i \\
  %       z_{i,j} t_{j,i} (e_{i,i} - e_{j,j}) & j = k, \ell = i \\
  %       0 & \text{ else }
  %     \end{cases} \\
  %   & = \sum_{j > i} \sum_{k > \ell} 
  %     \begin{cases}
  %       z_{i,j} t_{j,\ell} e_{i,\ell} & j=k, \ell \neq i, \ell > j
  %       \\
  %       -z_{i,j} t_{k,i} e_{k,j} & j \neq k, \ell = i, i > k \\
  %       0 & \text{ else }
  %     \end{cases} & \text{ since } t_{j,i} = 0
  % \end{align*}
  % However, since \(j > i\) and \(k > \ell\), then \([z,t]\) must have
  % \(0\)'s on the diagonal and the super-diagonal. 
\end{example}
\begin{example}
  A specific and important Lie algebra in this family of Lie algebras is
  \(\n_3(F)\), which is isomorphic to \de{the Heisenberg algebra},
  \(H\), with 
  basis \(\{f,g,z\}\) and relation \([f,g] = z\), as well as the properties
  that \([H,H] \subset Z(H)\) and \(z \in Z(H)\). The isomorphism
  between these two algebras is exhibited by \[
    f \mapsto e_{1,2}, \ g \mapsto e_{2,3}, \ z \mapsto e_{1,3}
  \]
\end{example}
\begin{prop}
  Let \(\g\) be an arbitrary Lie algebra. Then, \(\g\) has a unique
  maximal solvable ideal.
\end{prop}
\begin{proof}
  Let \(S \ideal \g\) be a maximal solvable ideal. We know that, if
  \(I,J \ideal \g\) are both solvable, then \(I+J\) is solvable. So,
  given another solvable ideal \(I \ideal \g\), it must be that
  \(S+I\) is solvable, but since \(S\) is maximal, \(S+I = S\). Thus,
  \(I \subset S\). 
\end{proof}
\begin{defn}
  If \(\g\) is a Lie algebra, we call its unique maximal solvable
  ideal the \de{radical of \(\g\)} denoted \(\Rad \g\). 
\end{defn}
\begin{defn}
  If \(\g\) is a Lie algebra and \(\Rad \g = 0\), we call \(\g\)
  \de{semisimple}. 
\end{defn}
\begin{rmk}
  Since simple Lie algebras have no non-trivial ideals, their radical
  is \(0\) and thus any simple Lie algebra is also semisimple.
\end{rmk}
\begin{prop}\label{solvable-has-non-trivial-abelian-ideal}
  Any non-abelian solvable Lie algebra has a non-trivial abelian
  ideal.
\end{prop}
\begin{proof}
  Consider that the derived series of \(\g\) has some minimal \(n \in
  \N\) 
  such that \(\g^{(n)} = 0\). Then, \(\g^{(n-1)} \neq 0\) but
  \([\g^{(n-1)}, \g^{(n-1)}] = 0\), so \(\g^{(n-1)}\) is abelian and
  is an ideal of \(\g\) by repeated application of the fact that the
  Lie bracket of two ideals is an ideal of \(\g\).
\end{proof}
\begin{prop}
  Let \(\g\) be a Lie algebra. Then \(\g / \Rad \g\) is semisimple.
\end{prop}
\begin{proof}
  Consider the short exact sequence \[
    0 \to \Rad \g \to \g \to \g / \Rad \g \to 0
  \]
  If \(\g \neq \Rad \g\), then it must be that \(\g / \Rad \g\) is not
  solvable, otherwise \(\g\) would be solvable and thus \(\g = \Rad
  \g\). Now, consider an solvable ideal of \(\g/\Rad \g\) must have
  the form \(I/\Rad \g\) for an ideal \(I \ideal \g\). Then we would
  have short exact sequence of ideals \[
    0 \to I \intersect \Rad \g \to I \to I/\Rad \g \to 0
  \]
  and \(I\) would be solvable, so \(I = \Rad \g\) and \(I/\Rad \g =
  0\). Thus, \(\g / \Rad \g\) is semisimple.
\end{proof}
\begin{defn}\label{levi-decomp}
  Let \(\g\) be a Lie algebra. Then, the short exact sequence used in
  the proof above, namely \[
    0 \to \Rad \g \to \g \to \g/\Rad \g \to 0
  \]
  is the \de{Levi decomposition} of \(\g\). That is, \(\g\) is the
  extension of a semisimple Lie algebra by a solvable algebra. 
\end{defn}
\begin{defn}
  Let \(\g\) be a Lie algebra, \(\g^0 := \g, \g^1 := [\g,\g]\), and \(\g^i := [\g, \g^{i-1}]\). We
  call \(\{\g^i\}_{i \geq 0}\) the \de{lower central series of \(\g\)}
  or the \de{descending central series of \(\g\)}.
\end{defn}
\begin{defn}
  Let \(\g\) be a Lie algebra. If there exists an \(n \in \N\) such
  that \(\g^n = 0\), we say that \(\g\) is \de{nilpotent}.
\end{defn}
Just like solvability, nilpotency behaves similarly to the group
theoretic version with respect to homomorphisms, ideals, and short
exact sequences. See \cite{humph}*{p 12} for more details.
\begin{example}
  Consider \(\g = \sl_2(F)\) where \(\Char F = 2\). Such a Lie algebra
  is nilpotent since \[
    [h,e] = 2e = 0, \ [h,f] = -2f = 0, \ \ [e,f] = h
  \]
  and so \(\g^2 = \langle h \rangle\) and thus \(\g^3 = 0\).
\end{example}
\begin{defn}
  Let \(g \in \g\), a Lie algebra. Then, if \(\ad_g\) is a nilpotent
  endomorphism, that is, there is an \(n\) such that \((\ad_g)^n =
  0\), we say that \(g\) is \de{ad-nilpotent}. 
\end{defn}
\begin{prop}
  Given a Lie algebra \(\g\), if \(\g\) is nilpotent,
  then all \(g \in \g\) are ad-nilpotent.
\end{prop}
\begin{proof}
  Let \(\g^n = [\g, \g^{n-1}] = 0\). Now, for any \(g \in \g\), we
  know \([g,\g] \subset [\g,\g]\). So, for any \(h \in \g\),
  \((\ad_g)^i(h) \subset \g^i\). Therefore, \((\ad_g)^n(h) \in \g^n = 0
  \implies (\ad g)^n = 0\). 
\end{proof}
\section{Engel's Theorem and Consequences}
We have now laid out the appropriate groundwork to state and prove
Engel's Theorem, which is a somewhat surprising partial converse to
the proposition above.
\begin{thm}[Engel's Theorem]
  Let \(\g\) be a Lie algebra such that, for all \(g \in \g\), \(g\)
  is ad-nilpotent. Then, \(\g\) is nilpotent.
\end{thm}
To prove Engel's theorem, we will make use of the following lemma, in
the style of \cite{humph}*{p 12}
\begin{lem}\label{nilp-implies-ad-nilp}
  Let \(g \in \gl(V)\) be a nilpotent endomorphism of \(V\), then
  \(\ad_g\) is nilpotent.
\end{lem}
\begin{proof}[Proof of Lemma]
  Consider that \(\ad_g = \lambda_g + \rho_{-g}\) where \(\lambda_g \from
  \g \to \g\) is given by \(x \mapsto gx\) and \(\rho_{-g} \from \g
  \to \g\) is given by \(x \mapsto -xg\). Now, since \(g\) is a
  nilpotent element of \(\End(V)\), it must be that \(\lambda_g,
  \rho_{-g}\) are nilpotent elements of \(\End(\End(V))\). Using the
  binomial theorem, it is straightforward to prove that, in any ring,
  the sum of 
  two nilpotent elements is again nilpotent, so \(\ad_g\) must be nilpotent.
\end{proof}
We will also use the following results to prove Engel's Theorem
\begin{thm}\label{nilp-has-nontrivial-simult-0-eigenspace}
  Let \(\g\) be a subalgebra of \(\gl(V)\), with \(V \neq 0\)
  finite-dimensional. If all \(g \in \g\) are nilpotent, then there
  exists \(0 \neq v \in V\) such that \(\g.v = 0\). 
\end{thm}
\begin{proof}
  See \cite{humph}*{p 13}
\end{proof}
\begin{proof}[Proof of Engel's Theorem]
  We proceed by induction on the dimension of \(\g\). If \(\g\) is
  1-dimensional, we are done, since \(\g\) must be abliean. So, assume
  the result is true for all Lie algebras with dimension less than
  \(n\) and let \(\g\) have dimension \(n\). Since \(\g\) consists of
  ad-nilpotent elements, then \(\ad \g \subset 
  \gl(\g)\) satisfies the above theorem. Thus, there is a \(0 \neq x \in
  \g\) such that \(ad_g x \neq 0\) for all \(g \in \g\), that is to
  say\([\g, x] = 0\). Therefore, \(Z(\g) \neq 0\) since \(x \in
  Z(\g)\). However, \(\g / Z(\g)\) must also consist of ad-nilpotent
  elements and have dimension strictly less than \(\g\), since \(Z(\g)
  \neq 0\). Thus, \(\g / Z(\g)\) is nilpotent, which implies that
  \(\g\) is nilpotent. 
\end{proof}
These result have some interesting corollaries which give us insight
into the structure of nilpotent Lie algebras. \todo{Add some
  consequences.}
\begin{cor}
  Let \(\g\) be a subalgebra of \(\gl(V)\), with \(V \neq 0\) finite
  dimensional, and all \(g \in \g\) are nilpotent. Then, there exists
  a flag \((V_i)\) that is stable under \(\g\), with \(g.V_i \subset
  V_{i-1}\) for all \(i\). In other words, there is a basis of \(V\)
  relative to which the matrices of \(\g\) are all strictly upper
  triangular.
\end{cor}
\begin{proof}
  By Engel's theorem, there is a \(v \in V\) such that \(\g.v =
  0\). So, proceeding by induction, \(V/ \Span\{v\}\) has such a
  flag and we are done.
\end{proof}
Thus, we see that any nilpotent Lie algebra will ``look like'' the Lie
algebra of strictly upper triangular matrices with the appropriate
choice of basis.
\begin{cor}\label{nilpotent-has-nontrivial-center}
  Let \(\g\) be a nilpotent Lie algebra. If \(0 \neq I \ideal \g\),
  then \(I \intersect Z(\g) \neq 0\). In particular, \(Z(\g) \neq 0\)
  if \(\g\) is not trivial. 
\end{cor}
\begin{proof}
  This essentially follows from the proof of Engel's theorem given
  above. By \ref{nilp-has-nontrivial-simult-0-eigenspace}, there is
  a \(0 \neq x \in I\) such that \([g,x] = 0\) for all \(g \in
  \g\). Thus, \(x \in Z(\g) \implies 0 \neq x \in I \intersect
  Z(\g)\). 
\end{proof}
\section{Lie's Theorem and Consequences}
Engel's theorem provided us with insight into the structure of
nilpotent Lie algebras. Solvability is a slightly weaker condition,
but Lie's theorem will still give us some useful information. For this
section, we must assume that \(F\) is algebraically closed and \(\Char
F = 0\). In essence, Lie's theorem is a corollary to the following
theorem which most closely resembles
\ref{nilp-has-nontrivial-simult-0-eigenspace}, and its proof will
follow the same rough structure.
\begin{thm}
  Let \(\g\) be a solvable sub-algebra of \(\gl(V)\), where \(0 \neq
  V\) is finite dimensional. Then, \(V\) contains a common eigenvector
  for all the endomorphisms in \(\g\). 
\end{thm}
\begin{proof}
  See \cite{humph}*{pp 15--16}
\end{proof}
\begin{cor}[Lie's Theorem]
  Let \(\g\) be a solvable subalgebra of \(\gl(V)\), \(V\) finite
  dimensional. Then, \(\g\) stabilizes some flag in \(\g\), that is to
  say that the matrices of \(\g\) are upper triangular relative to
  some suitable basis of \(V\). 
\end{cor}
\begin{rmk}
  Lie's theorem also holds in prime characteristic provided \(\dim V <
  \Char F\). 
\end{rmk}
\begin{proof}
  Let \(\dim V = 1\). Then, \(\g\) trivially stabilizes the flag \(0
  \subset V\). Now, assume the theorem is true for all \(V\) with
  dimension less than \(n\). Then, consider that, by the theorem
  above, there is a \(v \in V\) such that \(v\) is an eigenvector for
  all \(g \in \g\), that is, \(g.v = \lambda(g)v\) for some \(\lambda
  \from \g \to F\). Then, consider that \(\g\) stabalizes
  some flag \(V/\Span\{v\}\) by the inductive hypothesis, say \(0 = V_0
  \subset V_1
  \subset \cdots \subset V_{n-1} = V/\Span\{v\}\). Thus, \(\g\)
  stabalizes the flag \[
    0 = V_0 \subset V_1 \subset \cdots \subset V_{n-1} = V/\Span\{v\}
    \subset V_n = V.
  \]
\end{proof}
Using this result, we can get an interesting fact about the structure
of solvable Lie algebras.
\begin{cor}
  Let \(\g\) be a solvable Lie algebra. Then, there exist a chain of
  ideals of \(\g\), \[
    0 = \g_0 \subset \g_1 \subset \cdots \subset \g_n = \g
  \]
  such that \(\dim \g_i = i\). 
\end{cor}
\begin{proof}
  Let \(\g\) be a solvable Lie algebra and \(\phi \from \g \to
  \gl(V)\) be a finite-dimensional representation of \(\g\). Then,
  \(\phi(\g)\) must be solvable and, by Lie's theorem, stabilize some
  flag in \(V\). So, if \(\phi\) is the adjoint representation, then
  \(V = \g\) and the subspaces of the flag are ideals of \(\g\), each with
  codimension 1 in the next.
\end{proof}
\begin{cor}
  Let \(\g\) be a solvable Lie algebra. Then, \(\ad_\g x\) is
  nilpotent for any \(x \in [\g,\g]\). 
\end{cor}
\begin{proof}
  \(\g\) has a flag of ideals as in the corollary above. Then, take a
  basis \((x_1, \ldots, x_n)\) of \(\g\) for which \((x_1, \ldots,
  x_i)\) spans \(\g_i\). Relative to this basis, the matrices of \(\ad
  \g\) must be upper triangular since \(\g.x_i \in \Span \{x_1,
  \ldots, x_i\}\). Thus, \(ad_g\) is upper triangular for all \(g \in
  \g\). Thus, since the Lie algebra of upper triangular matrices has
  derived subalgebra the Lie algebra of all \emph{strictly} upper
  triangular matrices, it must be that \([\ad \g, \ad \g] = \ad_\g [\g,\g]\) contains
  only upper triangular matrices. Thus, since every strictly upper
  triangular matrix is nilpotent, every element of \([\g,\g]\) is
  ad-nilpotent. 
\end{proof}
\section{Representation Theory of Solvable and Nilpotent Lie Algebras}
We can also consider the following:
\begin{defn}
  A \de{1-dimensional representation} of a Lie algebra \(\g\) is a
  linear map \(\rho \from \g \to \End(\C) \isom \C\) such that \(\rho([x,y]) =
  [\rho(x),\rho(y)]\) for all \(x,y \in \g\).
\end{defn}
Then, we have
\begin{lem}
  A linear map \(\rho \from \g \to \C\) is a 1-dimensional
  representation of \(\g\) if and only if \(\rho\) vanishes on \([\g,\g]\).
\end{lem}
\begin{proof}
  This is a straightforward exercise using definitions since
  \(\End(\C) \isom \C\) is \(1\)-dimensional as a Lie algebra.
\end{proof}
\begin{cor}[Alternative Lie's Theorem]
  Let \(\g\) be a solvable Lie algebra and \(V\) a finite dimensional
  irreducible \(\g\)-module. Then \(\dim V = 1\).
\end{cor}
\begin{proof}
  Let \(V\) be an irreducible \(\g\)-module. We know by Lie's theorem that \(\g\) stabalizes some flag \(0 = V_0
  \subset V_1 \subset \cdots \subset V_n = V\). However, \(V\) is
  irreducible, so the only \(\g\)-submodule of \(V\) is itself and
  \(0\). Thus, the flag must be \(0 \subset V\) and thus \(V\) is
  one-dimensional. \\

  Conversely, one could prove Lie's theorem from this corollary since
  it implies that every \(\g\)-module would contain a
  \(1\)-dimensional subrepresentation.
\end{proof}
This reinforces the picture that all solvable Lie algebra modules can
be given a basis such that the action of \(\g\) is given by an
upper-triangular matrix. However, we can say even more about the
action of a nilpotent \(\g\), namely,
\begin{thm}
  Let \(\g\) be a nilpotent Lie algebra and \(V\) be a
  \(\g\)-module. Let \(y \in \g\) and \(\rho(y) \from V \to V\) be the
  map \(v \mapsto yv\). Then, the generalized eigenspaces of the map
  \(\rho(y)\) are submodules of \(V\).
\end{thm}
\begin{proof}
  The proof of this fact is rather ugly but relies on the fact that \[
    (\rho(y)-(\alpha+\beta)1)^n xv = \sum_{i=0}^n \binom{n}{i} ((\ad_y
    - \beta 1)^i x)((\rho(y)-\alpha 1)^{n-1}v)
  \]
  for \(v \in V, x,y \in \g, \alpha, \beta \in \C\). Then, one takes
  \(\alpha = \lambda_i\) and \(\beta = 0\). See
  \cite{carter}*{pp 17--18} for the full proof.
\end{proof}
Essentially, this theorem is saying that, if \(\rho(y)\) is put in
Jordan-Canonical form, then the action \(y \mapsto \) Jordan block of \(\rho(y)\)
forms a Lie algebra submodule.
% \begin{cor}
%   Let \(\g\) be a nilpotent Lie algebra and let \(V\) be a finite
%   dimensional indecomposable \(\g\)-module. Then, we can choose a
%   basis for \(V\) such that we obtain a matrix representation \(\rho\)
%   of \(\g\) of the form \[
%     \rho(x) = \left(
%       \begin{tikzcd}
%         0 \ar[rd, dash] & * \\
%         0 & 0
%       \end{tikzcd}
%     \right), \ \ \ \forall x \in \g
%   \]
% \end{cor}
% \begin{proof}
%   This is simply a reformulation of Engel's theorem, since there is a
%   \(v\) such that \(\g.v = 0\) and thus, \(x.v = 0\) and so \(x\) has
%   \(0\) as an eigenvalue. But then, since each generalized eigenspace
%   is a submodule
% \end{proof}
\section{Cartan's Criterion and Consequences}
While Lie's theorem gives us some idea of the structure of a solvable
Lie algebra, we still only have the actual definition of a solvable
Lie algebra to prove a Lie algebra is, in fact, solvable. Cartan's
criterion provides us a way to test a Lie algebra \(\g\) for
solvability.
\begin{thm}[Cartan's Criterion]
  Let a Lie algebra \(\g\) be a subalgebra of \(\gl(V)\), \(V\)
  finite-dimensional. Then, if \(\tr(xy) = 0\) for all \(x \in
  [\g,\g]\) and \(y \in \g\), \(\g\) is solvable.
\end{thm}
To prove this, we will use a somewhat technical lemma from
\cite{humph}*{p 19}.
\begin{lem}
  Let \(A \subset B\) be two subspaces of \(\gl(V)\) with \(\dim V <
  \infty\). Set \(M = \{x \in \gl(V) \st [x,B] \subset A\}\). If \(x
  \in M\) satisfies \(\tr(xy) = 0\) for all \(y \in M\), then \(x\) is
  nilpotent. 
\end{lem}
\begin{proof}
  See \cite{humph}*{p 19}.  
\end{proof}
\begin{lem}
  Given \(x,y,z \in \gl_n(F)\), we get \[
    \tr([x,y]z) = \tr(x[y,z])
  \]
\end{lem}
\begin{proof}[Proof of Cartan's Criterion]
  If \(\g^{(1)} = [\g,\g]\) is nilpotent, we know that \(\g\) is
  solvable since \(\g^{(i+1)} = [\g,\g]^{(i)} \subset [\g,\g]^i\). To
  do this, we will show that all \(x \in [\g,\g]\) are nilpotent
  endomorphisms and appeal to \ref{nilp-implies-ad-nilp} followed by
  Engel's theorem to conclude \([\g,\g]\) is nilpotent. Now, using the
  first lemma above, take \(A = [\g,\g]\) and \(B = \g\). Then, \(\g \subset
  M = \{x \in \gl(V) \st [x,\g] \subset \g\}\) trivially. Now, take
  \([x,y] \in [\g,\g]\) and \(z \in M\). The second lemma above tells
  us \[
    \tr([x,y]z) = \tr(x[y,z]) = \tr([y,z]x) = 0
  \]
  where the last equality follows from \([y,z] \in M\) and the given
  hypothesis on trace. Thus, we have satisfied the hypotheses for the first
  lemma above and get that any generator \([x,y] \in [\g,\g]\) is nilpotent.
\end{proof}
\begin{cor}\label{killing-form-degenerate-on-derived-implies-solvable}
  Let \(\g\) be a Lie algebra. If \(\tr(\ad_x, \ad_y) = 0\) for
  all \(x \in [\g,\g], y \in \g\), then \(\g\) is solvable.
\end{cor}
\begin{proof}
  We can apply Cartan's criterion to the adjoint representation of
  \(\g\) to get that \(\ad \g\) is solvable using the given trace
  condition. Then, since \(\ker \ad = Z(\g)\), by definition of
  \(Z(\g)\), and \(Z(\g)\) is an abelian Lie algebra, \(\ker \ad\) is
  also solvable. Thus, we have short exact sequence \[
    0 \to Z(\g) \to \g \to \ad \g \to 0
  \]
  with \(Z(\g), \ad \g\) solvable and so \(\g\) must be solvable.
\end{proof}
This corollary will be useful due to the way the Killing form is defined.
\section{The Killing Form}
\begin{defn}
  Given a Lie algebra \(\g\), we define the \de{Killing form of
    \(\g\)}, denoted \(\kappa \from \g \times \g \to \g\) to be
  defined by, for \(x,y \in \g\) \[
    \kappa(x,y) = \tr(\ad_x \ad_y)
  \]
\end{defn}
\begin{prop}
  \begin{enumerate}
  \item The Killing form on a Lie algebra \(\g\) is a symmetric bilinear
  form on \(\g\). 
  \item The Killing for is associative, that is \[
    \kappa([x,y],z) = \kappa(x,[y,z])
  \]
  \item The radical of the Killing form of \(\g\) is an ideal of
    \(\g\).
  \item Given \(x,y \in I \ideal \g\), then \[
      \kappa_\g(x,y) = \kappa_I(x,y)
    \]
    that is, the Killing form of \(\g\) restricted to \(I\) is the
    same as the Killing form on \(I\).
  \item If \(I \ideal \g\), then \[
      I^\perp := \{g \in \g \st \kappa(x,g) = 0, \forall x \in I\}
      \text{ is an ideal of }\g
    \]
  \end{enumerate}
\end{prop}
\begin{proof}
  For (a), observe that, for arbitrary matrices \(A,B \in M_n(F)\) and
  \(c \in F\), we
  have \[
    \tr(A+B) = \sum_{i=1}^n (A+B)_{i,i} = A_{i,i} + B_{i,i} = \tr(A) +
    \tr(B) \text{ and } \tr(cA) = \sum_{i=1}^n cA_{i,i} =
    c\sum_{i=1}^n A_{i,i} = c \tr(A)
  \]
  and also that
  \[
    \tr(AB) = \sum_{i=1}^n (AB)_{i,i} = \sum_{i=1}^n \sum_{j=1}^n A_{i,j} B_{j,i} = \sum_{j=1}^n
    \sum_{i=1}^n B_{j,i} A_{i,j} = \sum_{j=1}^n (BA)_{j,j} = \tr(BA)
  \]
  Thus, it must be that the Killing form is symmetric and bilinear.\\

  For (b), we note  \[
    \kappa([x,y],z) = \tr(\ad([x,y])\ad(z)) = \tr((xy-yx)z) = \tr(xyz)-\tr(yxz)
  \]
  and also \[
    \kappa(x,[y,z]) = \tr(\ad(x)\ad([y,z])) = \tr(x(yz-zy)) =
    \tr(xyz)-\tr(xzy).
  \]
  Since \(\tr(y(xz)) = \tr((xz)y)\), we are done. \\

  For (c), recall that the radical of \(\kappa\) is given by \[
    S := \{x \in \g \st \kappa(x,y) = 0, \forall y \in \g\}
  \]
  Then, since \(\kappa\) is bilinear, it is clear that, for \(x,z \in
  S\), \(x+z \in S\). Furthermore, for \(g \in \g\) and \(x \in S\),
  we get, using the lemma above,
  \[
    \kappa([g,x],y) = -\kappa(x,[g,y]) = 0 \implies [g,x] \in S 
  \]
  Thus, \(S\) is an ideal. \\

  For (d), pick a basis for \(I\) and extend it to a basis of
  \(\g\). Then, since \(x,y \in I\), the linear maps are given by \[
    \ad_x = \left(
      \begin{array}{cc}
        A_1 & A_2 \\
        0 & 0
      \end{array}
\right), \ \ \ad_y = \left(
  \begin{array}{cc}
    B_1&B_2\\
    0&0
  \end{array}
\right)
  \]
  Thus, is matrix form \[
    \ad_x \ad_y = \left(
      \begin{array}{cc}
        A_1 B_1 & A_2 B_2 \\
        0 & 0
      \end{array}
  \right) \implies \tr_\g(\ad_x \ad_y) = \tr A_1 B_1 = \tr_I(\ad_x \ad_y)
  \]
  Thus, \(\kappa_\g(x,y) = \kappa_I(x,y)\). \\ 

  For (e), if \(g \in \g\) and \(x \in I^\perp\), we wish to show
  \([x,g] \in I^\perp\). Given \(y \in I\), we check \[
    \kappa([x,g],y) = \kappa(x,[g,y]) = 0 \text{ since }[g,y] \in I
  \]
  Thus, \([x,g] \in I^\perp \implies I^\perp \ideal \g\).
\end{proof}
However, the most useful fact about the Killing form is the
following. To prove it, we will use the following lemma.
\begin{thm}\label{cartan-semisimplicity-criterion}
  Let \(\g\) be a Lie algebra. Then, \(\g\) is semisimple if and only
  if its Killing form is nondegenerate. 
\end{thm}
\begin{proof}
  \((\implies)\) Let \(\Rad \g = 0\). Then, we wish to show the ideal
  given by the radical of \(\kappa\), say \(S\), is contained in
  \(\Rad \g\). Now, by Cartan's criterion, \(\ad_\g S\) must be
  solvable since \(\tr(\ad_x \ad_y) = 0\) for all \(x \in S, y \in
  \g\) and therefore, since \(\ker \ad\) is solvable, \(S\) is
  solvable. Therefore, \(S \subset \Rad \g = 0\), so \(S = 0\). \\

  \((\impliedby)\) Let \(S = 0\). We wish to show that every abelian
  ideal of \(\g\) is contained in \(S\). Let \(x \in I\), an abelian
  ideal of \(\g\), and \(y \in \g\). Then, \(\ad x \ad y \from \g \to
  I\) and thus \((\ad 
  x \ad y)^2 \from \g \to I \to[] [I,I] = 0\). Thus, \(\ad x \ad y\) is
  a nilpotent linear transformation and so its only eigenvalues are
  0. So, \[
    0 = \tr(\ad x \ad y) = \kappa(x,y) \implies I \subset S = 0.
  \]
  Therefore, \(\g\) has no nontrivial abelian ideals. Furthermore,
  nontrivial \(\Rad \g\) 
  always contains a nontrivial abelian ideal by
  \ref{solvable-has-non-trivial-abelian-ideal}. So, \(\g\) must
  therefore be semisimple. 
\end{proof}
\begin{rmk}
  Note that, in \((\implies)\) direction of the proof, we did not use
  \(S = 0\) to show \(S \subset \Rad \g\), so this is always true.
\end{rmk}
\begin{example}
  Let \(\g = \sl_2\) with standard basis \(\beta = \{e,h,f\}\). Then, we wish
  to compute the dual basis to \(\beta\) under the Killing
  form. Consider that \(A = (\kappa(\beta_i
  \beta_j))_{i,j}\) has \[
    A = \left(
      \begin{array}{ccc}
        0 & 0 & 4 \\
        0 & 8 & 0 \\
        4 & 0 & 0
      \end{array}
    \right)
  \]
  and so its non-zero determinant tells us the form is nonzero and
  that the dual basis \(\gamma\) is given by \[
    A^{-1} = \left(
      \begin{array}{ccc}
        0 & 0 & \frac{1}{4} \\
        0 & \frac{1}{8} & 0 \\
        \frac{1}{4} & 0 & 0
      \end{array}
    \right)
  \]
  So, \(\kappa(\beta_i, \gamma_j) = e_i^t A (A^{-1} e_j) = \delta_{ij} \implies
  \gamma = \{\frac{1}{4}f, \frac{1}{8}h, \frac{1}{4}e\}\).
  % \begin{align*}
  %   1 = \kappa(e, x)
  %   & = \tr (\ad_e \ad_x) \\
  %   & = \tr \left( (-2e_{1,2} + e_{2,3}) \ad_{x} \right) \\
  %   & = \tr \left(\sum_{k,\ell} \left( -2x_{k,\ell} \delta_{2,k} e_{1,\ell}
  %     + x_{k,\ell} \delta_{3,k} e_{2,\ell} \right)\right) \\
  %   & = -2 x_{2,1} + x_{3,2} \\
  %   0 = \kappa(h, x)
  %   & = \tr(\ad_h \ad_x) \\
  %   & = \tr\left( (2 e_{1,1} - 2 e_{3,3}) \ad_x \right) \\
  %   & = 2 x_{1,1} - 2 x_{3,3} \\
  %   0 = \kappa(f, x)
  %   & = \tr(\ad_f \ad_x) \\
  %   & = \tr((2 e_{3,2} - e_{2,1}) \ad_x) \\
  %   & = 2 x_{2,3} - x_{1,2}
  % \end{align*}
  % Thus, using some simple algebra shows that \(x = \lambda h +
  % \frac{1}{4} f\) for some \(\lambda\). We similarly compute 
\end{example}
\begin{thm}
  If the Killing form of \(\g\) is identically zero, then \(\g\) is
  solvable.
\end{thm}
\begin{proof}
  In this case, \(\g = \Rad \kappa\) by definition, and by the remark
  above, \(\Rad \kappa = S \subset \Rad \g\), so \(\g\) is
  itself solvable. \todo{Check this proof; seems too easy}
\end{proof}
Using the Killing form, we can then arrive at the following
characterization of semisimple Lie algebras, mirroring the one for
semisimple Artinian rings.
\begin{thm}\label{structure-theorem-of-ss-las}
  A Lie algebra \(\g\) is semisimple if and only if it is isomorphic
  to a direct sum of non-trivial simple Lie algebras. 
\end{thm}
\begin{proof}
  First, assume \(\g\) is semisimple. If \(\g\) is nontrivial and
  simple, we are done, so assume \(\g\) is not simple. Then, consider
  a minimal non-zero ideal \(I \ideal \g\). Now, \(I^\perp\) is also
  an ideal and, since \(\kappa\) is non-degenerate, \(I \intersect
  I^\perp = \emptyset\). Thus, \[
    \g = I \oplus I^\perp
  \]
  \todo{This seems too fast. Carter spends much more time to do this.}
  Thus, we must show that \(I\) is a \emph{simple} Lie algebra. Take
  \(J \ideal I\). Then, it must be \[
    [J,\g] \subset [J,I] + [J,I^\perp] \subset [J,I] + [I,I^\perp] \subset J
  \]
  Thus, \(J \ideal \g\). However, \(I\) is minimal, so \(J = I\) or
  \(J=0\), so \(I\) is simple.

  Next, we wish to see that \(I^\perp\) is semisimple by showing all its
  solvable ideals are trivial. If \(J \ideal I^\perp\) is
  solvable, then \[
    [J,\g] \subset [J,I]+[J,I^\perp] \subset [J,I^\perp] \subset J
  \]
  and so \(J \ideal \g \implies J = 0\) since \(\g\) is
  semisimple. Thus, \(I^\perp\) is semisimple. \\

  For the converse, suppose \[
    \g = \g_1 \oplus \cdots \oplus \g_k
  \]
  where each \(\g_i\) is non-trivial and simple. We wish to show that
  the Killing form 
  on \(\g\) is non-degenerate. Each \(\g_i\) has non-degenerate
  Killing form. Take \(x \in \Rad \kappa \subset \g \implies x = x_1 +
  \cdots + x_k\) with \(x_i \in \g_i\). For \(y_i \in \g_i\), we
  have \[
    \langle x_i, y_i \rangle = \langle x,y_i \rangle = 0 \implies x_i
    = 0
  \]
  Thus, \(x = 0\) since the above is true for all \(i\). Therefore,
  \(\Rad \kappa = 0 \implies \g\) is semisimple.
\end{proof}
Thus, in summary, we have shown,
\begin{thm}
  Let \(\g\) be a Lie algebra. The following are equivalent.
  \begin{enumerate}
  \item \(\g\) is semisimple, that is, \(\Rad \g = 0\).
  \item \(\g\) does not contain any non-trivial, abelian ideals.
  \item The Killing form is non-degenerate on \(\g\).
  \item \(\g\) is isomorphic to the direct sum of finitely many
    non-trivial simple Lie algebras.
  \end{enumerate}
\end{thm}
\begin{proof}
  \((a) \implies (b)\) is immediate since any abelian ideal is also a
  solvable ideal, but the maximal solvable ideal must be trivial. For
  \((b) \implies (a)\), let \(\g\) have no non-trivial, abelian
  ideals. Then, \(\g\) does not have any solvable ideals, since the
  last ideal of a derived series must be abelian. \\

  \((a) \iff (c)\) is \ref{cartan-semisimplicity-criterion}

  \((a) \iff (d)\) is \ref{structure-theorem-of-ss-las}
\end{proof}
\section{Further Directions}
In light of the above discussion, we see that semisimple Lie algebras
break up into a direct sum of simple Lie algebras. Furthermore, in a
strengthening of the Levi decomposition presented above
(\ref{levi-decomp}), one gets that an arbitrary Lie algebra in
characteristic \(0\) breaks up as \(\g =
\mathfrak{l} \ltimes \rad \g\) where
\(\mathfrak{l} \isom \g/\rad \g\) 
is a semisimple subalgebra of \(\g\) called a \de{Levi
  subalgebra}. The existence of Levi subalgebras was proved by Levi in
1905 and Malcev proved that any two Levi subalgebras are conjugate in
1942; the combination of these results is sometimes called the
Levi-Malcen Theorem. \todo{This could be expanded upon. \cite{milne}
  has more details.}

Thus, in characteristic \(0\), if we could classifly \emph{all} simple
Lie 
algebras, we would understand all the possible pieces an arbitrary Lie
algebra could decompose into. An incredibly
important tool for doing this is the representation theory of
semisimple Lie algebras. In fact, using this representation,
mathematicisna developed a beautiful and complete
classification of simple Lie algebras.
\begin{bibdiv}
  \begin{biblist}
    \bib{carter}{book}{
      author={Carter, Roger}
      title={Lie Algebras of Finite and Affine Type}
      year={2005}
    }
    \bib{erdmann}{book}{
      author={Erdmann, Karin}
      author={Wildon, Mark J.}
      title={Introduction to Lie Algebras}
      year={2006}
    }
    \bib{humph}{book}{
      author={Humphreys, James E.}
      title={Introduction to Lie Algebras and Representaton Theory}
      year={1972}
      note={Third printing, revised}
    }
    \bib{cat-o}{book}{
      author={Humphreys, James E.}
      title={Representations of Semisimple Lie Algebras in the BGG
        Category $\mathcal{O}$}
      year={2008}
    }
    \bib{milne}{misc}{
      author={Milne, James S.},
      title={Lie Algebras, Algebraic Groups, and Lie Groups},
      year={2013},
      note={Available at www.jmilne.org/math/}
    }
    \bib{tao}{misc}{
      author={Tao, Terrence}
      title={Notes on the classification fo complex Lie algebras}
      year={2013}
      note={[Online; accessed 29-December-2017] \url{https://terrytao.wordpress.com/2013/04/27/notes-on-the-classification-of-complex-lie-algebras/}}
%      note={}
    }
    % \bib{wiki}{misc}{
    %   author={Wikipedia}
    %   title={Root system --- Wikipedia{,} The Free Encyclopedia}
    %   year={2017}
    %   url={https://en.wikipedia.org/w/index.php?title=Root_system&oldid=759729605}
    %   note={[Online; accessed 21-February-2017]}
    % }
  \end{biblist}
\end{bibdiv}

\end{document}