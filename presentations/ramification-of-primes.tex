\documentclass[11pt,leqno,oneside]{amsart}
\usepackage[alphabetic,abbrev]{amsrefs} % use AMS ref scheme
\usepackage{./presentation-notes}
\usepackage{../ReAdTeX/readtex-core}
\usepackage{../ReAdTeX/readtex-abstract-algebra}
\usepackage{caption}
\usepackage{subcaption}
\usepackage{todonotes}
\usepackage{ytableau}
\usepackage{xcolor}
\usepackage{mathtools}
\usepackage{tikz-cd}
\numberwithin{thm}{section}

\renewcommand{\O}{\mathcal{O}}
\newcommand{\disc}{\operatorname{disc}}
\newcommand{\Tr}{\operatorname{Tr}}

\title[Ramification of Primes]{Ramification of Primes: A Presentation
  for Math 8600: Commutative Algebra}
\author{George H. Seelinger}
\date{May 2018}
\begin{document}
\maketitle
\section{Introduction}
Let \(K|\Q\) be a finite field extension. Then, we may consider the
integral closure of \(\Z\) in \(K\), say \(\O_K\). Thus, we have the
following setup.
\[\begin{tikzcd}
    K & \ar[l, hookrightarrow]\O_K \\
    \Q \ar[u,dash] & \ar[l, hookrightarrow] \ar[u,dash]\Z
\end{tikzcd}
\]
where \(\O_K|\Z\) is an integral ring extension. Now, recall the
following facts.
\begin{prop}\label{intro-prop}
  Given the setup above
  \begin{enumerate}
  \item \(\O_K\) is a Dedekind domain.
  \item Given a prime \(p \in \Z\), the ideal \((p) = p\O_K \ideal
    \O_K\) has a unique decomposition \[
      (p) = \prod_{i=1}^n P_i^{e_i}
    \]
    for prime ideals \(P_i \ideal \O_K\) and \(e_i \in \N_0\).
  \item \(\O_K\) is a finitely-generated, free \(\Z\)-module, say \[
      \O_K \isom \Z \alpha_1 \oplus \cdots \oplus \Z \alpha_n \text{
        as a }\Z\text{-module.}
    \]
    Thus, \(\O_K/p\O_K\) is a finitely-generated \(\Z/p\Z\)-modules,
    that is \[
      \O_K/p\O_K \isom (\Z/p\Z) \ov{\alpha_1} \oplus \cdots \oplus (\Z/p\Z)\ov{\alpha_n}
    \]
  \end{enumerate}
\end{prop}
This leads us to the following definition:
\begin{defn}
  We say a prime \(p \in \Z\) is \de{ramified} in \(\O_K\) if \[
    p\O_k = \prod_{i=1}^n P_i^{e_i}
  \]
  has some \(e_i > 1\) for prime ideals \(P_i \ideal \O_K\). If every
  \(e_i = 1\), then \(p\) is
  \de{unramified} in \(\O_K\).
\end{defn}
\begin{example}
  Consider \(2 \in \Z[i]\). Then, since \[
    -i(1+i)(1+i) = -i(1+2i-1) = -i2i = 2,
  \] we have that \((2) \subset (1+i)^2\). However, \((2)
  \neq (1+i)\), so it must be that \((2) = (1+i)^2\). Therefore, \(2\)
  ramifies in \(\Z[i]\).
\end{example}
We wish to come up with some method to determine when a prime will
ramify in \(\O_K\). One such characterization uses the notion of the ``discriminant.''
\begin{defn}
  Let \(B|A\) be an integral ring extension where \(A\) is a PID. Then, we may view \(B\)
  as a free \(A\)-module with integral basis \(\{x_1, \ldots,
  x_n\}\). We define
  \begin{enumerate}
  \item The \de{trace} \(\Tr_{B|A}(z) := \tr m_z\) where, if \[
      zx_i = \sum_{j=1}^n a_{i,j} x_j
    \]
    then \(m_z = (a_{i,j})_{1 \leq i,j \leq n}\).
  \item The \de{trace form} \(T \from B \times B \to A\) is given
    by \[
      T(x,y) = \Tr_{B|A}(xy)
    \]
  \item The \de{discriminant} of an integral basis \(\{x_1, \ldots,
    x_n\}\) is \[
      \disc_A(x_1, \ldots, x_n) = \det \left( T(x_i, x_j) \right)
    \] 
  \end{enumerate}
\end{defn}
\begin{example}
  Let \(A = \Z\) and \(B = \Z[i]\). Then, if we take integral basis
  \(\{1,i\}\), we get \[
    m_1 = \left(
      \begin{array}{cc}
        1&0\\
        0&1
      \end{array}
\right), m_i = \left(
  \begin{array}{cc}
    0&1\\
    -1&0
  \end{array}
\right), \text{ and } m_{-1} = -m_1
\]
Thus, \[
  \Tr(1) = 2, \Tr(i) = 0, \Tr(-1) = -2
\]
and so \[
  \disc_\Z(\{1,i\}) = \det \left(
    \begin{array}{cc}
      \Tr(1)&\Tr(i)\\
      \Tr(i)&\Tr(-1)
    \end{array}
\right) = \det \left(
  \begin{array}{cc}
    2&0\\
    0&-2
  \end{array}
\right) = -4
\]
\end{example}
This paper seeks to prove the following useful characterization for when a
prime \(p\) ramifies in \(\O_K\).
\begin{thm}\label{main-thm}
  A prime \(p \in \Z\) ramifies in \(\O_K\) if and only if \(p
  \divides \disc_\Z(\O_K)\). 
\end{thm}
From this result, we also have the useful corollary
\begin{cor}
  Only a finite number of primes \(p \in \Z\) ramify in \(\O_K\).
\end{cor}
Thus, from our running example, \(2\) is the only prime that ramifies
in \(\Z[i]\). In the next section, we will follow a synthesis of the
programs by \cite{ash}*{4.2} and \cite{conrad} to prove this theorem.
\section{Structure and trace of the quotient \(\O_K/p\O_k\)}
Using our same setup, let \((p) = p\O_k = \prod_i P_i^{e_i}\) for
prime ideals \(P_i \ideal \O_K\) and \(e_i \in \N\). 
\begin{lem}\label{p-ramifies-iff-nilp}
  \(p\) ramifies if and only if the ring \(\O_K/(p)\) has nonzero
  nilpotent elements.
\end{lem}
\begin{proof}
  \begin{itemize}
    \item \((\implies)\). Let \(p\) ramify in \(\O_K\). Then, \(B/pB
      \isom B/P_1^{e_1}\times \cdots \times B/P_n^{e_n}\) by the
      Chinese Remainder Theorem, where at
      least one \(e_i > 1\), let us say \(e_1\). Then, the quotient
      ring \(\O_K/P_1^{e_1}\) has a nilpotent element since, for \(x \in
      P_1 \setminus P_1^{e_1}\), we get \((x+P_1^{e_1})^{e_1} =
      x^{e_1} P_1^{e_1} = P_1^{e_1}\).
    \item \((\impliedby)\). If \(p\) does not ramify in \(\O_K\), then
      \(\O_K/p\O_K \isom \O_K/P_1 \times \cdots \times \O_K/P_n\), each of which
      is a (finite) field since each \(P_i\) is maximal in
      \(\O_K\) and \(\O_K\) is a finitely-generated \(\Z\)-module. Thus,
      \(\O_K/p\O_K\) cannot have any nilpotent elements. 
    \end{itemize}
  \end{proof}
We also have, as a corollary to the proof, that
\begin{cor}\label{unramified-gives-prod-of-finite-fields}
  If \(p\) is unramified in \(\O_K\), then \(\O_K\) is a product of
  finite fields. 
\end{cor}
This is a useful fact since
\begin{lem}\label{nilp-elt-has-zero-trace}
  A nilpotent element has zero trace.
\end{lem}
\begin{proof}
  Let \(x^n = 0\) for some \(n \in \N\). Then, since \(m_{x^k} =
  (m_x)^k\), it must be that \((m_x)^n = 0\), so \(m_x\) is a
  nilpotent matrix, which has trace \(0\) since its mimimal
  polynomial \(\mu_{m_x}(t) \divides t^n\). Therefore, \[
    \Tr_{B|A}(x) = \tr m_{x} = 0
  \]
\end{proof}
And so, we get
\begin{lem}\label{nilp-elt-gives-disc-zero-mod-p}
  For prime \(p\), let \((p) \subset
  P_i^{e_i}\) such that \(e_i > 1\). Then,
  \(\disc_{\Z/p\Z}(\O_K/P_i^{e_i}) = \ov{0}\).
\end{lem}
\begin{proof}
  By the above, we know \(\O_K/P_i^{e_i}\) has a nilpotent element,
  say \(x\). Then, extend \(\{x\}\) to a basis of \(\O_K/P_i^{e_i}\)
  over \(\Z/p\Z\), say \(\{x,x_2, \ldots,
  x_k\}\). Each \(xx_i\) is nilpotent, so, for all \(i\), \[
    \Tr_{\O_K/(p)|\Z/p\Z}(xx_i) = \ov{0}
  \]
  and so, since the trace form matrix will have a row of all zeros, it
  must have determinant equal to \(\ov{0}\) and so the discriminant is \(0\).
\end{proof}
\begin{lem}\label{trace-form-nondeg-on-field-ext}
  Let \(p\) be an unramified prime in \(\O_K\) such that \((p) = \prod_i
  P_i\). Then, the trace form on \(\O_K/P_i\) is nondegenerate.
\end{lem}
\begin{proof}
  By the arguments above, we already know that \(\O_K/P_i\) is a
  finite field, and since \(\F_p\) is perfect, we have that
  \(\O_K/P_i|\F_p\) is a separable field extension. Therefore, by a
  lemma in class \todo{cite this}, it must be that the trace form is
  nondegenerate. 
\end{proof}
\section{Discriminant Behaves  Well with Reduction \(\mod p\) and Products}
\begin{lem}\label{disc-plays-nice-with-mod-p}
  For an appropriate choice of bases, \[
    \disc_\Z(\O_K) \mod p = \disc_{\Z/p\Z}(\O_K/pO_k)
  \]
\end{lem}
\begin{proof}
  Let \(\{\alpha_1, \ldots, \alpha_n\}\) be an integral basis for
  \(\O_K|\Z\). Then, for \(x \in \O_K\), we have \[
    x \alpha_i = \sum_j a_{i,j} \alpha_j \implies x \alpha_i +
    \O/p\O_K = \sum_j \ov{a_{i,j}} \alpha_j + \O/p\O_k
  \]
  where \(\ov{a_{i,j}} = a_{i,j}\). Thus, \(m_x\) with the entries
  reduced mod \(p\) is equal to \(m_{x+\O_K/p\O_K}\). Thus, \[
    \Tr_{\O_K/p\O_K | \Z/p\Z}(x+\O_K/p\O_K) = \tr(m_{x+\O_K/p\O_K}) =
    \tr(m_x) \mod p = \Tr_{\O_K|\Z}(x) \mod p
  \]
  giving us that \[
    (\Tr_{\O_K|\Z}(\alpha_i \alpha_j))_{1 \leq i,j \leq n} \mod p =
    \Tr_{\O_K/(p)|\Z/p\Z}(\ov{\alpha}_i \ov{\alpha_j})
  \]
  and so, taking determinants of both sides gives the desired result.
\end{proof}
\begin{lem}\label{disc-plays-nice-with-prod}
  Let \(A\) be a commutative ring with \(B_1, B_2\) ring
  extensions of \(A\) that are finitely-generated, free
  \(A\)-modules. Then, up to appropriate choice of basis, \[
    \disc_A(B_1 \times B_2) = \disc_A(B_1) \disc_A(B_2)
  \]
\end{lem}
\begin{proof}
  Let \[
    B_1 = \bigoplus_{i=1}^m A e_i, \ \ B_2 = \bigoplus_{j=1}^n A f_i
  \]
  Then, take the standard choice of \(A\)-basis of \(B_1 \times B_2\),
  \(\{e_1, \ldots, e_m, f_1, \ldots, f_m\}\). Since \(e_i f_j = 0\) in
  \(B_1 \times B_2\), we get that \[
    \disc_A(B_1 \times B_2) = \det \left(
      \begin{array}{cc}
        \Tr_{B_1 \times B_2 | A}(e_i e_k)& 0 \\
        0 & \Tr_{B_1 \times B_2 | A}(f_j f_\ell)
      \end{array}
    \right)
  \]
  Also, for \(x \in B_1\), since \(xy = 0\) for all \(y \in B_2\), we
  have \[
    \Tr_{B_1 \times B_2|A}(x) = \Tr_{B_1|A}(x)
  \]
  and similarly for \(y \in B_2\) \[
    \Tr_{B_1 \times B_2|A}(y) = \Tr_{B_2|A}(y)
  \]
  Thus, \[
\left(
      \begin{array}{cc}
        \Tr_{B_1 \times B_2 | A}(e_i e_k)& 0 \\
        0 & \Tr_{B_1 \times B_2 | A}(f_j f_\ell)
      \end{array}
    \right) = \left(
      \begin{array}{cc}
        \Tr_{B_1 | A}(e_i e_k)& 0 \\
        0 & \Tr_{B_2 | A}(f_j f_\ell)
      \end{array}
    \right)
  \]
  and so, taking the determinant of both sides, we get the desired result.
\end{proof}
\section{Proof of the Ramification Theorem}
We now prove our theorem.
\begin{proof}[Proof of \ref{main-thm}]
  We first observe that
  \begin{align*}
    p \divides \disc_\Z(\O_K)
    & \iff \disc_\Z(\O_K) \equiv 0 \mod p\\
    & \iff \disc_{\Z/p\Z}(\O_K/(p)) = \ov{0} & \text{by Lemma \ref{disc-plays-nice-with-mod-p}}\\
    & \iff \prod \disc_{\Z/p\Z}(\O_K/P_i^{e_i}) = \ov{0}
      & \text{by Lemma \ref{disc-plays-nice-with-prod}}
  \end{align*}
  Thus, if any \(e_i > 1\), we get that \(\O_K/P_i^{e_i}\) has a
  nilpotent element by \ref{p-ramifies-iff-nilp}, and so \(\disc_{\Z/p\Z}(\O_K/P_i^{e_i})
  = \ov{0}\) by \ref{nilp-elt-gives-disc-zero-mod-p}, thus giving \(p
  \divides \disc_\Z(\O_K)\) by the 
  equivalences above. \\

  If all \(e = 1\), then each \(\O_K/P_i\) is a finite field, so
  \(\disc_{\Z/p\Z}(\O_K/P_i) \neq 0\) because, \[
    \hspace{-0.5in} T \text{ on } \O_K/P_i \text{ nondegenerate by \ref{trace-form-nondeg-on-field-ext}} \iff (T(x_i
    x_j))_{1 \leq i,j \leq n} \text{ is invertible } \iff
    \disc_{\Z/p\Z}(\O_K/P_i) \neq \ov{0}
  \]
  Therefore, it must be that \(p \notdivides \disc_\Z(\O_K)\).
\end{proof}
\section{Factorization in Quadratic Number Fields}
In this section, we follow \cite{ash} to determine some results about
factorization of primes in quadratic number fields. First, recall the theorem
\begin{thm}[Ram-Rel Identity]
  Given an integral domain \(A\) with field of fractions \(K\), finite
  field extension \(L|K\), and \(B\) the integral closure of \(A\) in
  \(L\), then, given a prime ideal \(P \ideal A\), then, if \[
    PB = \prod_{i=1}^g P_i^{e_i} \ \ f_i = [B/P_i:A/P]
  \]
  we have that \[
    \sum_{i=1}^g e_i f_i = [B/PB:A/P] = n
  \]
\end{thm}
Thus, given \(\Q(\sqrt{m})|\Q\) has degree \(2\), if must be that, for
a prime \(p \in \Z\), there are only three possible situations.
\begin{enumerate}
\item \(g=2,e_1=e_2=f_1=f_2=1\), that is, \[
    (p) = P_1 P_2
  \]
  In this situation, we say that \(p\) \emph{splits} in \(\O_K\).
\item \(g=1,e_1=1,f_1=2\), that is, \((p)\) is a prime ideal of
  \(\O_K\). In this situations, we say that \((p)\) is \emph{inert}.
\item \(g=1,e_1=2,f_1=1\), that is, \[
    (p) = P_1^2
  \]
  so \(p\) ramifies.
\end{enumerate}
Furthermore, we recall the following result about the discriminant of
\(\Q(\sqrt{m})\).
\begin{prop}
  The discriminant of \(\Q(\sqrt{m})\) is \(m\) if \(m \equiv 1 \mod 4\)
  and it is \(4m\) if \(m \equiv 2,3 \mod 4\). In particular, the
  discriminant is always \(0\) or \(1 \mod 4\).
\end{prop}

We then have the following result.
\begin{thm}
  \cite{mack-crane} Let prime \(p \neq 2\) and \(D = \disc_\Q(\Q(\sqrt{m}))\). Then,
  \begin{enumerate}
  \item \((p)\) ramifies as \((p,\sqrt{m})^2\) in \(\Q(\sqrt{m})\) if
    and only if \(D \equiv 
    0 \mod p\).
  \item \((p)\) splits as \((p) = (p,a+\sqrt{m})(p,a-\sqrt{m})\) in
    \(\Q(\sqrt{m})\) if and only if \(D \equiv 
    a^2 \mod p\) for some \(a \not \equiv 0 \mod p\).
  \item \((p)\) is inert in \(\Q(\sqrt{m})\) if and only if \(D \not
    \equiv a^2 \mod p\) for all \(a\).
  \end{enumerate}
  If \(p = 2\), then
  \begin{enumerate}
  \item \((2)\) ramifies in \(\Q(\sqrt{m})\) if and only if \(D \equiv
    0,4 \mod 8\).
  \item \((2)\) splits in \(\Q(\sqrt{m})\) if and only if \(D \equiv 1
    \mod 8\).
  \item \((2)\) is inert in \(\Q(\sqrt{m})\) if and only if \(D \equiv
    5 \mod 8\).
  \end{enumerate}
\end{thm}
\begin{proof}
  We break down the various situations.
  \begin{itemize}
  \item Assume \(p\) is an odd prime with \(p\) not dividing \(m\).
    \begin{itemize}
    \item \(p\) does not divide the discriminant, so \(p\) cannot
      ramify.
    \item If \(m \equiv a^2 \mod p \iff D \equiv (2a)^2 \mod p\), then
      \(x^2-m\) factors \(\mod 
      p\) as \((x+a)(x-a)\) and so \((p) =
      (p,a+\sqrt{m})(p,a-\sqrt{m})\).
    \item If \(m \not \equiv a^2 \mod p\), then \(x^2-m\) cannot be
      the product of two linear factors, so \(x^2-m\) is irreducible
      \(\mod p\) and so \((p)\) is inert.
    \end{itemize}
  \item Let \(p\) divide \(m\). Then, \(p\) divides the discriminant
    and so \((p)\) ramifies. In fact, since \(x^2-m \equiv x^2 = xx
    \mod p\), we get \((p) = (p,\sqrt{m})^2\).
  \item Let \(p = 2\) and \(m\) be odd.
    \begin{itemize}
    \item If \(m \equiv 3 \mod 4 \iff m \equiv 3 \text{ or } 7 \mod 8
      \implies D = 4m\), then \(2\) divides the 
      discriminant \(4m\), so \((2)\) ramifies. We also get \(x^2-m
      \equiv (x+1)^2 \mod 2\) and so \((2) = (2,1+\sqrt{m})^2\).
    \item If \(m \equiv 1 \mod 8 \iff D \equiv 1 \mod 8\), then \(m
      \equiv 1 \mod 4\). We get 
      an integral basis \(\{1, \frac{1+\sqrt{m}}{2}\}\) and the
      discriminant is \(D = m\). Therefore, \(2 \notdivides D\), so
      \((2)\) does not ramify. We then compute, \[
        (2,\frac{1+\sqrt{m}}{2})(2,\frac{1-\sqrt{m}}{2}) =
        (2,1-\sqrt{m},1+\sqrt{m},
        \underbrace{\frac{1-m}{4}}_{\text{Even}}) = (2)
      \]
    \item If \(m \equiv 5 \mod 8 \iff D \equiv 5 \mod 8\), then \(m
      \equiv 1 \mod 4\), so \(D 
      = m\), meaning \(2\) does not ramify. Consider \[
        f(x) = x^2-x+\frac{1-m}{4} \in (\O_K/P)[x]
      \]
      where \((2) \subset P\) a prime ideal in \(\O_K\). The roots of
      \(f\) are \(\frac{1\pm\sqrt{m}}{2}\), so \(f\) has a root in
      \(\O_K\) and hence in \(\O_K/P\). However, since \(\frac{1-m}{4}
      \equiv 1 \mod 2\), \(f\) has no root in \(\F_2\). Therefore,
      \(\O_K/P\) and \(\F_2\) cannnot be isomorphic. If \((2) =
      P_1P_2\) in \(\O_K\), then the norm of \((2)\) is \(4\) and so
      \(P_1,P_2\) each have norm \(2\). Therefore, \(\O_K/P_i \isom
      \F_2\), which is a contradiction. Thus, \((2)\) must remain prime.
    \end{itemize}
  \end{itemize}
\end{proof}
\begin{bibdiv}
  \begin{biblist}
    \bib{ash}{book}{
      author={Ash, Robert B.}
      title={A Course In Algebraic Number Theory}
      year={2003}
      note={\url{https://faculty.math.illinois.edu/~r-ash/ANT.html}}
    }
    \bib{conrad}{article}{
      author={Conrad, Keith}
      title={Discriminants and Ramified Primes}
      note={\url{http://www.math.uconn.edu/~kconrad/blurbs/gradnumthy/disc.pdf}}
    }
    \bib{mack-crane}{article}{
      author={Mack-Crane. Sander}
      title={Prime Splitting in Quadratic Extensions I: One PRime,
        Many Fields}
      year={2016}
      note={\url{https://algebrateahousejmath.wordpress.com/2016/11/23/prime-splitting-in-quadratic-extensions-i-one-prime-many-fields/}}
    }
  \end{biblist}
\end{bibdiv}
\end{document}