\documentclass[11pt,leqno,oneside]{amsart}
\usepackage[alphabetic,abbrev]{amsrefs} % use AMS ref scheme
\usepackage{./presentation-notes}
\usepackage{../ReAdTeX/readtex-core}
\usepackage{../ReAdTeX/readtex-dangerous}
\usepackage{../ReAdTeX/readtex-abstract-algebra}
\usepackage{../ReAdTeX/readtex-lie-algebras}
\usepackage{caption}
\usepackage{subcaption}
\usepackage{todonotes}
\usepackage{ytableau}
\usepackage{xcolor}
\usepackage{mathtools}
\usepackage{tikz-cd}
\numberwithin{thm}{section}

\renewcommand{\W}{\mathcal{W}}
\renewcommand{\H}{\mathcal{H}} % Hecke algebra
\DeclareMathOperator{\ch}{ch}

\newcommand{\trace}{\operatorname{Tr}}
\newcommand{\transpose}{t}
\DeclareMathOperator{\SSYT}{SSYT}
\DeclareMathOperator{\wt}{wt}

\newtheorem{goal}[thm]{Goal}

\title[Kazhdan-Lusztig Basis]{Kazhdan-Lusztig basis for Hecke algebras
  \\ A class presentation for Quantum Groups} 
\author{George H. Seelinger}
\date{December 9, 2019}
\begin{document}
\maketitle
\section{Introduction}
The Kazhdan-Lusztig basis was introduced in~\cite{kl}. We will define
the basis and give 
a proof of its existence and uniqueness, although we will mainly
follow the proof in~\cite{soergel}. Since their introduction, the
so-called Kazhdan-Lusztig polynomials, which appear in the definition
of the basis, have appeared in many other fields of mathematics. For a
more detailed overview of connections, see~\cite{brenti}*{p 5}.
\section{Preliminaries}
We work with the Hecke algebra, for which we will give two
presentations.
\begin{defn}
  \cite{humphreys}*{Section 7.4} Let \(A = \Z[q,q^{-1}]\). Then, the
  \de{Hecke algebra} \(\H\) associated to 
  a Weyl group \(\W\) has a basis \(\{T_w \st w \in \W\}\) with
  relations
  \begin{enumerate}
  \item \(T_x T_y = T_{xy}\) if \(\ell(x) + \ell(y) = \ell(xy)\) and
  \item \(T_s^2 = (q-1)T_s + qT_{id}\) for all simple reflections \(s
    \in \W\).
  \end{enumerate}
\end{defn}
\begin{rmk}
  We need not restrict \(\W\) to be a Weyl group. In full generality,
  we can replace \(\W\) with any Coxeter group.
\end{rmk}
For our purposes, it will also be convenient to work with the Hecke
algebra over an enlarged ring. Let \(v := q^{-\frac{1}{2}}\). Then, we
have the following.
\begin{prop}
  The Hecke algebra over \(\Z[v,v^{-1}]\) is given as the associative algebra
  with generators \(\{H_s\}\) for \(H_s = vT_s\) and relations
  \begin{enumerate}
  \item \(H_s^2 = 1 + (v^{-1}-v)H_s\) and 
  \item \(H_s H_t \cdots H_s = H_t H_s \cdots H_t\) or \(H_s H_t H_s
    \cdots H_t = H_t H_s H_t \cdots H_s\) if \(st\cdots s = ts \cdots
    t\) or \(sts \cdots t = tst \cdots s\), repsectively, for simple
    reflections \(s,t \in \W\)
  \end{enumerate}
\end{prop}
\begin{prop}\label{H-basis}
  The Hecke algebra over \(\Z[v,v^{-1}]\) has a basis given by \(\{H_w
  \st w \in \W\}\) where \(H_w = v^{\ell(w)} T_w\). Furthermore, this
  basis has relation \(H_x H_y = H_{xy}\) if \(\ell(x)+\ell(y) = \ell(xy)\).
\end{prop}
\begin{lem}
  We have \(H_s^{-1} = H_s + (v-v^{-1})\) and so all the \(H_x\) basis
  elements are units in \(\H\).
\end{lem}
\begin{proof}
  \[
    H_s^2-(v^{-1}-v)H_s = 1 \implies H_s(H_s+(v-v^{-1})) = 1 
  \]
\end{proof}
\begin{lem}\label{lem:length-lowering-mult}
  For simple reflection \(s \in \W\), if \(\ell(xs) < \ell(x)\),
  then \[
    H_x H_s = H_{xs} + (v^{-1}-v)H_x \,. 
  \]
\end{lem}
\begin{proof}
  We know that \[
    H_x = H_{xs} H_s \implies H_x H_s = H_{xs} H_s^2 =
    H_{xs}(1+(v^{-1}-v)H_s) = H_{xs} + (v^{-1}-v)H_x
  \]
\end{proof}
\section{The Kazhdan-Lusztig Basis}
\begin{defn}
  Recall that \(A = \Z[q,q^{-1}]\).
  \begin{enumerate}
  \item We define the \(\Z\)-linear map, called the \de{bar
      involution}, \(\ov{\ } \from A \to A\)
    given by sending \(q \mapsto q^{-1}\)
  \item   The Hecke algebra \(\H\) admits an extension of the bar
    involution, say \(\iota \from \H
  \to \H\), given by \[
    \iota(T_w) := T_{w^{-1}}^{-1}
  \]
  for any \(w \in \W\). For convenience, we will overload notation and
  write \[
    \ov{T_w} := \iota(T_w)
  \]
  \end{enumerate}
\end{defn}
Note that \(\iota(H_s) = v^{-1} T_{s}^{-1} = v^{-1}(v^2 T_s - 1 + v^2
) = H_s - v^{-1} + v = H_s^{-1} \) and, similarly, \(\iota(H_w) = H_{w^{-1}}^{-1}\).
% Following~\cite{williamson}, we note that if one wants to discuss elements fixed by \(\iota\), say of the form
% \(F(q)T_s + G(q)T_{id}\), one finds that \[
%    \hspace{-0.5in}\iota(F(q)T_s + G(q)T_{id}) = F(q^{-1})(q^{-1}T_s+(q^{-1}-1)T_{id})
%    + G(q^{-1})T_{id} \implies
%    \begin{cases}
%      F(q) = q^{-1}F(q^{-1}) \\
%      G(q) = G(q^{-1})+F(q^{-1})(q^{-1}-1)
%    \end{cases}
%  \]
%  and so the simplest solution would be \(F(q) = q^{-1}+1\) and
%  \(G(q)=-q\). However, if we allow \(F(q),G(q) \in \Z[q^{\frac{1}{2}},
%  q^{-\frac{1}{2}}]\), then we can have \(F(q) = q^{-\frac{1}{2}}\) and \(G(q) =
%  -q^{\frac{1}{2}}\). So, let us redefine \(A\) to be the (larger) ring
%  \(\Z[q^{\frac{1}{2}},q^{-\frac{1}{2}}]\). 
 Then, we have an \(\iota\)-invariant
 of the form \[
   C_s := q^{-\frac{1}{2}} T_s - q^{\frac{1}{2}} T_{id} = H_s-v^{-1} H_{id}
 \]
 We can also introduce a similar \(\iota\)-invariant of the form \[
   C_s' := H_s+v H_{id}
 \]
 This justifies why we introduced the \(H\)-basis in
 Proposition~\ref{H-basis}. 
 In~\cite{humphreys}*{p 158}, it is noted that it could be tempting to
 construct further \(\iota\)-invariants by taking products of these
 \(C_s\) elements. However, if one has a word \(sts = tst\) with \(s,t
 \in \W\) both simple reflections and \(\ell(sts) = 3 = \ell(tst)\),
 then one can check that \(C_s C_t C_s \neq C_t C_s C_t\). However, if
 we compute (still assuming \(\ell(sts) = 3\)) \[
   C_s C_t C_s - C_t = q^{-\frac{3}{2}}(T_{sts}-q(T_{st}-T_{ts})+q^2(1+q^{-1})(T_s+T_t)-q^3(1+2q^{-1})T_{id})
 \]
 we get an \(\iota\)-invariant expression where the \(s\) and \(t\)'s
 are interchangeable. Similarly, we can compute \[
   C_s' C_t' C_s' - C_s' = H_{sts} + v(H_{ts}+H_{st})+v^2(H_s+H_t) + v^3
   H_{id}
 \]
 since \(vH_s^2 = H_s-v^2H_s+vH_{id}\) and so \(vH_s^2 - C_s' = -v^2 H_s\).
 
 This illustrates the problem more generally we wish to solve. For
 every \(w \in \W\), we want to associate an \(\iota\)-invariant
 element, \(C_w\), which is a linear combination of \(T_x\) for \(x
 \leq w\), thus giving us a basis. In order to follow~\cite{soergel},
 we will actually produce elements \(C_w'\) as a linear combination of
 \(H_x\)'s, but the idea remains the same. To do this, we first recall
 a partial ordering on the Weyl group.
 \begin{defn}
   For \(u,v \in \W\), we say \(u \leq v\) in the (strong) \de{Bruhat
     order} on \(\W\) if some substring of some reduced word
   for \(v\) is a reduced word for \(u\).
 \end{defn}
 \begin{example}
   Let \(\W = \Sym_3 = \langle s_1, s_2 \rangle\). Then, the Bruhat
   order is given by the following poset. \[
     \begin{tikzcd}[row sep=tiny, column sep=tiny]
       & s_1s_2s_1 & \\
       s_1 s_2 \ar[ur] & & s_2 s_1 \ar[ul]\\
       s_1 \ar[u] \ar[urr]& & s_2 \ar[u] \ar[ull]\\
       & id \ar[ul] \ar[ur]&
     \end{tikzcd}
   \]
 \end{example}
\begin{thm}\label{thm:KL}
  \cite{soergel}*{Theorem 2.1} For each \(w \in \W\), there exists a
  unique element \(C_w' \in \H\) having the following properties:
  \begin{enumerate}
  \item \(\iota(C_w') = C_w'\) 
  \item \(C_w' \in H_w + \sum_{x < w} v \Z[v] H_x\) where \(x < w\) in
    the (strong) Bruhat order.
  % \item \(C_w' = \epsilon_w q_w^{\frac{1}{2}} \sum_{x \leq w} \epsilon_x
  %   q_x^{-1} \ov{P}_{x,w} T_x\) where \(P_{w,w} = 1\) and \(P_{x,w}(q)
  %   \in \Z[q]\) has degree less than
  %   \(\frac{1}{2}(\ell(w)-\ell(x)-1)\) if \(x < w\).
  \end{enumerate}
\end{thm}
Then, one may wish to construct 
\begin{example}
  \begin{enumerate}
  \item From the above, we already see that if \(s \in \W\) is a simple
    reflection, then it must be that \[
      C_s' = H_s+v H_{id}
    \]
  \item We can compute the basis for \(\Sym_3 = \langle s_1, s_2
    \rangle\) by hand. We know that the simple reflections must be of
    the form.
    \begin{align*}
      C_{s_1}' = H_{s_1} +vH_{id}\\
      C_{s_2}' = H_{s_2}+vH_{id}
    \end{align*}
    Then, to form \(\iota\)-invariants of length \(2\), we check \[
      C_{s_1}' C_{s_2}' = H_{s_1 s_2} + v(H_{s_1}+H_{s_2}) + v^2 H_{id}
    \]
    is \(\iota\)-invariant. If we apply \(\iota\) to this, we get
    \begin{align*}
      \iota (C_{s_1}' C_{s_2}')
      & = H_{s_1 s_2}+(v-v^{-1})(H_{s_1} + H_{s_2})+(v-v^{-1})^2 +
        v^{-1}(H_{s_1} + H_{s_2}+2(v-v^{-1})) + v^{-2} \\
      & = H_{s_1 s_2} + v(H_{s_1}+H_{s_2}) + (v-v^{-1})^2 +
        2(1-v^{-2}) + v^{-2} \\
      & = H_{s_1 s_2} + v(H_{s_1}+H_{s_2}) + v^2
    \end{align*}
    So, by uniqueness, it must be \(C_{s_1 s_2}' = C_{s_1}'
    C_{s_2}'\). A similar computation gives \(C_{s_2 s_1}'\). For
    length \(3\), we already computed above that \[
      C_{s_1 s_2 s_1}' = C_{s_1}' C_{s_2}' C_{s_1}' - C_{s_1}' =
      H_{s_1 s_2 s_1} + v(H_{s_1 s_2} + H_{s_2 s_1}) +
      v^2(H_{s_1}+H_{s_2}) + v^3
    \]
  \end{enumerate}
\end{example}
\begin{proof}[Proof of Theorem~\ref{thm:KL}]
  We have already established the formula for \(C_s'\) for \(s\) a
  simple reflection. Now, we compute \[
    H_x C_s' =
    \begin{cases}
      H_{xs} + vH_x & \text{if } xs > x;\\
      H_{xs} + v^{-1} H_x & \text{if } xs < x
    \end{cases}
  \]
  where the first case is immediate from the definition of the Hecke
  algebra and the second case is a straightforward application
  of Lemma~\ref{lem:length-lowering-mult}. 
  To show existence, we proceed by induction on
  the Bruhat order. Certainly, \(C_{id}' = H_{id} = 1\). Now, let \(x
  \in \W\) be 
  given and suppose we know \(C_y'\) exists for all \(y < x\). If
  \(x \neq id\), we can find a simple reflection \(s\) such that \(xs
  < x\) and by induction, we get \[
    C_{xs}' C_s' = H_x + \sum_{y < x} h_y H_y
  \]
  for some \(h_y \in \Z[v]\). Then, we say \[
    C_x' = C_{xs}' C_s' - \sum_{y < x} h_y(0) C_y' \,.
  \]
  \(C_x'\) is \(\iota\)-invariant because it is a sum of
  \(\iota\)-invariant elements and it lies in \(H_x + \sum_{y <
    x}v\Z[v] H_y \) since, if \(C_y' = H_y + \sum_{z < y} h_{z,y} H_x\) for
  \(h_{z,y} \in v\Z[v]\),
  then \[
    C_x' = H_x + \sum_{y < x} \left( (h_y - h_y(0)) H_y - \sum_{z<y}
      h_y(0) h_{z,y} H_z \right) \,. 
  \]

  For uniqueness, we prove the following.
  \begin{lem}
    If \(H \in \sum_y v \Z[v] H_y\) is \(\iota\)-invariant, then \(H =
    0\). 
  \end{lem}
  We have \(H_z \in C_z' + \sum_{y < z}\Z[v,v^{-1}] C_y'\) for the
  \(C_x'\) elements described earlier in the proof by the
  unitriangularity condition. Now, if \(H = \sum_y h_y H_y\) and we
  choose \(z\) maximal such that \(h_z \neq 0\), then \(\iota(H) = H\)
  implies that \(\ov{h_z} = h_z\). However, this contradicts \(h_z \in
  v\Z[v]\), so it must be that \(H = 0\).

  Thus, if there were two \(\iota\)-invariant elements \(C_w'\) and
  \(D_w'\) satisfying the hypotheses of Theorem~\ref{thm:KL}, then it must be
  that \(C_w'-D_w' \in v \Z[v]\) is \(\iota\)-invariant, but the lemma
  shows that \(C_w'-D_w' = 0\). Thus, uniqueness is established.
\end{proof}
\begin{defn}
  For \(x,y \in \W\), we define the \de{Kazhdan-Lusztig polynomials}
  \(h_{y,x} \in \Z[v,v^{-1}]\) by the 
  equality \[
    C_x' = \sum_y h_{y,x} H_y
  \]
\end{defn}
\begin{rmk}
  These polynomials are related to the Kazhdan-Lusztig polynomials
  in~\cite{kl}, denoted \(P_{y,x}\), by \[
    h_{y,x} = v^{\ell(x)-\ell(y)} P_{y,x}
  \]
\end{rmk}
\begin{prop}
  Let \(\W\) be finite, \(w_\circ \in \W\) be the longest element, and
  \(r = \ell(w_\circ)\) its length. Then, we have \(C_{w_\circ}' = \sum_{y \in \W}
  v^{r-\ell(y)} H_y\).
\end{prop}
\subsection{Further Properties of Kazhdan-Lusztig Polynomials}
Since their introduction, the Kazhdan-Lusztig polynomials have been an
area of intense research. Now, much more is known than when they were
first introduced.
\begin{prop}
  \cite{kl2} For any Weyl group \(\W\) and \(x,y \in \W\), we have that the
  coefficients \(a_i\) occurring in \[
    P_{y,x}(q) = \sum_i a_i q^i
  \]
  satisfy \(a_i \in \Z_{\geq 0}\).
\end{prop}
\begin{rmk}
  This has been proved by~\cite{elias-williamson} for general Coxeter
  systems. 
\end{rmk}
In~\cite{kl}, the following was conjectured. It was proven
in~\cite{bb} and~\cite{bk}.
\begin{prop}
  Given a semisimple Lie algebra \(\g\) with Weyl group \(\W\), for
  each \(w \in \W\), let \(M_w\) be the Verma module with heighest
  weight \(-w(\rho)-\rho\) and let \(L_w\) be its unique irreducible
  quotient. Then, we have the equivalent identities
  \begin{enumerate}
  \item \(\ch L_w = \sum_{y \leq w} (-1)^{\ell(w)+\ell(y)} P_{y,w}(1)
    \ch M_y\)
  \item \(\ch M_w = \sum_{y \leq w} P_{w_\circ w, w_\circ y}(1) \ch L_y\)
  \end{enumerate}
  where \(w_\circ\) is the longest element of \(\W\). 
\end{prop}
Finally, there exists a geometric interpretation of the
Kazhdan-Lusztig polynomials using perverse sheaves.
\subsection{Historical Note}
Kazhdan and Lusztig were originally interested in using the
Kazhdan-Lusztig basis to construct representations of the Hecke
algebra, but their significance has extended far beyond this goal. Our
expoistion here does not follow~\cite{kl} and our definitions do not
match those in~\cite{kl}, although it is straightforward to translate
between~\cite{kl} and these notes. The proof given for
existence and uniqueness here is simpler; notably, this exposition will not
include the \(R\)-polynomials. Such a proof can be found in~\cite{humphreys}.
\begin{bibdiv}
  \begin{biblist}
    \bib{bb}{article}{
      author={Beilinson, A.}
      author={Bernstein, J.}
      title={Localisation de \(\g\)-modules}
      journal={C. R. Acad. Sci. Paris S\'{e}r. I Math.}
      volume={292}
      year={1981}
      number={1}
      pages={15--18}
    }
    \bib{bk}{article}{
      author={Brylinski, J.-L.}
      author={Kashiwara, M.}
      title={Kazhdan-Lusztig conjecture and holonomic systems}
      journal={Invent. Math.}
      volume={64}
      year={1981}
      number={3}
      pages={387--410}
    }
    \bib{brenti}{article}{
      author={Brenti, Francesco}
      title={Kazhdan-Lusztig polynomials: History, Problems, and
        Combinatorial Invariance}
      journal={S\'{e}minaire Lotharingien de Combinatoire}
      volume={49}
      year={2003}
    }
    \bib{elias-williamson}{article}{
      author={Elias, Ben}
      author={Williamson, Geordie}
      title={The Hodge theory of Soergel bimodules}
      year={2014}
      volume={180}
      issue={3}
      pages={1089--136}
    }
    \bib{humphreys}{book}{
      author={Humphreys, James E.}
      title={Reflection Groups and Coxeter Groups}
      year={1990}
    }
    \bib{kl}{article}{
      author={Kazhdan, David}
      author={Lusztig, George}
      title={Representation of Coxeter Groups and Hecke Algebras}
      journal={Inventiones math.}
      volume={53}
      year={1979}
      pages={165--184}
    }
    \bib{kl2}{article}{
      author={Kazhdan, David}
      author={Lusztig, George}
      title={Schubert varieties and Poincar\'{e} duality}
      journal={Geometry of the Laplace operator (Proc. Sympos. Pure
        Math., Univ. Hawaii, Honolulu, Hawaii, 1979)}
      year={1980}
      pages={185--203}
    }

    \bib{soergel}{article}{
      author={Soergel, Wolfgang}
      title={Kazhdan-Lusztig polynomials and a combinatoric for
        Tilting modules}
      journal={Representation Theory}
      volume={1}
      pages={83--114}
      year={1997}
    }
    \bib{williamson}{book}{
      author={Williamson, Geordie}
      title={Mind your $P$ and $Q$-symbols: Why the Kazhdan-Lusztig
        basis of the Hecke algebra of type A is cellular}
      year={2003}
    }
  \end{biblist}
\end{bibdiv}
\end{document}
%%% Local Variables:
%%% mode: latex
%%% TeX-master: t
%%% End:
