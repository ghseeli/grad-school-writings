\documentclass[11pt,leqno,oneside]{amsart}
\usepackage[alphabetic,abbrev]{amsrefs} % use AMS ref scheme
\usepackage{./presentation-notes}
\usepackage{../ReAdTeX/readtex-core}
\usepackage{../ReAdTeX/readtex-dangerous}
\usepackage{../ReAdTeX/readtex-abstract-algebra}
\usepackage{../ReAdTeX/readtex-lie-algebras}
\usepackage{caption}
\usepackage{subcaption}
\usepackage{todonotes}
\usepackage{ytableau}
\usepackage{xcolor}
\usepackage{mathtools}
\usepackage{tikz-cd}
\numberwithin{thm}{section}

\renewcommand{\W}{\mathcal{W}}
\renewcommand{\H}{\mathcal{H}} % Hecke algebra

\newcommand{\trace}{\operatorname{Tr}}
\newcommand{\transpose}{t}
\DeclareMathOperator{\SSYT}{SSYT}
\DeclareMathOperator{\wt}{wt}

\newtheorem{goal}[thm]{Goal}

\title[Kazhdan-Lusztig Basis]{Kazhdan-Lusztig basis for Hecke algebras
  \\ A class presentation for Quantum Groups} 
\author{George H. Seelinger}
\date{December 9, 2019}
\begin{document}
\maketitle
\section{Preliminaries}

\begin{defn}
  Recall that \(A = \Z[q,q^{-1}]\).
  \begin{enumerate}
  \item We define the \(\Z\)-linear map, called the \de{bar
      involution}, \(\ov{\ } \from A \to A\)
    given by sending \(q \mapsto q^{-1}\)
  \item   The Hecke algebra \(\H\) admits an extension of the bar
    involution, say \(\iota \from \H
  \to \H\), given by \[
    \iota(T_w) = T_{w^{-1}}^{-1}
  \]
  for any \(w \in \W\).
  \end{enumerate}
\end{defn}
Following~\cite{williamson}, we note that if one wants to discuss elements fixed by \(\iota\), say of the form
\(F(q)T_s + G(q)T_{id}\), one finds that \[
   \hspace{-0.5in}\iota(F(q)T_s + G(q)T_{id}) = F(q^{-1})(q^{-1}T_s+(q^{-1}-1)T_{id})
   + G(q^{-1})T_{id} \implies
   \begin{cases}
     F(q) = q^{-1}F(q^{-1}) \\
     G(q) = G(q^{-1})+F(q^{-1})(q^{-1}-1)
   \end{cases}
 \]
 and so the simplest solution would be \(F(q) = q^{-1}+1\) and
 \(G(q)=-q\). However, if we allow \(F(q),G(q) \in \Z[q^{\frac{1}{2}},
 q^{-\frac{1}{2}}]\), then we can have \(F(q) = q^{-\frac{1}{2}}\) and \(G(q) =
 -q^{\frac{1}{2}}\). So, let us redefine \(A\) to be the (larger) ring
 \(\Z[q^{\frac{1}{2}},q^{-\frac{1}{2}}]\). Then, we have such an \(\iota\)-invariant
 of the form \[
   C_s := q^{-\frac{1}{2}} T_s - q^{\frac{1}{2}} T_{id}
 \]
 In~\cite{humphreys}*{p 158}, it is noted that it could be tempting to
 construct further \(\iota\)-invariants by taking products of these
 \(C_s\) elements. However, if one has a word \(sts = tst\) with \(s,t
 \in \W\) both simple reflections and \(\ell(sts) = 3 = \ell(tst)\),
 then one can check that \(C_s C_t C_s \neq C_t C_s C_t\). However, if
 we compute (still assuming \(\ell(sts) = 3\)) \[
   C_s C_t C_s - C_t = q^{-\frac{3}{2}}(T_{sts}-q(T_{st}-T_{ts})+q^2(1+q^{-1})(T_s+T_t)-q^3(1+2q^{-1})T_{id})
 \]
 we get an \(\iota\)-invariant expression where the \(s\) and \(t\)'s
 are interchangeable.

 This illustrates the problem more generally we wish to solve. For
 every \(w \in \W\), we want to associate an \(\iota\)-invariant
 element, \(C_w\), which is a linear combination of \(T_x\) for \(x
 \leq w\), thus giving us a basis.
\begin{thm}
  \cite{humphreys}*{Theorem 7.9} For each \(w \in \W\), there exists a
  unique element \(C_w \in \H\) having the following properties:
  \begin{enumerate}
  \item \(\iota(C_w) = C_w\)
  \item \(C_w = \epsilon_w q_w^{\frac{1}{2}} \sum_{x \leq w} \epsilon_x
    q_x^{-1} \ov{P}_{x,w} T_x\) where \(P_{w,w} = 1\) and \(P_{x,w}(q)
    \in \Z[q]\) has degree less than
    \(\frac{1}{2}(\ell(w)-\ell(x)-1)\) if \(x < w\).
  \end{enumerate}
\end{thm}
Of course, it is easy to see that if \(s \in \W\) is a simple
reflection, then it must be that \[
  C_s = q^{-\frac{1}{2}}(T_s-qT_{id})
\]
Then, one may wish to construct 
\begin{example}
  
\end{example}
\begin{bibdiv}
  \begin{biblist}
    \bib{humphreys}{book}{
      author={Humphreys, James E.}
      title={Reflection Groups and Coxeter Groups}
      year={1990}
    }
    \bib{williamson}{book}{
      author={Williamson, Geordie}
      title={Mind your $P$ and $Q$-symbols: Why the Kazhdan-Lusztig
        basis of the Hecke algebra of type A is cellular}
      year={2003}
    }
  \end{biblist}
\end{bibdiv}
\end{document}
%%% Local Variables:
%%% mode: latex
%%% TeX-master: t
%%% End:
