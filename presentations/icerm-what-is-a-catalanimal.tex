%%%%%%%%%%%%%%%%%%%%%%%%%%%%%%%%%%%%%%%%%
% Beamer Presentation
% LaTeX Template
% Version 1.0 (10/11/12)
%
% This template has been downloaded from:
% http://www.LaTeXTemplates.com
%
% License:
% CC BY-NC-SA 3.0 (http://creativecommons.org/licenses/by-nc-sa/3.0/)
%
%%%%%%%%%%%%%%%%%%%%%%%%%%%%%%%%%%%%%%%%%

%----------------------------------------------------------------------------------------
%	PACKAGES AND THEMES
%----------------------------------------------------------------------------------------

\documentclass[dvipsnames]{beamer}
\mode<presentation> {

% The Beamer class comes with a number of default slide themes
% which change the colors and layouts of slides. Below this is a list
% of all the themes, uncomment each in turn to see what they look like.

%\usetheme{default}
%\usetheme{AnnArbor}
%\usetheme{Antibes}
%\usetheme{Bergen}
%\usetheme{Berkeley}
%\usetheme{Berlin}
%\usetheme{Boadilla}
%\usetheme{CambridgeUS}
%\usetheme{Copenhagen}
%\usetheme{Darmstadt}
%\usetheme{Dresden}
%\usetheme{Frankfurt}
%\usetheme{Goettingen}
%\usetheme{Hannover}
%\usetheme{Ilmenau}
%\usetheme{JuanLesPins}
%\usetheme{Luebeck}
\usetheme{Madrid}
%\usetheme{Malmoe}
%\usetheme{Marburg}
%\usetheme{Montpellier}
%\usetheme{PaloAlto}
%\usetheme{Pittsburgh}
%\usetheme{Rochester}
%\usetheme{Singapore}
%\usetheme{Szeged}
%\usetheme{Warsaw}

% As well as themes, the Beamer class has a number of color themes
% for any slide theme. Uncomment each of these in turn to see how it
% changes the colors of your current slide theme.

%\usecolortheme{albatross}
%\usecolortheme{beaver}
%\usecolortheme{beetle}
%\usecolortheme{crane}
%\usecolortheme{dolphin}
%\usecolortheme{dove}
%\usecolortheme{fly}
%\usecolortheme{lily}
%\usecolortheme{orchid}
%\usecolortheme{rose}
%\usecolortheme{seagull}
%\usecolortheme{seahorse}
%\usecolortheme{whale}
%\usecolortheme{wolverine}

%\setbeamertemplate{footline} % To remove the footer line in all slides uncomment this line
%\setbeamertemplate{footline}[page number] % To replace the footer line in all slides with a simple slide count uncomment this line

\setbeamertemplate{navigation symbols}{} % To remove the navigation symbols from the bottom of all slides uncomment this line
\setbeamertemplate{footline}{}
}

\usepackage{graphicx} % Allows including images
%\usepackage{booktabs} % Allows the use of \toprule, \midrule and
                      % \bottomrule in tables
\usepackage{tikz}
\usepackage{pgfplots}
\usepackage{tikz-cd}
\usepackage{bm}
\usepackage{amsmath}
\usepackage[author-year]{amsrefs}
\usepackage{amsthm}
\usepackage{../ReAdTeX/readtex-core}
% \usepackage{../ReAdTeX/readtex-dangerous}
% \usepackage{../ReAdTeX/readtex-abstract-algebra}
\usepackage{ytableau}
%%%%%%%%%%%%%%%%%%%%%%%%%%%%%%%%%%%%%%%%%%%%%%%%%%%%%%%%%%%%%%%%%%% 
%%  MACRO DEFINITIONS:  Co-authors -- PLEASE use these! 
%%%%%%%%%%%%%%%%%%%%%%%%%%%%%%%%%%%%%%%%%%%%%%%%%%%%%%%%%%%%%%%%%%%
\definecolor{coralred}{rgb}{1.0, 0.25, 0.25}
% \definecolor{lightblue}{rgb}{.68,.85,.9} % 
\definecolor{lightblue}{rgb}{.3,.65,1.0} %
\definecolor{asparagus}{rgb}{0.53, 0.66, 0.42}

\DeclareMathOperator{\Gr}{Gr}
\newcommand{\cupprod}{\smile}
\newcommand{\sym}{\Lambda}
\newcommand{\lowers}{\mathcal{L}}
\newcommand{\mynone}{\ }
\newcommand{\zz}{{\boldsymbol z}}
\newcommand{\xx}{{\boldsymbol x}}
\newcommand{\hbold}{{\boldsymbol h}}
\newcommand{\sbold}{{\boldsymbol s}}
\newcommand{\sigmabold}{\boldsymbol \sigma}
\newcommand{\mubold}{{\boldsymbol \mu }}
\newcommand{\Htild}{\tilde{H}}
\DeclareMathSymbol{\shortminus}{\mathbin}{AMSa}{"39}
\newcommand{\Gcal}{{\mathcal G}}
\newcommand{\nubold}{{\boldsymbol \nu }}
\renewcommand{\Span}{\operatorname{span}}

\DeclareMathOperator{\conv}{conv}
\DeclareMathOperator{\des}{des}
\DeclareMathOperator{\fin}{fin}
\DeclareMathOperator{\aff}{aff}
\DeclareMathOperator{\ext}{ext}
\DeclareMathOperator{\dinv}{dinv}
\DeclareMathOperator{\inv}{inv}
\DeclareMathOperator{\pol}{pol}
\DeclareMathOperator{\supp}{supp}
\DeclareMathOperator{\Ind}{Ind}
\DeclareMathOperator{\Inv}{Inv}
\DeclareMathOperator{\GL}{GL}
\DeclareMathOperator{\SYT}{SYT}
\DeclareMathOperator{\SSYT}{SSYT}
\DeclareMathOperator{\Ad}{Ad}
\DeclareMathOperator{\Stab}{Stab}
\newcommand{\sgn}{\text{\rm sgn}}
\DeclareMathOperator{\sort}{sort}
\newcommand{\south}{{\mathrm{south}}}
\newcommand{\leg}{{\mathrm{leg}}}
\newcommand{\arm}{{\mathrm{arm}}}
\newcommand{\bx}[2]{{\boldsymbol {#1}[#2]}}
\newcommand{\Ecal}{\mathcal{E}}
\newcommand{\kk}{\Bbbk}
\newcommand{\Scal}{\mathcal{S}}
\newcommand{\tGamma}{{\check{\Gamma}}} %for Laurent reps
\newcommand{\bbb}{{s}}
\DeclareMathOperator{\area}{area}
\newcommand{\Acal}{{\mathcal A}}
\DeclareMathOperator{\Aut}{Aut}
\newcommand{\F}{\mathbb{F}}
\newcommand{\aA}{{\mathbf a}}
\newcommand{\bb}{{\mathbf b}}
\renewcommand{\sl}{\mathfrak{sl}}

\newtheorem{thm}{Theorem}
\newtheorem{prop}{Proposition}
\theoremstyle{definition}
\newtheorem{rmk}[thm]{Remark}
\newtheorem*{rmk*}{Remark}
\newtheorem{conjecture}[theorem]{Conjecture}

\newcounter{boxnum}

\newcommand{\qtrootcolor}{blue!45}
\newcommand{\colorgr}[1]{\textcolor{gray}{#1}}
\newcommand{\colorgrgr}[1]{\textcolor{gray!70}{#1}}
\newcommand{\colorb}[1]{\textcolor{blue}{#1}}
\newcommand{\colordb}[1]{\textcolor{DarkBlue}{#1}}
\newcommand{\colorblack}[1]{\textcolor{black}{#1}}
\newcommand{\colorr}[1]{\textcolor{red}{#1}}
\newcommand{\colorg}[1]{\textcolor{ForestGreen}{#1}}
\newcommand{\colorgg}[1]{\textcolor{green}{#1}}
\newcommand{\colorp}[1]{\textcolor{purple}{#1}}


\newcommand{\drawskewdg}[2]{
  \def\ptn{#1}
  \def\offset{#2}
    \setcounter{rownum}{1}
    \foreach \xstart\xend in \ptn {
      \draw[thick,fill=white] (\xstart+\offset,\therownum-1+\offset)
      grid (\xend+\offset,\therownum+\offset) rectangle (\xstart+\offset,\therownum-1+\offset);
      \addtocounter{rownum}{1}
    }
}

\newcounter{rownum}

\newcommand{\drawDgNotThick}[3]{
      \setcounter{rownum}{0}
      \def\b{#1};
      \def\xshift{#2};
      \def\yshift{#3};
      \foreach \c in \b {
        \foreach \xx in \xshift {
           \foreach \yy in \yshift {
              \draw[shift={(\xx,\yy)}] (\therownum,0) grid (\therownum+1, \c);
              \addtocounter{rownum}{1};
           }
        }
      }
    }

\newcommand{\drawDg}[3]{
      \setcounter{rownum}{0}
      \def\b{#1};
      \def\xshift{#2};
      \def\yshift{#3};
      \foreach \c in \b {
        \foreach \xx in \xshift {
           \foreach \yy in \yshift {
              \draw[thick, shift={(\xx,\yy)}] (\therownum,0) grid (\therownum+1, \c);
              \addtocounter{rownum}{1};
           }
        }
      }
    }

\newcounter{c}
\newcounter{cp}
\tikzset{
    invisible/.style={opacity=0},
    visible on/.style={alt={#1{}{invisible}}},
    alt/.code args={<#1>#2#3}{%
      \alt<#1>{\pgfkeysalso{#2}}{\pgfkeysalso{#3}}%
  }
}

%%%%%%%%%%%%%%%%%%%%%%%%%%%%%%%%%%%%%%%%%%%%%%%%%%%%%%%%%%%%%%%%%%%% 


%----------------------------------------------------------------------------------------
%	TITLE PAGE
%----------------------------------------------------------------------------------------

\title[]{What is a Catalanimal?} % The short title appears at the bottom of every slide, the full title is only on the title page

\author[George H. Seelinger]{George H. Seelinger} % Your name
\institute[UMich] % Your institution as it will appear on the bottom of every slide, may be shorthand to save space
{
ghseeli@umich.edu\\ %Your email address
\medskip
ICERM: What is...? Seminar \\ % Your institution for the title page
\medskip
}
\date{24 September 2025} % Date, can be changed to a custom date

\begin{document}
\begin{frame}
\titlepage % Print the title page as the first slide
\end{frame}
\begin{frame}{Resources}
  \begin{itemize}
  \item My IPAC 2024 Lectures:
    \url{https://www2.math.upenn.edu/~jhaglund/IPAC/2024.html}
  \item Anna Pun’s 2021 Lectures: \url{https://www2.math.upenn.edu/~jhaglund/IPAC/2021.html}
  \end{itemize}
\end{frame}
\begin{frame}{Hall-Littlewood Examples}
  \begin{eqnarray*}
    & H_{(4,2)} & = \pol\left( \sigmabold\left( \frac{x_1^4 x_2^2}{1-t
                x_1/x_2} \right) \right)\\
    &&= \pol\left( \sigmabold \left( x_1^4 x_2^2 + t x_1^5 x_2 + t^2
      x_1^6 x_2^0 + t^3 x_1^7 x_2^{-1} + \cdots \right) \right) \\
    &&= \pol\left( \chi_{42} + t \chi_{51} + t^2 \chi_{60} + t^3
      \chi_{7,{\text{-}1}} + \cdots\right) \\
    &&= s_{42} + t s_{51} + t^2 s_{6} \,. \\
    && \\
    & H_{(4,2,1)} & = \pol\left( \sigmabold\left( \frac{x_1^4 x_2^2
                    x_3^1}{(1-t x_1/x_2)(1-t x_1/x_3)(1-t x_2/x_3)}
                    \right) \right) \\
    && \cdots \\
    && = s_{421} + ts_{43}+ts_{511}+(t^2+t)s_{52}+(t^3+t^2)s_{61}+t^4 s_7
  \end{eqnarray*}
\end{frame}
\begin{frame}{Root Ideals}
              \ytableausetup{mathmode, boxsize=1em, centertableaux}
            \[
              \begin{tikzpicture}[inner sep=0in, outer sep=0in]
                \node (n) {
                \begin{ytableau}
                  \mynone &*(\qtrootcolor)\text{\tiny (12)}
                  &*(\qtrootcolor)\text{\tiny (13)} &*(\qtrootcolor)\text{\tiny (14)}
                  &*(\qtrootcolor)
                  \text{\tiny (15)}\\
                  \mynone &\mynone &*(\qtrootcolor) \text{\tiny (23)}
                  &*(\qtrootcolor)\text{\tiny (24)}
                  &*(\qtrootcolor) \text{\tiny (25)}\\
                  \mynone &\mynone &\mynone &*(\qtrootcolor) \text{\tiny
                    (34)}
                  &*(\qtrootcolor)\text{\tiny (35)} \\
                  \mynone &\mynone &\mynone&\mynone&*(\qtrootcolor) \text{\tiny (45)}\\
                  \mynone &\mynone &\mynone&\mynone&\mynone\\
                \end{ytableau}};
              \end{tikzpicture}
              \hspace{3em}
              \begin{tikzpicture}[inner sep=0in, outer sep=0in]
                \node (n) {
                \begin{ytableau}
                  \mynone &\text{\tiny (12)}
                  &*(\qtrootcolor)\text{\tiny (13)} &*(\qtrootcolor)\text{\tiny (14)}
                  &*(\qtrootcolor)
                  \text{\tiny (15)}\\
                  \mynone &\mynone &*(\qtrootcolor) \text{\tiny (23)}
                  &*(\qtrootcolor)\text{\tiny (24)}
                  &*(\qtrootcolor) \text{\tiny (25)}\\
                  \mynone &\mynone &\mynone & \text{\tiny
                    (34)}
                  &*(\qtrootcolor)\text{\tiny (35)} \\
                  \mynone &\mynone &\mynone&\mynone& \text{\tiny (45)}\\
                  \mynone &\mynone &\mynone&\mynone&\mynone\\
                \end{ytableau}};
              \draw[very thick,orange] (n.north
              west)--([xshift=2.1em]n.north
              west)--++(0,-2.1em)--++(2.1em,0)--++(0,-1.05em)--++(1.05em,0)--++(0,-2.10em);
              \draw[thick,lightgray] (n.north west)--(n.south east);
              \end{tikzpicture}
          \]
\end{frame}
\begin{frame}{Catalan Function Examples}
  \begin{itemize}
  \item Parabolic Hall-Littlewood polynomials \(H_{\eta,\lambda} = H(R^+_\eta,\lambda)\)
    \\
    \[
      \textcolor{orange}{\eta=(1,2,3,3)} \hspace{4em} R^+_\eta = 
      \begin{tikzpicture}[scale=.3]
        \draw[draw = none, fill = \qtrootcolor] (2,-1) rectangle
        (3,-2); \draw[draw = none, fill = \qtrootcolor] (3,-1)
        rectangle (4,-2); \draw[draw = none, fill = \qtrootcolor]
        (4,-1) rectangle (5,-2); \draw[draw = none, fill =
        \qtrootcolor] (4,-2) rectangle (5,-3); \draw[draw = none, fill
        = \qtrootcolor] (4,-3) rectangle (5,-4); \draw[draw = none,
        fill = \qtrootcolor] (5,-1) rectangle (6,-2); \draw[draw =
        none, fill = \qtrootcolor] (5,-2) rectangle (6,-3); \draw[draw
        = none, fill = \qtrootcolor] (5,-3) rectangle (6,-4);
        \draw[draw = none, fill = \qtrootcolor] (6,-1) rectangle
        (7,-2); \draw[draw = none, fill = \qtrootcolor] (6,-2)
        rectangle (7,-3); \draw[draw = none, fill = \qtrootcolor]
        (6,-3) rectangle (7,-4); \draw[draw = none, fill =
        \qtrootcolor] (7,-1) rectangle (8,-2); \draw[draw = none, fill
        = \qtrootcolor] (7,-2) rectangle (8,-3); \draw[draw = none,
        fill = \qtrootcolor] (7,-3) rectangle (8,-4); \draw[draw =
        none, fill = \qtrootcolor] (7,-4) rectangle (8,-5); \draw[draw
        = none, fill = \qtrootcolor] (7,-5) rectangle (8,-6);
        \draw[draw = none, fill = \qtrootcolor] (7,-6) rectangle
        (8,-7); \draw[draw = none, fill = \qtrootcolor] (8,-1)
        rectangle (9,-2); \draw[draw = none, fill = \qtrootcolor]
        (8,-2) rectangle (9,-3); \draw[draw = none, fill =
        \qtrootcolor] (8,-3) rectangle (9,-4); \draw[draw = none, fill
        = \qtrootcolor] (8,-4) rectangle (9,-5); \draw[draw = none,
        fill = \qtrootcolor] (8,-5) rectangle (9,-6); \draw[draw =
        none, fill = \qtrootcolor] (8,-6) rectangle (9,-7); \draw[draw
        = none, fill = \qtrootcolor] (9,-1) rectangle (10,-2);
        \draw[draw = none, fill = \qtrootcolor] (9,-2) rectangle
        (10,-3); \draw[draw = none, fill = \qtrootcolor] (9,-3)
        rectangle (10,-4); \draw[draw = none, fill = \qtrootcolor]
        (9,-4) rectangle (10,-5); \draw[draw = none, fill =
        \qtrootcolor] (9,-5) rectangle (10,-6); \draw[draw = none,
        fill = \qtrootcolor] (9,-6) rectangle (10,-7); \draw[thin,
        black!31] (1,-1) -- (10,-1); \draw[thin, black!31] (2,-1) --
        (2,-1); \draw[thin, black!31] (2,-2) -- (10,-2); \draw[thin,
        black!31] (3,-2) -- (3,-1); \draw[thin, black!31] (3,-3) --
        (10,-3); \draw[thin, black!31] (4,-3) -- (4,-1); \draw[thin,
        black!31] (4,-4) -- (10,-4); \draw[thin, black!31] (5,-4) --
        (5,-1); \draw[thin, black!31] (5,-5) -- (10,-5); \draw[thin,
        black!31] (6,-5) -- (6,-1); \draw[thin, black!31] (6,-6) --
        (10,-6); \draw[thin, black!31] (7,-6) -- (7,-1); \draw[thin,
        black!31] (7,-7) -- (10,-7); \draw[thin, black!31] (8,-7) --
        (8,-1); \draw[thin, black!31] (8,-8) -- (10,-8); \draw[thin,
        black!31] (9,-8) -- (9,-1); \draw[thin, black!31] (9,-9) --
        (10,-9); \draw[thin, black!31] (10,-9) -- (10,-1);
        % \draw[draw = none, fill = black!100] (3,-2) rectangle
        % (4,-3); \draw[draw = none, fill = black!100] (5,-4)
        % rectangle (6,-5); \draw[draw = none, fill = black!100]
        % (6,-4) rectangle (7,-5); \draw[draw = none, fill =
        % black!100] (6,-5) rectangle (7,-6); \draw[draw = none, fill
        % = black!100] (8,-7) rectangle (9,-8); \draw[draw = none,
        % fill = black!100] (9,-7) rectangle (10,-8); \draw[draw =
        % none, fill = black!100] (9,-8) rectangle (10,-9);
        \draw[thin] (1,-1) -- (2,-1); \draw[thin] (2,-1) -- (2,-2);
        \draw[thin] (2,-2) -- (3,-2); \draw[thin] (3,-2) -- (3,-3);
        \draw[thin] (3,-3) -- (4,-3); \draw[thin] (4,-3) -- (4,-4);
        \draw[thin] (4,-4) -- (5,-4); \draw[thin] (5,-4) -- (5,-5);
        \draw[thin] (5,-5) -- (6,-5); \draw[thin] (6,-5) -- (6,-6);
        \draw[thin] (6,-6) -- (7,-6); \draw[thin] (7,-6) -- (7,-7);
        \draw[thin] (7,-7) -- (8,-7); \draw[thin] (8,-7) -- (8,-8);
        \draw[thin] (8,-8) -- (9,-8); \draw[thin] (9,-8) -- (9,-9);
        \draw[thin] (9,-9) -- (10,-9); \draw[thin] (10,-9) --
        (10,-10); \draw[thick, densely dotted, orange] (1,-1)
        rectangle (2,-2); \draw[thick, densely dotted, orange] (2,-2)
        rectangle (4,-4); \draw[thick, densely dotted, orange] (4,-4)
        rectangle (7,-7); \draw[thick, densely dotted, orange] (7,-7)
        rectangle (10,-10);
      \end{tikzpicture}
    \]
  \item \(k\)-Schur functions \(s^{(k)}_\lambda = H(\Delta^k(\lambda),\lambda)\)
    \[
      \Delta^4(432211111) =
      \begin{tikzpicture}[scale=.3]
\foreach \r/\c in {1/2,1/3,1/4,1/5,1/6,1/7,1/8,1/9,2/4,2/5,2/6,2/7,2/8,2/9,3/6,3/7,3/8,3/9,4/7,4/8,4/9,5/9} {
  \draw[draw = none, fill=\qtrootcolor] (\c,-\r) rectangle (\c+1,-\r-1);
}
         \draw[thin, black!31] (1,-1) -- (10,-1);
\draw[thin, black!31] (2,-1) -- (2,-1);
\draw[thin, black!31] (2,-2) -- (10,-2);
\draw[thin, black!31] (3,-2) -- (3,-1);
\draw[thin, black!31] (3,-3) -- (10,-3);
\draw[thin, black!31] (4,-3) -- (4,-1);
\draw[thin, black!31] (4,-4) -- (10,-4);
\draw[thin, black!31] (5,-4) -- (5,-1);
\draw[thin, black!31] (5,-5) -- (10,-5);
\draw[thin, black!31] (6,-5) -- (6,-1);
\draw[thin, black!31] (6,-6) -- (10,-6);
\draw[thin, black!31] (7,-6) -- (7,-1);
\draw[thin, black!31] (7,-7) -- (10,-7);
\draw[thin, black!31] (8,-7) -- (8,-1);
\draw[thin, black!31] (8,-8) -- (10,-8);
\draw[thin, black!31] (9,-8) -- (9,-1);
\draw[thin, black!31] (9,-9) -- (10,-9);
\draw[thin, black!31] (10,-9) -- (10,-1);
%\draw[draw = none, fill = black!100] (3,-2) rectangle (4,-3);
% \draw[draw = none, fill = black!100] (5,-4) rectangle (6,-5);
% \draw[draw = none, fill = black!100] (6,-4) rectangle (7,-5);
% \draw[draw = none, fill = black!100] (6,-5) rectangle (7,-6);
% \draw[draw = none, fill = black!100] (8,-7) rectangle (9,-8);
% \draw[draw = none, fill = black!100] (9,-7) rectangle (10,-8);
% \draw[draw = none, fill = black!100] (9,-8) rectangle (10,-9);
 \draw[thin] (1,-1) -- (2,-1);
\draw[thin] (2,-1) -- (2,-2);
\draw[thin] (2,-2) -- (3,-2);
\draw[thin] (3,-2) -- (3,-3);
\draw[thin] (3,-3) -- (4,-3);
\draw[thin] (4,-3) -- (4,-4);
\draw[thin] (4,-4) -- (5,-4);
\draw[thin] (5,-4) -- (5,-5);
\draw[thin] (5,-5) -- (6,-5);
\draw[thin] (6,-5) -- (6,-6);
\draw[thin] (6,-6) -- (7,-6);
\draw[thin] (7,-6) -- (7,-7);
\draw[thin] (7,-7) -- (8,-7);
\draw[thin] (8,-7) -- (8,-8);
\draw[thin] (8,-8) -- (9,-8);
\draw[thin] (9,-8) -- (9,-9);
\draw[thin] (9,-9) -- (10,-9);
\draw[thin] (10,-9) -- (10,-10);
\node at (1.5,-1.5) {\scriptsize $4$};
\node at (2.5,-2.5) {\scriptsize $3$};
\node at (3.5,-3.5) {\scriptsize $2$};
\node at (4.5,-4.5) {\scriptsize $2$};
\node at (5.5,-5.5) {\scriptsize $1$};
\node at (6.5,-6.5) {\scriptsize $1$};
\node at (7.5,-7.5) {\scriptsize $1$};
\node at (8.5,-8.5) {\scriptsize $1$};
\node at (9.5,-9.5) {\scriptsize $1$};
      \end{tikzpicture}
    \]
  \end{itemize}
\end{frame}
\begin{frame}{Catalanimal Example}

  {\small
    Special case: \(R_+ = R_q = R_t \supset R_{qt}\)
\begin{itemize}
\setlength{\itemsep}{-0.4mm}
\item[] \raisebox{-1mm}{
\begin{tikzpicture}[scale = .4]
\draw[thin, black!0] (2,-1) -- (3,-1);
\draw[thin, black!0] (2,-1) -- (2,-2);
\draw[thin, black!0] (2,-2) -- (3,-2);
\draw[thin, black!0] (3,-2) -- (3,-1);
\draw[draw = none, fill = gray!100] (2+0.5, -1-0.5) circle (.2);
\end{tikzpicture}} \
 $R_t \setminus R_{qt} $% pairs going between adjacent diagonals,
\item[]
\raisebox{-1mm}{
 \begin{tikzpicture}[scale = .4]
\draw[draw = none, fill = \qtrootcolor] (2,-1) rectangle (3,-2);
\end{tikzpicture}} \
 $R_{qt} $ %all other pairs,
\end{itemize}
}
\vspace{3ex}
    \hspace{6ex}\begin{tikzpicture}[scale=.47]
      \draw[draw = none, fill = \qtrootcolor] (3,-1) rectangle (4,-2);
      \draw[draw = none, fill = gray!100] (2+0.5, -1-0.5) circle (.2);
      \draw[draw = none, fill = gray!100] (3+0.5, -2-0.5) circle (.2);
      \draw[thin, black!31] (1,-1) -- (4,-1);
      \draw[thin, black!31] (2,-2) -- (4,-2);
      \draw[thin, black!31] (3,-3) -- (4,-3);
      \draw[thin, black!31] (4,-1) -- (4,-4);
      \draw[thin, black!31] (3,-1) -- (3,-3);
      \draw[thin, black!31] (2,-1) -- (2,-2);
      \draw[thin] (1,-1) -- (2,-1);
      \draw[thin] (2,-1) -- (2,-2);
      \draw[thin] (2,-2) -- (3,-2);
      \draw[thin] (3,-2) -- (3,-3);
      \draw[thin] (3,-3) -- (4,-3);
      \draw[thin] (4,-3) -- (4,-4);
    \end{tikzpicture}
  \begin{eqnarray*}
    &&H(R_+,R_+,\{\alpha_{13}\},111) \\
    &&= \pol \sigmabold\left( \frac{x_1x_2x_3 (1-qt x_1/x_3)}{\prod_{1
       \leq i < j \leq 3}(1-q x_i/x_j)(1-tx_i/x_j)} \right)\\
    &&= \cdots\\
    &&= s_{111} + (q^2+qt+t^2+q+t)s_{21}
       + (q^3+q^2t+qt^2+t^3+qt)s_3\\
    &&= \omega \nabla e_3
  \end{eqnarray*}
\end{frame}
\begin{frame}{Elliptic Hall Algebra \(\Ecal\)}
  \(\Lambda^{(m,n)} = \Lambda(X^{m,n}) \isom \Lambda\)
  \begin{center}
    \begin{tikzpicture}[scale=0.5]
      \draw[help lines, color=gray!30, dashed] (-4.9,-4.9) grid
      (4.9,4.9); \draw[->,thick] (0,0)--(0,5)
      node[above]{$\scriptstyle \Lambda^{(0,1)}$}; \draw[->,thick]
      (0,0)--(5/3,5) node[above
      ]{$\scriptstyle \phantom{++}\Lambda^{(1,3)}$}; \draw[->,thick]
      (0,0)--(10/3,5) node[above
      right]{$\scriptstyle \Lambda^{(2,3)}$}; \draw[->,thick]
      (0,0)--(5,5) node[right]{$\scriptstyle \Lambda^{(1,1)}$};
      \draw[->,thick] (0,0)--(5,10/3)
      node[right]{$\scriptstyle \Lambda^{(3,2)}$};
      % \draw[->,thick] (0,0)--(5,2.5)
      % node[right]{$\scriptstyle \Lambda^{(2,1)}$};
      \draw[->,thick] (0,0)--(5,5/3)
      node[right]{$\scriptstyle \Lambda^{(3,1)}$}; \draw[->,thick]
      (0,0)--(5,0) node[right]{$\scriptstyle \Lambda^{(1,0)}$};
      \draw[->,thick] (0,0)--(-5,0)
      node[above]{\(\scriptstyle \Lambda^{(-1,0)}\)}; \draw[->,thick]
      (0,0)--(0,-5) node[right]{\(\scriptstyle \Lambda^{(0,-1)}\)};
    \end{tikzpicture}
  \end{center}
\end{frame}
\begin{frame}{Negut Catalanimal}

   {\small
    Special case: \(R_+ = R_q = R_t \supset R_{qt}\)
\begin{itemize}
\setlength{\itemsep}{-0.4mm}
\item[] \raisebox{-1mm}{
\begin{tikzpicture}[scale = .4]
\draw[thin, black!0] (2,-1) -- (3,-1);
\draw[thin, black!0] (2,-1) -- (2,-2);
\draw[thin, black!0] (2,-2) -- (3,-2);
\draw[thin, black!0] (3,-2) -- (3,-1);
\draw[draw = none, fill = gray!100] (2+0.5, -1-0.5) circle (.2);
\end{tikzpicture}} \
 $R_t \setminus R_{qt} $% pairs going between adjacent diagonals,
\item[]
\raisebox{-1mm}{
 \begin{tikzpicture}[scale = .4]
\draw[draw = none, fill = \qtrootcolor] (2,-1) rectangle (3,-2);
\end{tikzpicture}} \
 $R_{qt} $ %all other pairs,
\end{itemize}
}
\vspace{3ex}
    \hspace{6ex}\begin{tikzpicture}[scale=.47]
      \foreach \i in {1,...,4} {
        \foreach \j in {\i,...,4} {
          \draw[draw = none, fill = \qtrootcolor] (\j+2,-\i) rectangle (\j+3,-\i-1);
        }
      }
      \foreach \i in {1,...,5} {
        \draw[draw = none, fill = gray!100] (\i+1.5, -\i-0.5) circle (.2);
      }
      \foreach \x in {1,...,7} {
        \draw[thin, black!31] (\x,-\x) -- (7,-\x);
      }
      \foreach \y in {1,...,7} {
        \draw[thin, black!31] (\y,-1) -- (\y,-\y);
      }
      \foreach \n in {1,...,6} {
        \draw[thin] (\n,-\n) -- (\n+1,-\n);
        \draw[thin] (\n+1,-\n) -- (\n+1,-\n-1);
      }
      \foreach \i in {1,...,6} {
        \node at (\i+0.5,-\i-0.5) {\small $\gamma_\i$};
      }
      % \draw[thin] (1,-1) -- (2,-1);
      % \draw[thin] (2,-1) -- (2,-2);
      % \draw[thin] (2,-2) -- (3,-2);
      % \draw[thin] (3,-2) -- (3,-3);
      % \draw[thin] (3,-3) -- (4,-3);
      % \draw[thin] (4,-3) -- (4,-4);
    \end{tikzpicture}
    \[
      H(R_+,R_+,[R_+,R_+],\gamma) = \pol \sigmabold\left( \frac{x^\gamma \prod_{i+1 < j}(1-qtx_i/x_j)}{\prod_{i<j}(1-qx_i/x_j)(q-tx_i/x_j)} \right)
    \]
\end{frame}
\begin{frame}{Some \((1,1)\)-Cuddly Catalanimals}

  {\small
    Special case: \(R_+ \supset R_q \supset R_t \supset R_{qt}\)
\begin{itemize}
\setlength{\itemsep}{-0.4mm}
\item[] \raisebox{-1mm}{
 \begin{tikzpicture}[scale = .4]
\draw[draw = none, fill = black!100] (2,-1) rectangle (3,-2);
\end{tikzpicture}} \
$R_+ \setminus R_q $ %pairs of boxes in the same diagonal in the same shape,
\item[] \raisebox{-1mm}{
 \begin{tikzpicture}[scale = .4]
\draw[draw = none, fill = red!68] (2,-1) rectangle (3,-2);
\end{tikzpicture}} \
$R_q \setminus R_t $ %the attacking pairs,
\item[] \raisebox{-1mm}{
\begin{tikzpicture}[scale = .4]
\draw[thin, black!0] (2,-1) -- (3,-1);
\draw[thin, black!0] (2,-1) -- (2,-2);
\draw[thin, black!0] (2,-2) -- (3,-2);
\draw[thin, black!0] (3,-2) -- (3,-1);
\draw[draw = none, fill = gray!100] (2+0.5, -1-0.5) circle (.2);
\end{tikzpicture}} \
 $R_t \setminus R_{qt} $% pairs going between adjacent diagonals,
\item[]
\raisebox{-1mm}{
 \begin{tikzpicture}[scale = .4]
\draw[draw = none, fill = \qtrootcolor] (2,-1) rectangle (3,-2);
\end{tikzpicture}} \
 $R_{qt} $ %all other pairs,
\end{itemize}
}
\vspace{-5mm}

\begin{align*}
\hspace{-1cm}
\begin{tikzpicture}[scale=.47]
       \begin{scope}
         \draw[draw = none, fill = \qtrootcolor] (4,-1) rectangle (5,-2);
         \draw[draw = none, fill = \qtrootcolor] (5,-1) rectangle (6,-2);
         \draw[draw = none, fill = \qtrootcolor] (6,-1) rectangle (7,-2);
         \draw[draw = none, fill = \qtrootcolor] (7,-1) rectangle (8,-2);
         \draw[draw = none, fill = \qtrootcolor] (7,-2) rectangle (8,-3);
         \draw[draw = none, fill = \qtrootcolor] (7,-3) rectangle (8,-4);
         \draw[draw = none, fill = \qtrootcolor] (8,-1) rectangle (9,-2);
         \draw[draw = none, fill = \qtrootcolor] (8,-2) rectangle (9,-3);
         \draw[draw = none, fill = \qtrootcolor] (8,-3) rectangle (9,-4);
         \draw[draw = none, fill = \qtrootcolor] (9,-1) rectangle (10,-2);
         \draw[draw = none, fill = \qtrootcolor] (9,-2) rectangle (10,-3);
         \draw[draw = none, fill = \qtrootcolor] (9,-3) rectangle (10,-4);
         \draw[draw = none, fill = \qtrootcolor] (9,-4) rectangle (10,-5);
         \draw[draw = none, fill = \qtrootcolor] (9,-5) rectangle (10,-6);
         \draw[draw = none, fill = \qtrootcolor] (9,-6) rectangle (10,-7);
         \draw[draw = none, fill = \qtrootcolor] (10,-1) rectangle
         (11,-2); \draw[draw = none, fill = \qtrootcolor] (10,-2)
         rectangle (11,-3); \draw[draw = none, fill = \qtrootcolor]
         (10,-3) rectangle (11,-4); \draw[draw = none, fill =
         \qtrootcolor] (10,-4) rectangle (11,-5); \draw[draw = none, fill
         = \qtrootcolor] (10,-5) rectangle (11,-6); \draw[draw = none,
         fill = \qtrootcolor] (10,-6) rectangle (11,-7); \draw[draw =
         none, fill = \qtrootcolor] (10,-7) rectangle (11,-8); \draw[draw
         = none, fill = \qtrootcolor] (10,-8) rectangle (11,-9);
         \draw[draw = none, fill = gray!100] (2+0.5, -1-0.5) circle
         (.2); \draw[draw = none, fill = gray!100] (3+0.5, -1-0.5)
         circle (.2); \draw[draw = none, fill = gray!100] (4+0.5,
         -2-0.5) circle (.2); \draw[draw = none, fill = gray!100]
         (5+0.5, -2-0.5) circle (.2); \draw[draw = none, fill =
         gray!100] (6+0.5, -2-0.5) circle (.2); \draw[draw = none,
         fill = gray!100] (4+0.5, -3-0.5) circle (.2); \draw[draw =
         none, fill = gray!100] (5+0.5, -3-0.5) circle (.2);
         \draw[draw = none, fill = gray!100] (6+0.5, -3-0.5) circle
         (.2); \draw[draw = none, fill = gray!100] (7+0.5, -4-0.5)
         circle (.2); \draw[draw = none, fill = gray!100] (8+0.5,
         -4-0.5) circle (.2); \draw[draw = none, fill = gray!100]
         (7+0.5, -5-0.5) circle (.2); \draw[draw = none, fill =
         gray!100] (8+0.5, -5-0.5) circle (.2); \draw[draw = none,
         fill = gray!100] (7+0.5, -6-0.5) circle (.2); \draw[draw =
         none, fill = gray!100] (8+0.5, -6-0.5) circle (.2);
         \draw[draw = none, fill = gray!100] (9+0.5, -7-0.5) circle
         (.2); \draw[draw = none, fill = gray!100] (9+0.5, -8-0.5)
         circle (.2); \draw[draw = none, fill = gray!100] (10+0.5,
         -9-0.5) circle (.2); \draw[thin, black!31] (1,-1) -- (11,-1);
         \draw[thin, black!31] (2,-1) -- (2,-1); \draw[thin, black!31]
         (2,-2) -- (11,-2); \draw[thin, black!31] (3,-2) -- (3,-1);
         \draw[thin, black!31] (3,-3) -- (11,-3); \draw[thin,
         black!31] (4,-3) -- (4,-1); \draw[thin, black!31] (4,-4) --
         (11,-4); \draw[thin, black!31] (5,-4) -- (5,-1); \draw[thin,
         black!31] (5,-5) -- (11,-5); \draw[thin, black!31] (6,-5) --
         (6,-1); \draw[thin, black!31] (6,-6) -- (11,-6); \draw[thin,
         black!31] (7,-6) -- (7,-1); \draw[thin, black!31] (7,-7) --
         (11,-7); \draw[thin, black!31] (8,-7) -- (8,-1); \draw[thin,
         black!31] (8,-8) -- (11,-8); \draw[thin, black!31] (9,-8) --
         (9,-1); \draw[thin, black!31] (9,-9) -- (11,-9); \draw[thin,
         black!31] (10,-9) -- (10,-1); \draw[thin, black!31] (10,-10)
         -- (11,-10); \draw[thin, black!31] (11,-10) -- (11,-1);
         \draw[draw = none, fill = black!100] (3,-2) rectangle (4,-3);
         \draw[draw = none, fill = black!100] (5,-4) rectangle (6,-5);
         \draw[draw = none, fill = black!100] (6,-4) rectangle (7,-5);
         \draw[draw = none, fill = black!100] (6,-5) rectangle (7,-6);
         \draw[draw = none, fill = black!100] (8,-7) rectangle (9,-8);
         \draw[thin] (1,-1) -- (2,-1); \draw[thin] (2,-1) -- (2,-2);
         \draw[thin] (2,-2) -- (3,-2); \draw[thin] (3,-2) -- (3,-3);
         \draw[thin] (3,-3) -- (4,-3); \draw[thin] (4,-3) -- (4,-4);
         \draw[thin] (4,-4) -- (5,-4); \draw[thin] (5,-4) -- (5,-5);
         \draw[thin] (5,-5) -- (6,-5); \draw[thin] (6,-5) -- (6,-6);
         \draw[thin] (6,-6) -- (7,-6); \draw[thin] (7,-6) -- (7,-7);
         \draw[thin] (7,-7) -- (8,-7); \draw[thin] (8,-7) -- (8,-8);
         \draw[thin] (8,-8) -- (9,-8); \draw[thin] (9,-8) -- (9,-9);
         \draw[thin] (9,-9) -- (10,-9); \draw[thin] (10,-9) --
         (10,-10); \draw[thin] (10,-10) -- (11,-10); \draw[thin]
         (11,-10) -- (11,-11); \node at (3/2,-3/2) {\small $2 $}; \node
         at (5/2,-5/2) {\small $2 $}; \node at (7/2,-7/2) {\small $2 $};
         \node at (9/2,-9/2) {\small $1 $}; \node at (11/2,-11/2)
         {\small $1 $}; \node at (13/2,-13/2) {\small $1 $}; \node at
         (15/2,-15/2) {\small $0 $}; \node at (17/2,-17/2) {\small
           $0 $}; \node (vv) at (19/2,-19/2) {\small $1 $}; \node at
         (21/2,-21/2) {\small $0 $};
       \end{scope}
  \end{tikzpicture}
\ \ \qquad
\begin{tikzpicture}[scale = .47]
\draw[draw = none, fill = \qtrootcolor] (3,-1) rectangle (4,-2);
 \draw[draw = none, fill = \qtrootcolor] (4,-1) rectangle (5,-2);
 \draw[draw = none, fill = \qtrootcolor] (5,-1) rectangle (6,-2);
 \draw[draw = none, fill = \qtrootcolor] (5,-2) rectangle (6,-3);
 \draw[draw = none, fill = \qtrootcolor] (6,-1) rectangle (7,-2);
 \draw[draw = none, fill = \qtrootcolor] (6,-2) rectangle (7,-3);
 \draw[draw = none, fill = \qtrootcolor] (7,-1) rectangle (8,-2);
 \draw[draw = none, fill = \qtrootcolor] (7,-2) rectangle (8,-3);
 \draw[draw = none, fill = \qtrootcolor] (7,-3) rectangle (8,-4);
 \draw[draw = none, fill = \qtrootcolor] (8,-1) rectangle (9,-2);
 \draw[draw = none, fill = \qtrootcolor] (8,-2) rectangle (9,-3);
 \draw[draw = none, fill = \qtrootcolor] (8,-3) rectangle (9,-4);
 \draw[draw = none, fill = \qtrootcolor] (8,-4) rectangle (9,-5);
 \draw[draw = none, fill = gray!100] (2+0.5, -1-0.5) circle (.2);
\draw[draw = none, fill = gray!100] (4+0.5, -2-0.5) circle (.2);
\draw[draw = none, fill = gray!100] (5+0.5, -3-0.5) circle (.2);
\draw[draw = none, fill = gray!100] (6+0.5, -3-0.5) circle (.2);
\draw[draw = none, fill = gray!100] (7+0.5, -4-0.5) circle (.2);
\draw[draw = none, fill = gray!100] (8+0.5, -5-0.5) circle (.2);
\draw[draw = none, fill = gray!100] (8+0.5, -6-0.5) circle (.2);
\draw[draw = none, fill = red!68] (3,-2) rectangle (4,-3);
 \draw[draw = none, fill = red!68] (4,-3) rectangle (5,-4);
 \draw[draw = none, fill = red!68] (5,-4) rectangle (6,-5);
 \draw[draw = none, fill = red!68] (6,-4) rectangle (7,-5);
 \draw[draw = none, fill = red!68] (7,-5) rectangle (8,-6);
 \draw[draw = none, fill = red!68] (7,-6) rectangle (8,-7);
 \draw[draw = none, fill = red!68] (8,-7) rectangle (9,-8);
 \draw[thin, black!31] (1,-1) -- (9,-1);
\draw[thin, black!31] (2,-1) -- (2,-1);
\draw[thin, black!31] (2,-2) -- (9,-2);
\draw[thin, black!31] (3,-2) -- (3,-1);
\draw[thin, black!31] (3,-3) -- (9,-3);
\draw[thin, black!31] (4,-3) -- (4,-1);
\draw[thin, black!31] (4,-4) -- (9,-4);
\draw[thin, black!31] (5,-4) -- (5,-1);
\draw[thin, black!31] (5,-5) -- (9,-5);
\draw[thin, black!31] (6,-5) -- (6,-1);
\draw[thin, black!31] (6,-6) -- (9,-6);
\draw[thin, black!31] (7,-6) -- (7,-1);
\draw[thin, black!31] (7,-7) -- (9,-7);
\draw[thin, black!31] (8,-7) -- (8,-1);
\draw[thin, black!31] (8,-8) -- (9,-8);
\draw[thin, black!31] (9,-8) -- (9,-1);
\draw[draw = none, fill = black!100] (6,-5) rectangle (7,-6);
 \draw[thin] (1,-1) -- (2,-1);
\draw[thin] (2,-1) -- (2,-2);
\draw[thin] (2,-2) -- (3,-2);
\draw[thin] (3,-2) -- (3,-3);
\draw[thin] (3,-3) -- (4,-3);
\draw[thin] (4,-3) -- (4,-4);
\draw[thin] (4,-4) -- (5,-4);
\draw[thin] (5,-4) -- (5,-5);
\draw[thin] (5,-5) -- (6,-5);
\draw[thin] (6,-5) -- (6,-6);
\draw[thin] (6,-6) -- (7,-6);
\draw[thin] (7,-6) -- (7,-7);
\draw[thin] (7,-7) -- (8,-7);
\draw[thin] (8,-7) -- (8,-8);
\draw[thin] (8,-8) -- (9,-8);
\draw[thin] (9,-8) -- (9,-9);
\node at (3/2,-3/2) {\small ${2} $};
\node at (5/2,-5/2) {\small ${0} $};
\node at (7/2,-7/2) {\small ${2} $};
\node at (9/2,-9/2) {\small ${2} $};
\node at (11/2,-11/2) {\small ${1} $};
\node at (13/2,-13/2) {\small ${1} $};
\node at (15/2,-15/2) {\small ${0} $};
\node at (17/2,-17/2) {\small ${0} $};
\end{tikzpicture}
\end{align*}
\end{frame}
\begin{frame}[fragile]{Catalanimals and shuffle theorems}
    \begin{center}
    \begin{tikzcd}[column sep=tiny]
      \Ecal^+ \ar[r,"\sim"] \ar[d] & \mathcal{S} \ar[d]\\
      \Ecal^+ \curvearrowright 1 \in \Lambda
      \ar[r,equals]\ar[d,equals]&\parbox[l]{20ex}{Tame Catalanimal
        (explicit description)} \ar[d,dashed,"\text{Cauchy formula}"]\\
      \text{Algebraic quantity}
      \ar[r,equals] & 
      \text{Combinatorial quantity}
    \end{tikzcd}
  \end{center}
\end{frame}
\begin{frame}[fragile]{Mystery}
  \begin{center}
    \begin{tikzpicture}[xscale = .7,yscale = .7]
      \draw[step=1cm,gray!20,very thin] (0,0) grid (11,6);
      \draw[thick] (0,5.45) to[out=-70,in=175] (11.4,0); \draw[very
      thick]
      (0,5)--(0,3)--(1,3)--(1,2)--(2,2)--(2,1)--(4,1)--(4,0)--(11,0);
      \draw[fill = black] (0,5.45) circle (0.15) node[left]
      {\((0,s)\)}; \draw[fill = black] (11.4,0) circle (0.15)
      node[below] {\((r,0)\)};
    \end{tikzpicture}
  \end{center}
\end{frame}
\end{document}
%%% Local Variables:
%%% mode: latex
%%% TeX-master: t
%%% End:
