%%%%%%%%%%%%%%%%%%%%%%%%%%%%%%%%%%%%%%%%%
% Beamer Presentation
% LaTeX Template
% Version 1.0 (10/11/12)
%
% This template has been downloaded from:
% http://www.LaTeXTemplates.com
%
% License:
% CC BY-NC-SA 3.0 (http://creativecommons.org/licenses/by-nc-sa/3.0/)
%
%%%%%%%%%%%%%%%%%%%%%%%%%%%%%%%%%%%%%%%%%

%----------------------------------------------------------------------------------------
%	PACKAGES AND THEMES
%----------------------------------------------------------------------------------------

\documentclass[dvipsnames]{beamer}
\mode<presentation> {

% The Beamer class comes with a number of default slide themes
% which change the colors and layouts of slides. Below this is a list
% of all the themes, uncomment each in turn to see what they look like.

%\usetheme{default}
%\usetheme{AnnArbor}
%\usetheme{Antibes}
%\usetheme{Bergen}
%\usetheme{Berkeley}
%\usetheme{Berlin}
%\usetheme{Boadilla}
%\usetheme{CambridgeUS}
%\usetheme{Copenhagen}
%\usetheme{Darmstadt}
%\usetheme{Dresden}
%\usetheme{Frankfurt}
%\usetheme{Goettingen}
%\usetheme{Hannover}
%\usetheme{Ilmenau}
%\usetheme{JuanLesPins}
%\usetheme{Luebeck}
\usetheme{Madrid}
%\usetheme{Malmoe}
%\usetheme{Marburg}
%\usetheme{Montpellier}
%\usetheme{PaloAlto}
%\usetheme{Pittsburgh}
%\usetheme{Rochester}
%\usetheme{Singapore}
%\usetheme{Szeged}
%\usetheme{Warsaw}

% As well as themes, the Beamer class has a number of color themes
% for any slide theme. Uncomment each of these in turn to see how it
% changes the colors of your current slide theme.

%\usecolortheme{albatross}
%\usecolortheme{beaver}
%\usecolortheme{beetle}
%\usecolortheme{crane}
%\usecolortheme{dolphin}
%\usecolortheme{dove}
%\usecolortheme{fly}
%\usecolortheme{lily}
%\usecolortheme{orchid}
%\usecolortheme{rose}
%\usecolortheme{seagull}
%\usecolortheme{seahorse}
%\usecolortheme{whale}
%\usecolortheme{wolverine}

%\setbeamertemplate{footline} % To remove the footer line in all slides uncomment this line
%\setbeamertemplate{footline}[page number] % To replace the footer line in all slides with a simple slide count uncomment this line

\setbeamertemplate{navigation symbols}{} % To remove the navigation symbols from the bottom of all slides uncomment this line
\setbeamertemplate{footline}{}
}

\usepackage{graphicx} % Allows including images
%\usepackage{booktabs} % Allows the use of \toprule, \midrule and
                      % \bottomrule in tables
\usepackage{tikz}
\usetikzlibrary{decorations.pathmorphing}
\usetikzlibrary{shapes,arrows,backgrounds,fit,positioning}
\usepackage{pgfplots}
\usepackage{tikz-cd}
\usepackage{bm}
\usepackage{amsmath}
\usepackage[author-year]{amsrefs}
\usepackage{amsthm}
\usepackage{../ReAdTeX/readtex-core}
% \usepackage{../ReAdTeX/readtex-dangerous}
% \usepackage{../ReAdTeX/readtex-abstract-algebra}
\usepackage{ytableau}
%%%%%%%%%%%%%%%%%%%%%%%%%%%%%%%%%%%%%%%%%%%%%%%%%%%%%%%%%%%%%%%%%%% 
%%  MACRO DEFINITIONS:  Co-authors -- PLEASE use these! 
%%%%%%%%%%%%%%%%%%%%%%%%%%%%%%%%%%%%%%%%%%%%%%%%%%%%%%%%%%%%%%%%%%%
\definecolor{coralred}{rgb}{1.0, 0.25, 0.25}
% \definecolor{lightblue}{rgb}{.68,.85,.9} % 
\definecolor{lightblue}{rgb}{.3,.65,1.0} %
\definecolor{asparagus}{rgb}{0.53, 0.66, 0.42}

\DeclareMathOperator{\Gr}{Gr}
\newcommand{\cupprod}{\smile}
\newcommand{\sym}{\Lambda}
\newcommand{\lowers}{\mathcal{L}}
\newcommand{\mynone}{\ }
\newcommand{\zz}{{\boldsymbol z}}
\newcommand{\xx}{{\boldsymbol x}}
\newcommand{\hbold}{{\boldsymbol h}}
\newcommand{\sbold}{{\boldsymbol s}}
\newcommand{\sigmabold}{\boldsymbol \sigma}
\newcommand{\mubold}{{\boldsymbol \mu }}
\newcommand{\Htild}{\tilde{H}}
\DeclareMathSymbol{\shortminus}{\mathbin}{AMSa}{"39}
\newcommand{\Gcal}{{\mathcal G}}
\newcommand{\nubold}{{\boldsymbol \nu }}
\renewcommand{\Span}{\operatorname{span}}

\DeclareMathOperator{\conv}{conv}
\DeclareMathOperator{\des}{des}
\DeclareMathOperator{\fin}{fin}
\DeclareMathOperator{\aff}{aff}
\DeclareMathOperator{\ext}{ext}
\DeclareMathOperator{\dinv}{dinv}
\DeclareMathOperator{\inv}{inv}
\DeclareMathOperator{\pol}{pol}
\DeclareMathOperator{\supp}{supp}
\DeclareMathOperator{\Ind}{Ind}
\DeclareMathOperator{\Inv}{Inv}
\DeclareMathOperator{\GL}{GL}
\DeclareMathOperator{\SYT}{SYT}
\DeclareMathOperator{\SSYT}{SSYT}
\DeclareMathOperator{\FT}{FT}
\DeclareMathOperator{\Ad}{Ad}
\DeclareMathOperator{\Stab}{Stab}
\newcommand{\sgn}{\text{\rm sgn}}
\DeclareMathOperator{\sort}{sort}
\newcommand{\south}{{\mathrm{south}}}
\newcommand{\leg}{{\mathrm{leg}}}
\newcommand{\arm}{{\mathrm{arm}}}
\newcommand{\bx}[2]{{\boldsymbol {#1}[#2]}}
\newcommand{\Ecal}{\mathcal{E}}
\newcommand{\kk}{\Bbbk}
\newcommand{\Scal}{\mathcal{S}}
\newcommand{\tGamma}{{\check{\Gamma}}} %for Laurent reps
\newcommand{\bbb}{{s}}
\DeclareMathOperator{\area}{area}
\newcommand{\Acal}{{\mathcal A}}
\newcommand{\Dcal}{{\mathcal D}}
\DeclareMathOperator{\Aut}{Aut}
\newcommand{\F}{\mathbb{F}}
\newcommand{\aA}{{\mathbf a}}
\newcommand{\bb}{{\mathbf b}}
\renewcommand{\sl}{\mathfrak{sl}}

\newtheorem{thm}{Theorem}
\newtheorem{prop}{Proposition}
\theoremstyle{definition}
\newtheorem{rmk}[thm]{Remark}
\newtheorem*{rmk*}{Remark}
\newtheorem{conjecture}[theorem]{Conjecture}

\newcounter{boxnum}

\newcommand{\qtrootcolor}{blue!45}
\newcommand{\colorgr}[1]{\textcolor{gray}{#1}}
\newcommand{\colorgrgr}[1]{\textcolor{gray!70}{#1}}
\newcommand{\colorb}[1]{\textcolor{blue}{#1}}
\newcommand{\colordb}[1]{\textcolor{DarkBlue}{#1}}
\newcommand{\colorblack}[1]{\textcolor{black}{#1}}
\newcommand{\colorr}[1]{\textcolor{red}{#1}}
\newcommand{\colorg}[1]{\textcolor{ForestGreen}{#1}}
\newcommand{\colorgg}[1]{\textcolor{green}{#1}}
\newcommand{\colorp}[1]{\textcolor{purple}{#1}}


\newcommand{\drawskewdg}[2]{
  \def\ptn{#1}
  \def\offset{#2}
    \setcounter{rownum}{1}
    \foreach \xstart\xend in \ptn {
      \draw[thick,fill=white] (\xstart+\offset,\therownum-1+\offset)
      grid (\xend+\offset,\therownum+\offset) rectangle (\xstart+\offset,\therownum-1+\offset);
      \addtocounter{rownum}{1}
    }
}

\newcounter{rownum}

\newcommand{\drawDgNotThick}[3]{
      \setcounter{rownum}{0}
      \def\b{#1};
      \def\xshift{#2};
      \def\yshift{#3};
      \foreach \c in \b {
        \foreach \xx in \xshift {
           \foreach \yy in \yshift {
              \draw[shift={(\xx,\yy)}] (\therownum,0) grid (\therownum+1, \c);
              \addtocounter{rownum}{1};
           }
        }
      }
    }

\newcommand{\drawDg}[3]{
      \setcounter{rownum}{0}
      \def\b{#1};
      \def\xshift{#2};
      \def\yshift{#3};
      \foreach \c in \b {
        \foreach \xx in \xshift {
           \foreach \yy in \yshift {
              \draw[thick, shift={(\xx,\yy)}] (\therownum,0) grid (\therownum+1, \c);
              \addtocounter{rownum}{1};
           }
        }
      }
    }

\newcounter{c}
\newcounter{cp}
\tikzset{
    invisible/.style={opacity=0},
    visible on/.style={alt={#1{}{invisible}}},
    alt/.code args={<#1>#2#3}{%
      \alt<#1>{\pgfkeysalso{#2}}{\pgfkeysalso{#3}}%
  }
}

%%%%%%%%%%%%%%%%%%%%%%%%%%%%%%%%%%%%%%%%%%%%%%%%%%%%%%%%%%%%%%%%%%%% 


%----------------------------------------------------------------------------------------
%	TITLE PAGE
%----------------------------------------------------------------------------------------

\title[]{Flagged LLT Polynomials} % The short title appears at the bottom of every slide, the full title is only on the title page

\author[George H. Seelinger]{George H. Seelinger} % Your name
\institute[UMich] % Your institution as it will appear on the bottom of every slide, may be shorthand to save space
{
ghseeli@umich.edu\\ %Your email address
\medskip
ICERM ECR Seminar \\ % Your institution for the title page
\medskip
joint work with Jonah Blasiak, Mark Haiman, Jennifer Morse, and Anna Pun
}
\date{18 November 2025} % Date, can be changed to a custom date

\begin{document}
\begin{frame}
\titlepage % Print the title page as the first slide
\end{frame}
\begin{frame}{Young Tableaux}
  \begin{definition}
    Filling of partition diagram of \(\lambda\) with numbers such that
    \begin{enumerate}
    \item strictly increasing up columns
    \item weakly increasing along rows
    \end{enumerate}
    Collection is called \(\SSYT(\lambda)\). 
  \end{definition}
  \begin{center}
    \begin{tikzpicture}
      \ytableausetup{boxsize=1em}
        \node at (0,0) { \(\ydiagram{1,1,3,4}\)};
        \node at (-.4,-.6) {\(\scriptstyle \leq\)};
        \node at (0,-.6) {\(\scriptstyle \leq\)};
        \node at (.4,-.6) {\(\scriptstyle \leq\)};
        \node at (.2,-.4) {\(\scriptstyle \vee\)};
        \node at (-.2,-.4) {\(\scriptstyle \vee\)};
        \node at (-.6,-.4) {\(\scriptstyle \vee\)};
        \node at (-.4,-.2) {\(\scriptstyle \leq\)};
        \node at (0,-.2) {\(\scriptstyle \leq\)};
        \node at (-.6,0) {\(\scriptstyle \vee\)};
        \node at (-.6,0.4) {\(\scriptstyle \vee\)};
      \end{tikzpicture}
  \end{center}
  For \(\lambda = (2,1)\), 
  \ytableausetup{aligntableaux=bottom,boxsize=1em}
\[
  \ytableaushort{2,11},\  \ytableaushort{3,11},\ \ytableaushort{3,22},\
    \ytableaushort{2,12},\ \ytableaushort{3,13},\ \ytableaushort{3,23},\
    \ytableaushort{2,13},\ \ytableaushort{3,12}
\]
\end{frame}
\begin{frame}{Schur polynomials}
  Associate a polynomial to \(\SSYT(\lambda)\).
 \[
   \quad \quad \quad \quad \quad \quad \quad \quad
  \ytableaushort{2,11},\  \ytableaushort{3,11},\ \ytableaushort{3,22},\
    \ytableaushort{2,12},\ \ytableaushort{3,13},\ \ytableaushort{3,23},\
    \ytableaushort{2,13},\ \ytableaushort{3,12}
  \]
  \vspace{-0.75em}
 \[
   \quad \quad \quad \quad \quad \quad \quad \to
  \ytableaushort{{z_{2}},{z_{1}}{z_{1}}},\  \ytableaushort{{z_{3}},{z_{1}}{z_{1}}},\ \ytableaushort{{z_{3}},{z_{2}}{z_{2}}},\
    \ytableaushort{{z_{2}},{z_{1}}{z_{2}}},\ \ytableaushort{{ z_{ 3 } },{ z_{ 1 } }{ z_{ 3 } }},\ \ytableaushort{{ z_{ 3 } },{ z_{ 2 } }{ z_{ 3 } }},\
    \ytableaushort{{ z_{ 2 } },{ z_{ 1 } }{ z_{ 3 } }},\ \ytableaushort{{ z_{ 3 } },{ z_{ 1 } }{ z_{ 2 } }}
  \]
  \vspace{-0.75em}
  \[
    s_{(2,1)}(z_1,z_2,z_3) = z_1^2z_2+z_1^2z_3+z_2^2z_3+z_1z_2^2+z_1z_3^2+z_2z_3^2+2z_1z_2z_3
  \]
  \begin{definition}
    For \(\lambda\) a partition \[
      s_\lambda = \sum_{T \in \SSYT(\lambda)} \zz^T \text{ for }\zz^T = \prod_{i
        \in T} z_i
    \]
  \end{definition}
  
  \begin{itemize}
  \item \(s_\lambda\) is a symmetric function.
  \item \(\{s_\lambda\}_\lambda\) forms a basis for \(\sym\).
  \end{itemize}
\end{frame}
\begin{frame}{Products of Schur polynomials}
  \begin{itemize}
  \item Littlewood-Richardson rule: \(s_\lambda s_\mu = \sum_\nu c_{\lambda \mu}^\nu s_\nu\) for
    \(c_{\lambda\mu}^\nu \in \N\). 
  \item Skew Schur function: \(s_{\nu/\lambda} = \sum_{T \in
      \SSYT(\nu/\lambda)} x^T = \sum_\mu c_{\lambda \mu}^\nu s_\mu\) .
  \item Straightforward observation: \(s_{\nu/\lambda} s_{\kappa/\mu} = \sum_{\substack{T \in \SSYT(\nu/\lambda) \\ S
        \in \SSYT(\kappa/\mu)}} x^T x^S\).
  \item \(q\)-deformation?
  \end{itemize}
\end{frame}
\begin{frame}{Key Object: LLT Polynomials}
  \vspace{-1.4mm}
{\small
Let  $\nubold= (\nu_{(1)}, \dots, \nu_{(k)})$ be a tuple of skew
shapes. (Skew shape = \(\lambda \setminus \mu\))
%\vspace{1mm}

\begin{itemize}
\onslide<2->{\item The \emph{content} of a box in row  $y$,  column  $x$ is  $x-y$.}
\onslide<3->{\item \emph{Reading order}: label boxes $b_1, \dots, b_n$ by scanning each diagonal from southwest to northeast, in order of increasing content.}
\onslide<4->{\item A pair  $(a,b) \in \nubold$ is \emph{attacking} if  $a$ precedes  $b$ in reading order and
\begin{itemize}
\item  ${\rm content}(b) = {\rm content}(a)$,  or
\item  ${\rm content}(b) = {\rm content}(a) + 1$ and $a \in \nu_{(i)}, b \in \nu_{(j)}$ with  $i > j$.
\end{itemize}}
\end{itemize}}

\vspace{-3mm}
\begin{equation*}
\begin{tikzpicture}[scale = .34]
\begin{scope}
\node[anchor=east] at (-0.1, 1) {$\nubold = \bigg( $};
%
\draw[thin, black!44]  (0,0) grid (1,1);
%\node at (0.5, 1.5) {\small $b_1$};
%\node at (1.5, 1.5) {\small $b_2$};
%\node at (1.5, 0.5) {\small $b_4$};
%\node at (2.5, 0.5) {\small $b_7$};
\node at (3.34, 0.02) { , };
\draw[very thick] (0,1) grid (2,2);
\draw[very thick] (1,0) grid (3,1);
\end{scope}
\begin{scope}[xshift = 117]
%
\draw[thin, black!44] (0,0) grid (1,2);
%\node at (1.5, 1.5) {\small $b_3$};
%\node at (2.5, 1.5) {\small $b_6$};
%\node at (1.5, 0.5) {\small $b_5$};
%\node at (2.5, 0.5) {\small $b_8$};
\draw[very thick] (1,0) grid (3,2);
\node at (3.58, 1) { $ \bigg)$};
\end{scope}
\end{tikzpicture}
\quad \quad \quad
\raisebox{-.94cm}{
\begin{tikzpicture}[scale = .51]
\begin{scope}
\only<5>{
\draw[draw = none, fill = red!47] (1,1) rectangle (2,2);
\draw[draw = none, fill = red!47] (4,4) rectangle (5,5);
}
\only<6>{
\draw[draw = none, fill = red!47] (4,4) rectangle (5,5);
\draw[draw = none, fill = red!47] (1,0) rectangle (2,1);
}
\only<7>{
\draw[draw = none, fill = red!47] (1,0) rectangle (2,1);
\draw[draw = none, fill = red!47] (4,3) rectangle (5,4);
}
\only<8>{
\draw[draw = none, fill = red!47] (1,0) rectangle (2,1);
\draw[draw = none, fill = red!47] (5,4) rectangle (6,5);
}
\only<9>{
\draw[draw = none, fill = red!47] (4,3) rectangle (5,4);
\draw[draw = none, fill = red!47] (2,0) rectangle (3,1);
}
\only<10>{
\draw[draw = none, fill = red!47] (5,4) rectangle (6,5);
\draw[draw = none, fill = red!47] (2,0) rectangle (3,1);
}
\only<11>{
\draw[draw = none, fill = red!47] (2,0) rectangle (3,1);
\draw[draw = none, fill = red!47] (5,3) rectangle (6,4);
}
%
\draw[help lines] (0,0) grid (6,5);
\draw[thick] (0,1) grid (2,2);
\draw[thick] (1,0) grid (3,1);
%
%
\draw[thick] (4,3) grid (6,5);
\only<3->{
\node at (0.5, 1.5) {\small $b_1$};
\node at (1.5, 1.5) {\small $b_2$};
\node at (1.5, 0.5) {\small $b_4$};
\node at (2.5, 0.5) {\small $b_7$};
%
\node at (3.0+1.5, 3.0+1.5) {\small $b_3$};
\node at (3.0+2.5, 3.0+1.5) {\small $b_6$};
\node at (3.0+1.5, 3.0+0.5) {\small $b_5$};
\node at (3.0+2.5, 3.0+0.5) {\small $b_8$};}
%
\only<2>{
\foreach \x in {0,...,5}
    \foreach \y in {0,...,4}
        {\pgfmathtruncatemacro{\myc}{\x - \y}
        \node at (\x+0.5,\y+0.5) {\footnotesize \myc };}
}
\end{scope}
\end{tikzpicture}}
\end{equation*}
{\small
\onslide<4->{
Attacking pairs:  $
{\color<5>{red}(b_2,b_3)},
{\color<6>{red}(b_3, b_4)},
{\color<7>{red}(b_4, b_5)},
{\color<8>{red}(b_4, b_6)},
{\color<9>{red}(b_5, b_7)},
{\color<10>{red}(b_6, b_7)},
{\color<11>{red}(b_7, b_8)} $}
}
\end{frame}
\begin{frame}{LLT Polynomials}
  \vspace{-2mm}
\small
\begin{itemize}
\item A \emph{semistandard tableau} on $\nubold  $ is a map
$T\colon \nubold  \rightarrow \Z _{+}$ which restricts to a
semistandard tableau on each $\nu_{(i)}$.
\onslide<2->{
\item An \emph{attacking inversion} in $T$ is
an attacking pair $(a,b)$ such that~$T(a)>T(b)$.}
\end{itemize}

The \emph{LLT polynomial} indexed by a tuple of skew shapes $\nubold
$ is
\begin{equation*}
\Gcal_{\nubold  }(\zz;q) = \sum _{T\in \SSYT (\nubold
)}\onslide<2->{q^{\inv (T)}}\zz ^{T},
\end{equation*}
\vspace{-1mm}
\onslide<2->{where $\inv (T)$ is the number of attacking inversions in $T$ and}
$\zz ^{T} = \prod _{a\in \nubold  } z_{T(a)}$.

\vspace{-2mm}
\begin{align*}
&
\begin{tikzpicture}[scale = .46]
\node[anchor = east] at (-1.4, 2.5) {$T \ \ = $};
\begin{scope}
\node[anchor = west] at (7,3) {\small \phantom{\colorb{non-inversion}}};
\only<3>{
\draw[draw = none, fill = blue!47] (1,1) rectangle (2,2);
\draw[draw = none, fill = blue!47] (4,4) rectangle (5,5);
\node[anchor = west] at (7,3) {\small \colorb{non-inversion}};
}
\only<4>{
\draw[draw = none, fill = red!47] (4,4) rectangle (5,5);
\draw[draw = none, fill = red!47] (1,0) rectangle (2,1);
\node[anchor = west] at (7,3) {\small \colorr{inversion}};
}
\only<5>{
\draw[draw = none, fill = red!47] (1,0) rectangle (2,1);
\draw[draw = none, fill = red!47] (4,3) rectangle (5,4);
\node[anchor = west] at (7,3) {\small \colorr{inversion}};
}
\only<6>{
\draw[draw = none, fill = blue!47] (1,0) rectangle (2,1);
\draw[draw = none, fill = blue!47] (5,4) rectangle (6,5);
\node[anchor = west] at (7,3) {\small \colorb{non-inversion}};
}
\only<7>{
\draw[draw = none, fill = blue!47] (4,3) rectangle (5,4);
\draw[draw = none, fill = blue!47] (2,0) rectangle (3,1);
\node[anchor = west] at (7,3) {\small \colorb{non-inversion}};
}
\only<8>{
\draw[draw = none, fill = red!47] (5,4) rectangle (6,5);
\draw[draw = none, fill = red!47] (2,0) rectangle (3,1);
\node[anchor = west] at (7,3) {\small \colorr{inversion}};
}
\only<9>{
\draw[draw = none, fill = red!47] (2,0) rectangle (3,1);
\draw[draw = none, fill = red!47] (5,3) rectangle (6,4);
\node[anchor = west] at (7,3) {\small \colorr{inversion}};
}
%
\draw[help lines] (0,0) grid (6,5);
\draw[thick] (0,1) grid (2,2);
\draw[thick] (1,0) grid (3,1);
\node at (0.5, 1.5) {\small $2$};
\node at (1.5, 1.5) {\small $4$};
\node at (1.5, 0.5) {\small $3$};
\node at (2.5, 0.5) {\small $5$};
\draw[thick] (4,3) grid (6,5);
\node at (3.0+1.5, 3.0+1.5) {\small $5$};
\node at (3.0+2.5, 3.0+1.5) {\small $6$};
\node at (3.0+1.5, 3.0+0.5) {\small $1$};
\node at (3.0+2.5, 3.0+0.5) {\small $1$};
\end{scope}
\end{tikzpicture}\\
&
\onslide<10>{\inv(T) = 4,}  \quad \zz^T = z_1^2z_2z_3z_4z_5^2z_6
\end{align*}
\end{frame}
\begin{frame}{LLT Polynomials \(\Gcal_\nubold(X;q)\)}
  \begin{itemize}
  \item \(\Gcal_\nubold(X;q)\) is a symmetric function\pause
  \item \(\Gcal_\nubold(X;1) = s_{\nu^{(1)}} \cdots s_{\nu^{(r)}}\)\pause
  \item \(\Gcal_\nubold\) were originally defined by Lascoux, Leclerc, and
    Thibon to explore connections to Fock space representations of \(U_q(\hat{\sl_r})\)\pause
  \item When \(\nu^{(i)}\) are partitions, the Schur-expansion
    coefficients are essentially parabolic Kazdhan-Luzstig polynomials.\pause
  \item \(\Gcal_\nubold\) is Schur-positive for any tuple of skew shapes \(\nubold\)
    [Grojnowski-Haiman, 2007].
  \end{itemize}
\end{frame}
\begin{frame}{Some notable occurrences of LLT Polynomials}
  \begin{itemize}
  \item Haglund-Haiman-Loehr formula for Macdonald polynomials:
    \[
      \ytableausetup{boxsize=0.75em}
      \Htild_\mu(X;q,t) =
      t^{n(\mu)} \sum_R \big( \prod_{\ytableaushort{u,{}}} q^{a+1}
        t^l\big) \Gcal_R(X;t^{-1})\,.
    \]\pause
  \item Shuffle theorem (and generalizations):
    \[
      \nabla e_n =
      \sum_{\lambda \in DP_n} t^{\area(\lambda)} q^{\dinv(\lambda)}
      \Gcal_{\nu(\lambda)}(X;q^{-1})\,.
    \]\pause
  \item For \(G = \) incomparability graph for natural unit interval
    order (encoded by Dyck path \(P\)),
    \[
      \chi_G(x;t) = {(1-t)}^{-|V|} \Gcal_{\nu(P)}[(1-t)x;t] \,.
    \]
  \end{itemize}
\end{frame}
\begin{frame}{Flagged Tableaux}
  \begin{itemize}
  \item Fix partition shape \(\lambda = (\lambda_1,\ldots,\lambda_l)\)
    and \emph{flag} \(\bb = (b_1 \leq \cdots \leq b_l) \in \N^l\).\pause
   \item A \emph{flagged semistandard Young tableau} of shape \(\lambda\)
    with flag \(\bb\) is a \(T \in \SSYT(\lambda)\) such that the
    entries of row \(i\) are bounded above by \(b_i\).
   \item Denote the set of such tableaux via \(\FT(\lambda,\bb)\).
   \item E.g., \(\lambda = (2,1), \bb = (1,3)\), \(\FT(\lambda,\bb) = \)
  \end{itemize}
  \begin{center}
    \begin{tikzpicture}[scale=0.5,baseline=0]
      \foreach \beta / \alpha / \y in {2/0/0,1/0/1} \draw (\alpha,\y)
      grid (\beta, \y+1); \node at (2.5,0.5) {\(\scriptstyle \leq 1\)}; \node at (1.5,1.5)
      {\(\scriptstyle \leq 3\)}; \node at (0.5,0.5) {1}; \node at (1.5,0.5) {1}; \node at
      (0.5,1.5) {2};
    \end{tikzpicture}
    \begin{tikzpicture}[scale=0.5,baseline=0]
      \foreach \beta / \alpha / \y in {2/0/0,1/0/1} \draw (\alpha,\y)
      grid (\beta, \y+1); \node at (2.5,0.5) {\(\scriptstyle \leq 1\)}; \node at (1.5,1.5)
      {\(\scriptstyle \leq 3\)}; \node at (0.5,0.5) {1}; \node at (1.5,0.5) {1}; \node at
      (0.5,1.5) {3};
    \end{tikzpicture}
    \begin{itemize}\pause
    \item Note, there is no issue with letting length of \(\bb\) be
      less than \(l\).
    \end{itemize}
  \end{center}
\end{frame}
\begin{frame}{Flagged Schur functions (Lascoux-Sch\"{u}tzenberger, Wachs)}
  \begin{itemize}
  \item For partition \(\lambda\) and flag \(\bb\), the \emph{flagged
      Schur function} is given by
    \[
      s_{\lambda,\bb}(x) = \sum_{T \in \FT(\lambda,\bb)} x^T
    \]\pause
  \item Jacobi-Trudi: \(s_{\lambda,\bb}(x) = \det(h_{\lambda_i-i+j}(x_1,\ldots,x_{b_i}))\)\pause
  \item E.g., \(s_{21,13}(x_1,x_2,x_3) = x_1^2x_2 + x_1^2 x_3 =
    h_2(x_1)h_1(x_1,x_2,x_3)-h_3(x_1)\) 
    \begin{center}
     \begin{tikzpicture}[scale=0.5,baseline=0]
      \foreach \beta / \alpha / \y in {2/0/0,1/0/1} \draw (\alpha,\y)
      grid (\beta, \y+1); \node at (2.5,0.5) {\(\scriptstyle \leq 1\)}; \node at (1.5,1.5)
      {\(\scriptstyle \leq 3\)}; \node at (0.5,0.5) {1}; \node at (1.5,0.5) {1}; \node at
      (0.5,1.5) {2};
    \end{tikzpicture}
    \begin{tikzpicture}[scale=0.5,baseline=0]
      \foreach \beta / \alpha / \y in {2/0/0,1/0/1} \draw (\alpha,\y)
      grid (\beta, \y+1); \node at (2.5,0.5) {\(\scriptstyle \leq 1\)}; \node at (1.5,1.5)
      {\(\scriptstyle \leq 3\)}; \node at (0.5,0.5) {1}; \node at (1.5,0.5) {1}; \node at
      (0.5,1.5) {3};
    \end{tikzpicture}
    \raisebox{.12in}{
    \(
\left(    \begin{matrix}
      h_2(x_1) & h_3(x_1) \\
      h_0(x_1,x_2,x_3) & h_1(x_1,x_2,x_3)
    \end{matrix}
\right)
    \)}
    \end{center}
    \pause
  \item \(\mathfrak{S}_w(x) = s_{\lambda,\bb}(x)\) for
    \emph{vexillary} \(w \in S_n\) (2143-avoiding).\pause
  \item \(s_{\lambda,\bb}(x)\) are examples of \emph{Demazure characters}.
  \end{itemize}
\end{frame}
\begin{frame}{Demazure characters and atoms}
  The \emph{Demazure operator} $\pi_i$
  acts on $f \in \Q(q,t)[x_1^{\pm 1}, \dots, x_N^{\pm 1}]$ by
  \vspace{-1mm}
  \begin{align*}
  \pi_i(f) &= \frac{x_if-x_{i+1}s_i(f)}{x_i-x_{i+1}}.
  \end{align*}\pause

  The \emph{Demazure characters} or \emph{key polynomials} are
  constructed from
  \begin{itemize}
  \item  $\Dcal_{\lambda} = x^\lambda := x_1^{\lambda_1}\cdots x_N^{\lambda_N}$ for partition  $\lambda$.
  \item $\Dcal_{s_i(\alpha)} = \pi_i \Dcal_{\alpha}$ for  $\alpha_i > \alpha_{i+1}$, for any  $\alpha \in \N^N$.

  \end{itemize}\pause
\emph{Demazure atoms} are defined the same as keys but with
$\hat{\pi}_i := \pi_i-1$ in place of  $\pi_i$:
\begin{itemize}
\item \(\Acal_\lambda = x^\lambda\) for partition \(\lambda\).
\item \(\Acal_{s_i(\alpha)} = \hat{\pi}_i \Acal_\alpha\) for \(\alpha_i >
  \alpha_{i+1}\), for any \(\alpha \in \N^N\).
\end{itemize}
\vspace{5mm}
\end{frame}
\begin{frame}{Examples}
  \begin{align*}
\Dcal_{520} & = x_1^5x_2^2  \\[.8mm] 
\Dcal_{250} & = \pi_{1}\Dcal_{520}=\pi_1(x_1^5x_2^2) = x_1^5x_2^2 + x_1^4 x_2^3 + x_1^3 x_2^4 + x_1^2x_2^5 \\[.8mm]
\Dcal_{205} & = \pi_2 \Dcal_{250} = \pi_2 \big(x_1^5x_2^2 + x_1^4 x_2^3 + x_1^3 x_2^4 + x_1^2x_2^5\big)
\end{align*}
\pause
\begin{align*}
  \Dcal_{520} & = \Acal_{520}\\
  \Dcal_{250} & = \Acal_{520} + \Acal_{250}\\
  \Dcal_{205} & = \Acal_{520} + \Acal_{250} + \Acal_{502} +
                \Acal_{205}\\
  \vdots
\end{align*}

\end{frame}
\begin{frame}{Recovering Symmetric Functions}
  \begin{itemize}
  \item If \(\bb = (n,\ldots,n)\) for \(n = |\lambda|\), then
    \(s_{\lambda,\bb}(x) = s_\lambda(x_1,\ldots,x_n)\).\pause
  \item If \(\bb = \emptyset\), \(s_{\lambda,\bb}(x) = s_\lambda(x)\).\pause
  \item For $w=s_{i_1}s_{i_2}\cdots s_{i_m} \in S_{N}$ reduced,
    $\pi_w := \pi_{i_1}\pi_{i_2}\cdots \pi_{i_m}.$\pause
  \item For \emph{Weyl symmetrization operator} \(\pi_{w_0}\), we have \[
        \pi_{w_0}(s_{\lambda,\bb}(x)) = s_\lambda(x), \hspace{0.1in}
        \pi_{w_0}(\Dcal_\alpha) = \Dcal_{\sort(\alpha)} = s_{\alpha_+},
      \]
      \[
        \pi_{w_0}(\Acal_\alpha) =
        \begin{cases}
          s_\alpha & \alpha \text{ a partition},\\
          0 & \text{else.}
        \end{cases}
    \]
  \end{itemize}
\end{frame}
\begin{frame}{Flagged LLT Polynomials (Blasiak-Haiman-Morse-Pun-S.)}
  \begin{itemize}
  \item Let  $e_1,\dots, e_l$ be the row ends of  $\nubold$, ordered in reverse reading order.
  \item Fix flag \(\bb = (b_1 \leq \cdots \leq b_l)\).
  \item \(\FT(\nubold,\bb) = \) set of semistandard tableaux \(T\)
    on \(\nubold\) satisfying \(T(e_i) \leq b_i\). 
  \item The \emph{flagged LLT polynomial} indexed by \(\nubold\) and
    \(\bb\) is
    \[
      \Gcal_{\nubold, \bb}(x;t) = \sum_{T \in \FT(\nubold,\bb)}
      t^{\inv (T)} x^T \,.
    \]
  \end{itemize}
  \begin{center}
    \begin{tikzpicture}[scale = .51]
      \begin{scope}
        \draw[help lines] (0,0) grid (6,5); \draw[thick] (0,1) grid
        (2,2); \draw[thick] (1,0) grid (3,1);
%
        \draw[thick] (4,3) grid (6,5); \only<1>{ \node at (0.5, 1.5)
          {\small $ $}; \node at (1.5, 1.5) {\small $e_4$}; \node at
          (1.5, 0.5) {\small $ $}; \node at (2.5, 0.5) {\small $e_2$};
%
          \node at (3.0+1.5, 3.0+1.5) {\small $ $}; \node at (3.0+2.5,
          3.0+1.5) {\small $e_3$}; \node at (3.0+1.5, 3.0+0.5) {\small
            $ $}; \node at (3.0+2.5, 3.0+0.5) {\small $e_1$};}

        \only<2->{ \node at (0.5, 1.5) {\small $2$}; \node at (1.5,
          1.5) {\small $4$}; \node at (1.5, 0.5) {\small $1$}; \node
          at (2.5, 0.5) {\small $2$}; \node at (3.0+1.5, 3.0+1.5)
          {\small $2$}; \node at (3.0+2.5, 3.0+1.5) {\small $3$};
          \node at (3.0+1.5, 3.0+0.5) {\small $1$}; \node at (3.0+2.5,
          3.0+0.5) {\small $1$};

          \node at (6.8, 3.0+1.5) {\small $\le 3$}; \node at (6.8,
          3.0+0.5) {\small $\le 1$}; \node at (6.8, 1.5) {\small
            $\le 4$}; \node at (6.8, 0.5) {\small $\le 2$}; }
      \end{scope}
    \end{tikzpicture}
  \end{center}
\end{frame}
\begin{frame}[fragile]{Flagged LLT Polynomials}
\begin{equation*}
\nubold =  \big(
\begin{tikzpicture}[scale = .34]
\draw[very thick] (0,0) grid (2,1);
\end{tikzpicture}
,
\begin{tikzpicture}[scale = .34]
\draw[very thick] (0,0) grid (2,1);
\end{tikzpicture}
\, \big), \bb = (1,2)
\end{equation*}
\vspace{0.2in}
\[
\begin{array}{cc@{\quad\quad\quad}c@{\quad\quad\quad}c}
\raisebox{5mm}{T}&
\begin{tikzpicture}[scale = .5]
\begin{scope}
\draw[help lines] (0,0) grid (4,3);
\draw[thick] (0,0) grid (2,1);
\draw[thick] (2,2) grid (4,3);
\node at (0.5, 0.5) {\small $1$};
\node at (1.5, 0.5) {\small $1$};
\node at (2.5, 2.5) {\small $1$};
\node at (3.5, 2.5) {\small $1$};
\node at (2.5, 0.5) {\tiny $\leq 2$};
\node at (4.5, 2.5) {\tiny $\leq 1$};
\end{scope}
\end{tikzpicture}
&
\begin{tikzpicture}[scale = .5]
\begin{scope}
\tikzstyle{rededge} = [draw, ->,red]
\draw[help lines] (0,0) grid (4,3);
\draw[thick] (0,0) grid (2,1);
\draw[thick] (2,2) grid (4,3);
\node at (0.5, 0.5) {\small $1$};
\node at (1.5, 0.5) {\small $2$};
\node at (2.5, 2.5) {\small $1$};
\node at (3.5, 2.5) {\small $1$};
\node at (2.5, 0.5) {\tiny $\leq 2$};
\node at (4.5, 2.5) {\tiny $\leq 1$};
\draw[rededge] (2+.1,1+0.1) to (3-.1,2-.1);
\end{scope}
\end{tikzpicture}
&
\begin{tikzpicture}[scale = .5]
\tikzstyle{rededge} = [draw, ->,red]
\begin{scope}
\draw[help lines] (0,0) grid (4,3);
\draw[thick] (0,0) grid (2,1);
\draw[thick] (2,2) grid (4,3);
\node at (0.5, 0.5) {\small $2$};
\node at (1.5, 0.5) {\small $2$};
\node at (2.5, 2.5) {\small $1$};
\node at (3.5, 2.5) {\small $1$};
\node at (2.5, 0.5) {\tiny $\leq 2$};
\node at (4.5, 2.5) {\tiny $\leq 1$};
\draw[rededge] (2+.1,1+0.1) to (3-.1,2-.1);
\draw[rededge] (1+.1,1+0.1) to (2-.1,2-.1);
\end{scope}
\end{tikzpicture}
\end{array}
\]
\[
\Gcal_{r, \nubold}(x;t) = x_1^4+ t\, x_1^3x_2 + t^2 x_1^2x_2^2
\]
\end{frame}

\begin{frame}{Flagged LLT Polynomials}
  \begin{itemize}
  \item \(\Gcal_{\nubold,\bb}(x;t)\) will Weyl symmetrize to
    \(\Gcal_\nubold(X;t)\).\pause
  \item For \(\nubold = (\lambda)\) a single shape, \(\Gcal_{\nubold,\bb}(x;t) = s_{\lambda,\bb}(x)\).\pause
  \item \(\Gcal_{\nubold,\bb}(x;t)\) have an algebraic formula via
    ``nonsymmetric Hall-Littlewood polynomials.'' \pause
  \item \(\Gcal_{\nubold,\bb}(x;t)\) are conjecturally positive in terms
    of \emph{Demazure atoms}.
  \end{itemize}
\end{frame}

\begin{frame}{Signed flagged LLT polynomials}
 \begin{itemize}
\item Signed alphabet $\Acal = 1 < \bar{1} < 2 < \bar{2} \, \cdots $\pause
\item $\FT^{\pm}(\nubold,\bb) = $ fillings of  $\nubold$ from $\Acal$ satisfying
\begin{itemize}
\item unbarred letters weakly increase in rows, strictly increase in columns.
\item barred letters strictly increase in rows, weakly increase in columns.
\item $T(e_{i}) \le b_i$ for  $i = 1,\dots, l$.\pause
\end{itemize}
\end{itemize}
The \emph{signed flagged LLT polynomial} indexed by \(\nubold\) and
flag \(\bb\) is 
 \begin{align*}
\Gcal^\pm_{\nubold,\bb}(\xx;t) & = \sum_{T \in \FT^{\pm}(\nubold,\bb)} t^{\inv(T)} (-t)^{-\#\text{bar}(T)} \xx^{|T|},
\end{align*}
\vspace{-3.6mm}
where $|T|$ is the result of removing all bars from $T$.
\end{frame}
\begin{frame}[fragile]{Signed flagged LLT polynomials}
\begin{equation*}
\nubold =  \big(
\begin{tikzpicture}[scale = .34]
\draw[very thick] (0,0) grid (2,1);
\end{tikzpicture}
,
\begin{tikzpicture}[scale = .34]
\draw[very thick] (0,0) grid (2,1);
\end{tikzpicture}
\, \big), \bb = (1,2)
\end{equation*}
\[
\begin{array}{cc@{\ \ \ }c@{\ \ \ }c@{\ \ \ }c@{\ \ \ }c}
T&
\begin{tikzpicture}[scale = .38]
\begin{scope}
\draw[help lines] (0,0) grid (4,3);
\draw[thick] (0,0) grid (2,1);
\draw[thick] (2,2) grid (4,3);
\node at (0.5, 0.5) {\scriptsize $1$};
\node at (1.5, 0.5) {\scriptsize $1$};
\node at (2.5, 2.5) {\scriptsize $1$};
\node at (3.5, 2.5) {\scriptsize $1$};
\end{scope}
\end{tikzpicture}
&
\begin{tikzpicture}[scale = .38]
\begin{scope}
\tikzstyle{rededge} = [draw, ->,red]
\draw[help lines] (0,0) grid (4,3);
\draw[thick] (0,0) grid (2,1);
\draw[thick] (2,2) grid (4,3);
\node at (0.5, 0.5) {\scriptsize $1$};
\node at (1.5, 0.5) {\scriptsize $2$};
\node at (2.5, 2.5) {\scriptsize $1$};
\node at (3.5, 2.5) {\scriptsize $1$};
\draw[rededge] (2+.1,1+0.1) to (3-.1,2-.1);
\end{scope}
\end{tikzpicture}
&
\begin{tikzpicture}[scale = .38]
\tikzstyle{rededge} = [draw, ->,red]
\begin{scope}
\draw[help lines] (0,0) grid (4,3);
\draw[thick] (0,0) grid (2,1);
\draw[thick] (2,2) grid (4,3);
\node at (0.5, 0.5) {\scriptsize $2$};
\node at (1.5, 0.5) {\scriptsize $2$};
\node at (2.5, 2.5) {\scriptsize $1$};
\node at (3.5, 2.5) {\scriptsize $1$};
\draw[rededge] (2+.1,1+0.1) to (3-.1,2-.1);
\draw[rededge] (1+.1,1+0.1) to (2-.1,2-.1);
\end{scope}
\end{tikzpicture}
 &
\begin{tikzpicture}[scale = .38]
\tikzstyle{rededge} = [draw, ->,red]
\begin{scope}
\draw[help lines] (0,0) grid (4,3);
\draw[thick] (0,0) grid (2,1);
\draw[thick] (2,2) grid (4,3);
\node at (0.5, 0.47) {\scriptsize $1$};
\node at (1.5, 0.47) {\scriptsize $\bar{1}$};
\node at (2.5, 2.5) {\scriptsize $1$};
\node at (3.5, 2.5) {\scriptsize $1$};
\draw[rededge] (2+.1,1+0.1) to (3-.1,2-.1);
\end{scope}
\end{tikzpicture}
 &
\begin{tikzpicture}[scale = .38]
\tikzstyle{rededge} = [draw, ->,red]
\begin{scope}
\draw[help lines] (0,0) grid (4,3);
\draw[thick] (0,0) grid (2,1);
\draw[thick] (2,2) grid (4,3);
\node at (0.5, 0.47) {\scriptsize $\bar{1}$};
\node at (1.5, 0.47) {\scriptsize $2$};
\node at (2.5, 2.5) {\scriptsize $1$};
\node at (3.5, 2.5) {\scriptsize $1$};
\draw[rededge] (2+.1,1+0.1) to (3-.1,2-.1);
\draw[rededge] (1+.1,1+0.1) to (2-.1,2-.1);
\end{scope}
\end{tikzpicture}
\end{array}
\]
\begin{align*}
\Gcal^{\pm}_{\nubold,\bb}(\xx;t)
\, \ & = \ \, x_1^4  \hspace{2.9mm}+ \hspace{2.9mm}tx_1^3x_2 \hspace{2.9mm}+\hspace{2.9mm}t^2 x_1^2x_2^2\hspace{2.9mm}-\hspace{2.9mm}x_1^4 \hspace{2.9mm}- \hspace{2.9mm}t x_1^3x_2 \\
\ \, & = \ \,  t^2 x_1^2 x_2^2
\end{align*}
\end{frame}
\begin{frame}{Flagged plethysm}
  Define \emph{flagged plethysm} \(\Pi_{t,x} \from \kk[x_1,\ldots,x_l]
  \to \kk[x_1,\ldots,x_l]\) to be the (over-determined) linear map \[
    \Pi_{t,x} \Gcal_{\nubold,\bb}^{\pm}(x;t^{-1}) =
    \Gcal_{\nubold,\bb}(x;t^{-1}) \,.
  \]
  \begin{thm}[Blasiak-Haiman-Morse-Pun-S., 2025+]
    \(\Pi_{t,x}\) is well-defined.
  \end{thm}
  Note, \(\Pi_{t,x}\) is a ``nonsymmetric analogue'' of the plethystic map
  \(f[X] \mapsto f[X/(1-t)]\) for symmetric function \(f\).
\end{frame}
\begin{frame}{Applications}
  \begin{enumerate}
  \item Can define \emph{modified \(r\)-nonsymmetric Macdonald
      polynomials} (or \emph{modern Macdonald polynomials})
    via a \emph{flagged} Haglund-Haiman-Loehr formula:
    \[
      H_{\eta|\lambda}(x;q,t) = t^{n} \sum_{R} \big( \prod_{\ytableaushort{u,{}}} q^{a+1}
        t^l\big) \Gcal_{R,(b_1,\ldots,b_r)}(x;t^{-1})\,.
    \]
    These symmetrize to Macdonald \(H\)-polynomials and are
    conjecturally Demazure atom positive.\pause
  \item Using \(H_{\eta|\lambda}\) to define a nonsymmetric analogue
    of \(\nabla\), we show a ``nonsymmetric shuffle theorem''
    expressed in terms of flagged LLT polynomials associated to
    flagged Dyck paths.\pause
  \item Tewari-Wilson-Zhang define chromatic nonsymmetric polynomials
    associated to \(d \times d\) Dyck paths starting with \(r\) north steps, \(\chi_{\bb,\pi}\). We show \[
      \chi_{(1,\ldots,r),\pi}(x;t) = (1-t)^{r-d} \Gcal^\pm_{\nu(\pi),(1,\ldots,r)}(x;t)\,.
    \]
  \end{enumerate}
\end{frame}
\end{document}
%%% Local Variables:
%%% mode: latex
%%% TeX-master: t
%%% End:
