%%%%%%%%%%%%%%%%%%%%%%%%%%%%%%%%%%%%%%%%%
% Beamer Presentation
% LaTeX Template
% Version 1.0 (10/11/12)
%
% This template has been downloaded from:
% http://www.LaTeXTemplates.com
%
% License:
% CC BY-NC-SA 3.0 (http://creativecommons.org/licenses/by-nc-sa/3.0/)
%
%%%%%%%%%%%%%%%%%%%%%%%%%%%%%%%%%%%%%%%%%

%----------------------------------------------------------------------------------------
%	PACKAGES AND THEMES
%----------------------------------------------------------------------------------------

\documentclass{beamer}

\mode<presentation> {

% The Beamer class comes with a number of default slide themes
% which change the colors and layouts of slides. Below this is a list
% of all the themes, uncomment each in turn to see what they look like.

%\usetheme{default}
%\usetheme{AnnArbor}
%\usetheme{Antibes}
%\usetheme{Bergen}
%\usetheme{Berkeley}
%\usetheme{Berlin}
%\usetheme{Boadilla}
%\usetheme{CambridgeUS}
%\usetheme{Copenhagen}
%\usetheme{Darmstadt}
%\usetheme{Dresden}
%\usetheme{Frankfurt}
%\usetheme{Goettingen}
%\usetheme{Hannover}
%\usetheme{Ilmenau}
%\usetheme{JuanLesPins}
%\usetheme{Luebeck}
\usetheme{Madrid}
%\usetheme{Malmoe}
%\usetheme{Marburg}
%\usetheme{Montpellier}
%\usetheme{PaloAlto}
%\usetheme{Pittsburgh}
%\usetheme{Rochester}
%\usetheme{Singapore}
%\usetheme{Szeged}
%\usetheme{Warsaw}

% As well as themes, the Beamer class has a number of color themes
% for any slide theme. Uncomment each of these in turn to see how it
% changes the colors of your current slide theme.

%\usecolortheme{albatross}
%\usecolortheme{beaver}
%\usecolortheme{beetle}
%\usecolortheme{crane}
%\usecolortheme{dolphin}
%\usecolortheme{dove}
%\usecolortheme{fly}
%\usecolortheme{lily}
%\usecolortheme{orchid}
%\usecolortheme{rose}
%\usecolortheme{seagull}
%\usecolortheme{seahorse}
%\usecolortheme{whale}
%\usecolortheme{wolverine}

%\setbeamertemplate{footline} % To remove the footer line in all slides uncomment this line
%\setbeamertemplate{footline}[page number] % To replace the footer line in all slides with a simple slide count uncomment this line

\setbeamertemplate{navigation symbols}{} % To remove the navigation symbols from the bottom of all slides uncomment this line
}

\usepackage{graphicx} % Allows including images
%\usepackage{booktabs} % Allows the use of \toprule, \midrule and
                      % \bottomrule in tables
\usepackage{tikz}
\usepackage{tikz-cd}
\usepackage{amsmath}
% \usepackage{../ReAdTeX/readtex-core}
% \usepackage{../ReAdTeX/readtex-dangerous}
% \usepackage{../ReAdTeX/readtex-abstract-algebra}
\usepackage{ytableau}
%%%%%%%%%%%%%%%%%%%%%%%%%%%%%%%%%%%%%%%%%%%%%%%%%%%%%%%%%%%%%%%%%%% 
%%  MACRO DEFINITIONS:  Co-authors -- PLEASE use these! 
%%%%%%%%%%%%%%%%%%%%%%%%%%%%%%%%%%%%%%%%%%%%%%%%%%%%%%%%%%%%%%%%%%%
\newcommand{\T}{\mathsf{T}} % use for tableaux
\newcommand{\U}{\mathsf{U}} % use for tableaux
\newcommand{\TA}{\mathsf{A}} % use for tableaux
\newcommand{\TB}{\mathsf{B}} % use for tableaux
\newcommand{\TC}{\mathsf{C}} % use for tableaux
\newcommand{\Std}{\operatorname{Std}} % set of standard tableaux
\newcommand{\StdB}{\operatorname{StdB}} % set of standard bitableaux
\newcommand{\Orb}{\mathcal{O}} % use for orbits
%%%%%%%%%%%%%%%%%%%%%%%%%%%%%%%%%%%%%%%%%%%%%%%%%%%%%%%%%%%%%%%%%%%%

%----------------------------------------------------------------------------------------
%	TITLE PAGE
%----------------------------------------------------------------------------------------

\title[Symmetric Functions]{Building Mathematical Bridges Between
  Symmetric Functions} % The short title appears at the bottom of every slide, the full title is only on the title page

\author[George H. Seelinger]{George H. Seelinger} % Your name
\institute[JSF] % Your institution as it will appear on the bottom of every slide, may be shorthand to save space
{
Jefferson Scholars Foundation \\ % Your institution for the title page
\medskip
\textit{ghs9ae@virginia.edu} % Your email address
}
\date{28 November 2018} % Date, can be changed to a custom date

\begin{document}

\begin{frame}
\titlepage % Print the title page as the first slide
\end{frame}
\begin{frame}
  \frametitle{Partitions of \(5\)}
  How many ways can we write a positive integer as a sum of positive
  integers? \pause
  \ytableausetup{boxsize=0.5em, aligntableaux=center}
    \begin{align*}
    5 \to &\ \ydiagram{5}\\
    4+1 \to &\ \ydiagram{4,1}\\
    3+2 \to &\ \ydiagram{3,2}\\
    3+1+1 \to &\ \ydiagram{3,1,1}\\
    2+2+1 \to&\ \ydiagram{2,2,1}\\
    2+1+1+1 \to&\ \ydiagram{2,1,1,1} \\
    1+1+1+1+1 \to&\ \ydiagram{1,1,1,1,1}
\end{align*}
\end{frame}
\begin{frame}
  \frametitle{Raising Operators}
  We can change partition diagrams by moving boxes.
  \[
    R_{1,3} \left( \ydiagram{3,1,1}*[*(red)]{0,0,1} \right) = \ydiagram{4,1}*[*(red)]{3+1}
  \]
\[
  R_{2,3} \left( \ydiagram{1,1,1}*[*(red)]{0,0,1} \right) = \ydiagram{1,2}*[*(red)]{0,1+1}
\]
\pause
If the result ``does not make sense'', we get \(0\): \[
  R_{1,4}\left( \ydiagram{1,1,1} \right) = 0
\]
\end{frame}
\begin{frame}
  \frametitle{Symmetric Functions}
  Let us introduce a rule for multiplication of partition diagrams by ``stacking.''
\begin{block}{Rule for Multiplication Example}
  \(
  \ydiagram{3} \cdot \ydiagram{2} = \ydiagram{3,2} =
  \ydiagram{2,3} = \ydiagram{2} \cdot \ydiagram{3}
  \)
\end{block}
\end{frame}
\begin{frame}
  \frametitle{Schur functions}
  We define a new class of functions. Given a partition diagram
  \(\lambda\) with 
  \(\ell\) rows, we say
  \begin{align*}
    s_\lambda = &(1-R_{1,2})\\
    &(1-R_{1,3})(1-R_{2,3})\\
    & \cdots\\
    &(1-R_{1,
      \ell})(1-R_{2,\ell})\cdots(1-R_{\ell-2,\ell})(1-R_{\ell-1,\ell})
    \lambda\\
  \end{align*}
  \begin{example}
    \[
      s_{\ydiagram{3,2,1}} = (1-R_{1,2})(1-R_{1,3})(1-R_{2,3})
      \ydiagram{3,2,1}
    \]
  \end{example}
    \pause For those who know about matrices, this is determinant of a matrix
    of partition diagrams.
\end{frame}
\begin{frame}
  \frametitle{Example continued}
    \begin{example}
    \[
      s_{\ydiagram{3,2,1}} = (1-R_{1,2})(1-R_{1,3})(1-R_{2,3})
      \ydiagram{3,2,1}
    \]
  \end{example}
  \ytableausetup{boxsize=0.25em}\pause
    \[
  \begin{array}[center]{ccc}
    &\ydiagram{3,2,1}&\\
    -R_{12}(\ydiagram{3,2,1}*[*(red)]{0,1+1})&-R_{13}(\ydiagram{3,2,1})&-R_{23}(\ydiagram{3,2,1})\\
    +R_{12}R_{13}(\ydiagram{3,2,1})&+R_{12}R_{23}(\ydiagram{3,2,1})&+R_{13}R_{23}(\ydiagram{3,2,1})\\
    &-R_{12}R_{13}R_{23}(\ydiagram{3,2,1})
  \end{array}
  \pause =
  \begin{array}{ccc}
    &\ydiagram{3,2,1}& \\
    -\ydiagram{4,1,1}*[*(red)]{3+1}&-\ydiagram{4,2}&-\ydiagram{3,3}\\
    +\ydiagram{5,1} & +\ydiagram{4,2}&+0\\
    & -0 &
  \end{array}
\]
\pause
Adding it all together, we get \[
  s_{\ydiagram{3,2,1}} = \ydiagram{3,2,1} - \ydiagram{4,1,1} -
  \ydiagram{3,3} + \ydiagram{5,1}
\]
\end{frame}
\begin{frame}
  \frametitle{Why Schur functions?}
  \begin{itemize}
  \item Schur functions encode the possible ways certain abstract
    algebraic objects appear in \(n\)-dimensional space.\pause
  \item Schur functions are ``orthogonal'' to each other in a certain
    sense, making computations easier.
  \end{itemize}
  \pause
  \begin{block}{Problem}
    However, the definition of Schur functions is complicated. If we
    had another formula for Schur functions, how can we prove they are
    the same thing?
  \end{block}
\end{frame}
\begin{frame}
  \frametitle{The Pieri Rule}
  Schur functions are defined in terms of partition
  diagrams, so it makes sense to multiply a diagram and a Schur function.
  \begin{example}
    \[
      \ydiagram[*(red)]{4} \cdot s_{\ydiagram{2,1}} =
      s_{\ydiagram{2,1}*[*(red)]{2+2,1+1,1}} +
      s_{\ydiagram{2,1}*[*(red)]{2+3,0,1}} +
      s_{\ydiagram{2,1}*[*(red)]{2+3,1+1}} +
      s_{\ydiagram{2,1}*[*(red)]{2+4}}
    \]
  \end{example}
  \pause We call this the Pieri rule and it is a fundamental property
  of Schur functions.
\end{frame}
\begin{frame}[fragile]
  \frametitle{Proof Technique}
  One approach to show two formulas for Schur functions are the same:
  \begin{block}{Proof technique}
    \[
      \begin{tikzcd}[cramped, column sep=0.01em]
        \text{Base cases are equal} \ar[rd]&\, & \ar[ld] \text{Pieri rules are the same}\\
        \ &\text{Linear algebra} \ar[d] & \ \\
        \ &\text{Functions are the same!} & \
      \end{tikzcd}
    \]
  \end{block}
\end{frame}
\begin{frame}
  \frametitle{What do I think about?}
  \begin{itemize}
  \item Schur functions are very well-understood.\pause
  \item Instead, I think about a class of functions called ``type C
    dual affine Stanley symmetric functions'' which serve a similar
    geometric role to the Schur functions, but in a different context.\pause
  \item However, the current formula for these functions is ``group
    theoretic'' instead of ``combinatorial,'' like the definition for
    Schur functions presented earlier.
  \end{itemize}
\end{frame}
\begin{frame}
  \frametitle{Type C dual affine Stanley symmetric functions}
  Start with ``word'' with letters given by colors,\ytableausetup{boxsize=0.5em}
  \(\{\ydiagram[*(red)]{1}, \ydiagram[*(blue)]{1},
  \ydiagram[*(green)]{1}\}\). For example, let's use \(w = 
  \begin{ytableau}
   *(green) & *(blue) & *(green)
  \end{ytableau}
  \). \pause

  We must find all ``subword decompositions'' of \(w\) that are also subwords of \(
  \rho = \begin{ytableau}
    *(blue) & *(green) & *(blue) & *(red)
  \end{ytableau}\) or any of its ``rotations'' \(\begin{ytableau}
    *(red)& *(blue) & *(green) & *(blue) 
  \end{ytableau}, \begin{ytableau}
     *(blue) &*(red)& *(blue) & *(green)
  \end{ytableau}, \begin{ytableau}
      *(green) & *(blue) &*(red)& *(blue)
  \end{ytableau}
\). \pause
\begin{example}
  \(
  \begin{ytableau}
    *(green) & *(blue)
  \end{ytableau} |
  \begin{ytableau}
    *(green)
  \end{ytableau}
\) is a subword decomposition of \(w\) where each part appears as a
subword of \(\rho = \begin{ytableau}
    *(blue) & *(green) & *(blue) & *(red)
  \end{ytableau}\), but \(
  \begin{ytableau}
    *(green) & *(blue) & *(green)
  \end{ytableau}
\) is not a subword of \(\rho\) or any of its rotations.
\end{example}
\end{frame}
\begin{frame}
  \frametitle{Example continued}
  Then, you take all such subword decompositions to get a formula \[
  \begin{array}{c}
    \begin{ytableau}
    *(green) & *(blue)
  \end{ytableau} |
  \begin{ytableau}
    *(green)
  \end{ytableau}  \\
  \begin{ytableau}
    *(green) 
  \end{ytableau} |
  \begin{ytableau}
    *(blue) & *(green)
  \end{ytableau} \\
  \begin{ytableau}
    *(green) 
  \end{ytableau} |
  \begin{ytableau}
    *(blue)  
  \end{ytableau} |
  \begin{ytableau}
    *(green)
  \end{ytableau}
  \end{array} \to
  \ytableausetup{aligntableaux=top}
  \begin{array}{c}
    \ydiagram{2,1} \\
    \ydiagram{1,1,1}
  \end{array}
  \ytableausetup{aligntableaux=center}
  \to Q^{(2)}_{
    \begin{ytableau}
      *(green) & *(blue) & *(green)
    \end{ytableau}
  } = 4* \ydiagram{2,1} + 8 * \ydiagram{1,1,1}\] \pause
But, you are not done! \pause You then have to take the ``dual'' of
this function to get the Type C dual affine Stanley symmetric
function, \(P^{(2)}_{
    \begin{ytableau}
      *(green) & *(blue) & *(green)
    \end{ytableau}
}\).
\end{frame}
\begin{frame}[fragile]
  \frametitle{What have I done?}
  \begin{itemize}
  \item I have a conjectured combinatorial formula that describes type
    C dual affine Stanley symetric functions directly (and
    combinatorially) using raising operators. \pause
  
   \item Computational evidence suggests my conjecture is correct. \pause
  
   \item However, proving the formulas are the same directly would be quite
    hard, so instead I am seeking to use the Peri rule approach \[
      \begin{tikzcd}[cramped, column sep=0.01em]
        \text{Base cases are equal} \ar[rd]&\, & \ar[ld, dashed] \text{Pieri rules are the same}\\
        \ &\text{Linear algebra} \ar[d] & \ \\
        \ &\text{Functions are the same!} & \
      \end{tikzcd}
    \]
  \end{itemize}
  \end{frame}
  \begin{frame}
    \begin{center}
      Thank you for listening!\\
      \includegraphics[scale=0.5]{images/jsf_horizontal_logo.pdf}
    \end{center}
  \end{frame}
  \begin{frame}[noframenumbering]
    \frametitle{Symmetric Functions?}
    I pulled the wool over your eyes. Our partition diagrams represent
    polynomial functions with an infinite number of variables and an infinite
    number of terms.
    \begin{block}{Dictionary}
      \[
        \begin{array}{ccc}
          \ydiagram{1}&\to h_{\ydiagram{1}}(x_1,x_2,x_3,\ldots) =
          & x_1 + x_2 + x_3 + \cdots\\
          \ydiagram{2}&\to h_{\ydiagram{2}}(x_1, x_2, x_3, \ldots) =
          & x_1^2 + x_1x_2 + x_1x_3 + \cdots \\
                      &&+ x_2^2 + x_2 x_3 + \cdots\\
                      && + x_3^2 + x_3 x_4 + \cdots \\
          \ydiagram{3}& \to h_{\ydiagram{3}}(x_1, x_2, x_3, \ldots) =
          & x_1^3 + x_1^2 x_2 + x_1^2 x_3 + \cdots\\
          \vdots
        \end{array}
      \]\[
        s_{\ydiagram{3,2,1}} = h_{\ydiagram{3,2,1}} - h_{\ydiagram{4,1,1}} -
  h_{\ydiagram{3,3}} + h_{\ydiagram{5,1}}
      \]
    \end{block}

  \end{frame}
  \begin{frame}[noframenumbering]
    \frametitle{Applications?}
\begin{picture}(150, 150)(-100, -100)
\put(-25,42){\line(-3,-5){25}}
\put(-25, 42){\line(1, 0){50}}
\put(25, 42){\line(3,-5){25}}
\put(-25, -42){\line(1,0){50}}
\put(-25, -42){\line(-3, 5){25}}
\put(25, -42){\line(3, 5){25}}

\put(-25, 42){\circle*{3}}
\put(-25, -42){\circle*{3}}
\put(25, 42){\circle*{3}}
\put(25, -42){\circle*{3}}
\put(-50, 0){\circle*{3}}

\put(50, 0){\circle*{3}}
\put(0, 3){\circle*{3}}
\put(0, -3){\circle*{3}}

\put(-28, 47){$K^{0}$}
\put(28, 47){$K^{+}$}
\put(-28, -55){$K^{-}$}
\put(28, -55){$\overline{K}^{0}$}
\put(55, 0){$\pi^{+}$}
\put(-65, 0){$\pi^{-}$}
\put(0, 10){$\pi^{0}$}
\put(0, -15){$\eta$}
\end{picture}

The ``eightfold way'' from particle physics is encoded in Schur functions by \[
  s_{\ydiagram{1,1}}(\epsilon_1, \epsilon_2)x+1
\]
\end{frame}
\end{document}