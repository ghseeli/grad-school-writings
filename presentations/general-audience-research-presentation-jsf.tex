%%%%%%%%%%%%%%%%%%%%%%%%%%%%%%%%%%%%%%%%%
% Beamer Presentation
% LaTeX Template
% Version 1.0 (10/11/12)
%
% This template has been downloaded from:
% http://www.LaTeXTemplates.com
%
% License:
% CC BY-NC-SA 3.0 (http://creativecommons.org/licenses/by-nc-sa/3.0/)
%
%%%%%%%%%%%%%%%%%%%%%%%%%%%%%%%%%%%%%%%%%

%----------------------------------------------------------------------------------------
%	PACKAGES AND THEMES
%----------------------------------------------------------------------------------------

\documentclass{beamer}

\mode<presentation> {

% The Beamer class comes with a number of default slide themes
% which change the colors and layouts of slides. Below this is a list
% of all the themes, uncomment each in turn to see what they look like.

%\usetheme{default}
%\usetheme{AnnArbor}
%\usetheme{Antibes}
%\usetheme{Bergen}
%\usetheme{Berkeley}
%\usetheme{Berlin}
%\usetheme{Boadilla}
%\usetheme{CambridgeUS}
%\usetheme{Copenhagen}
%\usetheme{Darmstadt}
%\usetheme{Dresden}
%\usetheme{Frankfurt}
%\usetheme{Goettingen}
%\usetheme{Hannover}
%\usetheme{Ilmenau}
%\usetheme{JuanLesPins}
%\usetheme{Luebeck}
\usetheme{Madrid}
%\usetheme{Malmoe}
%\usetheme{Marburg}
%\usetheme{Montpellier}
%\usetheme{PaloAlto}
%\usetheme{Pittsburgh}
%\usetheme{Rochester}
%\usetheme{Singapore}
%\usetheme{Szeged}
%\usetheme{Warsaw}

% As well as themes, the Beamer class has a number of color themes
% for any slide theme. Uncomment each of these in turn to see how it
% changes the colors of your current slide theme.

%\usecolortheme{albatross}
%\usecolortheme{beaver}
%\usecolortheme{beetle}
%\usecolortheme{crane}
%\usecolortheme{dolphin}
%\usecolortheme{dove}
%\usecolortheme{fly}
%\usecolortheme{lily}
%\usecolortheme{orchid}
%\usecolortheme{rose}
%\usecolortheme{seagull}
%\usecolortheme{seahorse}
%\usecolortheme{whale}
%\usecolortheme{wolverine}

%\setbeamertemplate{footline} % To remove the footer line in all slides uncomment this line
%\setbeamertemplate{footline}[page number] % To replace the footer line in all slides with a simple slide count uncomment this line

\setbeamertemplate{navigation symbols}{} % To remove the navigation symbols from the bottom of all slides uncomment this line
}

\usepackage{graphicx} % Allows including images
%\usepackage{booktabs} % Allows the use of \toprule, \midrule and
                      % \bottomrule in tables
\usepackage{tikz}
\usepackage{tikz-cd}
\usepackage{amsmath}
% \usepackage{../ReAdTeX/readtex-core}
% \usepackage{../ReAdTeX/readtex-dangerous}
% \usepackage{../ReAdTeX/readtex-abstract-algebra}
\usepackage{ytableau}
%%%%%%%%%%%%%%%%%%%%%%%%%%%%%%%%%%%%%%%%%%%%%%%%%%%%%%%%%%%%%%%%%%% 
%%  MACRO DEFINITIONS:  Co-authors -- PLEASE use these! 
%%%%%%%%%%%%%%%%%%%%%%%%%%%%%%%%%%%%%%%%%%%%%%%%%%%%%%%%%%%%%%%%%%%
\newcommand{\T}{\mathsf{T}} % use for tableaux
\newcommand{\U}{\mathsf{U}} % use for tableaux
\newcommand{\TA}{\mathsf{A}} % use for tableaux
\newcommand{\TB}{\mathsf{B}} % use for tableaux
\newcommand{\TC}{\mathsf{C}} % use for tableaux
\newcommand{\Std}{\operatorname{Std}} % set of standard tableaux
\newcommand{\StdB}{\operatorname{StdB}} % set of standard bitableaux
\newcommand{\Orb}{\mathcal{O}} % use for orbits
%%%%%%%%%%%%%%%%%%%%%%%%%%%%%%%%%%%%%%%%%%%%%%%%%%%%%%%%%%%%%%%%%%%%

%----------------------------------------------------------------------------------------
%	TITLE PAGE
%----------------------------------------------------------------------------------------

\title[Symmetric Functions]{Building Mathematical Bridges Between
  Symmetric Functions} % The short title appears at the bottom of every slide, the full title is only on the title page

\author[George H. Seelinger]{George H. Seelinger} % Your name
\institute[JSF] % Your institution as it will appear on the bottom of every slide, may be shorthand to save space
{
Jefferson Scholars Foundation \\ % Your institution for the title page
\medskip
\textit{ghs9ae@virginia.edu} % Your email address
}
\date{28 November 2018} % Date, can be changed to a custom date

\begin{document}

\begin{frame}
\titlepage % Print the title page as the first slide
\end{frame}
\section{Background}
\begin{frame}
  \frametitle{Partitions of \(5\)}
  How many ways can we write a positive integer as a sum of positive
  integers? \pause
  \ytableausetup{boxsize=0.5em, aligntableaux=center}
    \begin{align*}
    5 \to &\ \ydiagram{5}\\
    4+1 \to &\ \ydiagram{4,1}\\
    3+2 \to &\ \ydiagram{3,2}\\
    3+1+1 \to &\ \ydiagram{3,1,1}\\
    2+2+1 \to&\ \ydiagram{2,2,1}\\
    2+1+1+1 \to&\ \ydiagram{2,1,1,1} \\
    1+1+1+1+1 \to&\ \ydiagram{1,1,1,1,1}
    \end{align*}
  \pause We will use these diagrams to describe a type of symmetric
  function called a ``Schur function.''
\end{frame}
\begin{frame}
  \frametitle{Raising Operators}
  To do this, we will need functions that change partition diagrams
  called ``raising operators.'' \pause

  
  We can change partition diagrams by moving boxes.
  \[
    R_{1,3} \left( \ydiagram{3,1,1}*[*(red)]{0,0,1} \right) = \ydiagram{4,1}*[*(red)]{3+1}
  \]
\[
  R_{2,3} \left( \ydiagram{1,1,1}*[*(red)]{0,0,1} \right) = \ydiagram{1,2}*[*(red)]{0,1+1}
\]
\pause
If the result ``does not make sense'', we get \(0\): \[
  R_{1,4}\left( \ydiagram{1,1,1} \right) = 0
\]
\end{frame}
\section{Schur functions}
\begin{frame}
  \frametitle{Schur functions}
  We define a new class of functions. Given a partition diagram
  \(\lambda\) with 
  \(\ell\) rows, we have definition \pause
  \begin{block}{Definition}
    \begin{align*}
      s_\lambda = &(1-R_{1,2})\\
                  &(1-R_{1,3})(1-R_{2,3})\\
                  & \cdots\\
                  &(1-R_{1,
                    \ell})(1-R_{2,\ell})\cdots(1-R_{\ell-2,\ell})(1-R_{\ell-1,\ell})
                    \lambda\\
    \end{align*}
  \end{block}
  \pause
    \begin{example}
      \[
        s_{\ydiagram{3,2,1}} = (1-R_{1,2})(1-R_{1,3})(1-R_{2,3})
        \ydiagram{3,2,1}
      \]
    \end{example}
\end{frame}
\begin{frame}
  \frametitle{Example continued}
    \begin{example}
    \[
      s_{\ydiagram{3,2,1}} = (1-R_{1,2})(1-R_{1,3})(1-R_{2,3})
      \ydiagram{3,2,1}
    \]
  \end{example}\pause
  Recall the foil method from high school:
  \begin{align*}
    &(1-R_{1,2})(1-R_{1,3})(1-R_{2,3}) \\
     = &(1-R_{1,2}-R_{1,3}+R_{1,2}R_{1,3})(1-R_{2,3}) \\
     = &1-R_{1,2}-R_{1,3}-R_{2,3}+R_{1,2}R_{1,3}+R_{1,2}R_{2,3}+R_{1,3}R_{2,3}-R_{1,2}R_{1,3}R_{2,3}
  \end{align*}\pause
  So, we must compute \(s_{\ydiagram{3,2,1}} = \)\[
    (1-R_{1,2}-R_{1,3}-R_{2,3}+R_{1,2}R_{1,3}+R_{1,2}R_{2,3}+R_{1,3}R_{2,3}-R_{1,2}R_{1,3}R_{2,3})\ydiagram{3,2,1}
  \]
\end{frame}
\begin{frame}
  \frametitle{Example continued}
  \ytableausetup{boxsize=0.25em}
    \begin{example}
    \[
      s_{\ydiagram{3,2,1}} = (1-R_{1,2})(1-R_{1,3})(1-R_{2,3})
      \ydiagram{3,2,1}
    \]
  \end{example}
    \pause
    \[
  \begin{array}[center]{ccc}
    &\ydiagram{3,2,1}&\\
    -R_{1,2}(\ydiagram{3,2,1}*[*(red)]{0,1+1})&-R_{1,3}(\ydiagram{3,2,1})&-R_{2,3}(\ydiagram{3,2,1})\\
    +R_{1,2}R_{1,3}(\ydiagram{3,2,1})&+R_{1,2}R_{2,3}(\ydiagram{3,2,1})&+R_{1,3}R_{2,3}(\ydiagram{3,2,1})\\
    &-R_{1,2}R_{1,3}R_{2,3}(\ydiagram{3,2,1})
  \end{array}
  \pause =
  \begin{array}{ccc}
    &\ydiagram{3,2,1}& \\
    -\ydiagram{4,1,1}*[*(red)]{3+1}&-\ydiagram{4,2}&-\ydiagram{3,3}\\
    +\ydiagram{5,1} & +\ydiagram{4,2}&+0\\
    & -0 &
  \end{array}
\]
\pause
Adding it all together, we get
\begin{block}{Solution}
  \[
    s_{\ydiagram{3,2,1}} = \ydiagram{3,2,1} - \ydiagram{4,1,1} -
    \ydiagram{3,3} + \ydiagram{5,1}
  \]
\end{block}
\end{frame}
\begin{frame}
  \frametitle{Why Schur functions?}
  \begin{itemize}
  \item Schur functions encode the possible ways certain abstract
    algebraic objects appear in \(n\)-dimensional space. (If \(n=3\),
    we have \(3\)D space.)\pause
  \item Schur functions make computer computations easier.
  \end{itemize}
  \pause
  \begin{block}{Problem}
    However, the formula for Schur functions is complicated. If we
    have another formula for Schur functions, how can we prove they
    give the same result?
  \end{block}
\end{frame}
\section{Pieri rule}
\begin{frame}
  \frametitle{Multiplication for Symmetric Functions}
  Let us introduce a rule for multiplication of partition diagrams by ``stacking.''
\begin{block}{Rule for Multiplication (Example)}
  \(
  \ydiagram{3} \cdot \ydiagram{2} = \ydiagram{3,2} =
  \ydiagram{2,3} = \ydiagram{2} \cdot \ydiagram{3}
  \)
\end{block}
\pause
Schur functions are a sum of partition diagrams, so we can compute 
\begin{example}
  \begin{align*}
    \ydiagram[*(red)]{3} \cdot s_{\ydiagram{3,2,1}}
     = \ydiagram[*(red)]{3}  \cdot (&\ydiagram{3,2,1} - \ydiagram{4,1,1} -
      \ydiagram{3,3} + \ydiagram{5,1}) \\
     =\ &\ydiagram{3,3,2,1}*[*(red)]{3} -
      \ydiagram{4,3,1,1}*[*(red)]{0,3} -
      \ydiagram{3,3,3}*[*(red)]{3} +
      \ydiagram{5,3,1}*[*(red)]{0,3}
  \end{align*}
\end{example}\pause
\begin{block}{Problem}
  Result is in terms of partition diagrams, but we would like a result
  in terms of Schur functions.
\end{block}
\end{frame}
\begin{frame}
  \frametitle{The Pieri Rule}
  \begin{example}
    \[
      \ydiagram[*(red)]{4} \cdot s_{\ydiagram{2,1}} =
      s_{\ydiagram{2,1}*[*(red)]{2+2,1+1,1}} +
      s_{\ydiagram{2,1}*[*(red)]{2+3,0,1}} +
      s_{\ydiagram{2,1}*[*(red)]{2+3,1+1}} +
      s_{\ydiagram{2,1}*[*(red)]{2+4}}
    \]
  \end{example}
  \pause
  \begin{itemize}
  \item In general, we get the result in terms of Schur functions by
    finding all ways to add the red 
    boxes such that we only add at most one box to each column.
    \pause
    \item We call this method \emph{the Pieri rule} and it is a
    fundamental property of Schur functions.
  \end{itemize}
\end{frame}
\begin{frame}[fragile]
  \frametitle{Proof Technique}
  One approach to show two formulas for Schur functions are the same:
  \begin{block}{Proof technique}
    \[
      \begin{tikzcd}[cramped, column sep=0.01em]
        \text{Base cases are equal} \ar[rd]&\, & \ar[ld] \text{Pieri rules are the same}\\
        \ &\text{Linear algebra} \ar[d] & \ \\
        \ &\text{Functions are the same!} & \
      \end{tikzcd}
    \]
  \end{block}
\end{frame}
\section{Type C affine Stanley symmetric functions}
\begin{frame}
  \frametitle{What do I think about?}
  \begin{itemize}
  \item Most problems about Schur functions are solved. \pause
  \item Instead, I think about a class of functions called ``type C
    dual affine Stanley symmetric functions'' which have similar
    properties to Schur functions.\pause
  \item However, the current formula for these functions is not as
    concrete as the formula I gave you for Schur functions.
  \end{itemize}
\end{frame}
\begin{frame}
  \frametitle{Type C dual affine Stanley symmetric functions}
  Start with ``word'' with letters given by colors,\ytableausetup{boxsize=0.5em}
  \(\{\ydiagram[*(red)]{1}, \ydiagram[*(blue)]{1},
  \ydiagram[*(green)]{1}\}\). For example, let's use \(w = 
  \begin{ytableau}
   *(green) & *(blue) & *(green)
  \end{ytableau}
  \). \pause

  We must find all ``subword decompositions'' of \(w\) that are also subwords of \(
  \rho = \begin{ytableau}
    *(blue) & *(green) & *(blue) & *(red)
  \end{ytableau}\) or any of its ``rotations'' \(\begin{ytableau}
    *(red)& *(blue) & *(green) & *(blue) 
  \end{ytableau}, \begin{ytableau}
     *(blue) &*(red)& *(blue) & *(green)
  \end{ytableau}, \begin{ytableau}
      *(green) & *(blue) &*(red)& *(blue)
  \end{ytableau}
\). \pause
\begin{example}
  \(
  \begin{ytableau}
    *(green) & *(blue)
  \end{ytableau} |
  \begin{ytableau}
    *(green)
  \end{ytableau}
\) is a subword decomposition of \(w\) where each part appears as a
subword of \(\rho = \begin{ytableau}
    *(blue) & *(green) & *(blue) & *(red)
  \end{ytableau}\), but \(
  \begin{ytableau}
    *(green) & *(blue) & *(green)
  \end{ytableau}
\) is not a subword of \(\rho\) or any of its rotations.
\end{example}
\end{frame}
\begin{frame}
  \frametitle{Example continued}
  Then, you take all such subword decompositions to get a formula \[
  \begin{array}{c}
    \begin{ytableau}
    *(green) & *(blue)
  \end{ytableau} |
  \begin{ytableau}
    *(green)
  \end{ytableau}  \\
  \begin{ytableau}
    *(green) 
  \end{ytableau} |
  \begin{ytableau}
    *(blue) & *(green)
  \end{ytableau} \\
  \begin{ytableau}
    *(green) 
  \end{ytableau} |
  \begin{ytableau}
    *(blue)  
  \end{ytableau} |
  \begin{ytableau}
    *(green)
  \end{ytableau}
  \end{array} \to
  \ytableausetup{aligntableaux=top}
  \begin{array}{c}
    \ydiagram{2,1} \\
    \ydiagram{1,1,1}
  \end{array}
  \ytableausetup{aligntableaux=center}
  \to Q^{(2)}_{
    \begin{ytableau}
      *(green) & *(blue) & *(green)
    \end{ytableau}
  } = 4* \ydiagram{2,1} + 8 * \ydiagram{1,1,1}\] \pause
But, unfortunately, you are not done! \pause
\begin{block}{Problem}
  You then have to take the ``dual'' of this function to get the Type
  C dual affine Stanley symmetric function, \(P^{(2)}_{
    \begin{ytableau}
      *(green) & *(blue) & *(green)
    \end{ytableau}
  }\). This process is not direct and not computationally straightforward.
\end{block}
\end{frame}
\begin{frame}[fragile]
  \frametitle{What have I done?}
  \begin{itemize}
  \item I have a conjectured formula that describes type
    C dual affine Stanley symetric functions (\(P_w^{(n)}\)) directly
    using raising operators. \pause 
  
   \item Computational evidence suggests my conjecture is correct. \pause
  
   \item However, proving the formulas are the same directly would be quite
    hard, so instead I am seeking to use the Pieri rule approach \[
      \begin{tikzcd}[cramped, column sep=0.01em]
        \text{Base cases are equal} \ar[rd]&\, & \ar[ld, dashed] \text{Pieri rules are the same}\\
        \ &\text{Linear algebra} \ar[d] & \ \\
        \ &\text{Functions are the same!} & \
      \end{tikzcd}
    \]
  \end{itemize}
  \end{frame}
  \begin{frame}
    \begin{center}
      Thank you for your support and for listening!\\
      \includegraphics[scale=0.5]{images/jsf_horizontal_logo.pdf}
    \end{center}
  \end{frame}
  \begin{frame}[noframenumbering]
    \frametitle{Symmetric Functions?}
    I pulled the wool over your eyes. Our partition diagrams represent
    polynomial functions with an infinite number of variables and an infinite
    number of terms.
    \begin{block}{Dictionary}
      \[
        \begin{array}{ccc}
          \ydiagram{1}&\to h_{\ydiagram{1}}(x_1,x_2,x_3,\ldots) =
          & x_1 + x_2 + x_3 + \cdots\\
          \ydiagram{2}&\to h_{\ydiagram{2}}(x_1, x_2, x_3, \ldots) =
          & x_1^2 + x_1x_2 + x_1x_3 + \cdots \\
                      &&+ x_2^2 + x_2 x_3 + \cdots\\
                      && + x_3^2 + x_3 x_4 + \cdots \\
          \ydiagram{3}& \to h_{\ydiagram{3}}(x_1, x_2, x_3, \ldots) =
          & x_1^3 + x_1^2 x_2 + x_1^2 x_3 + \cdots\\
          \vdots
        \end{array}
      \]\[
        s_{\ydiagram{3,2,1}} = h_{\ydiagram{3,2,1}} - h_{\ydiagram{4,1,1}} -
  h_{\ydiagram{3,3}} + h_{\ydiagram{5,1}}
      \]
    \end{block}
  \end{frame}
  \begin{frame}[noframenumbering]
    \frametitle{Applications?}
\begin{picture}(150, 150)(-100, -100)
\put(-25,42){\line(-3,-5){25}}
\put(-25, 42){\line(1, 0){50}}
\put(25, 42){\line(3,-5){25}}
\put(-25, -42){\line(1,0){50}}
\put(-25, -42){\line(-3, 5){25}}
\put(25, -42){\line(3, 5){25}}

\put(-25, 42){\circle*{3}}
\put(-25, -42){\circle*{3}}
\put(25, 42){\circle*{3}}
\put(25, -42){\circle*{3}}
\put(-50, 0){\circle*{3}}

\put(50, 0){\circle*{3}}
\put(0, 3){\circle*{3}}
\put(0, -3){\circle*{3}}

\put(-28, 47){$K^{0}$}
\put(28, 47){$K^{+}$}
\put(-28, -55){$K^{-}$}
\put(28, -55){$\overline{K}^{0}$}
\put(55, 0){$\pi^{+}$}
\put(-65, 0){$\pi^{-}$}
\put(0, 10){$\pi^{0}$}
\put(0, -15){$\eta$}
\end{picture}

The ``eightfold way'' from particle physics is encoded in Schur functions by \[
  s_{\ydiagram{2,1}}(e_{\epsilon_1}, e_{\epsilon_2}, e_{\epsilon_3})
\]
\end{frame}
\end{document}