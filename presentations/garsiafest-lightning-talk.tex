%%%%%%%%%%%%%%%%%%%%%%%%%%%%%%%%%%%%%%%%%
% Beamer Presentation
% LaTeX Template
% Version 1.0 (10/11/12)
%
% This template has been downloaded from:
% http://www.LaTeXTemplates.com
%
% License:
% CC BY-NC-SA 3.0 (http://creativecommons.org/licenses/by-nc-sa/3.0/)
%
%%%%%%%%%%%%%%%%%%%%%%%%%%%%%%%%%%%%%%%%%

%----------------------------------------------------------------------------------------
%	PACKAGES AND THEMES
%----------------------------------------------------------------------------------------

\documentclass{beamer}

\mode<presentation> {

% The Beamer class comes with a number of default slide themes
% which change the colors and layouts of slides. Below this is a list
% of all the themes, uncomment each in turn to see what they look like.

%\usetheme{default}
%\usetheme{AnnArbor}
%\usetheme{Antibes}
%\usetheme{Bergen}
%\usetheme{Berkeley}
%\usetheme{Berlin}
%\usetheme{Boadilla}
%\usetheme{CambridgeUS}
%\usetheme{Copenhagen}
%\usetheme{Darmstadt}
%\usetheme{Dresden}
%\usetheme{Frankfurt}
%\usetheme{Goettingen}
%\usetheme{Hannover}
%\usetheme{Ilmenau}
%\usetheme{JuanLesPins}
%\usetheme{Luebeck}
\usetheme{Madrid}
%\usetheme{Malmoe}
%\usetheme{Marburg}
%\usetheme{Montpellier}
%\usetheme{PaloAlto}
%\usetheme{Pittsburgh}
%\usetheme{Rochester}
%\usetheme{Singapore}
%\usetheme{Szeged}
%\usetheme{Warsaw}

% As well as themes, the Beamer class has a number of color themes
% for any slide theme. Uncomment each of these in turn to see how it
% changes the colors of your current slide theme.

%\usecolortheme{albatross}
%\usecolortheme{beaver}
%\usecolortheme{beetle}
%\usecolortheme{crane}
%\usecolortheme{dolphin}
%\usecolortheme{dove}
%\usecolortheme{fly}
%\usecolortheme{lily}
%\usecolortheme{orchid}
%\usecolortheme{rose}
%\usecolortheme{seagull}
%\usecolortheme{seahorse}
%\usecolortheme{whale}
%\usecolortheme{wolverine}

%\setbeamertemplate{footline} % To remove the footer line in all slides uncomment this line
%\setbeamertemplate{footline}[page number] % To replace the footer line in all slides with a simple slide count uncomment this line

\setbeamertemplate{navigation symbols}{} % To remove the navigation symbols from the bottom of all slides uncomment this line
}

\usepackage{graphicx} % Allows including images
%\usepackage{booktabs} % Allows the use of \toprule, \midrule and
                      % \bottomrule in tables
\usepackage{tikz}
\usepackage{tikz-cd}
\usepackage{amsmath}
\usepackage{../ReAdTeX/readtex-core}
% \usepackage{../ReAdTeX/readtex-dangerous}
% \usepackage{../ReAdTeX/readtex-abstract-algebra}
\usepackage{ytableau}
%%%%%%%%%%%%%%%%%%%%%%%%%%%%%%%%%%%%%%%%%%%%%%%%%%%%%%%%%%%%%%%%%%% 
%%  MACRO DEFINITIONS:  Co-authors -- PLEASE use these! 
%%%%%%%%%%%%%%%%%%%%%%%%%%%%%%%%%%%%%%%%%%%%%%%%%%%%%%%%%%%%%%%%%%%
\definecolor{coralred}{rgb}{1.0, 0.25, 0.25}
\definecolor{lightblue}{rgb}{.68,.85,.9} % 
\DeclareMathOperator{\Gr}{Gr}
\newcommand{\cupprod}{\cup}
\newcommand{\sym}{\Lambda}
\newcommand{\lowers}{\mathcal{L}}
\newcommand{\mynone}{\ }
%%%%%%%%%%%%%%%%%%%%%%%%%%%%%%%%%%%%%%%%%%%%%%%%%%%%%%%%%%%%%%%%%%%% 


%----------------------------------------------------------------------------------------
%	TITLE PAGE
%----------------------------------------------------------------------------------------

\title[Raising operators in Schubert Calculus]{Raising operators in Schubert Calculus} % The short title appears at the bottom of every slide, the full title is only on the title page

\author[George H. Seelinger]{George H. Seelinger (joint with
  J. Blasiak and J. Morse)} % Your name
\institute[UVA] % Your institution as it will appear on the bottom of every slide, may be shorthand to save space
{
GarsiaFest 2019 \\ % Your institution for the title page
\medskip
\textit{ghs9ae@virginia.edu} % Your email address
}
\date{18 June 2019} % Date, can be changed to a custom date

\begin{document}

\begin{frame}
\titlepage % Print the title page as the first slide
\end{frame}
\begin{frame}{Overview of Schubert Calculus Combinatorics}
  Geometric problem: count intersections in projective space.
  \begin{example}
    How many lines intersect four given lines in \(\R^3\)? 
  \end{example}
  Solution given in cohomology of Grassmannian!
  \[
    \sigma_\lambda \cupprod \sigma_\mu = \sum_\nu c_{\lambda \mu}^\nu \sigma_\nu
  \]
  where \(c_{\lambda \mu}^\nu = \) number of points in intersection of
  Schubert varieties.
\end{frame}
\begin{frame}{Overview of Schubert Calculus Combinatorics (cont.)}
  We can realize these bases explicitly
  \begin{block}{Fact}
    There exists a surjection of \(\Z\)-algebras
    \begin{align*}
      \sym_m & \onto H^*(\Gr_{m,n})\\
      s_\lambda & \mapsto
                  \begin{cases}
                    \sigma_\lambda & \text{ if }\lambda \subset (n^m)\\
                    0 & \text{ otherwise }
                  \end{cases}
    \end{align*}
  \end{block}
  Upshot:
  \begin{itemize}
  \item Cohomology of Grassmannian \(\correspondsto\) symmetric
    function quotient
  \item Schubert structure constants \(\correspondsto\) Schur LR
    coefficients (\(c_{\lambda \mu}^\nu\))
  \end{itemize}
\end{frame}
\begin{frame}{Overview of Schubert Calculus Combinatorics (cont.)}
  \begin{block}{Combinatorial Program}
    \begin{enumerate}
    \item Describe explicit bases for cohomology ring.\\
      eg Schur basis for \(H^*(\Gr_{m,n})\)
    \item Describe Pieri rule combinatorially (completely describes
      structure constants). \\
      ie rule for summands in \(\sigma_{(r)} \sigma_\lambda = \sum_\mu \sigma_\mu\)
    \item Describe Littlewood-Richardson coefficients combinatorially. \\
      ie rule for \(c_{\lambda \mu}^\nu\) in \(\sigma_\mu
      \sigma_\lambda = \sum_\nu c_{\lambda,\mu}^\nu 
      \sigma_\nu\) 
    \end{enumerate}
  \end{block}
\end{frame}
\begin{frame}{Schubert Calculus Variations}
  \begin{tabular}{c|p{2.25cm}|p{2.5cm}|p{2.75cm}}
    Cohomology theory
    & Explicit basis
    & Pieri rule
    & LR coefficients \\
    \hline
    Type A
    & Schur
    & Horiz strips
    & Skew Yamanouchi tableaux \\
    \(K\)-theory
    & Grothendieck
    & Set-valued strips
    & Skew set-valued Yamanouchi tableaux \\
    Types BCD
    & Billey-Haiman
    & BH
    & ?? \\
    Quantum
    & ??
    & ??
    & ?? \\
    Affine Type A
    & \(t=1\) \(k\)-Schur 
    & Affine horiz strips
    & Gromov-Witten Invariants?\\
    Affine \(K\)-theory
    & \textcolor{red}{Only indirect}
    & Affine set-valued tableaux
    & Unkown \\
    Affine Type C
    & LSS
    & Unknown
    & Unknown
    \\
  \end{tabular}
\end{frame}
\begin{frame}{Raising Operators and Root Ideals}
  \begin{itemize}
  \item Raising operators
    \(R_{i,j}(h_\lambda) = h_{\lambda+\epsilon_i-\epsilon_j}\)
    \ytableausetup{boxsize=0.5em}
    \[
      R_{1,3} \left( \ydiagram{1,1,3}*[*(red)]{1} \right) =
      \ydiagram{1,4}*[*(red)]{0,3+1} \ \ \ R_{2,3} \left(
        \ydiagram{1,1,1}*[*(red)]{1} \right) =
      \ydiagram{2,1}*[*(red)]{1+1}
    \]
  \item Lowering Operators
     \(L_j(h_\lambda) = h_{\lambda-\epsilon_j}\)
            \ytableausetup{boxsize=0.6em}
            \[ L_3\left( \ydiagram{1,1,3}*[*(red)]{1} \right) =
              \ydiagram{1,3} \ \ L_{1}\left( \ydiagram{1,1,3}*[*(red)]{0,0,2+1} \right)
              = \ydiagram{1,1,2}
            \]
  \item Root ideal \(\Psi\): given by Dyck path.
            \ytableausetup{mathmode, boxsize=1em,centertableaux}
            \[
              \Psi =
              \begin{ytableau}
                \mynone &*(lightblue)\text{\tiny (12)}  &*(red)\text{\tiny (13)}   &*(red)\text{\tiny (14)}  &*(red)
                \text{\tiny (15)}\\
                \mynone &\mynone &*(red) \text{\tiny (23)}  &*(red)\text{\tiny (24)}
                &*(red) \text{\tiny (25)}\\
                \mynone &\mynone &\mynone &*(lightblue) \text{\tiny (34)}
                &*(red)\text{\tiny (35)} \\
                \mynone &\mynone &\mynone&\mynone&*(lightblue) \text{\tiny (45)}\\
                \mynone &\mynone &\mynone&\mynone&\mynone\\
              \end{ytableau}
              \ \ \ \overset{\text{\textcolor{red}{\scriptsize{Roots above Dyck
                    path}}}}{\text{\textcolor{blue}{\scriptsize{Non-roots below}}}}
            \]
  \end{itemize}
\end{frame}
\begin{frame}{Affine \(K\)-Theory Representatives with Raising Operators}
  \begin{definition}
    Let \(\Psi,\lowers \subset \Delta^+_\ell\) be order ideals of
    positive roots and \(\gamma \in \Z^\ell\), then \[
      K(\Psi;\lowers;\gamma) := \prod_{(i,j) \in \lowers} (1-L_j)
      \prod_{(i,j) \in \Delta^+_\ell \setminus \Psi} (1-R_{ij})
      Kh_\gamma
    \]
    for \(Kh_\gamma\) an inhomogeneous analogue of \(h_\gamma\).
  \end{definition}
  \begin{example}
    \(\textcolor{blue}{\text{non-roots of }\Psi},
              \textcolor{coralred}{\text{roots of }\lowers}\)
              \begin{columns}
                \begin{column}{0.35\textwidth}
                  \ytableausetup{mathmode,
                    boxsize=1em,centertableaux} \[
                    \begin{ytableau}
                      \mynone &*(lightblue)\text{\tiny (12)}
                      &*(white)\text{\tiny (13)}
                      &*(coralred)\text{\tiny (14)} &*(coralred)
                      \text{\tiny (15)}\\
                      \mynone &\mynone &*(white) \text{\tiny (23)}
                      &*(coralred)\text{\tiny (24)}
                      &*(coralred) \text{\tiny (25)}\\
                      \mynone &\mynone &\mynone &*(lightblue)
                      \text{\tiny (34)}
                      &*(white)\text{\tiny (35)} \\
                      \mynone &\mynone &\mynone&\mynone&*(lightblue) \text{\tiny (45)}\\
                      \mynone &\mynone &\mynone&\mynone&\mynone\\
                    \end{ytableau}
                  \]
                \end{column}
                \begin{column}{0.65\textwidth}
                  \begin{align*}
                    & K(\Psi;\lowers;54332) \\
                    & = \textcolor{coralred}{(1-L_{4})^2(1-L_{5})^2}
                    \\
                    & \cdot \textcolor{blue}{(1-R_{12})(1-R_{34})(1-R_{45})} Kh_{54332}
                  \end{align*}
                \end{column}
              \end{columns}
  \end{example}
\end{frame}
\begin{frame}{Affine \(K\)-Theory Representatives with Raising
    Operators}
  \begin{definition}
    The \emph{\(k\)-Schur root ideal}, \(\Delta^{(k)}(\lambda)\) is the
    unique root ideal with \(\lambda_i + \#\)non-roots in row \(i =
    k\).
  \end{definition}
              \[
              \Delta^{(4)}(332111) = 
              {\footnotesize
                \begin{ytableau}
                  *(white) 3     &*(lightblue)  &*(red)   &*(red)  &*(red)  &*(red) \\
                  \mynone & *(white)3 & *(lightblue) & *(red) & *(red)  &*(red)  \\
                  \mynone &*(white)  & *(white)2 & *(lightblue) & *(lightblue)  &*(red)  \\
                  \mynone &*(white)  & *(white)  & *(white)1 & *(lightblue) &*(lightblue) \\
                  \mynone &\mynone  &\mynone  &\mynone  & *(white)1 & *(lightblue) \\
                  \mynone &\mynone  &\mynone  &\mynone  &*(white)  & *(white) 1
                \end{ytableau}
              }
              \leftarrow\,\text{\small{4 - 2 non-roots}}
              \quad
            \]
  \begin{theorem}[Blasiak-Morse-S., 2019]
    \begin{enumerate}
    \item \(g_\lambda^{(k)} = K(\Delta^{(k)}(\lambda);
      \Delta^{(k+1)}(\lambda);\lambda)\)
      \item The \(g_\lambda^{(k)}\) are ``shift
        invariant'', ie for \(\ell = \ell(\lambda)\)
      \[
        G_{1^\ell}^\perp g_{\lambda+1^\ell}^{(k+1)} = g_\lambda^{(k)}
      \]
    \end{enumerate}
  \end{theorem}
\end{frame}
\end{document}