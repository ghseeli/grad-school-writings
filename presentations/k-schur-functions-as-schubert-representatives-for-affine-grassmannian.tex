\documentclass[11pt,leqno,oneside]{amsart}
\usepackage[alphabetic,abbrev]{amsrefs} % use AMS ref scheme
\usepackage{./presentation-notes}
\usepackage{../ReAdTeX/readtex-core}
\usepackage{../ReAdTeX/readtex-dangerous}
\usepackage{../ReAdTeX/readtex-abstract-algebra}
\usepackage{../ReAdTeX/readtex-lie-algebras}
\usepackage{caption}
\usepackage{subcaption}
\usepackage{todonotes}
\usepackage{ytableau}
\usepackage{xcolor}
\usepackage{mathtools}
\usepackage{tikz-cd}
\usepackage{stmaryrd}
\numberwithin{thm}{section}

\renewcommand{\W}{W}
\newcommand{\Waff}{\W_{\text{aff}}}
\newcommand{\Wfin}{\W_{\text{fin}}}
\newcommand{\A}{\mathbb{A}}
\newcommand{\Aaff}{\A_{\text{aff}}}
\newcommand{\B}{\mathbb{B}}
\renewcommand{\H}{\mathcal{H}} % Hecke algebra
\newcommand{\eS}{\tilde{S}}
\newcommand{\Gr}{\operatorname{Gr}}
\newcommand{\Bc}{\mathcal{B}}
\DeclareMathOperator{\ch}{ch}
\newcommand{\sym}{\Lambda}
\DeclareMathOperator{\dg}{dg}

\newcommand{\trace}{\operatorname{Tr}}
\newcommand{\transpose}{t}
\DeclareMathOperator{\SSYT}{SSYT}
\DeclareMathOperator{\wt}{wt}
\DeclareMathOperator{\Frac}{Frac}
\newcommand{\coveredby}{\mathrel{\lessdot}}

\newtheorem{goal}[thm]{Goal}

\title[\(k\)-Schur functions]{\(k\)-Schur functions as Schubert
  representatives for the affine Grassmannian} 
\author{George H. Seelinger}
\date{April 13, 2020}
\begin{document}
\maketitle
\section{Introduction}
Classically, the Grassmannian of \(m\)-planes in \(\C^{m+n}\), denoted
\(X = \Gr(m,n)\), has a decomposition into Schubert cells,
\(\Omega_\lambda\) for partitions \(\lambda = (\lambda_1, \ldots,
\lambda_m)\) with \(\lambda_i \leq n\). One can then define Schubert varieties
\(X_\lambda = \ov{\Omega_\lambda}\), the closure of a Schubert
cell. In turn, this gives a basis 
for the cohomology ring \(H^*(X) = \bigoplus_\lambda \Z
\sigma_\lambda\) for \(\sigma_\lambda\) a representative of
\(X_\lambda\). Then, the coefficients in the product of
\(\sigma_\lambda \sigma_\mu\) expressed as 
a linear combination of Schubert representatives contain useful
geometric information.

On the other hand, the ring of symmetric functions \(\sym = \Z[
h_1(x), h_2(x), \ldots]\) for \(h_d(x) = \sum_{i_1 \leq i_2 \leq \cdots \leq
  i_d} x_{i_1} x_{i_2} \cdots x_{i_d}\) has a distinguished basis of
Schur functions indexed by partitions \(\lambda\) given by \[
  s_\lambda = \det(h_{\lambda_i+j-i})_{1 \leq i,j \leq \ell}
\]
where \(h_0(x) = 1\) and \(h_r = 0\) for \(r < 0\) by convention. The
``Littlewood-Richardson coefficients''
\(c_{\lambda \mu}^\nu\) in the expansion \(s_\lambda s_\mu = \sum_\nu
c_{\lambda \mu}^\nu s_\nu\) have many combinatorial formulas and are
well-studied. Thus, the following theorem is quite useful for
connecting geometry and combinatorics.
\begin{thm}
  There is a surjection of rings \(f \from \sym \to H^*(\Gr(m,n))\)
  given by \[
    f(s_\lambda) =
    \begin{cases}
      \sigma_\lambda & \text{if }\lambda \subset (\underbrace{n,
        \ldots, n}_{m \text{ times}}) \\
      0 & \text{else}
    \end{cases}
  \]
\end{thm}
In this presentation, we seek to summarize how a special class of
symmetric functions called ``\(k\)-Schur functions'', denoted
\(s_\lambda^{(k)}(x)\), play a similar role for the affine Grassmannian
of \(SL_{k+1}\) and then explore their combinatorics. We will make
precise and summarize the
original proof of~\cite{lam} that \(k\)-Schur functions are the
Schubert representatives of the affine Grassmannian in
Sections~\ref{preliminaries}--\ref{schubert-representatives},
culminating in Theorem~\ref{k-schur-functions-are-schubert-reps}. Then, in
Section~\ref{combinatorics}, we will present a more combinatorial
treatment of \(k\)-Schur functions. These two parts can be read
more-or-less independently.
\section{Preliminaries}\label{preliminaries}
\begin{defn}
  The following is a summary of definitions in~\cite{lam}*{Section 2.1}.
  \begin{itemize}
  \item Let \(\W\) be a crystallographic Coxeter group with simple
    generators \(\{s_i \st i \in I\}\).
  \item Let \(\roots\) be the root system for \(\W\)
  \item Let \(\roots^+\) be the positive roots and \(\{\alpha_i \st i
    \in I\}\) be the simple roots.
  \item Let \(Q = \bigoplus_{i \in I} \Z \alpha_i\) be the root
    lattice and \(Q^\vee = \bigoplus_{i \in I} \Z \alpha_i^\vee\) be
    the co-root lattice.
  \item Let \(\h_\Z^*\) and \(\h_\Z\) be the weight and co-weight
    lattice.
  \item Let \(\langle \cdot, \cdot \rangle \from \h_\Z \times \h_\Z^*
    \to \Z\) be the pairing between \(\h_\Z\) and \(\h_\Z^*\)
  \item Let \(\Waff = \W \ltimes Q^\vee\) be the affine Weyl group with
    additional generator \(s_0\).
  \item For \(\lambda \in Q^\vee\), let \(t_\lambda \in \Waff\) be the
    corresponding translation element. Note \(t_\lambda \cdot t_\mu =
    t_{\lambda+\mu}\) and \(w t_\lambda w^{-1} = t_{w \cdot \lambda}\)
    for \(w \in \W\).
  \item For real root \(\alpha \in Q_{\text{aff}} = \bigoplus_{i \in I
    \union \{0\}} \Z \alpha_i\), let \(s_\alpha\) denote the
  corresponding reflection.
  \item Let \(\ell \from \Waff \to \N_{\geq 0}\) be the length
    function for \(\Waff\) and \(\epsilon(w) = (-1)^{\ell(w)}\) be the
    sign. 
  \item Let \(\W^0\) be the minimal length coset representatives of
    \(\Waff/\W\), called \de{Grassmannian} elements. 
  \end{itemize}
\end{defn}
\begin{prop}
  There exists a natural bijection between \(\W^0\) and \(Q^\vee\).
\end{prop}
\begin{proof}
  Each coset of \(\Waff/\W\) contains a unique element of \(Q^\vee\)
  and a unique minimal length coset representative.
\end{proof}
\subsection{The affine Grassmannian}
Let \(G\) be a simple and simply connected complex algebraic group
with Weyl group \(\W\). Let \(K\) be a maximal compact subgroup and
let \(T\) be a maximal torus in \(K\). Let \(\h_\Z^\vee\) be the
weight lattice and \(\h_\Z\) be the co-weight lattice of \(T\). In our
case, we are primarily interested in type \(A\) with \(G = SL_n\), \(\W = S_n\), \(K =
SU_n\), \(T = (\C^\times)^n\), \(\Waff = \eS_n\) is the affine symmetric group, and
affine simple roots are given by \(\{\alpha_i \st i \in \Z/n\Z\}\).

For \(F = \C((t))\) and \(O = \C \llbracket t \rrbracket\), we define
\(\Gr = \Gr_G = G(F)/G(O)\) to be the affine Grassmannian.
\begin{prop}
 \(\Gr\) is homotopy equivalent to \(\Omega K\). Thus, the two spaces
 have isomorphic (co)homology theories.
\end{prop}
Let \(S = \operatorname{Sym}(\h_\Z^*)\) be the
symmetric algebra of \(\h^*_\Z\). From my last presentation, we know
that \(G(F)\) has Bruhat decomposition \[
  G(F) = \Union_{w \in \Waff} \Bc w \Bc
\]
for Iwahori subgroup \(\Bc\). This induces a decomposition of \(\Gr\)
into Schubert cells \(\Omega_w = \Bc w G(O) \subset G(K)/G(O)\): \[
  \Gr = \Dunion_{w \in \W^0} \Omega_w = \Union_{w \in W^0} X_w
\]
where \(X_w = \ov{\Omega_w}\) are
Schubert varieties. Then, we define the following
\begin{defn}
  We denote the following Schubert classes in homology, cohomology,
  equivariant homology, and equivariant cohomology as follows.
  \begin{enumerate}
  \item Let \(\sigma_w \in H_*(\Gr)\);
  \item Let \(\sigma^w \in H^*(\Gr)\);
  \item Let \(\sigma_{(w)} \in H_T(\Gr)\), the \(T\)-equivariant
    homology of \(\Gr\);
  \item Let \(\sigma^{(w)} \in H^T(\Gr)\), the \(T\)-equivariant
    cohomology of \(\Gr\).
  \end{enumerate}
\end{defn}
\begin{rmk}
  Note that \(S = \operatorname{Sym}(\h_\Z^*) = H^T(\text{pt})\).
\end{rmk}
\section{Four Hopf Algebras}
To show that (dual) \(k\)-Schur functions are Schubert representatives for
the (co)homology of the affine Grassmannian, we will string together
various results concerning four different Hopf algebras.
\subsection{The affine nilHecke ring}
\begin{defn}
  \begin{enumerate}
  \item Define \(\Aaff\) to be the ring with \(1\) given by generators
    \(\{A_i \st i \in I \union \{0\}\} \union \{\lambda \st \lambda
    \in \h_\Z^*\}\) and relations
    \begin{align*}
      A_i \lambda
      & = (s_i \cdot \lambda) A_i + \langle \lambda,
        \alpha_i^\vee \rangle \cdot 1 & \text{for }\lambda \in \h_\Z^*\\
      A_i A_i & = 0 \\
      (A_i A_j)^m & = (A_jA_i)^m & \text{if }(s_i s_j)^m = (s_j s_i)^m
    \end{align*}
    where the elements of \(\h_\Z^*\) commute with each other.
  \item For \(w \in \Waff\), let \(w = s_{i_1} \cdots s_{i_\ell}\) be
    a reduced word for \(w\). Then, we define \(A_w := A_{i_1} \cdots A_{i_\ell}\).
  \item Let \(\A_0 = \Z[A_i \st i \in I \union \{0\}] \subset \Aaff\)
    be the \de{affine nilCoxeter algebra} generated by the \(A_i\)'s.
  \end{enumerate}
\end{defn}
There is a specialization map \(\phi_0 \from \Aaff \to \A_0\) given
by \[
  \phi_0\left( \sum_{w} a_w A_w\right) = \sum_w \phi_0(a_w) A_w
\]
where \(\phi_0(s)\) evaluates the polynomial \(s \in S =
\operatorname{Sym}(\h_\Z^*)\) at \(0\). (Recall elements in
\(\h_\Z^*\) are maps from \(\h_\Z\) to \(\Z\).)
\begin{prop}
  \(\Aaff\) is a Hopf algebra with coprodcut map \(\Delta \from \Aaff
  \to \Aaff \otimes_S \Aaff\) given by
  \begin{align*}
    \Delta(s) & = 1 \otimes s = s \otimes 1 & s \in S\\
    \Delta(A_i) & = A_i \otimes 1 + 1 \otimes A_i - A_i \otimes \alpha_i A_i
  \end{align*}
\end{prop}
\subsection{Hopf algebra structure of \(H_T(\Omega K)\) and Peterson's \(j\)-Homomorphism}
Over \(\Frac(S)\), \(H_T(\Omega K)\) is spanned by the classes
\(\{\psi_t \st t = t_\lambda \in Q^\vee \subset \Waff\}\). Since
\(\Omega K\) is a group with \(T\)-equivariant multiplications,
\begin{prop}
  \(H_T(\Omega K)\) and \(H^T(\Omega K)\) have the structure of dal
  Hopf algebras. Furthermore, \(H_T(\Omega K)\) has Hopf algebra
  structure given by
  \begin{align*}
    \psi_{id} & = 1& \epsilon(\psi_t) & = 1& c(\psi_t) = \psi_{t^{-1}} \\
    \omega(\psi_t) & = \psi_t \otimes \psi_t & \psi_t \psi_{t'} &=
                                             \psi_{tt'}
  \end{align*}
  with scalars \(s \in S \subset H_T(\Omega K)\) central and \(c(s) =
  s\). In particular, \(H_T(\Omega K)\) is commutative and
  co-commutative as a Hopf algebra.
\end{prop}
\begin{defn}
  Let the \de{Peterson subalgebra} be \(Z_{\Aaff}(S)\), the
  centralizer of \(S\) in \(\Aaff\).
\end{defn}
\begin{prop}
  There is a Hopf algebra isomorphism \(j \from H_T(\Omega K) \to
  Z_{\Aaff}(S)\) sending classes \(\psi_{t_\lambda} \mapsto
  t_\lambda\) where \(t_\lambda \in Q^\vee \subset \Waff\).
\end{prop}
\subsection{Affine Fomin-Stanley subalgebra}
\begin{defn}
  The \de{affine Fomin Stanley subalgebra} is the subalgebra \[
    \B' = \{a \in \A_0 \st \phi_0(as) = \phi_0(s)a \text{ for all }s
    \in S\}
  \]
\end{defn}
The algebra \(\B'\) is a model for the homology \(H_*(\Gr)\) in the
following sense.
\begin{prop}\label{affine-fomin-stanley-isom-to-homology}
  Granting that \(\phi_0(j(\sigma_{(u)})) \in \B'\),
  \begin{enumerate}
  \item The map
    \begin{align*}
      H_*(\Gr) & \to \B' \\
      \sigma_u & \mapsto \phi_0(j(\sigma_{(u)}))
    \end{align*}
    is an isomorphism of Hopf algebras.
  \item \(\phi_0(j(\sigma_{(u)}))\) is the unique element in \(\B'\)
    with unique Grassmannian term \(A_u\).
  \end{enumerate}
\end{prop}
\begin{cor}
  \(\B'\) is a commutative algebra.
\end{cor}
\subsection{Combiatorial affine Fomin-Stanley subalgebra and symmetric
  functions}
From now on, we will restrict ourselves to type \(A\).
\begin{defn}
  We construct the combinatorial affine Fomin-Stanley subalgebra:
  \begin{enumerate}
  \item We say a word \(a\) in alphabet \(\Z/n\Z\) is \de{cyclically
      decreasing} if no letter is repeated and, if \(i,i+1\) both
    occur in \(a\), then \(i+1\) occurs to the west of \(i\). 
  \item Define \(h_i \in \A_0 \subset \Aaff\) for \(i \in [0,n-1]\) by
    \[
      h_i = \sum_{\substack{w \in \Waff \\ w \text{ cyclically decreasing} \\ \ell(w) = i}} A_w
    \]
  \item Let \(\B = \langle h_i \st i \in [0,n-1] \rangle\) be the
    \de{combinatorial affine Fomin-Stanley subalgebra} of \(\A_0\).
  \end{enumerate}
\end{defn}
\begin{example}
  The word \(s_1s_0s_2 \in \eS_3\) is cyclically decreasing but \(s_1
  s_2 s_0\) is not. We also have that, for \(n=4\), \[
    h_2 = A_3 A_2 + A_3 A_1 + A_0 A_3 + A_2 A_1 + A_2 A_0 + A_1 A_0
  \]
\end{example}
\begin{prop}
  \(\B\) is commutative and isomorphic to \(\sym_n =
  \Z[h_0,h_1,\ldots,h_{n-1}]\) via
  \begin{align*}
    \psi \from \sym_n & \to \B \\
    h_i(x) & \mapsto h_i
  \end{align*}
\end{prop}
\begin{defn}\label{k-schur-def-1}
  Let \(\langle \cdot, \cdot \rangle \from \A_0 \times \A_0 \to
    \Z\) be given by \(\langle A_w, A_v \rangle = \delta_{w,v}\). We
    define the following symmetric functions.
  \begin{enumerate}
  \item Let \(w \in \Waff\). We define the \de{affine Stanley
      symmetric functions} \(\tilde{F}_w(x) \in \sym\) by
    \[ \tilde{F}_w(x) = \sum_{a = (a_1, a_2, \ldots, a_t)} \langle
      h_{a_t} h_{a_{t-1}} \cdots h_{a_1} \cdot 1, A_w \rangle
      x_1^{a_1} x_2^{a_2} \cdots x_t^{a_t}
    \]
    where the sum is over compositions of \(\ell(w)\) satisfying
    \(a_i \in [0,n-1]\).
  \item Define their image in \(\sym^n = \sym/\langle m_\lambda(x) \st
    \lambda_1 \geq n\rangle\) to be the \de{affine Schur functions}
  \item The \de{\(k\)-Schur functions} \(\{s_w^{(k)}(x) \st w \in \W^0\}\)
  are the dual basis of \(\sym_n\) to the affine Schur functions under
  the Hall inner product where \(k = n-1\).
  \item Let the \de{non-commutative \(k\)-Schur functions} be given
    by \[
      s_w^{(k)} := \psi(s_w^{(k)}(x)) \in \B
    \]
  \end{enumerate}
\end{defn}
\begin{rmk}
  Affine Stanley symmetric functions ``enumerate'' all the ways of
  factoring reduced words for \(w \in \Waff\) into groups of
  ``cyclically decreasing'' words. In particular, the coefficient of \(x_1
  \cdots x_\ell\) gives the number of reduced words for \(w \in \Waff\).
\end{rmk}
\begin{example}
  For \(n=3=k+1\), we compute some non-commutative \(k\)-Schur functions.
  \begin{align*}
    s_{id}^{(k)} & = 1 \\
    s_{s_0}^{(k)} & = h_1 = A_0+A_1+a_@ \\
    s_{s_1 s_0}^{(k)} & = h_2 = A_{02}+A_{21}+A_{10}\\
    s_{s_2 s_0}^{(k)} & = h_1^2-h_2=A_{20}+A_{12}+A_{01}\\
    s_{s_2 s_1 s_0}^{(k)} & = h_2h_1 = h_1 h_2 = A_{021}+A_{010}+A_{102}+A_{121}+A_{202}+A_{210}\\
    s_{s_1 s_2 s_0}^{(k)} & = h_1^3-h_2 h_1 = A_{120}+A_{010}+A_{201}+A_{121}+A_{202}+A_{012}
  \end{align*}
\end{example}
\begin{prop}\label{k-schur-subspace-props}
  We observe the following.
  \begin{enumerate}
  \item \(s_w^{(k)}\) has a unique Grassmannian term \(A_w\).
  \item \(\sym_n\) and \(\sym^n\) are dual Hopf-algebras under the
    Hall-inner product. Their Hopf algebra structure is inherited from
    the Hopf algebra structure of \(\sym\) with Hopf structure
    \begin{align*}
      \Delta_{\sym}(h_i(x)) & = \sum_{j \leq i} h_j(x) \otimes h_i(y)
      & \epsilon(f(x)) & = f(0) & c(h_i(x)) = (-1)^i e_i(x)
    \end{align*}
  \item \(\psi\) is an isomorphism of Hopf algebras. Thus, \(h_0 =
    A_{id}\) is the unit, \(\epsilon(b)\) is the coefficient of
    \(h_0\) when \(b\) is written as a polynomial in the \(h_i\)'s,
    and \[
      \Delta_\B(h_i) = \sum_{j \leq i} h_j \otimes h_{i-j} \,.
    \]
  \end{enumerate}
\end{prop}
\begin{thm}[\cite{lam}*{Theorem 7.4}]\label{affine-fomin-stanley-equal}
  \(\B\) and \(\B'\) are identical as subalgebras of
  \(\A_0\). Furthermore, the two Hopf structures agree and we have for
  each \(w \in W^0\), \[
    \phi_0(j(\sigma_{(w)})) = s_w^{(k)} \,.
  \]
\end{thm}
\begin{proof}[Sketch of subalgebra equality]
  We will grant that \(B \subset \B'\) (for the non-trivial proof,
  see~\cite{lam}*{Subsection
    7.1}). By Proposition~\ref{k-schur-subspace-props}(a), \(s_w^{(k)} \in \B\)
  has a 
  unique Grassmannain term \(A_w\) and by
  Proposition~\ref{affine-fomin-stanley-isom-to-homology}(b), 
  \(\phi_0(j(\sigma_{(w)}))\) is the unique element in \(\B'\) with
  unique Grassmannian term \(A_w\). Thus, \(\phi_0(j(\sigma_{(w)})) =
  s_w^{(k)}\) and so the \(s_w^{(k)}\)'s span \(\B'\). Thus, \(\B = \B'\).
\end{proof}
\section{\(k\)-Schur functions as Schubert representatives}\label{schubert-representatives}
\begin{thm}{\cite{lam}*{Theorem 7.1}}\label{k-schur-functions-are-schubert-reps}
  \begin{enumerate}
  \item The map \(\theta \from H_*(\Gr) \to \sym_n\) given by
    \[
      \theta(\sigma_w) \mapsto s_w^{(k)}(x)
    \]
    is an isomorphism of Hopf algebras.
  \item The map
    \(\theta' \from H^*(\Gr) \to \sym^n\) given by
    \[
      \theta'(\sigma^w) = \tilde{F}_w(x)
    \]
    is an isomorphism of Hopf algebras.
  \end{enumerate}
\end{thm}
\begin{proof}
  Part (a) follows by compositing the following chain of isomorphisms \[
    H_*(\Gr) \isomto \B' \overset{\ref{affine-fomin-stanley-equal}}{=}
    \B \to[\psi^{-1}] \sym_n 
  \]
  and the statement or part (b) follows by duality.
\end{proof}
\section{Combinatorics of \(k\)-Schur functions}\label{combinatorics}
Historically, \(k\)-Schur functions were first introduced for the
study of Macdonald polynomials in~\cite{llm}. From the symmetric function
perspective, \(k\)-Schur functions have an extra parameter \(t\) and
were constructed to satisfy the following conjecture.
\begin{conj}[\cite{llm}]
  For \(\mu\) a partition let \(H_\mu(x;q,t)\) be the ``modified''
  Macdonaly polynomials. Then, for any \(k \geq \mu_1\),
  \begin{align*}
    H_\mu(x;q,t)
    &= \sum_\nu K_{\nu\mu}^{(k)}(q,t) s_\nu^{(k)}(x;t)
    & s_\nu^{(k)} = \sum_\lambda \pi_{\lambda \nu}^{(k)}(t) s_\lambda(x)
  \end{align*}
  satisfy \(K_{\nu \mu}^{(k)}(q,t) \in \N[q,t]\) and \(\pi_{\lambda
    \nu}^{(k)}(t) \in \N[t]\). 
\end{conj}
Note that~\cite{catalans} established that \(\pi_{\lambda \nu}^{(k)}
\in \N[t]\), thus establishing one part of the conjecture. In this
presentation, we will specialize \(t=1\), thereby recovering the
\(k\)-Schur functions from the previous sections. The following
propositions and definitions come from~\cite{lm1, lm2}, but our
treatment is a summary of~\cite{k-schur-book}.
\begin{defn}
  We define the following types of partitions.
  \begin{enumerate}
  \item A \(k\)-bounded partition \(\lambda\) is a partition with
    \(\lambda_1 \leq k\).
  \item An \(r\)-core is a partition with where none of its cells
    have a hook-length equal to \(r\).
  \end{enumerate}
\end{defn}
\begin{example}
  Given two shapes \[
    \ydiagram{5,2,1} \ \ \ydiagram{4,3}*[\bullet]{1+3,1+1}
  \]
  the left shape is a \(4\)-core but the right shape is not. A hook of
  size \(4\) is given by the dotted cells.
\end{example}
\begin{prop}
  There is a bijection \(\mathfrak{p}\) between the set of \((k+1)\)-cores \(\kappa\)
  and \(k\)-bounded partitions given by deleting all cells of hook
  length greater than \(k+1\) in \(\kappa\) and left justifying the
  remaining cells.
\end{prop}
Such a bijection is best understood via an example.
\begin{example}
  For \(k+1=4\), highlight the cells of hook length \(\leq 3\) and
  remove the remaining ones. Then, left justify.
  \[
    \ydiagram{5,2,1}*[*(lightgray)]{2} \mapsto
    \ydiagram{2+3,2,1} \mapsto \ydiagram{3,2,1}
  \]
\end{example}
Now, recall the following.
\begin{defn}
  The \de{affine symmetric group} \(\eS_n\) is given by the generators
  \(\{s_i \st i \in \Z/n\Z\}\) satisfying the relations
  \begin{align*}
    s_i^2 & = 1, & s_i s_{i+1} s_i & = s_{i+1} s_i s_{i+1}, & s_i s_j
    & = s_j s_i & \text{for }i-j \not\congruent 0,1,n-1 \mod n
  \end{align*}
  with all indices considered modulo \(n\).
\end{defn}
\begin{prop}
  The Grassmannian elements of \(\eS_n\) are the minimal length coset
  representatives of \(\eS_n / S_n\) for \(S_n = \langle s_1, \ldots,
  s_n \rangle\) and are characterized by having all reduced words
  beginning with \(s_0\).
\end{prop}
\begin{defn}
  \begin{enumerate}
  \item Given a diagram representing a partition, say \(\dg(\mu)\), the \de{content} of
    a cell \(c = (i,j)\) is given by \(j-i\). 
  \item For a \(k+1\)-core
    \(\kappa\), the \de{residue} of a cell \((i,j)\) is given by
    \(j-i \mod k+1\).
  \item We say a cell \(c = (i,j)\) is an \de{addable corner} of a
    partition \(\mu\) if \(\dg(\mu) \union \{c\}\) is a diagram for a partition.
  \item We say a cell \(c = (i,j)\) is a \de{removable corner} of a
    partition \(\mu\) if \(\dg(\mu) \setminus \{c\}\) is a diagram for
    a partition.
  \item For a \((k+1)\)-core, we call an addable (resp.\ removable)
    corner of residue \(r\) and \de{addable (resp.\ removable) \(r\)-corner}.
  \end{enumerate}
\end{defn}
\begin{example}
  Consider the following \(4\)-core with labelled residues. \[
    \kappa = \ytableaushort{01230,30,2}
  \]
  \(\kappa\) has addable corners of residues \(1\) and \(3\) and
  removable corners of residues \(0\) and \(2\).
\end{example}
Then, we define the following action of \(\eS_{k+1}\) on
\(k+1\)-cores.
\begin{defn}
  Given \(s_r \in \eS_{k+1}\), we have \[
    s_r \cdot \kappa =
    \begin{cases}
      \kappa + \text{all its addable $r$-corners} & \text{if $\kappa$
        has at least one addable $r$-corner}\\
      \kappa - \text{all its removable $r$-corners} & \text{if
        $\kappa$ has at least one removable $r$-corner}\\
      \kappa & \text{else.}
    \end{cases}
  \]
\end{defn}
\begin{prop}
  There is a bijection \(\mathfrak{t}\) between \(\eS_{k+1}^0\) and \(k+1\)-cores given
  by sending \[
    w \mapsto s_{i_\ell} \cdots s_{i_1} \cdot \emptyset
  \]
  for \(s_{i_\ell} \cdots s_{i_1}\) a reduced word of \(w\).
\end{prop}
\begin{defn}\label{k-schur-def-2}
  The \(k\)-Schur functions \(s_\lambda^{(k)}(x)\) are the unique
  basis of \(\sym_{k+1} = \Z[h_1,\ldots,h_k]\) satisfying the
  following Pieri
  rule:
  for \(0 \leq r \leq k\) and \(w_\lambda\) such that
  \(\mathfrak{p}(\mathfrak{t}(w_\lambda)) = \lambda\)
  \[
    h_r s_\lambda^{(k)}(x) = \sum_{\substack{u \in \eS_{k+1} \\ u
        \text{ cyclically decreasing}\\ \ell(u)
    = r \\ \ell(uw_\lambda) = \ell(w_\lambda)+r}} s_{\mathfrak{p}(\mathfrak{t}(uw_\lambda))}^{(k)}(x)
  \]
\end{defn}
This definition is actually equivalent to
Definition~\ref{k-schur-def-1}(c), but is not very combinatorial, nor
very explicit. However, we can work through our bijections to make it
more combinatorial. To do this, we will define the notion of ``weak
order'' and ``weak
tableaux'' on cores.
\begin{defn}
  Consider the affine symmetric group \(\eS_{k+1}\).
  \begin{enumerate}
  \item The \de{(left) weak order} on \(\eS_{k+1}\) is given by saying
    \(w \leq v\) if there exists some \(u \in \eS_{k+1}\) such that
    \(uw = v\) and \(\ell(u) + \ell(w) = \ell(v)\).
  \item We says \(w\) is \de{covered by} \(v\), denoted \(w \coveredby v\),
    if there is some \(s_i \in \eS_{k+1}\) such that \(s_i w = v\) and
    \(\ell(v) = \ell(w)+1\).
  \item Given two \(k+1\)-cores \(\kappa\) and \(\tau\), we say
    \(\kappa \coveredby \tau\) if \(\mathfrak{t}^{-1}(\kappa)
    \coveredby \mathfrak{t}^{-1}(\tau)\).
  \end{enumerate}
\end{defn}
\begin{example}
  The following is the beginning of the Hasse diagram for the weak
  order on \(\eS_{3}^0\) and on the corresponding \(k+1\)-cores.
  \[
    \ytableausetup{boxsize=0.5em}
    \begin{tikzcd}[row sep=tiny, column sep=tiny]
      &&\emptyset\ar[d,dash]&&\\
      &&s_0\ar[ld,dash]\ar[rd,dash]&&\\
      &s_1 s_0 \ar[d,dash]&& s_2 s_0 \ar[d,dash]&\\
      &s_2 s_1 s_0 \ar[ld,dash]\ar[rd,dash]&&s_1 s_2 s_0\ar[ld,dash]\ar[rd,dash]&\\
      s_0 s_2 s_1 s_0 && s_2 s_1 s_2 s_0 && s_0 s_1 s_2 s_0
    \end{tikzcd}
    \correspondsto
    \begin{tikzcd}[row sep=tiny, column sep=tiny]
      &&\emptyset\ar[d,dash]&&\\
      &&\ydiagram{1}\ar[ld,dash]\ar[rd,dash]&&\\
      &\ydiagram{2}\ar[d,dash]&&\ydiagram{1,1}\ar[d,dash]&\\
      &\ydiagram{3,1}*[*(lightgray)]{1}\ar[ld,dash]\ar[rd,dash]&&\ydiagram{2,1,1}*[*(lightgray)]{1}\ar[ld,dash]\ar[rd,dash]&\\
      \ydiagram{4,2}*[*(lightgray)]{2}&&\ydiagram{3,1,1}*[*(lightgray)]{1}&&\ydiagram{2,2,1,1}*[*(lightgray)]{1,1}
    \end{tikzcd}
    % \correspondsto
    % \begin{tikzcd}[row sep=tiny, column sep=tiny]
    %   &&\emptyset \ar[d,dash]&&\\
    %   &&\ydiagram{1}\ar[ld,dash]\ar[rd,dash]&&\\
    %   &\ydiagram{2}\ar[d,dash]&&\ydiagram{1,1}\ar[d,dash]&\\
    %   &\ydiagram{2,1}\ar[ld,dash]\ar[rd,dash]&&\ydiagram{1,1,1}\ar[ld,dash]\ar[rd,dash]&\\ 
    %   \ydiagram{2,2}&&\ydiagram{2,1,1}&&\ydiagram{1,1,1,1}
    % \end{tikzcd}
  \]
\end{example}
\begin{prop}
  Given two \(k+1\)-cores \(\kappa\) and \(\tau\) with \(\kappa
  \subset \tau\), then \(\kappa \coveredby \tau\) if and only if
  \(\tau = s_i \kappa\).
\end{prop}
Using this perspective, we will build up a kind of semistandard
tableau using the notion of horizontal strips. To motivate, we will
first describe this notion for partitions and semistandard tableau.
\begin{defn}
  Given two partitions \(\lambda\) and \(\mu\), we say \(\lambda\) and
  \(\mu\) differ by a \de{horizontal strip of size \(r\)} if \(\mu \setminus
  \lambda\) has \(r\) cells lying in distinct columns.
\end{defn}
From this notion, one can construct a semistandard tableau from a
sequence of partitions differing by horizontal strips. E.g., consider
the following \[
  \ytableausetup{boxsize=1.0em}
  \emptyset \subset \ydiagram{3} \subset \ydiagram{5,2} \subset
  \ydiagram{6,3,1} \correspondsto \emptyset \subset
  \ytableaushort{111} \subset \ytableaushort{11122,22} \subset \ytableaushort{111223,223,3}
\]
Thus, the chain above corresponds to \[
  \ytableaushort{111223,223,3}
\]
\begin{defn}
  Given two \(k+1\)-cores \(\kappa\) and \(\tau\), we say they differ
  by a \de{weak horizontal strip} of size \(r \leq k\) if
  \begin{enumerate}
  \item \(\tau \setminus \kappa\) is a horizontal strip of partitions,
  \item \(|\mathfrak{p}(\tau)| = |\mathfrak{p}(\kappa)|+r\),
  \item and there are exactly \(r\) residues in the set of cells
    \(\tau \setminus \kappa\).
  \end{enumerate}
\end{defn}
\begin{prop}
  For \(0 \leq r \leq k\), we have that \[
    h_r s_\lambda^{(k)}(x) = \sum_{\substack{\tau \text{ a
        }k+1\text{-core} \\ \tau = \mathfrak{p}^{-1}(\lambda)+\text{a
          weak horizontal $r$-strip}}} s_{\mathfrak{p}(\tau)}^{(k)}(x) \,.
  \]
\end{prop}
\begin{defn}
  A \de{weak tableau} of shape \(\kappa\) and composition weight \(\alpha\) is a
  sequence of \(k+1\)-cores differing by weak horizontal strips of
  sizes given by \(\alpha\).
\end{defn}
\begin{example}
  For \(k+1=3\), a weak tableau of shape \((4,2)\) and weight
  \((1,1,1,1)\) is given by \[
    \ytableaushort{1234,34}
  \]
  Furthermore, a weak tableau of the same shape weight weight
  \((2,2)\) is given by \[
    \ytableaushort{1122,22}
  \]
\end{example}
From this, we get a combinatorial characterization of the affine
Stanley symmetric functions, also called dual \(k\)-Schur functions.
\begin{prop}
  The dual \(k\)-Schur function \(\tilde{F}_\lambda^{(k)}(x) \in
  \sym^{k+1} = \sym/\langle m_{\lambda} \st \lambda_1 \geq k+1
  \rangle\) is the weight generating function of weak tableau of shape
  \(\mathfrak{p}^{-1}(\lambda)\). More preciesly, \[
    F_\lambda^{(k)}(x) = \sum_\alpha \sum_{T \text{ a weak tableau of shape
      }\mathfrak{p}^{-1}(\lambda)} x^\alpha
  \]
  where \(x^{\alpha} = \prod_i x_i^\alpha\).
\end{prop}
\begin{bibdiv}
  \begin{biblist}
    \bib{catalans}{article}{
      author={Blasiak, Jonah}
      author={Morse, Jennifer}
      author={Pun, Anna}
      author={Summers, Daniel}
      title={Catalan Functions and \(k\)-Schur Positivity}
      year={2019}
      journal={J. Amer. Math. Soc.}
      volume={32}
      number={4}
      pages={921--963}
    }
    \bib{lam}{article}{
      author={Lam, Thomas}
      title={Schubert polynomials for the affine Grassmannian}
      journal={J. Amer. Math. Soc.}
      volume={21}
      number={1}
      year={2008}
      pages={259--281}
    }
    \bib{k-schur-book}{book}{
      author={Lam, Thomas}
      author={Lapointe, Luc}
      author={Morse, Jennifer}
      author={Schilling, Anne}
      author={Shimozono, Mark}
      author={Zabrocki, Mike}
      title={\(k\)-Schur functions and affine Schubert calculus}
      publisher={Springer}
      year={2014}
    }    
    \bib{llm}{article}{
      author={Lapointe, Luc}
      author={Lascoux, Alain}
      author={Morse, Jennifer}
      title={Tableau atoms and a new Macdonald positivity conjecture}
      journal={Duke Math. J.}
      volume={116}
      number={1}
      year={2003}
      pages={103--146}
    }
    \bib{lm1}{article}{
      author={Lapointe, Luc} 
      author={Morse, Jennifer}
      title={Schur function analogs for a filtration of the symmetric
        function space}
      journal={J. Combin. Theory Ser. A}
      volume={101}
      number={2}
      year={2003}
      pages={101--224}
    }
    \bib{lm2}{article}{
      author={Lapointe, Luc} 
      author={Morse, Jennifer}
      title={A \(k\)-tableaux characterization of \(k\)-Schur
        functions}
      journal={Adv. Math.}
      volume={213}
      number={1}
      year={2007}
      pages={183--204}
    }
  \end{biblist}
\end{bibdiv}
\end{document}
%%% Local Variables:
%%% mode: latex
%%% TeX-master: t
%%% End:
