\documentclass[11pt,leqno,oneside]{amsart}
\usepackage[alphabetic,abbrev]{amsrefs} % use AMS ref scheme
\usepackage{./presentation-notes}
\usepackage{../ReAdTeX/readtex-core}
\usepackage{../ReAdTeX/readtex-dangerous}
\usepackage{../ReAdTeX/readtex-abstract-algebra}
\usepackage{caption}
\usepackage{subcaption}
\usepackage{todonotes}
\usepackage{ytableau}
\usepackage{xcolor}
\usepackage{mathtools}
\usepackage{tikz-cd}
\numberwithin{thm}{section}

\newcommand{\trace}{\operatorname{Tr}}
\newcommand{\transpose}{t}
\newcommand{\T}{T} % Torus
\DeclareMathOperator{\SSYT}{SSYT}
\DeclareMathOperator{\wt}{wt}
\renewcommand{\P}{\mathbb{P}} % Probability

\title[Two Elementary Examples of Extreme Characters of
  \(U(\infty)\)]{Two Elementary Examples of Extreme Characters of
  \(U(\infty)\) \\ Integrable Probability Reading Seminar} 
\author{George H. Seelinger}
\date{February 1, 2019}
\begin{document}
\maketitle
\section{Introduction}
First, we recall some definitions.
\begin{defn}
  An \(N \times N\) matrix \(U\) is \de{unitary} if \(U U^* = I_N\)
  where \(U^*\) is the conjugate transpose of \(U\). Then, \(U(N)\) is
  the compact Lie group of all \(N \times N\) unitary matrices. Since
  \(U(N-1) \into U(N)\) via a canonical embedding, we also define \[
    U(\infty) := \Union_{N=1}^\infty U(N)
  \]
  that is, \(U(\infty)\) are all infinite \(\N \times \N\) unitary
  matrices that differ from the identity matrix only in a fixed number
  of positions.
\end{defn}
\begin{defn}
  A \de{normalized character} of \(U(N)\) is a function \(\chi
  \from U(N) \to \C\) such that
  \begin{enumerate}
  \item \(\chi(e) = 1\) (normalized),
  \item \(\chi(ab) = \chi(ba)\) (constant on conjugacy classes),
  \item \(\sum c_i \ov{c}_j \chi(a_i a_j^{-1}) \geq 0\) (nonnegative
    definite),
  \item \(\chi\) is continuous.
  \end{enumerate}
\end{defn}
Normalized characters form a convex set since \(t \chi_1 + (1-t)
\chi_t\) meets all the axioms of a normalized character for all \(t
\in [0,1]\). Then, we can discuss the following notion.
\begin{defn}
  An \de{extreme character} \(\chi \from U(N) \to \C\) is a normalized
  character such that \(\chi \neq t \chi_1 + (1-t) \chi_2\) for any
  \(t \in [0,1]\).
\end{defn}
Recall one of our main goals is to understand the following theorem.
\begin{thm}[Edrei-Voiculescu]
  Extreme characters of \(U(\infty)\) are functions \(\chi \from
  T^\infty_{fin} \to \C\) depending on countably many parameters \[
    \begin{cases}
      \alpha^{\pm} = (\alpha_1^\pm \geq \alpha_2^\pm \geq \cdots \geq
      0); \\
      \beta^\pm = (\beta_1^\pm \geq \beta_2^\pm \geq \cdots \geq 0);
      \\
      \gamma^\pm \geq 0
    \end{cases}
  \]
  such that \[
    \sum_i \alpha_i^+ + \sum_i \alpha_i^- + \sum_i \beta_i^+ + \sum_i
    \beta_i^- < \infty, \ \ \ \beta_1^+ + \beta_1^- \leq 1
  \]
\end{thm}
In this, we will outline two very special examples of this
parameterization.
\section{Symmetric Functions}
In the last lecture, we introduced the following.
\begin{defn}
  Given a sequence of integers \(\lambda_1 \geq \lambda_2 \geq \cdots
  \geq \lambda_N\), the \de{Schur polynomial} is given by \[
    s_\lambda(x_1, \ldots, x_N) =
    \frac{\det(x_j^{\lambda_i+N-i})_{i,j=1}^N}{\det(x_j^{N-i})_{i,j=1}^N}
  \]
\end{defn}
We also proved that
\begin{thm}
  The irreducible representations of \(U(N)\) are in one-to-one
  correspondence with \(\{\lambda \in \Z^N \st \lambda_1 \geq \cdots
  \geq \lambda_N\}\) where the character of representation
  \(T_\lambda\) of 
  \(U(N)\) corresponding to \(\lambda\) has character given by
  \[
    \trace\left( T_\lambda\left(
      \begin{array}{ccc}
        x_1&&\\
           &\ddots&\\
           &&x_N
      \end{array}
    \right) \right) = s_\lambda(x_1, \ldots x_N)
  \]
\end{thm}
We will work with two special cases of the Schur polynomials.
\begin{defn}
  Let \(e_m(x_1, \ldots, x_N) := s_{(1^m)}(x_1, \ldots, x_N)\) be the
  \de{elementary symmetric polynomials}. 
\end{defn}
\begin{example}
  Using the semistandard Young tableaux formula for Schur functions
  (Littlewood's combinatorial description), we compute
  \begin{enumerate}
  \item
    \begin{align*}
      e_2(x_1,x_2) =
      & x_1 x_2\\
      & \ytableaushort{1,2}
    \end{align*}
  \item
    \begin{align*}
      e_2(x_1,x_2,x_3) =
      & x_1 x_2 + x_1 x_3 + x_2 x_3 \\
      & \ytableaushort{1,2} + \ytableaushort{1,3} + \ytableaushort{2,3}
    \end{align*}
  \item
    \begin{align*}
      e_3(x_1,x_2,x_3) =
      & x_1 x_2 x_3 \\
      & \ytableaushort{1,2,3}
    \end{align*}
  \end{enumerate}
\end{example}
These function encode the ``determinant representation'' of \(U(N)\),
that is \[
  T(U)v = (\det U)v = x_1 x_2 \cdots x_N v
\]
since the determinant is just the product of the eigenvalues. Now, we
wish to take a sequence of these representations to get a
representation of \(U(\infty)\).
\begin{defn}
  We say that a sequence of central functions \(f_N\) on \(U(N)\)
  converge to a \de{central function} \(f\) on \(U(\infty)\) if, for
  every fixed \(K\), we have \[
    f_N(x_1, \ldots, x_K,1,1, \ldots, 1) \to f(x_1, \ldots, x_K, 1, 1,
    \ldots)
  \]
  uniformly on the \(K\)-torus \(\T^K\) of diagonal matrices. 
\end{defn}
\begin{prop}
  Let \(L \from \N \to \N\) be a sequence such that \(L(N)/N \to \beta
  \in [0,1]\) as \(N \to \infty\). Then, \[
    \frac{e_{L(N)}(x_1, \ldots, x_N)}{e_{L(N)}(1,\ldots,1)} \to
    \prod_{i=1}^\infty (1+\beta(x_i-1)), \ \ (x_1, x_2, \ldots) \in \T^\infty_{fin}
  \]
\end{prop}
\begin{proof}
  Consider that \[
    e_{L(N)}(x_1, \ldots, x_N) = \sum_{T \in \SSYT(1^{L(N)}) \text{filled
        with elements of }\{1, \ldots, N\}} x^{\wt(T)} = \sum_{I \subset \{1,\ldots,N\}, |I| =
    L(N)} x^{I}
  \]
  and thus also \[
    e_{L(N)}(\underbrace{1, \ldots, 1}_{N}) = \sum_{I \subset
      \{1,\ldots,N\}, |I|=L(N)} 1 = \binom{N}{L(N)}
  \]
  Now, for a fixed \(K \leq N\), we have
  \begin{align*}
    & e_{L(N)}(x_1, \ldots, x_K, 1, \ldots, 1)
    = \sum_{\text{binary }N\text{ sequences } \epsilon}
    |\SSYT(1^{L(N)},\epsilon)| x^{(\epsilon_1, \ldots, \epsilon_K)}
     \\
    \implies & \frac{e_{L(N)}(x_1, \ldots, x_K, 1, \ldots,
    1)}{e_{L(N)}(1,\ldots,1)} =\\
    & \sum_{\text{binary }N\text{
    sequences}} x^{(\epsilon_1,\ldots,\epsilon_K)} \P(\text{first
    }K\text{ coordinates of 
    the binary sequence are }(\epsilon_1, \ldots, \epsilon_K))
  \end{align*}
  where
  \begin{align*}
    & \P(\text{first
    }K\text{ coordinates of 
    the binary sequence are }(\epsilon_1, \ldots, \epsilon_K)) \\ & =
  \binom{N-K}{L(N)-\sum_{i=1}^K \epsilon_i} / \binom{N}{L(N)} \\
   & = \frac{(N-K)!}{N!} \times \frac{(L(N))!}{(L(N)-\sum_{i=1}^K
    \epsilon_i)!} \times \frac{(N-L(N))!}{(N-L(N)-(K-\sum_{i=1}^K
     \epsilon_i))!} \\
  & \overset{N \to \infty}{\longrightarrow} \beta^{\sum_{i=1}^K \epsilon_i}
    (1-\beta)^{\sum_{i=1}^K 
    \epsilon_i} & \text{ since } L(N)/N \to \beta
  \end{align*}
\end{proof}
\begin{bibdiv}
  \begin{biblist}
    \bib{ash}{book}{
      author={Ash, Robert B.}
      title={A Course In Algebraic Number Theory}
      year={2003}
      note={\url{https://faculty.math.illinois.edu/~r-ash/ANT.html}}
    }
    \bib{conrad}{article}{
      author={Conrad, Keith}
      title={Discriminants and Ramified Primes}
      note={\url{http://www.math.uconn.edu/~kconrad/blurbs/gradnumthy/disc.pdf}}
    }
    \bib{mack-crane}{article}{
      author={Mack-Crane, Sander}
      title={Prime Splitting in Quadratic Extensions I: One Prime,
        Many Fields}
      year={2016}
      note={\url{https://algebrateahousejmath.wordpress.com/2016/11/23/prime-splitting-in-quadratic-extensions-i-one-prime-many-fields/}}
    }
  \end{biblist}
\end{bibdiv}
\end{document}