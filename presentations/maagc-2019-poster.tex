\documentclass[final]{beamer}

\mode<presentation>
  {
    \usetheme{confposter} 
  }
\usepackage{../ReAdTeX/readtex-core}
\usepackage{amsmath, amsthm, amssymb, latexsym}
\usepackage{mathtools}
\usepackage[alphabetic,abbrev]{amsrefs} % use AMS ref scheme
\usepackage{anyfontsize}
\usepackage{scalefnt, scalerel}
\usepackage{array,booktabs,tabularx}  
\usepackage[orientation=portrait, size=a0, scale=1.5]{beamerposter} % Use the beamerposter package for laying out the poster
\RequirePackage{lmodern}
\usepackage{lmodern}
\usepackage{exscale} % Necessary for math symbols to be big enough
\usepackage{ytableau}

%% Blocks and Colors
\definecolor{pastelred}{rgb}{1.0, 0.41, 0.38}
\definecolor{tearose}{rgb}{0.96, 0.76, 0.76}
\definecolor{pastelpink}{rgb}{1.0, 0.82, 0.86}
\definecolor{palepink}{rgb}{0.98, 0.85, 0.87}
\definecolor{indianred}{rgb}{0.8, 0.36, 0.36}
\definecolor{coralred}{rgb}{1.0, 0.25, 0.25}

\tikzset{wei/.style=%{red,double=pink,thick,doubledistance=1.5pt}}
{red,double=red,double
distance=0.5pt}}

\tikzset{wei2/.style={red,double=red,double
    distance=0.5pt}}

\newenvironment<>{myexampleblock}[1]{%
  \setbeamercolor{block alerted title}{fg=white,bg=coralred}
  \setbeamercolor{block alerted body}{fg=black,bg=palepink}
  \begin{alertblock}#2{#1}}{\end{alertblock}}
\newenvironment<>{subtleblock}{
  \setbeamercolor{block alerted title}{fg=white,bg=white} 
  \setbeamercolor{block alerted body}{fg=black,bg=dblue!10}
  \begin{alertblock}#1{}}{\end{alertblock}}
\newenvironment<>{subtleexampleblock}{
  \setbeamercolor{block alerted title}{fg=white,bg=white} 
  \setbeamercolor{block alerted body}{fg=black,bg=palepink}
  \begin{alertblock}#1{}}{\end{alertblock}}

\setbeamercolor{block title}{fg=jblue,bg=white}  % Colors of the block
% titles
\setbeamercolor{block body}{fg=black,bg=white} % Colors of the body of blocks
\setbeamercolor{block alerted title}{fg=white,bg=dblue!70} % Colors of the highlighted block titles
\setbeamercolor{block alerted body}{fg=black,bg=dblue!10} % Colors of the body of highlighted blocks
\definecolor{melon}{rgb}{0.99, 0.74, 0.71}
\addtobeamertemplate{block end}{}{\vspace*{1ex}} % White space under blocks
\addtobeamertemplate{block alerted end}{}{\vspace*{1ex}} % White space
% under highlighted (alert) blocks

%% Math shortcuts
\newcommand{\lowers}{\mathcal{L}}
\newcommand{\g}{\mathfrak{g}}

%% Title matter
  \title[K-theoretic Catalan Functions]{K-theoretic Catalan Functions}
  \author[Seelinger]{George H. Seelinger (joint work with Jonah
    Blasiak and Jennifer Morse)}
  \institute[UVA]{University of Virginia}
  \date{May 4, 2019}
%% Document
  \begin{document}
  \begin{frame}[t] % The whole poster is in one Beamer frame
    \begin{columns}[t] % Align all column contents to the top
      \begin{column}{.3\paperwidth}
        \begin{block}{Background}
          \begin{itemize}
          \item Schur functions \(s_\lambda \in \Lambda\)
            \begin{itemize}
            % \item \[
            %   h_r s_\lambda = \sum_{\mathclap{\mu = \lambda + \text{ a
            %       horizontal }r\text{-strip}}} s_{\mu} \iff s_\lambda =
            %   \sum_\mu K_{\lambda,\mu} m_\mu
            % \]
            % for \(K_{\lambda,\mu}\) the Kostka numbers.
            % \item \ytableausetup{boxsize=0.3em}
            % \[
            %   h_3 s_{\ydiagram{1,2}} = s_{\ydiagram{1,2,3}}+s_{\ydiagram{1,1,4}}+s_{\ydiagram{2,4}}+s_{\ydiagram{1,5}}
            % \]
            \item
              \begin{align*}
                s_{21} & = m_{21} + 2m_{111}\\
               & \ytableaushort{2,11} \ \ytableaushort{2,13} \ \ytableaushort{3,12}
              \end{align*}
            \item \(\Lambda_m \onto H^*(Gr(m,n))\) via \[
                s_\lambda \mapsto
                \begin{cases}
                  \sigma_\lambda & \lambda \text{ in an }m \times
                  n\text{ rectangle}\\
                  0 & \text{else}
                \end{cases}
            \]
            \end{itemize}
          \item Grothendieck polynomials, \(G_\lambda \in
            \hat{\Lambda}^{(n)}\)
            \begin{itemize}
            \item \(G_\lambda = s_\lambda + \text{ higher order
                terms}\)
            \item
              \begin{align*}
                G_{21} & = m_{21}+2m_{111}-m_{22}-3m_{211}-\cdots
                \\
                & \ytableaushort{2,11} \ \ytableaushort{2,13} \
                  \ytableaushort{3,12} \\
                & \\
                & \ytableaushort{2,1{\scriptstyle\{1,2\}}} \
                  \ytableaushort{2,1{\scriptstyle\{1,3\}}} \
                  \ytableaushort{3,1{\scriptstyle\{1,2\}}} \
                  \ytableaushort{{\scriptstyle\{2,3\}},11} \cdots
              \end{align*}
            \item Gives Schubert basis representatives for \(K^*(Gr_{SL_n})\).
            % \item \(K^*(Gr_{SL_n}) \isom \hat{\Lambda}^{(n)}\)
            %   identifies Schubert basis to \(G_\lambda\)
            \end{itemize}
            \item dual Grothendieck polynomials \(g_\lambda \in
              \Lambda_{(n)}\)
            \begin{itemize}              
            \item \(g_\lambda = s_\lambda + \text{ lower order
                terms}\)
            \item Gives Schubert basis representatives for \(K_*(Gr_{SL_n})\).
            \end{itemize}
          \item \(k\)-Schur functions, \(s_\lambda^{(k)} \in \Lambda_t\) for
            \(k\)-bounded partitions \(\lambda\)
            \begin{itemize}
            \item When \(t=1\),
              \begin{align*}
                s_{21}^{(2)} & = m_3+2m_{21}+3m_{111}\\
                & \ytableaushort{{1},{1^*}{1^*}{1^*}} \
                  \ytableaushort{{2},{1^*}{1^*}{2^*}} \
                  \ytableaushort{{2^*},{1^*}{1^*}{2}} \\
                & \ytableaushort{{3},{1^*}{2^*}{3^*}} \ 
                  \ytableaushort{{3^*},{1^*}{2^*}{3}} \
                  \ytableaushort{{2^*},{1^*}{3}{3^*}} \ 
              \end{align*}
            \item Schubert representatives for (co)homology of affine Grassmannian.
            \item \(s_\lambda^{(k)} = s_\lambda\) when \(k \geq \lambda_1+\ell(\lambda)\).
            % \item When \(t=1\)
            %   \begin{align*}
            %      h_r s_\lambda^{(k)} = \; \displaystyle \sum_{\mathclap{\kappa =
            %     core(\lambda)+\text{strong marked }r\text{-strip}}}
            %   \; s_{part(\kappa)}^{(k)} \\
            %   \iff s_\lambda^{(k)} = \sum_\mu
            %   K_{core(\lambda),\mu}^{(k)} m_\mu
            %   \end{align*}
            %   for \(K_{core(\lambda),\mu}^{(k)}\) the \(k\)-Kostka
            %   coefficients.
            % \item \[
            %   h_3 s_{\ydiagram{1,2}}^{(3)} = s_{\ydiagram{1,2,3}}^{(3)}
            % \]
            \end{itemize}
          \item The \emph{affine Grothendieck polynomials},
            \(G_\lambda^{(k)}\), and \emph{dual affine Grothendieck polynomials},
            \(g_\lambda^{(k)}\).
            \begin{itemize}
            \item \(G_\lambda^{(k)} = s_\lambda^{(k)} + \text{
                higher order terms}\)
            \item \(g_\lambda^{(k)} = s_\lambda^{(k)} + \text{ lower order
                terms}\)
            \item \(g_\lambda^{(k)} = g_\lambda\) when \(k \geq
              \lambda_1+\ell(\lambda)\)
            \end{itemize}
            % play the analogous K-theoretic
            % role of the \(k\)-Schur functions. Furthermore, the Pieri
            % rule for \(g_\lambda^{(k)}\) is controlled by affine
            % set-valued tableaux, but there is no known combinatoric
            % for the Pieri rule of \(G_\lambda^{(k)}\). By duality,
            % this is
            % equivalent to finding a dual Pieri rule for \(g_\lambda^{(k)}\)
          \end{itemize}
        \end{block}
        \begin{alertblock}{Problems}
          \begin{enumerate}
          \item What are the branching coefficients for
            \(g_\lambda^{(k)}\), ie what are the coefficients
            \(c_{\lambda,\mu}^{(k)}\) for \[
              g_\lambda^{(k)} = \sum_\mu (-1)^{|\lambda|-|\mu|} c_{\lambda,\mu}^{(k)} g_\mu^{(k+1)}?
            \]
          \item What is the dual (vertical) Pieri rule for
            \(g_\lambda^{(k)}\), 
            ie what are the coefficients \(a_{\lambda,\mu}^{(k)}\) for
            \[
              G_{1^r}^\perp g_\lambda^{(k)} = \sum_\mu (-1)^{|\lambda|-|\mu|}
            a_{\lambda,\mu}^{(k)} g_\mu^{(k)}?
            \] 
          \end{enumerate}
        \end{alertblock}
        \begin{alertblock}{Conjectures}
          \begin{enumerate}
          \item~\cite{lss}*{7.20(iii)}: \(c_{\lambda,\mu}^{(k)} \in \Z_{\geq 0}\)
          \item~\cite{lss}*{7.21(iii)}: \(a_{\lambda,\mu}^{(k)} \in
            \Z_{\geq 0}\)
          \end{enumerate}
        \end{alertblock}
        \end{column}
        \begin{column}{.3\paperwidth}
          \begin{block}{Raising Operators and Catalan Functions}
            \begin{itemize}
            \item Raising operators \(R_{i,j}\) \ytableausetup{boxsize=0.4em}
              \begin{itemize}
              \item \[
                  R_{1,3} \left( \ydiagram{1,1,3}*[*(red)]{1} \right)
                  = \ydiagram{1,4}*[*(red)]{0,3+1} \ \ \ R_{2,3}
                  \left( \ydiagram{1,1,1}*[*(red)]{0,0,1} \right) =
                  \ydiagram{1,2}*[*(red)]{0,1+1} 
                \]
              \item Jacobi-Trudi: \(s_\lambda
                = \prod_{i < j}(1-R_{ij})h_\lambda\) for
                \(R_{ij}h_\lambda =
                h_{\lambda+\epsilon_i-\epsilon_j}\). 
              \end{itemize}
            \item Catalan functions
              \begin{itemize}
              \item \[
                  \begin{tikzpicture}[every node/.style={minimum size=0.95cm-\pgflinewidth, outer sep=0pt}]
                    \draw[step=1cm,color=black] (0,0) grid (4.0,-4.0);
                    \node[fill=red] at (1.5,-0.5) {};
                    \node[fill=red] at (2.5,-0.5) {};
                    \node[fill=red] at (3.5,-0.5) {};
                    \node[fill=red] at (3.5,-1.5) {};
                    \node at (0.5,-0.5) {$3$};
                    \node at (1.5,-1.5) {$4$};
                    \node at (2.5,-2.5) {$2$};
                    \node at (3.5,-3.5) {$1$};
                  \end{tikzpicture}
                  := (1-tR_{2,3})(1-tR_{3,4}) H_{3421}(x;t)
                \]
              \item \[
                  \begin{tikzpicture}[every node/.style={minimum size=0.95cm-\pgflinewidth, outer sep=0pt}]
                    \draw[step=1cm,color=black] (0,0) grid (4.0,-4.0);
                    \node at (0.5,-0.5) {$3$};
                    \node at (1.5,-1.5) {$2$};
                    \node at (2.5,-2.5) {$1$};
                    \node at (3.5,-3.5) {$1$};
                  \end{tikzpicture}
                  = s_{3211} \ \ \ \ \ 
                  \begin{tikzpicture}[every node/.style={minimum size=0.95cm-\pgflinewidth, outer sep=0pt}]
                    \draw[step=1cm,color=black] (0,0) grid (4.0,-4.0);
                    \node[fill=red] at (1.5,-0.5) {};
                    \node[fill=red] at (2.5,-0.5) {};
                    \node[fill=red] at (3.5,-0.5) {};
                    \node[fill=red] at (3.5,-1.5) {};
                    \node at (0.5,-0.5) {$4$};
                    \node at (1.5,-1.5) {$3$};
                    \node at (2.5,-2.5) {$1$};
                    \node at (3.5,-3.5) {$1$};
                  \end{tikzpicture}
                  = s^{(4)}_{4311}
                \]
             \item~\cite{bmps} use Catalan functions to show
               \emph{shift invariance}:
                 \[
                   e_\ell^\perp s_{\lambda+1^\ell}^{(k+1)} =
                   s_\lambda^{(k)}
                 \]
            \end{itemize}
          \end{itemize}
          \begin{alertblock}{Technique}
            Shift invariance + dual Pieri rule yields
              \(s^{(k)}_\lambda\) branching coefficients.
                \[
                  s_\lambda^{(k)} = e_\ell^\perp
                  s_{\lambda+1^\ell}^{(k+1)} = \, \sum_{\mathclap{T \in
                  VSMT^{k+1}_{(d)}(\lambda+1^\ell)}} \, t^{spin(T)} s_{inside(T)}^{(k+1)}
               \]
          \end{alertblock}
          \end{block}
          \begin{block}{K-theory and Symmetric Functions}
            \begin{itemize}
            \item \(m \in \Z, r \in \Z_{\geq
                0}\)\[ Kh_m^{(r)} := \sum_{i=0}^m \binom{r+i-1}{i}
                h_{m-i}
              \]
            \item \(Kh_m^{(0)} = h_m\)
            \item \(Kh_4^{(2)} = h_4+2h_3+3h_2+4h_1+5h_0\)
            \item \(\gamma \in \Z^\ell\),
              \[
                Kh_\lambda := Kh_{\lambda_1}^{(0)}
                Kh_{\lambda_2}^{(1)} \cdots
                Kh_{\lambda_\ell}^{(\ell-1)}
              \]
            \end{itemize}
          \end{block}
      \end{column}
      \begin{column}{0.3\paperwidth}
        \begin{block}{K-theoretic Catalan Functions}
          \begin{itemize}
          \item Lowering operators \(L_j\)
\[            L_3\left( \ydiagram{1,1,3} \right) =
              \ydiagram{1,3}, L_{1}\left( \ydiagram{1,1,3} \right)
              = \ydiagram{1,1,2}
            \]
          \item K-theoretic Catalan functions
            \begin{itemize}
            \item \[
                  \begin{tikzpicture}[every node/.style={minimum size=0.95cm-\pgflinewidth, outer sep=0pt}]
                    \draw[step=1cm,color=black] (0,0) grid (4.0,-4.0);
                    \node[fill=red] at (1.5,-0.5) {};
                    \node[fill=red] at (2.5,-0.5) {};
                    \node[fill=violet] at (3.5,-0.5) {};
                    \node[fill=violet] at (3.5,-1.5) {};
                    \node at (0.5,-0.5) {$3$};
                    \node at (1.5,-1.5) {$4$};
                    \node at (2.5,-2.5) {$2$};
                    \node at (3.5,-3.5) {$1$};
                  \end{tikzpicture}
                  := (1-L_4)^2(1-R_{23})(1-R_{34})Kh_{3421}
                \]
              \item \[
                                    \begin{tikzpicture}[every node/.style={minimum size=0.95cm-\pgflinewidth, outer sep=0pt}]
                    \draw[step=1cm,color=black] (0,0) grid (4.0,-4.0);
                    \node at (0.5,-0.5) {$3$};
                    \node at (1.5,-1.5) {$2$};
                    \node at (2.5,-2.5) {$1$};
                    \node at (3.5,-3.5) {$1$};
                  \end{tikzpicture}
                  = g_{3211} \ \ \ \ 
                  \begin{tikzpicture}[every node/.style={minimum size=0.95cm-\pgflinewidth, outer sep=0pt}]
                    \draw[step=1cm,color=black] (0,0) grid (4.0,-4.0);
                    \node[fill=red] at (1.5,-0.5) {};
                    \node[fill=violet] at (2.5,-0.5) {};
                    \node[fill=violet] at (3.5,-0.5) {};
                    \node[fill=red] at (3.5,-1.5) {};
                    \node at (0.5,-0.5) {$4$};
                    \node at (1.5,-1.5) {$3$};
                    \node at (2.5,-2.5) {$1$};
                    \node at (3.5,-3.5) {$1$};
                  \end{tikzpicture} = g_{4311}^{(4)}
                \]
              \end{itemize}
            \end{itemize}
          \begin{alertblock}{Conjectures}
            \begin{enumerate}
            \setcounter{enumi}{2}
            \item K-theoretic Catalan functions with matching root
              ideals (all roots are purple) and partition weight are
              \(g_\lambda\) positive.
            \item All \(g_\lambda^{(k)}\)'s arise as K-theoretic
              Catalan functions with same pattern as above.
            \end{enumerate}
          \end{alertblock}
        \end{block}
        \begin{block}{Shift Invariance for K-Catalan Functions}
          \begin{alertblock}{Theorem}
            \[
              G_{1^\ell}^\perp
                 \begin{tikzpicture}[every node/.style={minimum size=1.95cm-\pgflinewidth, outer sep=0pt}]
                    \draw[step=2cm,color=black] (0,0) grid (8.01,-8.01);
                    \node[fill=red] at (3,-1) {};
                    \node[fill=violet] at (5,-1) {};
                    \node[fill=violet] at (7,-1) {};
                    \node[fill=red] at (7,-3) {};
                    \node at (1,-1) {$\gamma_1$};
                    \node at (3,-3) {$\gamma_2$};
                    \node at (5,-5) {$\ddots$};
                    \node at (7,-7) {$\gamma_\ell$};
                  \end{tikzpicture}
                  =
                  \begin{tikzpicture}[every node/.style={minimum size=1.95cm-\pgflinewidth, outer sep=0pt}]
                    \draw[step=2cm,color=black] (0,0) grid (8.01,-8.01);
                    \node[fill=red] at (3,-1) {};
                    \node[fill=violet] at (5,-1) {};
                    \node[fill=violet] at (7,-1) {};
                    \node[fill=red] at (7,-3) {};
                    \node at (1,-1) [font=\fontsize{20}{0}\selectfont]
                    {$\tiny \gamma_1-1$};
                    \node at (3,-3) [font=\fontsize{20}{0}\selectfont]
                    {$\tiny \gamma_2-1$};
                    \node at (5,-5) {$\ddots$};
                    \node at (7,-7) [font=\fontsize{20}{0}\selectfont] {$\gamma_\ell-1$};
                  \end{tikzpicture}
                \]
                Granting conjecture \(2\) above, this implies \[
                  G_{1^\ell}^\perp g_{\lambda+1^\ell}^{(k+1)} = g_\lambda^{(k)}
                \]
          \end{alertblock}
        \end{block}
        \begin{block}{Towards a Vertical Dual Pieri Rule}
          \begin{itemize}
          \item Conjecture 4 gives Catalan formula for \(g_\lambda^{(k)}\).
          \item Catalan-theoretic techniques are promising for dual
            Pieri rule. Could yield an answer to Conjecture 2.
          \item Theorem + Conjecture 2 answers Conjecture 1.
          \end{itemize}
        \end{block}
        \setbeamercolor{block title}{fg=jblue,bg=white} % Change the block title color

        \begin{block}{References}
          \begin{bibdiv}
            \begin{biblist}
              \bib{bmps}{article}{
                author={Blasiak, Jonah}
                author={Morse, Jennifer}
                author={Pun, Anna}
                author={Summers, Daniel}
                title={Catalan Functions and \(k\)-Schur Positivity}
                year={2019}
                journal={Journal of the AMS}
              }
              \bib{lss}{article}{
                author={Lam, Thomas}
                author={Schilling, Anne}
                author={Shimozono, Mark}
                title={K-theory Schubert calculus of the affine
                  Grassmannian}
                year={2010}
                journal={Compositio Math.}
                volume={146}
                pages={811--852}
              }
              \bib{morse}{article}{
                author={Morse, Jennifer}
                title={Combinatorics of the K-theory of affine
                  Grassmannians}
                year={2011}
                journal={Advances in Mathematics}
              }
            \end{biblist}
          \end{bibdiv}
        \end{block}
      \end{column}
    \end{columns}
  \end{frame} % End the wrapping frame
  \end{document}
