\documentclass[final]{beamer}

\mode<presentation>
  {
    \usetheme{confposter} 
  }
\usepackage{../ReAdTeX/readtex-core}
\usepackage{amsmath, amsthm, amssymb, latexsym}
\usepackage{mathtools}
\usepackage{mathdots}
\usepackage[alphabetic,abbrev]{amsrefs} % use AMS ref scheme
\usepackage{anyfontsize}
\usepackage{scalefnt, scalerel}
\usepackage{array,booktabs,tabularx}  
\usepackage[orientation=landscape, size=a0, scale=1.5]{beamerposter} % Use the beamerposter package for laying out the poster
\RequirePackage{lmodern}
\usepackage{lmodern}
\usepackage{exscale} % Necessary for math symbols to be big enough
\usepackage{ytableau}
\usepackage{tikz}
\usetikzlibrary{tikzmark,decorations.pathreplacing}

%% Blocks and Colors
\definecolor{pastelred}{rgb}{1.0, 0.41, 0.38}
\definecolor{tearose}{rgb}{0.96, 0.76, 0.76}
\definecolor{pastelpink}{rgb}{1.0, 0.82, 0.86}
\definecolor{palepink}{rgb}{0.98, 0.85, 0.87}
\definecolor{indianred}{rgb}{0.8, 0.36, 0.36}
\definecolor{coralred}{rgb}{1.0, 0.25, 0.25}

\tikzset{wei/.style=%{red,double=pink,thick,doubledistance=1.5pt}}
{red,double=red,double
distance=0.5pt}}

\tikzset{wei2/.style={red,double=red,double
    distance=0.5pt}}

\newenvironment<>{myexampleblock}[1]{%
  \setbeamercolor{block alerted title}{fg=white,bg=coralred}
  \setbeamercolor{block alerted body}{fg=black,bg=palepink}
  \begin{alertblock}#2{#1}}{\end{alertblock}}
\newenvironment<>{subtleblock}{
  \setbeamercolor{block alerted title}{fg=white,bg=white} 
  \setbeamercolor{block alerted body}{fg=black,bg=dblue!10}
  \begin{alertblock}#1{}}{\end{alertblock}}
\newenvironment<>{subtleexampleblock}{
  \setbeamercolor{block alerted title}{fg=white,bg=white} 
  \setbeamercolor{block alerted body}{fg=black,bg=palepink}
  \begin{alertblock}#1{}}{\end{alertblock}}

\setbeamercolor{block title}{fg=jblue,bg=white}  % Colors of the block
% titles
\setbeamercolor{block body}{fg=black,bg=white} % Colors of the body of blocks
\setbeamercolor{block alerted title}{fg=white,bg=dblue!70} % Colors of the highlighted block titles
\setbeamercolor{block alerted body}{fg=black,bg=dblue!10} % Colors of the body of highlighted blocks
\definecolor{melon}{rgb}{0.99, 0.74, 0.71}
\addtobeamertemplate{block end}{}{\vspace*{1ex}} % White space under blocks
\addtobeamertemplate{block alerted end}{}{\vspace*{1ex}} % White space
% under highlighted (alert) blocks

%% Math shortcuts
\newcommand{\lowers}{\mathcal{L}}
\newcommand{\g}{\mathfrak{g}}
\newcommand{\covers}{\mathrel{\gtrdot}}
\newcommand{\coveredby}{\mathrel{\lessdot}}


%% Title matter
  \title[K-theoretic Catalan Functions]{K-theoretic Catalan Functions  George H. Seelinger (joint with Jonah Blasiak and Jennifer Morse)}
  \author[Seelinger]{\ \vspace{-3ex}}
  \institute[UVA]{\ \vspace{-3ex}}
  \date{\ \vspace{-1ex}}
%% Document
\begin{document}
  \begin{frame}[t] % The whole poster is in one Beamer frame
    \begin{columns}[t] % Align all column contents to the top
      \begin{column}{.19\paperwidth}
        \begin{block}{Background}
          \begin{itemize}
          \item Schur functions and Grothendieck polynomials give
            representatives for cohomology and \(K\)-theory of the
            Grassmannian.
          \item Pieri rules determine the structure constants of these
            rings.
          \item We seek to do the same for the affine Grassmannian.
          \end{itemize}
        \end{block}
        \begin{block}{Affine combinatorics}
        Grassmannian words \(w \in \tilde{S}_n\) correspond to
        \(n\)-cores, ie shapes that have no cell with hook length
        \(n\):
        \ytableausetup{boxsize=0.6em}
        \[
          \ydiagram{1,2,3,4,5,5}*[\bullet]{0,1+1,1+1,1+1,1+4}*[*(red)]{0,0,0,0,1+1}
          \ \ \text{red cell has hook-length }7
        \]
        Weak order
        \begin{itemize}
        \item Ordered by adding all cells of same reside.
          \ytableausetup{boxsize=1.0em} \[
            \emptyset \coveredby \ytableaushort{0}*[*(red)]{1} \coveredby
            \ytableaushort{{}1}*[*(red)]{1+1} \coveredby \ytableaushort{2,{}{}2}*[*(red)]{1,2+1}
            \coveredby \ytableaushort{{}0,{}{}{}0}*[*(red)]{1+1,3+1}
          \]\[
            \to \ytableaushort{34,1234}
          \]
        \item Weak tableaux = chain in weak order
        \end{itemize}
        Strong (Bruaht) order
        \begin{itemize}
        \item Ordered by containment
              \ytableausetup{boxsize=0.3em}
              \[
                \ytableausetup{centertableaux}
                % \ydiagram{1,1,4,4} \coveredby \ydiagram{1,1,1,4,4}
                % \coveredby \ydiagram{1,1,2,4,5} \coveredby
                % \ydiagram{2,2,2,4,5} \coveredby \ydiagram{2,2,2,5,5}
                % \coveredby \ydiagram{1,2,2,3,5,6} \coveredby
                \ydiagram{1,1,2,2,3,5,6} \coveredby
                \ydiagram{1,1,1,1,3,3,3,5,6} \coveredby
                \ydiagram{1,1,1,1,3,3,3,6,6} \coveredby
                \ydiagram{1,2,2,3,4,4,5,6,6}*[*(red)]{0,1+1,1+1,1+2,3+1,3+1,3+2}
              \]
            \item Strong strip = copies of a ribbon
            \item Strong marked strip = one copy of ribbon marked
            \item Strong marked tableaux = chain is strong order with markings
         \end{itemize}
         \(k+1\)-core to \(k\)-bounded partition
         \begin{itemize}
         \item Highlight all cells with hook length \(\leq k\) and
           justify highlighted cells \[
             \ydiagram{1,2,2,5,5,8,12}*[*(blue)]{1,2,2,2+3,2+3,5+3,8+4}
             \to \ydiagram[*(blue)]{1,2,2,3,3,3,4}
           \]
         \item Process is bijective
         \end{itemize}

           % \item Dual \(k\)-Schur functions, \(F_\lambda^{(k)}\) for
           %   \(k\)-bounded partitions \(\lambda\).
           %   \begin{itemize}
           %   \item Underlying combinatorics by weak order on
           %     Grassmannian elements of \(\tilde{S}_n\)
           %   \item Ordered by ``residue order'' (Describe this better)
           %   \item \[
           %       \emptyset \subset \ydiagram{1} \subset
           %      \ydiagram{1,1} \subset
           %      \ydiagram{1,1,2}
           %      \to \ytableaushort{3,2,13} \to x_1x_2x_3
           %    \]
           %  \item Weight generating function \(F_\lambda^{(k)} =
           %    \sum_T x^{weight(T)}\) eg \[
           %      F_{111}^{(2)} = x_1 x_2 x_3
           %    \]
           %   \item Equivalently defined by Pieri rule:
           %   \end{itemize}

            % play the analogous K-theoretic
            % role of the \(k\)-Schur functions. Furthermore, the Pieri
            % rule for \(g_\lambda^{(k)}\) is controlled by affine
            % set-valued tableaux, but there is no known combinatoric
            % for the Pieri rule of \(G_\lambda^{(k)}\). By duality,
            % this is
            % equivalent to finding a dual Pieri rule for \(g_\lambda^{(k)}\)
%          \end{itemize}
        \end{block}
      \end{column}
      \begin{column}{.38\paperwidth}
        \vspace{-0.1in}
        \begin{columns}[t]
          \begin{column}{.18\paperwidth}
            \begin{block}{Dual \(k\)-Schur Functions}
              Weight generating functions of weak tableaux of shape \(c(\lambda)\). \[
                F_\lambda^{(k)} := \sum_T x^{weight(T)}
              \]
              Pieri rule: 
            \end{block}
          \end{column}
          \begin{column}{.18\paperwidth}
            \begin{block}{\(k\)-Schur Functions}
              Weight generating functons of strong marked tabelaux of shape \(c(\lambda)\). \[
                s_\lambda^{(k)} := \sum_T x^{weight(T)}
              \]
              Dual Pieri rule:
            \end{block}
          \end{column}
        \end{columns}
        \[
          e_r F_\lambda^{(k)} = \ \ \ \sum_{\mathclap{\substack{\mu = \lambda + \text{ strong
              marked vertical }\\\text{strip of size }r}}} \ \ \ F_\mu^{(k)}
          \iff e_r^\perp s_\mu^{(k)} = \ \ \ \sum_{\mathclap{\substack{\lambda = \mu - \text{
              strong marked vertical }\\\text{strip of size }r}}} \ \
      \ s_\lambda^{(k)}
        \]
        \begin{columns}[t]
          \begin{column}{.18\paperwidth}
            \begin{block}{Affine Grothendieck Polynomials}
            \(G_\lambda^{(k)}\) given as a weight generating function of
            affine set-valued tableaux. \[
                G_\lambda^{(k)} = \sum_{T}
                (-1)^{|\lambda|+|weight(T)|} x^{weight(T)}
              \]
            \begin{itemize}
            \item Each \(T_{\geq x}\) is a \(k+1\)-core
              \ytableausetup{boxsize=1.0em}
              \[
                \hspace{-0.6in} T =
                \ytableaushort{7,{\scriptstyle{2,5}}6,1{\scriptstyle{2,3}}4{\scriptstyle{4,6}}}
                \ \
                T_{\leq 4} = \ytableaushort{2,1{\scriptstyle{2,3}}44}
              \]
            \item \(G_\lambda^{(k)} = F_\lambda^{(k)} + \text{
                higher order terms}\)
            \end{itemize}
            \end{block}
          \end{column}
          \begin{column}{.18\paperwidth}
            \begin{block}{Dual Affine Grothendieck Polynomials}
              \(g_\lambda^{(k)}\) defined by duality with respect to
              \(G_\lambda^{(k)}\).
              \begin{subtleblock}
                There is no known direct definition of
                \(g_\lambda^{(k)}\).
                \end{subtleblock}
              \begin{itemize}
              \item \(g_\lambda^{(k)} = s_\lambda^{(k)} + \text{ lower
                order terms}\)
            \end{itemize}
            \begin{subtleblock}
              \textbf{Remarkable Property:}
              \begin{itemize}
              \item k-Schur branching:
                \(s_\lambda^{(k)} = \sum s_\mu^{(k+1)}\).
              \item Similarly,
                \(g_\lambda^{(k)} = \sum_\mu a_{\lambda \mu}^{(k)}
                g_\mu^{(k)}\) for unknown \(a_{\lambda \mu}^{(k)}\).
              \end{itemize}
            \end{subtleblock}
            \end{block}
          \end{column}
        \end{columns}
        \begin{alertblock}{Open Problem}
          What is the \(G_\lambda^{(k)}\) Pieri rule? \(\iff\)
          What is the the \(g_\lambda^{(k)}\) dual Pieri rule?
        \end{alertblock}


        % \begin{block}{Grothendieck polynomials}
        %   \begin{itemize}

        %   \item Grothendieck polynomials, \(G_\lambda \in
        %     \hat{\Lambda}\)
        %     \begin{itemize}
        %     \item \(G_\lambda = s_\lambda + \text{ higher order
        %         terms}\)
        %     \item Set-valued tableaux \ytableausetup{boxsize=normal}
        %       \begin{align*}
        %         G_{21} & = m_{21}+2m_{111}-m_{22}-3m_{211}-\cdots
        %         \\
        %         & \ytableaushort{2,11} \ \ytableaushort{2,13} \
        %           \ytableaushort{3,12} \\
        %         & \\
        %         & \ytableaushort{2,1{\scriptstyle\{1,2\}}} \
        %           \ytableaushort{2,1{\scriptstyle\{1,3\}}} \
        %           \ytableaushort{3,1{\scriptstyle\{1,2\}}} \
        %           \ytableaushort{{\scriptstyle\{2,3\}},11} \cdots
        %       \end{align*}
        %     \item Equivalently defined by Pieri rule:
        %     % \item Gives Schubert basis representatives for \(K(Gr(d,n))\).
        %     % \item \(K^*(Gr_{SL_n}) \isom \hat{\Lambda}^{(n)}\)
        %     %   identifies Schubert basis to \(G_\lambda\)
        %     \end{itemize}
        %     \item dual Grothendieck polynomials \(g_\lambda \in
        %       \Lambda\)
        %     \begin{itemize}              
        %     \item \(g_\lambda = s_\lambda + \text{ lower order
        %         terms}\)
        %     \item Reverse plane partitions
        %       \begin{align*}
        %         g_{21} = x_1^2x_2 + 2 x_1 x_2 x_3 + x_1^2 + x_1 x_2 +
        %         \cdots \\
        %         \ytableaushort{2,11} \
        %         \ytableaushort{2,13} \
        %         \ytableaushort{3,12} \
        %         \ytableaushort{1,11} \
        %         \ytableaushort{1,12} 
        %       \end{align*}
        %     \item Equivalently defined by Pieri rule:            
        %     \end{itemize}
        %   \end{itemize}
        % \end{block}
        \end{column}
%         \begin{column}{.19\paperwidth}

%             % \item \(G_\lambda^{(k)}\) gives Schubert basis
%             %   representatives for \(K^*(Gr_{SL_k})\).
%             \begin{itemize}
%             \item \(g_\lambda^{(k)}\) defined by its Pieri rule (from
%               duality): \[
%                 h_r g_\lambda^{(k)} = \sum_{\mu = \lambda + \text{
%                     horizontal affine sv }r\text{-strip}} g_\mu^{(k)}
%               \]
%             \item \(g_\lambda^{(k)} = s_\lambda^{(k)} + \text{ lower order
%                 terms}\)
%             % \item \(g_\lambda^{(k)}\) gives Schubert basis
%             %   representatives for \(K_*(Gr_{SL_k})\).
%             \end{itemize}
%             \begin{alertblock}{Open Problem}
%               Is there any definition for \(g_\lambda^{(k)}\) other
%               than by duality?
%             \end{alertblock}
%             \begin{alertblock}{Bonus Problem: Branching}
%               The coefficients \(s_\lambda^{(k)} = \sum_\mu a_{\lambda
%               \mu}^{(k)} s_\mu^{(k+1)}\) are given by counting certain
%             strong tableaux. What are the coefficients
%             \(g_\lambda^{(k)} = \sum_\mu b_{\lambda \mu}^{(k)}
%             g_\mu^{(k+1)}\) ?
%             \end{alertblock}
%         %     \begin{alertblock}{Problems}
%         %   \begin{enumerate}
%         %   \item What are the branching coefficients for
%         %     \(g_\lambda^{(k)}\), ie what are the coefficients
%         %     \(c_{\lambda,\mu}^{(k)}\) for \[
%         %       g_\lambda^{(k)} = \sum_\mu (-1)^{|\lambda|-|\mu|} c_{\lambda,\mu}^{(k)} g_\mu^{(k+1)}?
%         %     \]
%         %   \item What is the dual (vertical) Pieri rule for
%         %     \(g_\lambda^{(k)}\), 
%         %     ie what are the coefficients \(a_{\lambda,\mu}^{(k)}\) for
%         %     \[
%         %       G_{1^r}^\perp g_\lambda^{(k)} = \sum_\mu (-1)^{|\lambda|-|\mu|}
%         %     a_{\lambda,\mu}^{(k)} g_\mu^{(k)}?
%         %     \] 
%         %   \end{enumerate}
%         % \end{alertblock}
%         % \begin{alertblock}{Conjectures}
%         %   \begin{enumerate}
%         %   \item~\cite{lss}*{7.20(iii)}: \(c_{\lambda,\mu}^{(k)} \in \Z_{\geq 0}\)
%         %   \item~\cite{lss}*{7.21(iii)}: \(a_{\lambda,\mu}^{(k)} \in
%         %     \Z_{\geq 0}\)
%         %   \end{enumerate}
%         % \end{alertblock}
% %        \end{block}
%           % \begin{block}{K-theory and Symmetric Functions}
%           %   \begin{itemize}
%           %   \item \(m \in \Z, r \in \Z_{\geq
%           %       0}\)\[ Kh_m^{(r)} := \sum_{i=0}^m \binom{r+i-1}{i}
%           %       h_{m-i}
%           %     \]
%           %   \item \(Kh_m^{(0)} = h_m\)
%           %   \item \(Kh_4^{(2)} = h_4+2h_3+3h_2+4h_1+5h_0\)
%           %   \item \(\gamma \in \Z^\ell\),
%           %     \[
%           %       Kh_\gamma := Kh_{\gamma_1}^{(0)}
%           %       Kh_{\gamma_2}^{(1)} \cdots
%           %       Kh_{\gamma_\ell}^{(\ell-1)}
%           %     \]
%           %   \item Similar role to \(h_\lambda\), but \(Kh_{25} \neq Kh_{52}\)
%           %   \item \(g_\lambda = \prod_{i<j}(1-R_{ij})Kh_\lambda\)
%           %   \end{itemize}
%           % \end{block}
%       \end{column}
      \begin{column}{0.19\paperwidth}
        \begin{block}{Catalan Functions}
          For \(\gamma \in \Z^\ell\), \[
            H(\Psi;\gamma) := \prod_{(i,j) \in \Delta^+\setminus
              \Psi} (1-R_{ij}) h_\gamma
          \]
          \begin{itemize}
          \item Raising operators \(R_{i,j}\) \ytableausetup{boxsize=0.5em}
            \[
              R_{1,3} \left( \ydiagram{1,1,3}*[*(red)]{1} \right)
              = \ydiagram{1,4}*[*(red)]{0,3+1} \ \ \ R_{2,3}
              \left( \ydiagram{1,1,1}*[*(red)]{1} \right) =
              \ydiagram{2,1}*[*(red)]{1+1} 
            \]
          \item Root ideal: given by Dyck path. \[
              \Psi = \begin{tikzpicture}[every node/.style={minimum size=1.95cm-\pgflinewidth, outer sep=0pt}]
                \draw[step=2cm,color=black] (0,0) grid (8.01,-8.01);
                \node[fill=red] at (3,-1) {1,2};
                \node[fill=red] at (5,-1) {1,3};
                \node[fill=red] at (7,-1) {1,4};
                \node[fill=red] at (7,-3) {2,4};
              \end{tikzpicture}
              \]
            \item \(H(\emptyset;\lambda) = s_\lambda\) (Jacobi-Trudi
              Identity) \[
                \begin{tikzpicture}[every node/.style={minimum size=0.95cm-\pgflinewidth, outer sep=0pt}]
                  \draw[step=1cm,color=black] (0,0) grid (4.0,-4.0);
                  \node at (0.5,-0.5) {$3$};
                  \node at (1.5,-1.5) {$2$};
                  \node at (2.5,-2.5) {$1$};
                  \node at (3.5,-3.5) {$1$};
                \end{tikzpicture}
                = s_{3211}
              \]
            \end{itemize}
            \end{block}
            \begin{block}{\(k\)-Schur Catalans}
              If \(\Psi\) has \(\lambda_i + \)non-roots in row
                \(i = k\) (ie bandwidth\(=k\)), then \(H(\Psi;\lambda) = s_\lambda^{(k)}\) \[
                  \begin{tikzpicture}[every node/.style={minimum size=0.95cm-\pgflinewidth, outer sep=0pt}]
                    \draw[step=1cm,color=black] (0,0) grid (4.0,-4.0);
                    \node[fill=red] at (1.5,-0.5) {};
                    \node[fill=red] at (2.5,-0.5) {};
                    \node[fill=red] at (3.5,-0.5) {};
                    \node[fill=red] at (3.5,-1.5) {};
                    \node at (0.5,-0.5) {$4$};
                    \node at (1.5,-1.5) {$3$};
                    \node at (2.5,-2.5) {$1$};
                    \node at (3.5,-3.5) {$1$};
                  \end{tikzpicture}
                  = s^{(4)}_{4311}
                \]
                \vspace{-0.5in}
                \begin{subtleblock}
                  \textbf{Key Feature:} Catalan function definition of \(s_\lambda^{(k)}\) yields \emph{shift
                    invariance}:
                  \[
                    e_\ell^\perp s_{\lambda+1^\ell}^{(k+1)} =
                    s_\lambda^{(k)}
                  \] 
                  \[
                    e_{\ell}^\perp \ 
                 \begin{tikzpicture}[every node/.style={minimum size=1.95cm-\pgflinewidth, outer sep=0pt}]
                    \draw[step=2cm,color=black] (0,0) grid (8.01,-8.01);
                    \node[fill=red] at (3,-1) {};
                    \node[fill=red] at (5,-1) {};
                    \node[fill=red] at (7,-1) {};
                    \node[fill=red] at (7,-3) {};
                    \node at (1,-1) {$\gamma_1$};
                    \node at (3,-3) {$\gamma_2$};
                    \node at (5,-5) {$\ddots$};
                    \node at (7,-7) {$\gamma_\ell$};
                  \end{tikzpicture}
                  =
                  \begin{tikzpicture}[every node/.style={minimum size=1.95cm-\pgflinewidth, outer sep=0pt}]
                    \draw[step=2cm,color=black] (0,0) grid (8.01,-8.01);
                    \node[fill=red] at (3,-1) {};
                    \node[fill=red] at (5,-1) {};
                    \node[fill=red] at (7,-1) {};
                    \node[fill=red] at (7,-3) {};
                    \node at (1,-1) [font=\fontsize{20}{0}\selectfont]
                    {$\tiny \gamma_1-1$};
                    \node at (3,-3) [font=\fontsize{20}{0}\selectfont]
                    {$\tiny \gamma_2-1$};
                    \node at (5,-5) {$\ddots$};
                    \node at (7,-7) [font=\fontsize{20}{0}\selectfont] {$\gamma_\ell-1$};
                  \end{tikzpicture}
                \]
              \end{subtleblock}
              \vspace{-0.3cm}
          \begin{alertblock}{Technique}
              \(s^{(k)}_\lambda\) branching coefficients are dual
              Pieri rule!
              
              \[
                  s_\lambda^{(k)} \tikzmarknode{shift}{=} e_\ell^\perp
                  s_{\lambda+1^\ell}^{(k+1)} \tikzmarknode{pieri}{=} \, \sum_{\mu} \, s_{\mu}^{(k+1)}
            \]
            \begin{tikzpicture}[remember picture,overlay]
              \draw[->,>=latex,ultra thick]
              ++(173pt,30pt) 
              node[anchor=north,text width=8cm] 
              {Shift invariance
              } -| ([shift={(0pt,0pt)}]shift)
              ;

              \draw[->,>=latex,ultra thick]
              ++(319pt,125pt) 
              node[anchor=south,text width=5cm] 
              {dual Pieri
              } -| ([shift={(0pt,0pt)}]pieri)
              ;              
            \end{tikzpicture}
          \end{alertblock}
          \end{block}
      \end{column}
      \begin{column}{0.19\paperwidth}
        \begin{block}{K-theoretic Catalan Functions}
          \begin{itemize}
          \item For \(\gamma \in \Z^\ell\), root ideals
            \(\Psi,\lowers\) \[
              \hspace{-0.5in} K(\Psi;\lowers;\gamma) := \prod_{\mathclap{(i,j) \in
                \lowers}}(1-L_j)\prod_{\mathclap{(i,j) \in \Delta^+ \setminus
                \Psi}}(1-R_{ij}) Kh_\gamma
            \]
          \item Lowering operators \(L_j\)
            \[ L_3\left( \ydiagram{1,1,3} \right) =
              \ydiagram{1,3} \ \ L_{1}\left( \ydiagram{1,1,3} \right)
              = \ydiagram{1,1,2}
            \]
            \begin{itemize}
            \item For example (\(\textcolor{red}{\Psi},
              \textcolor{blue}{\lowers}, \textcolor{violet}{\Psi
                \intersect \lowers}\)) \[
                  \begin{tikzpicture}[every node/.style={minimum size=0.95cm-\pgflinewidth, outer sep=0pt}]
                    \draw[step=1cm,color=black] (0,0) grid (4.0,-4.0);
                    \node[fill=red] at (1.5,-0.5) {};
                    \node[fill=red] at (2.5,-0.5) {};
                    \node[fill=violet] at (3.5,-0.5) {};
                    \node[fill=violet] at (3.5,-1.5) {};
                    \node[fill=blue] at (3.5,-2.5) {};
                    \node at (0.5,-0.5) {$3$};
                    \node at (1.5,-1.5) {$4$};
                    \node at (2.5,-2.5) {$2$};
                    \node at (3.5,-3.5) {$1$};
                  \end{tikzpicture}
                  := (1-L_4)^3(1-R_{23})(1-R_{34})Kh_{3421}
                \]            
              \item \(K(\emptyset;\emptyset;\lambda) = g_\lambda\) \[
                                    \begin{tikzpicture}[every node/.style={minimum size=0.95cm-\pgflinewidth, outer sep=0pt}]
                    \draw[step=1cm,color=black] (0,0) grid (4.0,-4.0);
                    \node at (0.5,-0.5) {$3$};
                    \node at (1.5,-1.5) {$2$};
                    \node at (2.5,-2.5) {$1$};
                    \node at (3.5,-3.5) {$1$};
                  \end{tikzpicture}
                  = g_{3211}\]
              \end{itemize}
            \end{itemize}
              \begin{subtleblock}
                \(\g_\lambda^{(k)} = K(\Psi;\lowers;\lambda)\) with
                \(band(\Psi) = k\), \(band(\lowers)=k+1\) \[
                  \begin{tikzpicture}[every node/.style={minimum
                      size=0.95cm-\pgflinewidth, outer sep=0pt}]
                    \draw[step=1cm,color=black] (0,0) grid (4.0,-4.0);
                    \node[fill=red] at (1.5,-0.5) {};
                    \node[fill=violet] at (2.5,-0.5) {};
                    \node[fill=violet] at (3.5,-0.5) {};
                    \node[fill=red] at (3.5,-1.5) {}; \node at
                    (0.5,-0.5) {$4$}; \node at (1.5,-1.5)
                    {$3$}; \node at (2.5,-2.5)
                    {$1$}; \node at (3.5,-3.5) {$1$};
                  \end{tikzpicture}
                  = \g_{4311}^{(4)}
                \]
              \end{subtleblock}
              \begin{alertblock}{Conjecture}
                \[\g_\lambda^{(k)} = g_\lambda^{(k)}\]
              \end{alertblock}              
              \begin{alertblock}{Theorem: Shift Invariance}
                \[
                  G_{1^\ell}^\perp \g_{\lambda+1^\ell}^{(k+1)} =
                  \g_\lambda^{(k)}
                \]
                Thus, understanding dual Pieri rule gives the
                \(\g_\lambda^{(k)}\) branching coefficients. 
              \end{alertblock}
            \end{block}
        % \begin{block}{Vertical Pieri Rule}
        %   \begin{itemize}
        %   \item \[
        %       g_{11} \ \begin{tikzpicture}[every node/.style={minimum
        %           size=0.95cm-\pgflinewidth, outer sep=0pt}]
        %         \draw[step=1cm,color=black] (0,0) grid (4.0,-4.0);
        %         \node[fill=red] at (1.5,-0.5) {}; \node[fill=violet]
        %         at (2.5,-0.5) {}; \node[fill=violet] at (3.5,-0.5) {};
        %         \node[fill=red] at (3.5,-1.5) {}; \node at (0.5,-0.5)
        %         {$4$}; \node at (1.5,-1.5)
        %         {$3$}; \node at (2.5,-2.5)
        %         {$1$}; \node at (3.5,-3.5) {$1$};
        %       \end{tikzpicture}
        %       = \begin{tikzpicture}[every node/.style={minimum
        %           size=.95cm-\pgflinewidth, outer sep=0pt}]
        %         \draw[step=1cm,color=black] (0,0) grid (6.0,-6.0);
        %         \node[fill=red] at (1.5,-0.5) {}; \node[fill=violet]
        %         at (2.5,-0.5) {}; \node[fill=violet] at (3.5,-0.5) {};
        %         \node[fill=violet] at (4.5,-0.5) {};
        %         \node[fill=violet] at (5.5,-0.5) {}; \node[fill=red]
        %         at (3.5,-1.5) {}; \node[fill=violet] at (4.5,-1.5) {};
        %         \node[fill=violet] at (5.5,-1.5) {};
        %         \node[fill=violet] at (4.5,-2.5) {};
        %         \node[fill=violet] at (5.5,-2.5) {};
        %         \node[fill=violet] at (4.5,-3.5) {};
        %         \node[fill=violet] at (5.5,-3.5) {}; \node at
        %         (0.5,-0.5) {$4$}; \node at (1.5,-1.5)
        %         {$3$}; \node at (2.5,-2.5)
        %         {$1$}; \node at (3.5,-3.5)
        %         {$1$}; \node at (4.5,-4.5)
        %         {$1$}; \node at (5.5,-5.5) {$1$};
        %       \end{tikzpicture}
        %     \]
        %   \item Apply root removal identities, eg
        %     \begin{align*}
        %                     \begin{tikzpicture}[every node/.style={minimum
        %           size=.95cm-\pgflinewidth, outer sep=0pt}]
        %         \draw[step=1cm,color=black] (0,0) grid (6.0,-6.0);
        %         \node[fill=red] at (1.5,-0.5) {}; \node[fill=violet]
        %         at (2.5,-0.5) {}; \node[fill=violet] at (3.5,-0.5) {};
        %         \node[fill=violet] at (4.5,-0.5) {};
        %         \node[fill=violet] at (5.5,-0.5) {}; \node[fill=red]
        %         at (3.5,-1.5) {}; \node[fill=violet] at (4.5,-1.5) {};
        %         \node[fill=violet] at (5.5,-1.5) {};
        %         \node[fill=violet] at (4.5,-2.5) {};
        %         \node[fill=violet] at (5.5,-2.5) {};
        %         \node[fill=violet] at (4.5,-3.5) {};
        %         \node[fill=violet] at (5.5,-3.5) {}; \node at
        %         (0.5,-0.5) {$4$}; \node at (1.5,-1.5)
        %         {$3$}; \node at (2.5,-2.5)
        %         {$1$}; \node at (3.5,-3.5)
        %         {$1$}; \node at (4.5,-4.5)
        %         {$1$}; \node at (5.5,-5.5) {$1$};
        %       \end{tikzpicture}
        %       & = \begin{tikzpicture}[every node/.style={minimum
        %           size=.95cm-\pgflinewidth, outer sep=0pt}]
        %         \draw[step=1cm,color=black] (0,0) grid (6.0,-6.0);
        %         \node[fill=red] at (1.5,-0.5) {}; \node[fill=violet]
        %         at (2.5,-0.5) {}; \node[fill=violet] at (3.5,-0.5) {};
        %         \node[fill=violet] at (4.5,-0.5) {};
        %         \node[fill=violet] at (5.5,-0.5) {}; \node[fill=red]
        %         at (3.5,-1.5) {}; \node[fill=violet] at (4.5,-1.5) {};
        %         \node[fill=violet] at (5.5,-1.5) {};
        %         \node[fill=violet] at (4.5,-2.5) {};
        %         \node[fill=violet] at (5.5,-2.5) {};
        %         \node[fill=violet] at (5.5,-3.5) {}; \node at
        %         (0.5,-0.5) {$4$}; \node at (1.5,-1.5)
        %         {$3$}; \node at (2.5,-2.5)
        %         {$1$}; \node at (3.5,-3.5)
        %         {$1$}; \node at (4.5,-4.5)
        %         {$1$}; \node at (5.5,-5.5) {$1$};
        %       \end{tikzpicture}
        %         +
        %         \underbrace{
        %         \begin{tikzpicture}[every node/.style={minimum
        %           size=.95cm-\pgflinewidth, outer sep=0pt}]
        %         \draw[step=1cm,color=black] (0,0) grid (6.0,-6.0);
        %         \node[fill=red] at (1.5,-0.5) {}; \node[fill=violet]
        %         at (2.5,-0.5) {}; \node[fill=violet] at (3.5,-0.5) {};
        %         \node[fill=violet] at (4.5,-0.5) {};
        %         \node[fill=violet] at (5.5,-0.5) {}; \node[fill=red]
        %         at (3.5,-1.5) {}; \node[fill=violet] at (4.5,-1.5) {};
        %         \node[fill=violet] at (5.5,-1.5) {};
        %         \node[fill=violet] at (4.5,-2.5) {};
        %         \node[fill=violet] at (5.5,-2.5) {};
        %         \node[fill=violet] at (4.5,-3.5) {};
        %         \node[fill=violet] at (5.5,-3.5) {}; \node at
        %         (0.5,-0.5) {$4$}; \node at (1.5,-1.5)
        %         {$3$}; \node at (2.5,-2.5)
        %         {$1$}; \node at (3.5,-3.5)
        %         {$2$}; \node at (4.5,-4.5)
        %         {$0$}; \node at (5.5,-5.5) {$1$};
        %       \end{tikzpicture}
        %       }_{=0}\\
        %       & -
        %       \underbrace{
        %       \begin{tikzpicture}[every node/.style={minimum
        %           size=.95cm-\pgflinewidth, outer sep=0pt}]
        %         \draw[step=1cm,color=black] (0,0) grid (6.0,-6.0);
        %         \node[fill=red] at (1.5,-0.5) {}; \node[fill=violet]
        %         at (2.5,-0.5) {}; \node[fill=violet] at (3.5,-0.5) {};
        %         \node[fill=violet] at (4.5,-0.5) {};
        %         \node[fill=violet] at (5.5,-0.5) {}; \node[fill=red]
        %         at (3.5,-1.5) {}; \node[fill=violet] at (4.5,-1.5) {};
        %         \node[fill=violet] at (5.5,-1.5) {};
        %         \node[fill=violet] at (4.5,-2.5) {};
        %         \node[fill=violet] at (5.5,-2.5) {};
        %         \node[fill=violet] at (4.5,-3.5) {};
        %         \node[fill=violet] at (5.5,-3.5) {}; \node at
        %         (0.5,-0.5) {$4$}; \node at (1.5,-1.5)
        %         {$3$}; \node at (2.5,-2.5)
        %         {$1$}; \node at (3.5,-3.5)
        %         {$1$}; \node at (4.5,-4.5)
        %         {$0$}; \node at (5.5,-5.5) {$1$};
        %       \end{tikzpicture}
        %         }_{=0} \\
        %     & \vdots \\
        %     & =
        %       g_{431111}^{(4)}+\textcolor{red}{g_{43121}^{(4)}}+g_{43211}^{(4)}+g_{4322}^{(4)}\\
        %     & -2g_{43111}^{(4)}-g_{4321}^{(4)}-\textcolor{red}{g_{4312}^{(4)}}+g_{4311}^{(4)}
        %     \end{align*}
        %     \item Apply ``straightening identities.''
        %       \begin{itemize}
        %       \item \(g_{43121}^{(4)} = g_{44111}^{(4)}+g_{43111}^{(4)}\)
        %       \item \(g_{4312}^{(4)} = g_{4411}^{(4)}+g_{4311}^{(4)}\)
        %       \end{itemize}
        %     \end{itemize}
        %     \begin{alertblock}{Technique}
        %       Use above approach to show formula of Conjecture 3 and
        %       \(g_\lambda^{(k)}\) satisfy the same vertical Pieri
        %       rule, thus proving Conjecture 3.
        %     \end{alertblock}
        % \end{block}
        % \begin{block}{Towards a Vertical Dual Pieri Rule}
        %   \begin{itemize}
        %   \item Conjecture 4 gives Catalan formula for \(g_\lambda^{(k)}\).
        %   \item Catalan-theoretic techniques are promising for dual
        %     Pieri rule. Could yield an answer to Conjecture 2.
        %   \item Theorem + Conjecture 2 answers Conjecture 1.
        %   \end{itemize}
        % \end{block}
        
        \setbeamercolor{block title}{fg=jblue,bg=white} % Change the block title color
        \begin{block}{References}
          \begin{bibdiv}
            \begin{biblist}
              \bib{bmps}{article}{
                author={Blasiak, Jonah}
                author={Morse, Jennifer}
                author={Pun, Anna}
                author={Summers, Daniel}
                title={Catalan Functions and \(k\)-Schur Positivity}
                year={2019}
                journal={Journal of the AMS}
              }
              \bib{lss}{article}{
                author={Lam, Thomas}
                author={Schilling, Anne}
                author={Shimozono, Mark}
                title={K-theory Schubert calculus of the affine
                  Grassmannian}
                year={2010}
                journal={Compositio Math.}
                volume={146}
                pages={811--852}
              }
              \bib{morse}{article}{
                author={Morse, Jennifer}
                title={Combinatorics of the K-theory of affine
                  Grassmannians}
                year={2011}
                journal={Advances in Mathematics}
              }
            \end{biblist}
          \end{bibdiv}
        \end{block}
      \end{column}
    \end{columns}
  \end{frame} % End the wrapping frame
  \end{document}
