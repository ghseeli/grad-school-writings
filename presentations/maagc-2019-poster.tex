\documentclass[final]{beamer}

\mode<presentation>
  {
    \usetheme{confposter} 
  }
\usepackage{../ReAdTeX/readtex-core}
\usepackage{amsmath, amsthm, amssymb, latexsym}
\usepackage{mathtools}
\usepackage{mathdots}
\usepackage[alphabetic,abbrev]{amsrefs} % use AMS ref scheme
\usepackage{anyfontsize}
\usepackage{scalefnt, scalerel}
\usepackage{array,booktabs,tabularx}  
\usepackage[orientation=landscape, size=a0, scale=1.5]{beamerposter} % Use the beamerposter package for laying out the poster
\RequirePackage{lmodern}
\usepackage{lmodern}
\usepackage{exscale} % Necessary for math symbols to be big enough
\usepackage{ytableau}
\usepackage{tikz}
\usetikzlibrary{tikzmark,decorations.pathreplacing}

%% Blocks and Colors
\definecolor{pastelred}{rgb}{1.0, 0.41, 0.38}
\definecolor{tearose}{rgb}{0.96, 0.76, 0.76}
\definecolor{pastelpink}{rgb}{1.0, 0.82, 0.86}
\definecolor{palepink}{rgb}{0.98, 0.85, 0.87}
\definecolor{indianred}{rgb}{0.8, 0.36, 0.36}
\definecolor{coralred}{rgb}{1.0, 0.25, 0.25}
\definecolor{lightblue}{rgb}{.68,.85,.9} % 

\tikzset{wei/.style=%{red,double=pink,thick,doubledistance=1.5pt}}
{red,double=red,double
distance=0.5pt}}

\tikzset{wei2/.style={red,double=red,double
    distance=0.5pt}}

\newenvironment<>{myexampleblock}[1]{%
  \setbeamercolor{block alerted title}{fg=white,bg=coralred}
  \setbeamercolor{block alerted body}{fg=black,bg=palepink}
  \begin{alertblock}#2{#1}}{\end{alertblock}}
\newenvironment<>{subtleblock}{
  \setbeamercolor{block alerted title}{fg=white,bg=white} 
  \setbeamercolor{block alerted body}{fg=black,bg=dblue!10}
  \begin{alertblock}#1{}}{\end{alertblock}}
\newenvironment<>{subtleexampleblock}{
  \setbeamercolor{block alerted title}{fg=white,bg=white}
  \setbeamercolor{block alerted body}{fg=black,bg=palepink}
  \begin{alertblock}#1{\vspace{-20ex}}}{\end{alertblock}}

\setbeamercolor{block title}{fg=jblue,bg=white}  % Colors of the block
% titles
\setbeamercolor{block body}{fg=black,bg=white} % Colors of the body of blocks
\setbeamercolor{block alerted title}{fg=white,bg=dblue!70} % Colors of the highlighted block titles
\setbeamercolor{block alerted body}{fg=black,bg=dblue!10} % Colors of the body of highlighted blocks
\definecolor{melon}{rgb}{0.99, 0.74, 0.71}
\addtobeamertemplate{block end}{}{\vspace*{1ex}} % White space under blocks
\addtobeamertemplate{block alerted end}{}{\vspace*{1ex}} % White space
% under highlighted (alert) blocks

%% Math shortcuts
\newcommand{\lowers}{\mathcal{L}}
\newcommand{\g}{\mathfrak{g}}
\newcommand{\covers}{\mathrel{\gtrdot}}
\newcommand{\coveredby}{\mathrel{\lessdot}}
\DeclareMathOperator{\weight}{weight}
\DeclareMathOperator{\band}{band}
\newcommand{\mynone}{\ }

%% Title

\setbeamertemplate{headline}{
 \leavevmode
  \begin{columns}
   \begin{column}{\linewidth}
    \vskip1cm
    \raggedright
    \usebeamercolor{title in
      headline}{\color{jblue}\Large{\textbf{\inserttitle}}\\[0.5ex]}
    \raggedleft
    \vspace{-2cm}
    \usebeamercolor{author in headline}{\color{fg}\large{\insertauthor}\\[0.5ex]}
%    \usebeamercolor{institute in headline}{\color{fg}\large{\insertinstitute}\\[1ex]}
%    \vskip1cm
   \end{column}
   \vspace{1cm}
  \end{columns}
 \vspace{0.5in}
 \hspace{0.5in}\begin{beamercolorbox}[wd=47in,colsep=0.15cm]{cboxb}\end{beamercolorbox}
 \vspace{0.1in}
}

%% Title matter
  \title[K-theoretic Catalan Functions]{\(K\)-THEORETIC CATALAN FUNCTIONS}
  \author{George H. Seelinger (joint with Jonah Blasiak and Jennifer
      Morse)}
  \institute[UVA]{\ \vspace{-3ex}}
  \date{\ \vspace{-1ex}}
%% Document
\begin{document}
  \begin{frame}[t] % The whole poster is in one Beamer frame
    \begin{columns}[t] % Align all column contents to the top
      \begin{column}{.19\paperwidth}
        \begin{block}{Overview}
          \begin{itemize}
          \item Schur functions, \(s_\lambda\), and Grothendieck
            polynomials, \(G_\lambda\), give
            representatives for cohomology and \(K\)-theory of the
            Grassmannian.
          \item Pieri rules determine the structure constants of these
            rings.
          \item Representatives and Pieri rules are known for
            cohomology of affine Grassmannian.
          \item We seek to better understand the corresponding picture
            for the \(K\)-theory of the affine Grassmannian.
          \end{itemize}
        \end{block}
        \begin{block}{Affine combinatorics}
          \(w \in \tilde{S}_n \correspondsto\) 
        \(n\)-cores
        \ytableausetup{boxsize=0.6em}
        \[
          \ydiagram{1,2,3,4,5,5}*[\bullet]{0,1+1,1+1,1+1,1+4}*[*(red)]{0,0,0,0,1+1}
          \ \ \overset{n\text{-core}=\text{partition with no cell of
              hook-length }n}{\scriptstyle \text{\emph{red cell has hook-length }}7}
        \]
        \textbf{Weak order}
        \begin{itemize}
        \item Covers differ by boxes of same color.
          \ytableausetup{boxsize=0.8em}
          \[
            \ydiagram{1,1,2,4,6,8,10,12}*[*(blue)]{1}*[*(red)]{0,0,1}*[*(blue)]{0,0,1+1,1}*[*(red)]{0,0,0,2+1,1+1,1}*[*(blue)]{0,0,0,3+1,2+1,1+1,1}*[*(red)]{0,0,0,0,4+1,3+1,2+1,1+1}*[*(blue)]{0,0,0,0,5+1,4+1,3+1,2+1}*[*(red)]{0,0,0,0,0,6+1,5+1,4+1}*[*(blue)]{0,0,0,0,0,7+1,6+1,5+1}*[*(red)]{0,0,0,0,0,0,8+1,7+1}*[*(blue)]{0,0,0,0,0,0,9+1,8+1}*[*(red)]{0,0,0,0,0,0,0,10+1}*[*(blue)]{0,0,0,0,0,0,0,11+1}
          \]
          \\
          \[
            \emptyset \coveredby \ydiagram{1}*[*(black)]{1} \coveredby
            \ydiagram{2}*[*(black)]{1+1} \coveredby \ydiagram{1,3}*[*(black)]{1,2+1}
            \coveredby \ydiagram{2,4}*[*(black)]{1+1,3+1}
          \]
        \end{itemize}
        \textbf{Strong (Bruaht) order}
        \begin{itemize}
        \item Ordered by containment of shapes.
        \item Covers differ by a ribbon \(+\) its copies.
          \ytableausetup{boxsize=0.45em}
              \[
                \ytableausetup{centertableaux}
                % \ydiagram{1,1,4,4} \coveredby \ydiagram{1,1,1,4,4}
                % \coveredby \ydiagram{1,1,2,4,5} \coveredby
                % \ydiagram{2,2,2,4,5} \coveredby \ydiagram{2,2,2,5,5}
                % \coveredby \ydiagram{1,2,2,3,5,6} \coveredby
                \ydiagram{1,1,2,2,3,5,6} \coveredby
                \ydiagram{1,1,1,1,3,3,3,5,6} \coveredby
                \ydiagram{1,1,1,1,3,3,3,6,6} \coveredby
                \ydiagram{1,2,2,3,4,4,5,6,6}*[*(red)]{0,1+1,1+1,1+2,3+1,3+1,3+2}
              \]
         \end{itemize}
        \end{block}
      \end{column}
      \begin{column}{.38\paperwidth}
        \vspace{-2cm}
        \begin{columns}[t]
          \begin{column}{.18\paperwidth}
            \begin{block}{Dual \(k\)-Schur Functions}
              Generating functions of weak tableaux. \[
                F_\lambda^{(k)} := \sum x^{\weight(T)}
              \]
              \vspace{-1,5cm}
              \begin{subtleblock}
                \textbf{Weak Tableaux:} \\
                Maximal chains in the weak order.
                \[
                  \ytableausetup{boxsize=0.6em} \emptyset \subset
                  \ydiagram{1} \subset \ydiagram{1,1} \subset
                  \ydiagram{1,1,2} \subset \ydiagram{1,1,3} \to
                  \ytableausetup{boxsize=1em} \ytableaushort{3,2,134}
                \]
              \end{subtleblock}

            \end{block}
          \end{column}
          \begin{column}{.18\paperwidth}
            \begin{block}{\(k\)-Schur Functions}
              Generating functons of strong tabelaux. \[
                s_\lambda^{(k)} := \sum x^{\weight(T)}
              \]
              \vspace{-1.5cm}
              \begin{subtleblock}
                \textbf{Marked cover:} Strong cover with selection of
                one ribbon \\
                \textbf{Strong Tableaux:} Maximal strong order chains of marked covers
                \ytableausetup{boxsize=0.3em}
                \begin{align*}
                  \emptyset \subset \ydiagram[*(black)]{1} \subset
                  \ydiagram{1,1}*[*(black)]{1} \subset
                  \ydiagram{1,1,2}*[*(black)]{1} & \\
                  \emptyset \subset \ydiagram[*(black)]{1} \subset
                  \ydiagram{1,1}*[*(black)]{1} \subset
                  \ydiagram{1,1,2}*[*(black)]{0,0,1+1} 
                   \\
                  \emptyset \subset \ydiagram[*(black)]{1} \subset
                  \ydiagram{2}*[*(black)]{1+1} \subset
                  \ydiagram{1,1,2}*[*(black)]{1,1}
                \end{align*}
              \end{subtleblock}
            \end{block}
          \end{column}
        \end{columns}
        \vspace{-2cm}
        \textbf{Pieri rule:}
        \[
          e_r F_\lambda^{(k)} = \ \ \ \sum_{\mathclap{\substack{\mu = \lambda + \text{ strong
              marked vertical }\\\text{strip of size }r}}} \ \ \ F_\mu^{(k)}
          \iff e_r^\perp s_\mu^{(k)} = \ \ \ \sum_{\mathclap{\substack{\lambda = \mu - \text{
              strong marked vertical }\\\text{strip of size }r}}} \ \
      \ s_\lambda^{(k)}
    \]
    where a \textbf{strong vertical strip} is a chain of marked covers
    with marking proceeding north to south.
        \begin{columns}[t]
          \begin{column}{.18\paperwidth}
            \begin{block}{Affine Grothendieck Polynomials}
            Generating functions of affine SVTs.\[
                G_\lambda^{(k)} := \sum
                (-1)^{|\lambda|+|\weight(T)|} x^{\weight(T)}
              \]
              \vspace{-1.5cm}
              \begin{subtleblock}
                \textbf{Affine Set-Valued Tableaux:} \\
                Each \(T_{\leq x}\) is a \(k+1\)-core.
                  \ytableausetup{boxsize=1.0em}
                  \[
                    \hspace{-0.6in} T =
                    \ytableaushort{7,{\scriptstyle{2,5}}6,1{\scriptstyle{2,3}}4{\scriptstyle{4,6}}}
                    \ \ T_{\leq 4} =
                    \ytableaushort{2,1{\scriptstyle{2,3}}44}
                  \]                
                \end{subtleblock}
                \begin{itemize}
                \item \(G_\lambda^{(k)} = F_\lambda^{(k)} + \text{ higher
                    order terms}\)
                \item \(G_\lambda^{(k)} = G_\lambda\) for
                  large \(k\).
                \end{itemize}
            \end{block}
          \end{column}
          \begin{column}{.18\paperwidth}
            \begin{block}{Dual Affine Grothendieck Polynomials}
              \(g_\lambda^{(k)}\) is dual basis to
              \(G_\lambda^{(k)}\).
              \begin{itemize}
              \item \(g_\lambda^{(k)} = s_\lambda^{(k)} + \text{ lower
                  order terms}\)
              \item \(g_\lambda^{(k)} = g_\lambda\), dual to
                \(G_\lambda\), for large \(k\).
            \end{itemize}
            \begin{alertblock}{Open Problem}
              Find a direct definition of \(g_\lambda^{(k)}\).
            \end{alertblock}
            \end{block}
          \end{column}
        \end{columns}
        \begin{alertblock}{Open Problem}
          \begin{center}
            Describe the \(G_\lambda^{(k)}\) Pieri rule. \(\iff\) Describe the \(g_\lambda^{(k)}\) dual Pieri rule.
          \end{center}
        \end{alertblock}
        \begin{block}{Branching}
          \begin{itemize}
          \item \(k\)-Schur functions are \(k+1\)-Schur positive:
            \(s_\lambda^{(k)} = \sum a_{\lambda \mu}^{(k)} s_\mu^{(k+1)}\).
          \item Iteration gives Schur positivity of \(k\)-Schur functions.
          \item \textbf{Conjecture:} \(g_\lambda^{(k)}\) is Schur positive.
          \item \textbf{Conjecture:} \(g_\lambda^{(k)} = \sum_\mu (-1)^{|\lambda|-|\mu|} a_{\lambda \mu}^{(k)}
            g_\mu^{(k+1)}\) for \(a_{\lambda \mu}^{(k)} \in \Z_{\geq 0}\).
          \end{itemize}
        \end{block}
        \end{column}
        \begin{column}{0.19\paperwidth}
        \begin{block}{Catalan Functions}
          For \(\gamma \in \Z^\ell\), \[
            H(\Psi;\gamma) := \prod_{(i,j) \not\in 
              \Psi} (1-R_{ij}) h_\gamma
          \]
          \begin{itemize}
          \item Raising operators \(R_{i,j}(h_\lambda) = h_{\lambda+\epsilon_i-\epsilon_j}\) \ytableausetup{boxsize=0.5em}
            \[
              R_{1,3} \left( \ydiagram{1,1,3}*[*(red)]{1} \right)
              = \ydiagram{1,4}*[*(red)]{0,3+1} \ \ \ R_{2,3}
              \left( \ydiagram{1,1,1}*[*(red)]{1} \right) =
              \ydiagram{2,1}*[*(red)]{1+1} 
            \]
          \item Root ideal \(\Psi\): given by Dyck path.
            \ytableausetup{mathmode, boxsize=1em,centertableaux}
            \[
              \Psi =
              \begin{ytableau}
                \mynone &*(lightblue)\text{\tiny (12)}  &*(red)\text{\tiny (13)}   &*(red)\text{\tiny (14)}  &*(red)
                \text{\tiny (15)}\\
                \mynone &\mynone &*(red) \text{\tiny (23)}  &*(red)\text{\tiny (24)}
                &*(red) \text{\tiny (25)}\\
                \mynone &\mynone &\mynone &*(lightblue) \text{\tiny (34)}
                &*(red)\text{\tiny (35)} \\
                \mynone &\mynone &\mynone&\mynone&*(lightblue) \text{\tiny (45)}\\
                \mynone &\mynone &\mynone&\mynone&\mynone\\
              \end{ytableau}
              \ \ \ \overset{\text{\textcolor{red}{\scriptsize{Roots above Dyck
                    path}}}}{\text{\textcolor{blue}{\scriptsize{Non-roots below}}}}
            \]
          \end{itemize}
          \vspace{-0.5cm}
          \begin{align*}
            & H(\Psi;54332) \\
            & = \textcolor{blue}{(1-R_{12})(1-R_{34})(1-R_{45})} h_{54332} \\
            & = h_{54332}-h_{45332}-h_{54422}-h_{54341}\\
            & + h_{45422} + h_{45341} + h_{54431} - h_{45431}
          \end{align*}
          \vspace{-1cm}
          \begin{itemize}
            \item \(H(\emptyset;\lambda) = s_\lambda\) (Jacobi-Trudi
              Identity)
            \end{itemize}
          \end{block}
          \begin{block}{\(k\)-Schur Catalans}
            \(s_\lambda^{(k)} = H(\Psi;\lambda)\) for particular
            \(\Psi\), \\
            defined by \(\underbrace{\lambda_i+\#
              \text{non-roots in row } i}_{\text{bandwidth}}~=~k\).
            \ytableausetup{mathmode, boxsize=1em,centertableaux}
            \[
              s_{33211}^{(4)} = 
              {\footnotesize
                \begin{ytableau}
                  *(white) 3     &*(blue)  &*(red)   &*(red)  &*(red)  &*(red) \\
                  \mynone & *(white)3 & *(blue) & *(red) & *(red)  &*(red)  \\
                  \mynone &*(white)  & *(white)2 & *(blue) & *(blue)  &*(red)  \\
                  \mynone &*(white)  & *(white)  & *(white)1 & *(blue) &*(blue) \\
                  \mynone &\mynone  &\mynone  &\mynone  & *(white)1 & *(blue) \\
                  \mynone &\mynone  &\mynone  &\mynone  &*(white)  & *(white) 1
                \end{ytableau}
              }
              \leftarrow\,\text{\small{4 - 2 non-roots}}
              \quad
            \]\[
              s_{44322}^{(5)} =
              {\footnotesize
                \begin{ytableau}
                  *(white) 4     &*(blue)  &*(red)   &*(red)  &*(red)  &*(red) \\
                  \mynone & *(white)4 & *(blue) & *(red) & *(red)  &*(red)  \\
                  \mynone &*(white)  & *(white)3 & *(blue) & *(blue)  &*(red)  \\
                  \mynone &*(white)  & *(white)  & *(white)3 & *(blue) &*(blue) \\
                  \mynone &\mynone  &\mynone  &\mynone  & *(white)2 & *(blue) \\
                  \mynone &\mynone  &\mynone  &\mynone  &*(white)  & *(white) 2
                \end{ytableau}
              }
              \leftarrow\, \text{\small{5 - 3 non-roots}}
            \]
            \textbf{Why use Catalan \(k\)-Schurs?} \\
                \begin{subtleblock}
                  \textbf{Shift Invariance:}
                  \[
                    e_\ell^\perp s_{\lambda+1^\ell}^{(k+1)} =
                    s_\lambda^{(k)}
                  \]
              \end{subtleblock}
              \vspace{-0.3cm}
              \begin{alertblock}{Corollary}
              \(s^{(k)}_\lambda\) branching follows from dual
              Pieri rule!
              \vspace{0.5cm}
              \[
                  s_\lambda^{(k)} \tikzmarknode{shift}{=} e_\ell^\perp
                  s_{\lambda+1^\ell}^{(k+1)} \tikzmarknode{pieri}{=} \, \sum_{\mu} \, s_{\mu}^{(k+1)}
            \]
            \begin{tikzpicture}[remember picture,overlay]
              \draw[->,>=latex,ultra thick]
              ++(173pt,30pt) 
              node[anchor=north,text width=8cm] 
              {Shift invariance
              } -| ([shift={(0pt,0pt)}]shift)
              ;

              \draw[->,>=latex,ultra thick]
              ++(319pt,100pt) 
              node[anchor=south,text width=5cm] 
              {dual Pieri
              } -| ([shift={(0pt,0pt)}]pieri)
              ;              
            \end{tikzpicture}
          \end{alertblock}
          \end{block}
      \end{column}
      \begin{column}{0.19\paperwidth}
        \begin{block}{K-theoretic Catalan Functions}
          For \(\gamma \in \Z^\ell\), root ideals
            \(\Psi,\lowers\) \[
              \hspace{-0.5in} K(\Psi;\lowers;\gamma) := \prod_{\mathclap{(i,j) \in
                \lowers}}(1-L_j)\prod_{\mathclap{(i,j) \not\in
                \Psi}}(1-R_{ij}) Kh_\gamma
            \]
          \begin{itemize}
          \item Lowering operators \(L_j(Kh_\lambda) = Kh_{\lambda-\epsilon_j}\)
            \ytableausetup{boxsize=0.6em}
            \[ L_3\left( \ydiagram{1,1,3} \right) =
              \ydiagram{1,3} \ \ L_{1}\left( \ydiagram{1,1,3} \right)
              = \ydiagram{1,1,2}
            \]
          \item \(\textcolor{blue}{\text{non-roots of }\Psi},
              \textcolor{coralred}{\text{roots of }\lowers}\)
              \ytableausetup{mathmode,
                boxsize=1em,centertableaux} \[
              \begin{ytableau}
                \mynone &*(lightblue)\text{\tiny (12)}  &*(white)\text{\tiny (13)}   &*(coralred)\text{\tiny (14)}  &*(coralred)
                \text{\tiny (15)}\\
                \mynone &\mynone &*(white) \text{\tiny (23)}  &*(coralred)\text{\tiny (24)}
                &*(coralred) \text{\tiny (25)}\\
                \mynone &\mynone &\mynone &*(lightblue) \text{\tiny (34)}
                &*(white)\text{\tiny (35)} \\
                \mynone &\mynone &\mynone&\mynone&*(lightblue) \text{\tiny (45)}\\
                \mynone &\mynone &\mynone&\mynone&\mynone\\
              \end{ytableau}
            \]
            \begin{align*}
              & K(\Psi;\lowers;54332) \\
              & = \textcolor{coralred}{(1-L_{4})^2(1-L_{5})^2}\\
              & \cdot \textcolor{blue}{(1-R_{12})(1-R_{34})(1-R_{45})} Kh_{54332}
            \end{align*}
              \item \(K(\emptyset;\emptyset;\lambda) = g_\lambda\).
              \end{itemize}
              \begin{subtleblock}
                \(\g_\lambda^{(k)} := K(\Psi;\lowers;\lambda)\) with
                \(\band(\Psi) = k\), \(\band(\lowers)=k+1\) \[
                  \g_{33211}^{(4)} = {\footnotesize
                \begin{ytableau}
                  *(white) 3     &*(blue)  &*(white)   &*(coralred)  &*(coralred)  &*(coralred) \\
                  \mynone & *(white)3 & *(blue) & *(white) & *(coralred)  &*(coralred)  \\
                  \mynone &*(white)  & *(white)2 & *(blue) & *(blue)  &*(white)  \\
                  \mynone &*(white)  & *(white)  & *(white)1 & *(blue) &*(blue) \\
                  \mynone &\mynone  &\mynone  &\mynone  & *(white)1 & *(blue) \\
                  \mynone &\mynone  &\mynone  &\mynone  &*(white)  & *(white) 1
                \end{ytableau}
              }
                \]
              \end{subtleblock}
              \begin{alertblock}{Theorem: Shift Invariance}
                \[
                  G_{1^\ell}^\perp\, \g_{\lambda+1^\ell}^{(k+1)} =
                  \g_\lambda^{(k)}
                \]
              \end{alertblock}
              \begin{alertblock}{Corollary}
                \(\g_\lambda^{(k)}\) branching follows from dual Pieri rule.
              \end{alertblock}
              \begin{alertblock}{Conjecture}
                \[\g_\lambda^{(k)} = g_\lambda^{(k)}\]
              \end{alertblock}              
            \end{block}        
        % \setbeamercolor{block title}{fg=jblue,bg=white} % Change the block title color
        % \begin{block}{References}
        %   \begin{bibdiv}
        %     \begin{biblist}
        %       \bib{bmps}{article}{
        %         author={Blasiak, Jonah}
        %         author={Morse, Jennifer}
        %         author={Pun, Anna}
        %         author={Summers, Daniel}
        %         title={Catalan Functions and \(k\)-Schur Positivity}
        %         year={2019}
        %         journal={Journal of the AMS}
        %       }
        %       \bib{lss}{article}{
        %         author={Lam, Thomas}
        %         author={Schilling, Anne}
        %         author={Shimozono, Mark}
        %         title={K-theory Schubert calculus of the affine
        %           Grassmannian}
        %         year={2010}
        %         journal={Compositio Math.}
        %         volume={146}
        %         pages={811--852}
        %       }
        %       \bib{morse}{article}{
        %         author={Morse, Jennifer}
        %         title={Combinatorics of the K-theory of affine
        %           Grassmannians}
        %         year={2011}
        %         journal={Advances in Mathematics}
        %       }
        %     \end{biblist}
        %   \end{bibdiv}
        % \end{block}
      \end{column}
    \end{columns}
  \end{frame} % End the wrapping frame
  \end{document}
