\documentclass[final]{beamer}

\mode<presentation>
  {
    \usetheme{confposter} 
  }
\usepackage{amsmath, amsthm, amssymb, latexsym}
\usepackage{mathtools}
\usepackage[alphabetic,abbrev]{amsrefs} % use AMS ref scheme
\usepackage{anyfontsize}
\usepackage{scalefnt, scalerel}
\usepackage{array,booktabs,tabularx}  
\usepackage[orientation=portrait, size=a0, scale=1.5]{beamerposter} % Use the beamerposter package for laying out the poster
\RequirePackage{lmodern}
\usepackage{lmodern}
\usepackage{exscale} % Necessary for math symbols to be big enough
\usepackage{ytableau}

%% Blocks and Colors
\definecolor{pastelred}{rgb}{1.0, 0.41, 0.38}
\definecolor{tearose}{rgb}{0.96, 0.76, 0.76}
\definecolor{pastelpink}{rgb}{1.0, 0.82, 0.86}
\definecolor{palepink}{rgb}{0.98, 0.85, 0.87}
\definecolor{indianred}{rgb}{0.8, 0.36, 0.36}
\definecolor{coralred}{rgb}{1.0, 0.25, 0.25}

\tikzset{wei/.style=%{red,double=pink,thick,doubledistance=1.5pt}}
{red,double=red,double
distance=0.5pt}}

\tikzset{wei2/.style={red,double=red,double
    distance=0.5pt}}

\newenvironment<>{myexampleblock}[1]{%
  \setbeamercolor{block alerted title}{fg=white,bg=coralred}
  \setbeamercolor{block alerted body}{fg=black,bg=palepink}
  \begin{alertblock}#2{#1}}{\end{alertblock}}
\newenvironment<>{subtleblock}{
  \setbeamercolor{block alerted title}{fg=white,bg=white} 
  \setbeamercolor{block alerted body}{fg=black,bg=dblue!10}
  \begin{alertblock}#1{}}{\end{alertblock}}
\newenvironment<>{subtleexampleblock}{
  \setbeamercolor{block alerted title}{fg=white,bg=white} 
  \setbeamercolor{block alerted body}{fg=black,bg=palepink}
  \begin{alertblock}#1{}}{\end{alertblock}}

\setbeamercolor{block title}{fg=jblue,bg=white}  % Colors of the block
% titles
\setbeamercolor{block body}{fg=black,bg=white} % Colors of the body of blocks
\setbeamercolor{block alerted title}{fg=white,bg=dblue!70} % Colors of the highlighted block titles
\setbeamercolor{block alerted body}{fg=black,bg=dblue!10} % Colors of the body of highlighted blocks
\definecolor{melon}{rgb}{0.99, 0.74, 0.71}
\addtobeamertemplate{block end}{}{\vspace*{1ex}} % White space under blocks
\addtobeamertemplate{block alerted end}{}{\vspace*{1ex}} % White space
% under highlighted (alert) blocks

%% Math shortcuts
\newcommand{\Z}{\mathbb{Z}}
\newcommand{\lowers}{\mathcal{L}}
\newcommand{\g}{\mathfrak{g}}

%% Title matter
  \title[K-theoretic Catalan Functions]{K-theoretic Catalan Functions}
  \author[Seelinger]{George H. Seelinger (joint work with Jonah
    Blasiak and Jennifer Morse)}
  \institute[UVA]{University of Virginia}
  \date{May 4, 2019}
%% Document
  \begin{document}
  \begin{frame}[t] % The whole poster is in one Beamer frame
    \begin{columns}[t] % Align all column contents to the top
      \begin{column}{.3\paperwidth}
        \begin{block}{Background}
          \begin{itemize}
          \item Schur functions \(s_\lambda \in \Lambda\) give an
            explicit understanding of Schubert classes in the
            cohomology ring of the complex Grassmannian,
            \(H^*(Gr(m,n))\). Furthermore, their Pieri rule is
            controlled by semistandard Young tableaux. That is \[
              h_r s_\lambda = \sum_{\mathclap{\mu = \lambda + \text{ a
                  horizontal }r\text{-strip}}} s_{\mu} \iff s_\lambda =
              \sum_\mu K_{\lambda,\mu} m_\mu
            \]
            where \(K_{\lambda\mu}\) is the number of semistandard
            Young tableaux of shape \(\lambda\) and weight \(\mu\).
            \ytableausetup{boxsize=0.3em}
            \[
              h_3 s_{\ydiagram{1,2}} = s_{\ydiagram{1,2,3}}+s_{\ydiagram{1,1,4}}+s_{\ydiagram{2,4}}+s_{\ydiagram{1,5}}
            \]
          \item Grothendieck polynomials, \(G_\lambda\), play a
            similar role as representations for the \(K\)-theory
            classes determined by the structure sheaves of Schubert
            varieties.  Unlike Schur functions, these are not self-dual. 
            So, the dual Grothendieck polynomials are denoted
            \(g_\lambda\). Furthermore, the Pieri rule for \(G_\lambda\) is controlled by
            reverse plane partitions. Thus, we have \[
              G_r G_\lambda = \sum_{\mu }
            \]
          \item \(k\)-Schur functions, \(s_\lambda^{(k)}\) for
            \(k\)-bounded partitions \(\lambda\) also play
            an analogous role for the structure of quantum cohomology
            of the Grassmanian and homology of the affine
            Grassmannians. When \(t=1\), their Pieri rule is controlled by strong
            marked tableaux. Thus, \[
              h_r s_\lambda^{(k)} = \; \sum_{\mathclap{\kappa =
                core(\lambda)+\text{strong marked }r\text{-strip}}}
              \; s_{part(\kappa)}^{(k)} 
            \]\[
              \iff s_\lambda^{(k)} = \sum_\mu
              K_{core(\lambda),\mu}^{(k)} m_\mu
            \]
            where \(K_{core(\lambda),\mu}^{(k)}\) is the number of strong
            marked tableaux of shape \(core(\lambda)\) and weight
            \(\mu\). \[
              h_3 s_{\ydiagram{1,2}}^{(3)} = s_{\ydiagram{1,2,3}}^{(3)}
            \]
          \item The \emph{affine Grothendieck polynomials},
            \(G_\lambda^{(k)}\), and \emph{dual affine Grothendieck polynomials},
            \(g_\lambda^{(k)}\), play the analogous K-theoretic
            role of the \(k\)-Schur functions. Furthermore, the Pieri
            rule for \(g_\lambda^{(k)}\) is controlled by affine
            set-valued tableaux, but there is no known combinatoric
            for the Pieri rule of \(G_\lambda^{(k)}\). By duality,
            this is
            equivalent to finding a dual Pieri rule for \(g_\lambda^{(k)}\)
          \end{itemize}
        \end{block}
        \begin{alertblock}{Problems}
          \begin{enumerate}
          \item What are the branching coefficients for
            \(g_\lambda^{(k)}\), ie what are the coefficients \[
              g_\lambda^{(k)} = \sum_\mu c_{\lambda,\mu} g_\mu?
            \] 
          \item What is the dual (vertical) Pieri rule for
            \(g_\lambda^{(k)}\), 
            ie what are the coefficients \(a_{\lambda,\mu}^{(k)}\) for
            \[
              G_{1^r}^\perp g_\lambda^{(k)} = \sum_\mu
            a_{\lambda,\mu}^{(k)} g_\mu^{(k)}?
            \] 
          \end{enumerate}
        \end{alertblock}
        \end{column}
        \begin{column}{.3\paperwidth}
                  \begin{block}{Raising Operators and Catalan Functions}
          For \(1 \leq i < j \leq \ell\), a raising operator
            \(R_{i,j}\) acts on \(\gamma \in \Z^\ell\) by
            \begin{subtleblock}
            \[
              R_{i,j} \gamma = \gamma+\epsilon_i-\epsilon_j
            \]
            \end{subtleblock}
            for \(\epsilon_i\) the standard basis vector with a \(1\)
            in coordinate \(i\).\\
          \begin{alertblock}{Example: Jacobi-Trudi}
            One can re-state the Jacobi-Trudi identity as \(s_\lambda
            = \prod_{i < j}(1-R_{ij})h_\lambda\) where 
            \(R_{ij}h_\lambda = h_{\lambda+\epsilon_i-\epsilon_j}\). 
          \end{alertblock}

          Given an order
          ideal of positive roots in type A, \(\Psi \subset
          \Delta^+_\ell\) and an index \(\gamma \in \Z^\ell\), the
          \emph{Catalan function} corresponding to this data is
          \begin{subtleblock}
          \[
            H(\Psi;\gamma)(x;t) = \prod_{(i,j) \in \Delta^+_\ell
              \setminus \Psi} (1-tR_{ij}) H_\gamma(x;t)
          \]
          \end{subtleblock}
          for a ``compositional Hall-Littlewood''
          \(H_\gamma(x;t)\).
          \begin{itemize}
          \item \(H(\emptyset;\lambda) = s_\lambda\) for any partition
            \(\lambda\).
          \item \(H(\Delta^{(k)}(\lambda);\lambda) = s_\lambda^{(k)}\)
            for \(\lambda\) a \(k\)-bounded partition of length
            \(\ell\) and \(\Delta^{(k)}(\lambda) := \{(i,j) \in
            \Delta^+_\ell \mid k-\lambda_i+i<j\}\).
          \end{itemize}
          Catalan functions were recently used in~\cite{bmps} to
            prove an incredibly useful property called
            \emph{shift invariance}:
            \begin{subtleblock}
              \[
                e_\ell^\perp s_{\lambda+1^\ell}^{(k+1)} =
                s_\lambda^{(k)}
              \]
            \end{subtleblock}
            Using the known dual vertical pieri rule, one gets the
            branching coefficients:
            \begin{subtleblock}
              \[
                s_\lambda^{(k)} = e_\ell^\perp
                s_{\lambda+1^\ell}^{(k+1)} = \, \sum_{\mathclap{T \in
                  VSMT^{k+1}_{(d)}(\lambda+1^\ell)}} \, t^{spin(T)} s_{inside(T)}^{(k+1)}
              \]
            \end{subtleblock}
          \end{block}
          \begin{block}{K-theory and Symmetric Functions}
            When working with (co)homology representatives, the
            symmetric functions tend to be homogeneous in degree. With
            K-theory representatives, this no longer holds. We define
            a few inhomogeneous symmetric functions. For \(m \in \Z, r
            \in \Z_{\geq 0}\)\[
              Kh_m^{(r)} := \sum_{i=0}^m \binom{r+i-1}{i} h_{m-i}
            \]
            and for \(\gamma \in \Z^\ell\),
            \[
              Kh_\lambda := Kh_{\lambda_1}^{(0)} Kh_{\lambda_2}^{(1)} \cdots Kh_{\lambda_\ell}^{(\ell-1)}
            \]
            
          \end{block}
      \end{column}
      \begin{column}{0.3\paperwidth}
          \begin{block}{K-theoretic Catalan Functions}
            We must introduce another operator called a \emph{lowering
            operator} acting via
          \begin{subtleblock}
            \[
              L_i \gamma = \gamma - \epsilon_i
            \]
          \end{subtleblock}
          Then, for two order ideals of positive roots in type A,
          \(\Psi,\lowers \subset \Delta_\ell^+\) and \(\gamma \in
          \Z^\ell\), we get 
          \begin{subtleblock}
            \[
              K(\Psi;\lowers;\gamma) := \prod_{\mathclap{(i,j) \in \lowers}}
              (1-L_j) \prod_{\mathclap{(i,j) \in \Delta^+_\ell \setminus \Psi}} (1-R_{ij}) Kh_\gamma
            \]
          \end{subtleblock}
          \begin{itemize}
          \item The top degree term of \(K(\Psi;\lowers;\gamma)\) is
            \(H(\Psi;\gamma)\).
          \item \(K(\Psi;\Delta^+_\ell;\gamma) =
            H(\Psi;\gamma)\). 
          \end{itemize}
          \begin{alertblock}{Conjectures}
            \begin{enumerate}
            \item If \(\Psi = \lowers\), then the coefficients in \(K(\Psi;\lowers;\lambda) = \sum
              (-1)^{|\lambda|-|\mu|} b_{\lambda,\mu} g_\mu\) are such 
              that \(b_{\lambda,\mu} \in \Z_{\geq 0}\).
            \item For any \(k\) bounded partition \(\lambda\), \[
                K(\Delta^{(k)}(\lambda);\Delta^{(k+1)}(\lambda);\lambda)
                = g_\lambda^{(k)}
              \]
            \end{enumerate}
          \end{alertblock}
          We will say \(\g_\lambda^{(k)} :=
          K(\Delta^{(k)}(\lambda);\Delta^{(k+1)}(\lambda);\lambda)\). 
        \end{block}
        \begin{block}{Shift Invariance for K-Catalan Functions}
          For \(s \geq 0\), \(\Psi,\lowers \subset \Delta^+_\ell\),
          \(\gamma \in \Z^\ell\), we have \[
            e_s^\perp K(\Psi;\lowers;\lambda) = \sum_{S \subset \{1,
              \ldots, \ell\}, |S| = d} K(\Psi;\lambda-\epsilon_S)
          \]
          for \(\epsilon_S = \sum_{j \in S} \epsilon_j\). It follows
          that
          \begin{alertblock}{Theorem}
            For \(\gamma \in \Z^\ell\), 
            \[
              G_{1^\ell}^\perp K(\Psi;\lowers;\gamma) =
              K(\Psi;\lowers;\gamma-1^\ell)
            \]
            Thus, for a \(k\)-bounded partition \(\lambda\)
            \[ G_{1^\ell}^\perp \g_{\lambda+1^\ell}^{(k)} = \g_\lambda^{(k)}
            \]
          \end{alertblock}
        \end{block}
        \begin{block}{Towards a Vertical Dual Pieri Rule}
          Unlike with \(k\)-Schur functions, there is no known
          combinatoric to guide what the dual Pieri rule
          \(G_{1^r}^\perp g_\lambda^{(k)}\) should be. However, we have made
          progress towards understanding \(G_{1^r}^\perp
          \g_\lambda^{(k)}\) using the combinatorics of K-theoretic
          Catalan functions.
        \end{block}
        \setbeamercolor{block title}{fg=jblue,bg=white} % Change the block title color

        \begin{block}{References}
          \begin{bibdiv}
            \begin{biblist}
              \bib{bmps}{article}{
                author={Blasiak, Jonah}
                author={Morse, Jennifer}
                author={Pun, Anna}
                author={Summers, Daniel}
                title={Catalan Functions and \(k\)-Schur Positivity}
                year={2019}
                journal={Journal of the AMS}
              }
              \bib{morse}{article}{
                author={Morse, Jennifer}
                title={Combinatorics of the K-theory of affine
                  Grassmannians}
                year={2011}
                journal={Advances in Mathematics}
              }
            \end{biblist}
          \end{bibdiv}
        \end{block}
      \end{column}
    \end{columns}
  \end{frame} % End the wrapping frame
  \end{document}
