%%%%%%%%%%%%%%%%%%%%%%%%%%%%%%%%%%%%%%%%%
% Beamer Presentation
% LaTeX Template
% Version 1.0 (10/11/12)
%
% This template has been downloaded from:
% http://www.LaTeXTemplates.com
%
% License:
% CC BY-NC-SA 3.0 (http://creativecommons.org/licenses/by-nc-sa/3.0/)
%
%%%%%%%%%%%%%%%%%%%%%%%%%%%%%%%%%%%%%%%%%

%----------------------------------------------------------------------------------------
%	PACKAGES AND THEMES
%----------------------------------------------------------------------------------------

\documentclass[dvipsnames,handout]{beamer}
\mode<presentation> {

% The Beamer class comes with a number of default slide themes
% which change the colors and layouts of slides. Below this is a list
% of all the themes, uncomment each in turn to see what they look like.

%\usetheme{default}
%\usetheme{AnnArbor}
%\usetheme{Antibes}
%\usetheme{Bergen}
%\usetheme{Berkeley}
%\usetheme{Berlin}
%\usetheme{Boadilla}
%\usetheme{CambridgeUS}
%\usetheme{Copenhagen}
%\usetheme{Darmstadt}
%\usetheme{Dresden}
%\usetheme{Frankfurt}
%\usetheme{Goettingen}
%\usetheme{Hannover}
%\usetheme{Ilmenau}
%\usetheme{JuanLesPins}
%\usetheme{Luebeck}
\usetheme{Madrid}
%\usetheme{Malmoe}
%\usetheme{Marburg}
%\usetheme{Montpellier}
%\usetheme{PaloAlto}
%\usetheme{Pittsburgh}
%\usetheme{Rochester}
%\usetheme{Singapore}
%\usetheme{Szeged}
%\usetheme{Warsaw}

% As well as themes, the Beamer class has a number of color themes
% for any slide theme. Uncomment each of these in turn to see how it
% changes the colors of your current slide theme.

%\usecolortheme{albatross}
%\usecolortheme{beaver}
%\usecolortheme{beetle}
%\usecolortheme{crane}
%\usecolortheme{dolphin}
%\usecolortheme{dove}
%\usecolortheme{fly}
%\usecolortheme{lily}
%\usecolortheme{orchid}
%\usecolortheme{rose}
%\usecolortheme{seagull}
%\usecolortheme{seahorse}
%\usecolortheme{whale}
%\usecolortheme{wolverine}

%\setbeamertemplate{footline} % To remove the footer line in all slides uncomment this line
%\setbeamertemplate{footline}[page number] % To replace the footer line in all slides with a simple slide count uncomment this line

\setbeamertemplate{navigation symbols}{} % To remove the navigation symbols from the bottom of all slides uncomment this line
\setbeamertemplate{footline}{}
}

\usepackage{graphicx} % Allows including images
%\usepackage{booktabs} % Allows the use of \toprule, \midrule and
                      % \bottomrule in tables
\usepackage{tikz}
\usepackage{tikz-cd}
\usepackage{amsmath}
\usepackage[author-year]{amsrefs}
\usepackage{amsthm}
\usepackage{../ReAdTeX/readtex-core}
% \usepackage{../ReAdTeX/readtex-dangerous}
% \usepackage{../ReAdTeX/readtex-abstract-algebra}
\usepackage{ytableau}
%%%%%%%%%%%%%%%%%%%%%%%%%%%%%%%%%%%%%%%%%%%%%%%%%%%%%%%%%%%%%%%%%%% 
%%  MACRO DEFINITIONS:  Co-authors -- PLEASE use these! 
%%%%%%%%%%%%%%%%%%%%%%%%%%%%%%%%%%%%%%%%%%%%%%%%%%%%%%%%%%%%%%%%%%%
\definecolor{coralred}{rgb}{1.0, 0.25, 0.25}
% \definecolor{lightblue}{rgb}{.68,.85,.9} % 
\definecolor{lightblue}{rgb}{.3,.65,1.0} %
\DeclareMathOperator{\Gr}{Gr}
\newcommand{\cupprod}{\smile}
\newcommand{\sym}{\Lambda}
\newcommand{\lowers}{\mathcal{L}}
\newcommand{\mynone}{\ }
\newcommand{\zz}{{\boldsymbol z}}
\newcommand{\xx}{{\boldsymbol x}}
\newcommand{\hbold}{{\boldsymbol h}}
\newcommand{\sbold}{{\boldsymbol s}}
\newcommand{\sigmabold}{\boldsymbol \sigma}
\newcommand{\mubold}{{\boldsymbol \mu }}
\newcommand{\Htild}{\tilde{H}}
\DeclareMathSymbol{\shortminus}{\mathbin}{AMSa}{"39}
\newcommand{\Gcal}{{\mathcal G}}
\newcommand{\nubold}{{\boldsymbol \nu }}
\renewcommand{\Span}{\operatorname{sp}}

\DeclareMathOperator{\conv}{conv}
\DeclareMathOperator{\des}{des}
\DeclareMathOperator{\fin}{fin}
\DeclareMathOperator{\aff}{aff}
\DeclareMathOperator{\ext}{ext}
\DeclareMathOperator{\dinv}{dinv}
\DeclareMathOperator{\inv}{inv}
\DeclareMathOperator{\pol}{pol}
\DeclareMathOperator{\supp}{supp}
\DeclareMathOperator{\Ind}{Ind}
\DeclareMathOperator{\Inv}{Inv}
\DeclareMathOperator{\GL}{GL}
\DeclareMathOperator{\SYT}{SYT}
\DeclareMathOperator{\SSYT}{SSYT}
\DeclareMathOperator{\Ad}{Ad}
\DeclareMathOperator{\Stab}{Stab}
\newcommand{\sgn}{\text{\rm sgn}}
\DeclareMathOperator{\sort}{sort}
\newcommand{\south}{{\mathrm{south}}}
\newcommand{\leg}{{\mathrm{leg}}}
\newcommand{\arm}{{\mathrm{arm}}}
\newcommand{\bx}[2]{{\boldsymbol {#1}[#2]}}
\newcommand{\Ecal}{\mathcal{E}}
\newcommand{\kk}{\Bbbk}
\newcommand{\Scal}{\mathcal{S}}
\newcommand{\tGamma}{{\check{\Gamma}}} %for Laurent reps
\newcommand{\bbb}{{s}}

\newtheorem{thm}{Theorem}
\newtheorem{prop}{Proposition}
\theoremstyle{definition}
\newtheorem{rmk}[thm]{Remark}
\newtheorem*{rmk*}{Remark}
\newtheorem{conjecture}[theorem]{Conjecture}

\newcounter{boxnum}

\newcommand{\qtrootcolor}{blue!45}
\newcommand{\colorgr}[1]{\textcolor{gray}{#1}}
\newcommand{\colorgrgr}[1]{\textcolor{gray!70}{#1}}
\newcommand{\colorb}[1]{\textcolor{blue}{#1}}
\newcommand{\colordb}[1]{\textcolor{DarkBlue}{#1}}
\newcommand{\colorblack}[1]{\textcolor{black}{#1}}
\newcommand{\colorr}[1]{\textcolor{red}{#1}}
\newcommand{\colorg}[1]{\textcolor{ForestGreen}{#1}}
\newcommand{\colorgg}[1]{\textcolor{green}{#1}}
\newcommand{\colorp}[1]{\textcolor{purple}{#1}}


\newcommand{\drawskewdg}[2]{
  \def\ptn{#1}
  \def\offset{#2}
    \setcounter{rownum}{1}
    \foreach \xstart\xend in \ptn {
      \draw[thick,fill=white] (\xstart+\offset,\therownum-1+\offset)
      grid (\xend+\offset,\therownum+\offset) rectangle (\xstart+\offset,\therownum-1+\offset);
      \addtocounter{rownum}{1}
    }
}

\newcounter{rownum}

\newcommand{\drawDgNotThick}[3]{
      \setcounter{rownum}{0}
      \def\b{#1};
      \def\xshift{#2};
      \def\yshift{#3};
      \foreach \c in \b {
        \foreach \xx in \xshift {
           \foreach \yy in \yshift {
              \draw[shift={(\xx,\yy)}] (\therownum,0) grid (\therownum+1, \c);
              \addtocounter{rownum}{1};
           }
        }
      }
    }

\newcommand{\drawDg}[3]{
      \setcounter{rownum}{0}
      \def\b{#1};
      \def\xshift{#2};
      \def\yshift{#3};
      \foreach \c in \b {
        \foreach \xx in \xshift {
           \foreach \yy in \yshift {
              \draw[thick, shift={(\xx,\yy)}] (\therownum,0) grid (\therownum+1, \c);
              \addtocounter{rownum}{1};
           }
        }
      }
    }

%%%%%%%%%%%%%%%%%%%%%%%%%%%%%%%%%%%%%%%%%%%%%%%%%%%%%%%%%%%%%%%%%%%% 


%----------------------------------------------------------------------------------------
%	TITLE PAGE
%----------------------------------------------------------------------------------------

\title[Macdonald Catalanimals]{A raising operator formula for
  Macdonald polynomials via LLT polynomials in the Schiffmann algebra} % The short title appears at the bottom of every slide, the full title is only on the title page

\author[George H. Seelinger]{George H. Seelinger \\ joint work with
  J. Blasiak, M. Haiman, J. Morse, and A. Pun} % Your name
\institute[UMich] % Your institution as it will appear on the bottom of every slide, may be shorthand to save space
{
ghseeli@umich.edu\\ %Your email address
\medskip
Purdue Mathematical Physics Seminar\\ % Your institution for the title page
\medskip
Based on arXiv:2112.07063 and arXiv:2307.06517
}
\date{September 13, 2023} % Date, can be changed to a custom date

\begin{document}
\begin{frame}
\titlepage % Print the title page as the first slide
\end{frame}
\begin{frame}
  \frametitle{Outline}
  \begin{enumerate}
  \item {\bf Background on symmetric functions and Macdonald polynomials}
  \item A new formula for Macdonald polynomials
  \item LLT polynomials in the elliptic Hall algebra
  \end{enumerate}
\end{frame}
\begin{frame}
  \frametitle{Symmetric Polynomials}
  \begin{itemize}
  \item Polynomials \(f \in \Q(q,t)[x_1,\ldots,x_n]\) satisfying \(\sigma.f
    = f\) for all \(\sigma \in S_n\).\pause
    \begin{block}{Generators}
    \[
      e_r =
      \sum_{i_1 < i_2 < \cdots < i_r} x_{i_1} x_{i_2} \cdots x_{i_r}
      \text { or }
      h_r = 
      \sum_{i_1 \leq i_2 \leq \cdots \leq i_r} x_{i_1} x_{i_2} \cdots x_{i_r}
    \]\pause 
  \end{block}
    \item E.g.\,for \(n=3\),
    \begin{align*}
      e_1 = x_1 + x_2 + x_3 = & h_1  \\
      e_2 = x_1 x_2 + x_1 x_3 + x_2 x_3 \quad & h_2 = x_1^2 + x_1 x_2 + x_1
                                          x_3 + x_2^2 +  x_2 x_3 +x_3^2  \\
      e_3 = x_1 x_2 x_3 \quad & h_3 = x_1^3 + x_1^2 x_2 + x_1^2 x_3 + x_1
                          x_2^2 + \cdots
    \end{align*} \pause
    \item Let \(\sym =
      \Q(q,t)[e_1,e_2,\ldots] = \Q(q,t)[h_1,h_2,\ldots]\). Call these
      ``symmetric functions.''\pause
    \item \(\sym\) is a \(\Q(q,t)\)-algebra.
  \end{itemize}
\end{frame}
\begin{frame}{Symmetric functions and Schur functions}
  \begin{itemize}
  \item \(h_d = h_d(X) = \sum_{i_1 \leq \cdots \leq i_d} x_{i_1}
    \cdots x_{i_d}\) with \(h_0 = 1\) and \(h_d = 0\) for \(d < 0\).
  \item For any \(\gamma = (\gamma_1,\ldots,\gamma_n) \in \Z^n \),
    \[
      s_\gamma = s_\gamma(X) = \det(h_{\gamma_i+j-i}(X))_{1 \leq i,j
        \leq n}
    \] \pause
  \end{itemize}
  Then,
  \[
    s_\gamma = 
\begin{cases}
\sgn(\gamma+\rho) s_{\sort(\gamma+\rho) -\rho} & \text{if $\gamma +
                                                 \rho$ has distinct
                                                 nonnegative parts,}\\
0                                                          & \text{otherwise,}
\end{cases}
  \]
\begin{itemize}
\item $\sort(\beta) = $ weakly decreasing sequence obtained by sorting $\beta$,
\vspace{-1mm}
\item $\sgn(\beta) =$ sign of the shortest permutation taking $\beta$ to $\sort(\beta)$.
\end{itemize}
\end{frame}
\begin{frame}
  \frametitle{Schur Polynomials}
  \begin{itemize}\pause
  \item \(\{s_\mu\}_\mu\) forms a basis of symmetric polynomials indexed by integer partitions
    \(\mu = (\mu_1,\ldots,\mu_l) \in \Z^l\) where \(\mu_1 \geq \cdots \geq
    \mu_l \geq 0\).\pause
  \item \(s_\lambda = \sum_\mu \frac{\chi_\lambda(\mu)}{z_\mu} p_\mu\) for
    irreducible \(S_n\)-character \(\chi_\lambda\). \pause
  \end{itemize}
  \begin{block}{Hidden Guide: Schur Positivity}
    ``Naturally occurring'' symmetric functions which are non-negative
    (coefficients in \(\N[q,t]\))
    linear combinations in Schur polynomial basis
     are interesting since they could have representation-theoretic models.
  \end{block}
\end{frame}
\begin{frame}{Harmonic polynomials}
  \begin{block}{Harmonic polynomials}
   \(M =\) polynomials killed by all symmetric differential
   operators.
  \end{block}\pause
  Explicitly, for
   \[
     \Delta = \det \left|
       \begin{matrix}
         x_1^2 & x_1 & 1\\
         x_2^2 & x_2 & 1\\
         x_3^2 & x_3 & 1
       \end{matrix}
     \right| = x_1^2(x_2-x_3) - x_2^2 (x_1 - x_3) + x_3^2(x_1-x_2)
   \]\pause
   \(M\) is the vector space given by\pause
   \begin{align*}
       M  = & \Span\left\{
\left(           \partial_{x_1}^a
           \partial_{x_2}^b  \partial_{x_3}^c
\right)         \Delta \st a,b,c \geq 0\right\} \\
        = & \Span\{\Delta, 2x_1(x_2-x_3)-x_2^2+x_3^2,
            2x_2(x_3-x_1)-x_3^2+x_1^2, \\
       & \phantom{\Span\{\}}x_3-x_1, x_2-x_3,1\}
   \end{align*}
\end{frame}
\begin{frame}{Harmonic polynomials}
\[
\Span\{\Delta, 2x_1(x_2-x_3)-x_2^2+x_3^2,
            2x_2(x_3-x_1)-x_3^2+x_1^2, 
       x_3-x_1, x_2-x_3,1\}
  \]\pause 
  \begin{enumerate}
\item Break \(M\) up into irreducible \(S_n\)-representations. \pause
  \ytableausetup{boxsize=0.75em,aligntableaux=top}
  \[
    \hspace{-2.9em}
    \scalebox{0.95}{\(
      \underbrace{\Span\{\Delta\}}_{\ydiagram{1,1,1}} {\oplus} \underbrace{\Span\{2x_1(x_2{-}x_3){-}x_2^2{+}x_3^2,
        2x_2(x_3{-}x_1){-}x_3^2{+}x_1^2\}}_{\ydiagram{2,1}} {\oplus}
      \underbrace{\Span\{x_3{-}x_1, x_2{-}x_3\}}_{\ydiagram{2,1}} {\oplus} \underbrace{\Span\{1\}}_{\ydiagram{3}}\)}
  \]\pause
  \item How many times does an irreducible \(S_n\)-representation occur? \pause
    Frobenius: \pause
    \ytableausetup{boxsize=0.5em}
    \[
      e_1^3 = (x_1+x_2+x_3)^3 = s_{\ydiagram{1,1,1}} + s_{\ydiagram{2,1}} +
      s_{\ydiagram{2,1}} + s_{\ydiagram{3}}
    \]
  \end{enumerate}
  \pause
  Remark: \(M \isom \C[x_1,x_2,x_3]/(\C[x_1,x_2,x_3]^{S_3}_+)\).
\end{frame}
\begin{frame}{Getting more information}
  \pause
  Break \(M\) up into smallest \(S_n\) fixed subspaces 
  \ytableausetup{boxsize=0.75em,aligntableaux=top}
  \[
    \hspace{-0.75em}
    \scalebox{0.95}{\(
      \underbrace{\Span\{\Delta\}}_{\ydiagram{1,1,1}} {\oplus} \underbrace{\Span\{2x_1(x_2{-}x_3){-}x_2^2{+}x_3^2,
        2x_2(x_3{-}x_1){-}x_3^2{+}x_1^2\}}_{\substack{\ydiagram{2,1}\\\deg
        = 2}} {\oplus}
      \underbrace{\Span\{x_3{-}x_1, x_2{-}x_3\}}_{\substack{\ydiagram{2,1}\\\deg=1}} {\oplus} \underbrace{\Span\{1\}}_{\ydiagram{3}}\)}
  \]
  \pause
  Solution: irreducible \(S_n\)-representation of polynomials of degree \(d\) \(\mapsto q^d
  s_\lambda\) (graded Frobenius)\[
    ?? = q^3 s_{\ydiagram{1,1,1}} + q^2 s_{\ydiagram{2,1}} + q
    s_{\ydiagram{2,1}} + s_{\ydiagram{3}}
  \]\pause
  Answer: Hall-Littlewood polynomial \(H_{\ydiagram{1,1,1}}(X;q)\).
\end{frame}
\begin{frame}{A Problem}
  \begin{itemize}
  \item In 1988, Macdonald introduces a family of symmetric
    polynomials with coefficients in \(\Q(q,t)\) generalizing
    Hall-Littlewood polynomials.\pause
  \item Garsia modifies these polynomials so 
    \[
      \tilde{H}_\lambda(X;q,t) = \sum_\mu \tilde{K}(q,t) s_\mu \text{
        conjecturally satisfies }\tilde{K}(q,t) \in \N[q,t]
    \]\pause
  \item \(\tilde{H}_\lambda(X;1,1) = e_1^{|\lambda|}\).\pause
  \item Does there
    exist a family of \(S_n\)-regular representations whose bigraded
    Frobenius characteristics equal \(\tilde{H}_\lambda(X;q,t)\)?
  \end{itemize}
\end{frame}
\begin{frame}{Garsia-Haiman modules}
  \begin{itemize}
  \item \(\Q[x_1,\ldots,x_n,y_1,\ldots,y_n]\) satisfying
    \(\sigma(x_i) = x_{\sigma(i)}\), \(\sigma(y_j) = y_{\sigma(j)}\).\pause
  \item Garsia-Haiman (1993): \(M_\mu = \) span of partial derivatives of
    \(\Delta_\mu = \det_{(i,j) \in \mu, k \in [n]} (x_k^{i-1} y_k^{j-1})\) \pause \[
      \Delta_{\ydiagram{2,1}} = \det \left|
        \begin{matrix}
          1 & y_1 & x_1 \\
          1 & y_2 & x_2 \\
          1 & y_3 & x_3
        \end{matrix}
      \right| = x_3 y_2 - y_3 x_2 - y_1 x_3 + y_1 x_2 + y_3 x_1 - y_2 x_1
    \]
    \pause
  \[
    \hspace{-3em}
      M_{2,1} = \underbrace{\Span\{\Delta_{2,1}\}}_{\deg = (1,1)}
      \oplus \underbrace{\Span\{y_3-y_1, y_1 - y_2\}}_{\deg = (0,1)}
      \oplus \underbrace{\Span\{x_3-x_1, x_1 - x_2\}}_{\deg = (1,0)}
      \oplus \underbrace{\Span \{1\}}_{\deg = (0,0)}
    \]
    \pause
    Irreducible \(S_n\)-representation with bidegree \((a,b) \mapsto
    q^at^b s_\lambda\) \pause \[
      \tilde{H}_{\ydiagram{2,1}} = qt s_{\ydiagram{1,1,1}} + t
      s_{\ydiagram{2,1}} + q s_{\ydiagram{2,1}} + s_{\ydiagram{3}}
    \]
  \end{itemize}
\end{frame}
\begin{frame}
  \frametitle{Garsia-Haiman modules}
  \begin{theorem}[Haiman, 2001]
    The Garsia-Haiman module \(M_\lambda\) has bigraded Frobenius
    characteristic given by \(\tilde{H}_\lambda(X;q,t)\)
  \end{theorem}\pause
  \begin{itemize}
  \item Proved via connection to the Hilbert Scheme \(Hilb^n(\C^2)\).\pause
  \end{itemize}
  \begin{corollary}
    \(\tilde{H}_\lambda(X;q,t) = \tilde{K}_{\lambda \mu}(q,t) s_\mu\)
    satisfies \(\tilde{K}_{\lambda \mu}(q,t) \in \N[q,t]\).
  \end{corollary}\pause
  \begin{itemize}
  \item No combinatorial description of \(\tilde{K}_{\lambda \mu}(q,t)\).
  \end{itemize}
  \end{frame}
\begin{frame}
  \frametitle{Outline}
  \begin{enumerate}
  \item Background on symmetric functions and Macdonald polynomials
  \item {\bf A new formula for Macdonald polynomials}
  \item LLT polynomials in the elliptic Hall algebra
  \end{enumerate}
\end{frame}
\begin{frame}{Root ideals}
    $R_+ =  \big\{\alpha_{ij} \mid 1 \le i < j \le n\big\}$ denotes the set of positive roots for $GL _{n}$, where  $\alpha_{ij} = \epsilon_i - \epsilon_j$.
            \ytableausetup{mathmode, boxsize=1em, centertableaux}
            \[
              \begin{tikzpicture}[inner sep=0in, outer sep=0in]
                \node (n) {
                \begin{ytableau}
                  \mynone &*(red)\text{\tiny (12)}
                  &*(red)\text{\tiny (13)} &*(red)\text{\tiny (14)}
                  &*(red)
                  \text{\tiny (15)}\\
                  \mynone &\mynone &*(red) \text{\tiny (23)}
                  &*(red)\text{\tiny (24)}
                  &*(red) \text{\tiny (25)}\\
                  \mynone &\mynone &\mynone &*(red) \text{\tiny
                    (34)}
                  &*(red)\text{\tiny (35)} \\
                  \mynone &\mynone &\mynone&\mynone&*(red) \text{\tiny (45)}\\
                  \mynone &\mynone &\mynone&\mynone&\mynone\\
                \end{ytableau}};
              \end{tikzpicture}
          \]
          \pause
          A root ideal \(\Psi \subset R_+\) is an upper order ideal of positive roots.
            \ytableausetup{mathmode, boxsize=1em, centertableaux}
            \[
              \begin{tikzpicture}[inner sep=0in, outer sep=0in]
                \node (n) {
                \begin{ytableau}
                  \mynone &\text{\tiny (12)}
                  &*(red)\text{\tiny (13)} &*(red)\text{\tiny (14)}
                  &*(red)
                  \text{\tiny (15)}\\
                  \mynone &\mynone &*(red) \text{\tiny (23)}
                  &*(red)\text{\tiny (24)}
                  &*(red) \text{\tiny (25)}\\
                  \mynone &\mynone &\mynone & \text{\tiny
                    (34)}
                  &*(red)\text{\tiny (35)} \\
                  \mynone &\mynone &\mynone&\mynone& \text{\tiny (45)}\\
                  \mynone &\mynone &\mynone&\mynone&\mynone\\
                \end{ytableau}};
              \draw[thick,lightblue] (n.north
              west)--([xshift=2.1em]n.north
              west)--++(0,-2.1em)--++(2.1em,0)--++(0,-1.05em)--++(1.05em,0)--++(0,-2.10em);
              \draw[thick,lightgray] (n.north west)--(n.south east);
              \end{tikzpicture}
              \ \ \
              \scalebox{1.3}{$ 
              \text{\textcolor{red}{\scriptsize{
                      $\Psi =$ Roots above Dyck
                      path}
                  }
                }
            $}
          \]
\end{frame}
\begin{frame}{Weyl symmetrization}
 Define the \emph{Weyl symmetrization operator} \(\sigmabold \from
 \Q[z_1^{\pm 1},\ldots, z_n^{\pm 1}] \to \Lambda(X)\) by linearly
 extending
 \[
   \zz^\gamma \mapsto s_\gamma(X)
 \]
 where \(\zz^\gamma = z_1^{\gamma_1} \cdots z_n^{\gamma_n}\).
 \begin{definition}
   A \emph{Catalan function} is a symmetric function indexed by a root ideal \(\Psi \subset
   R_+\) and \(\gamma \in \Z^n\) given by
   \[
     H(\Phi;\gamma) = \sigmabold \left(
       \frac{\zz^\gamma}{\prod_{(i,j) \in \Psi} (1-t z_i/z_j)} \right) 
   \]
 \end{definition}
Denominator factors are understood as geometric series \((1-t
   z_i/z_j)^{-1} = 1 + t z_i/z_j + t^2 (z_i/z_j)^2 + \cdots\)
\end{frame}
\begin{frame}{Catalan functions}
  \begin{definition}
   A \emph{Catalan function} is a symmetric function indexed by a root ideal \(\Psi \subset
   R_+\) and \(\gamma \in \Z^n\) given by
   \[
     H(\Phi;\gamma) = \sigmabold \left(
       \frac{\zz^\gamma}{\prod_{(i,j) \in \Psi} (1-t z_i/z_j)} \right) 
   \]
 \end{definition}
\begin{align*} 
  \Psi = 
\begin{tikzpicture}[scale=.213]
\draw[draw = none, fill = \qtrootcolor] (2,-1) rectangle (3,-2);
\draw[draw = none, fill = \qtrootcolor] (3,-1) rectangle (4,-2);
\draw[thin, black!31] (1,-1) -- (4,-1);
\draw[thin, black!31] (2,-2) -- (4,-2);
\draw[thin, black!31] (3,-3) -- (4,-3);
\draw[thin, black!31] (2,-2) -- (2,-1);
\draw[thin, black!31] (3,-3) -- (3,-1);
\draw[thin, black!31] (4,-1) -- (4,-4);
\draw[thin] (1,-1) -- (2,-1);
\draw[thin] (2,-1) -- (2,-2);
\draw[thin] (2,-2) -- (3,-2);
\draw[thin] (3,-2) -- (3,-3);
\draw[thin] (3,-3) -- (4,-3);
\draw[thin] (4,-3) -- (4,-4);
\end{tikzpicture} & \phantom{=} \gamma = (1,1,1)\\
H(\Psi;\gamma) &= \sigma\left( (1+t\frac{z_1}{z_2} + t^2
  \frac{z_1^2}{z_2^2} + \cdots)(1+t\frac{z_1}{z_3} +
  t^2\frac{z_1^2}{z_3^2} + \cdots)x_1 x_2 x_3 \right)\\
  & =  s_{111} + t( s_{201} +  s_{210}) + t^2(
    s_{3\text{-}10} + s_{300} + s_{31\text{-}1}  ) + \cdots \\
  & = s_{111} + t s_{210}
\end{align*}
\end{frame}
\begin{frame}{A Catalan function for modified Hall-Littlewoods}

  {\small \text{$B_\mu$ = set of roots above block diagonal matrix with block sizes $\mu_{\ell(\mu)}, \dots, \mu_1$}}
\vspace{.4mm}
\vspace{-4.4mm}
\begin{align*}
\raisebox{6.7mm}{${B_{3321} = }$}
\begin{tikzpicture}[scale=.213]
\draw[draw = none, fill = \qtrootcolor] (2,-1) rectangle (3,-2);
 \draw[draw = none, fill = \qtrootcolor] (3,-1) rectangle (4,-2);
 \draw[draw = none, fill = \qtrootcolor] (4,-1) rectangle (5,-2);
 \draw[draw = none, fill = \qtrootcolor] (4,-2) rectangle (5,-3);
 \draw[draw = none, fill = \qtrootcolor] (4,-3) rectangle (5,-4);
 \draw[draw = none, fill = \qtrootcolor] (5,-1) rectangle (6,-2);
 \draw[draw = none, fill = \qtrootcolor] (5,-2) rectangle (6,-3);
 \draw[draw = none, fill = \qtrootcolor] (5,-3) rectangle (6,-4);
 \draw[draw = none, fill = \qtrootcolor] (6,-1) rectangle (7,-2);
 \draw[draw = none, fill = \qtrootcolor] (6,-2) rectangle (7,-3);
 \draw[draw = none, fill = \qtrootcolor] (6,-3) rectangle (7,-4);
 \draw[draw = none, fill = \qtrootcolor] (7,-1) rectangle (8,-2);
 \draw[draw = none, fill = \qtrootcolor] (7,-2) rectangle (8,-3);
 \draw[draw = none, fill = \qtrootcolor] (7,-3) rectangle (8,-4);
 \draw[draw = none, fill = \qtrootcolor] (7,-4) rectangle (8,-5);
 \draw[draw = none, fill = \qtrootcolor] (7,-5) rectangle (8,-6);
 \draw[draw = none, fill = \qtrootcolor] (7,-6) rectangle (8,-7);
 \draw[draw = none, fill = \qtrootcolor] (8,-1) rectangle (9,-2);
 \draw[draw = none, fill = \qtrootcolor] (8,-2) rectangle (9,-3);
 \draw[draw = none, fill = \qtrootcolor] (8,-3) rectangle (9,-4);
 \draw[draw = none, fill = \qtrootcolor] (8,-4) rectangle (9,-5);
 \draw[draw = none, fill = \qtrootcolor] (8,-5) rectangle (9,-6);
 \draw[draw = none, fill = \qtrootcolor] (8,-6) rectangle (9,-7);
 \draw[draw = none, fill = \qtrootcolor] (9,-1) rectangle (10,-2);
 \draw[draw = none, fill = \qtrootcolor] (9,-2) rectangle (10,-3);
 \draw[draw = none, fill = \qtrootcolor] (9,-3) rectangle (10,-4);
 \draw[draw = none, fill = \qtrootcolor] (9,-4) rectangle (10,-5);
 \draw[draw = none, fill = \qtrootcolor] (9,-5) rectangle (10,-6);
 \draw[draw = none, fill = \qtrootcolor] (9,-6) rectangle (10,-7);
 \draw[thin, black!31] (1,-1) -- (10,-1);
\draw[thin, black!31] (2,-1) -- (2,-1);
\draw[thin, black!31] (2,-2) -- (10,-2);
\draw[thin, black!31] (3,-2) -- (3,-1);
\draw[thin, black!31] (3,-3) -- (10,-3);
\draw[thin, black!31] (4,-3) -- (4,-1);
\draw[thin, black!31] (4,-4) -- (10,-4);
\draw[thin, black!31] (5,-4) -- (5,-1);
\draw[thin, black!31] (5,-5) -- (10,-5);
\draw[thin, black!31] (6,-5) -- (6,-1);
\draw[thin, black!31] (6,-6) -- (10,-6);
\draw[thin, black!31] (7,-6) -- (7,-1);
\draw[thin, black!31] (7,-7) -- (10,-7);
\draw[thin, black!31] (8,-7) -- (8,-1);
\draw[thin, black!31] (8,-8) -- (10,-8);
\draw[thin, black!31] (9,-8) -- (9,-1);
\draw[thin, black!31] (9,-9) -- (10,-9);
\draw[thin, black!31] (10,-9) -- (10,-1);
%\draw[draw = none, fill = black!100] (3,-2) rectangle (4,-3);
% \draw[draw = none, fill = black!100] (5,-4) rectangle (6,-5);
% \draw[draw = none, fill = black!100] (6,-4) rectangle (7,-5);
% \draw[draw = none, fill = black!100] (6,-5) rectangle (7,-6);
% \draw[draw = none, fill = black!100] (8,-7) rectangle (9,-8);
% \draw[draw = none, fill = black!100] (9,-7) rectangle (10,-8);
% \draw[draw = none, fill = black!100] (9,-8) rectangle (10,-9);
 \draw[thin] (1,-1) -- (2,-1);
\draw[thin] (2,-1) -- (2,-2);
\draw[thin] (2,-2) -- (3,-2);
\draw[thin] (3,-2) -- (3,-3);
\draw[thin] (3,-3) -- (4,-3);
\draw[thin] (4,-3) -- (4,-4);
\draw[thin] (4,-4) -- (5,-4);
\draw[thin] (5,-4) -- (5,-5);
\draw[thin] (5,-5) -- (6,-5);
\draw[thin] (6,-5) -- (6,-6);
\draw[thin] (6,-6) -- (7,-6);
\draw[thin] (7,-6) -- (7,-7);
\draw[thin] (7,-7) -- (8,-7);
\draw[thin] (8,-7) -- (8,-8);
\draw[thin] (8,-8) -- (9,-8);
\draw[thin] (9,-8) -- (9,-9);
\draw[thin] (9,-9) -- (10,-9);
\draw[thin] (10,-9) -- (10,-10);
\draw[thick, densely dotted, orange] (1,-1) rectangle (2,-2);
\draw[thick, densely dotted, orange] (2,-2) rectangle (4,-4);
\draw[thick, densely dotted, orange] (4,-4) rectangle (7,-7);
\draw[thick, densely dotted, orange] (7,-7) rectangle (10,-10);
\end{tikzpicture}
\end{align*}

\vspace{-2mm}\pause
\begin{theorem}[Weyman, Shimozono-Weyman]
  \begin{align*}
\Htild_\mu(X;0,t) & = \omega\sigmabold 
\Big( \frac{z_1\cdots z_n}{\prod_{\alpha \in
B_\mu}(1 - t \zz^\alpha)} \Big),
\end{align*}
where \(\zz^\alpha = z_i/z_j\).
\end{theorem}
\end{frame}
\begin{frame}{Catalan functions for modified Hall-Littlewoods}
\begin{align*}
\quad \ \ \
\begin{tikzpicture}[scale = .49]
\begin{scope}
\draw[thick] (0,0) grid (1,4);
\draw[thick] (0,0) grid (2,3);
\draw[thick] (0,0) grid (3,2);
\node at (0.5, 3.5) { \footnotesize $b_1$};
\node at (0.5, 2.5) { \footnotesize $b_2$};
\node at (1.5, 2.5) { \footnotesize $b_3$};
\node at (0.5, 1.5) { \footnotesize $b_4$};
\node at (1.5, 1.5) { \footnotesize $b_5$};
\node at (2.5, 1.5) { \footnotesize $b_6$};
\node at (0.5, 0.5) { \footnotesize $b_7$};
\node at (1.5, 0.5) { \footnotesize $b_8$};
\node at (2.5, 0.5) { \footnotesize $b_9$};
\node at (1,-.54) {\small row reading order };
\node at (1,-1.44) {\small $b_1 \prec b_2 \prec \cdots  \prec b_n$};
\end{scope}
\end{tikzpicture}
\quad  \quad \ \
\raisebox{14mm}{\parbox{7cm}{
$R_\mu  :=  \big\{ \alpha_{ij} \in R_+ \mid  \south(b_i) \preceq b_j \big\}$.}}
\end{align*}
\vspace{-1cm}
\begin{align*}
\raisebox{6.7mm}{$\colorb{R_{3321} = }$}
\begin{tikzpicture}[scale=.213]
\draw[draw = none, fill = \qtrootcolor] (2,-1) rectangle (3,-2);
 \draw[draw = none, fill = \qtrootcolor] (3,-1) rectangle (4,-2);
 \draw[draw = none, fill = \qtrootcolor] (4,-1) rectangle (5,-2);
 \draw[draw = none, fill = \qtrootcolor] (4,-2) rectangle (5,-3);
 \draw[draw = none, fill = \qtrootcolor] (5,-1) rectangle (6,-2);
 \draw[draw = none, fill = \qtrootcolor] (5,-2) rectangle (6,-3);
 \draw[draw = none, fill = \qtrootcolor] (5,-3) rectangle (6,-4);
 \draw[draw = none, fill = \qtrootcolor] (6,-1) rectangle (7,-2);
 \draw[draw = none, fill = \qtrootcolor] (6,-2) rectangle (7,-3);
 \draw[draw = none, fill = \qtrootcolor] (6,-3) rectangle (7,-4);
 \draw[draw = none, fill = \qtrootcolor] (7,-1) rectangle (8,-2);
 \draw[draw = none, fill = \qtrootcolor] (7,-2) rectangle (8,-3);
 \draw[draw = none, fill = \qtrootcolor] (7,-3) rectangle (8,-4);
 \draw[draw = none, fill = \qtrootcolor] (7,-4) rectangle (8,-5);
 \draw[draw = none, fill = \qtrootcolor] (8,-1) rectangle (9,-2);
 \draw[draw = none, fill = \qtrootcolor] (8,-2) rectangle (9,-3);
 \draw[draw = none, fill = \qtrootcolor] (8,-3) rectangle (9,-4);
 \draw[draw = none, fill = \qtrootcolor] (8,-4) rectangle (9,-5);
 \draw[draw = none, fill = \qtrootcolor] (8,-5) rectangle (9,-6);
 \draw[draw = none, fill = \qtrootcolor] (9,-1) rectangle (10,-2);
 \draw[draw = none, fill = \qtrootcolor] (9,-2) rectangle (10,-3);
 \draw[draw = none, fill = \qtrootcolor] (9,-3) rectangle (10,-4);
 \draw[draw = none, fill = \qtrootcolor] (9,-4) rectangle (10,-5);
 \draw[draw = none, fill = \qtrootcolor] (9,-5) rectangle (10,-6);
 \draw[draw = none, fill = \qtrootcolor] (9,-6) rectangle (10,-7);
 \draw[thin, black!31] (1,-1) -- (10,-1);
\draw[thin, black!31] (2,-1) -- (2,-1);
\draw[thin, black!31] (2,-2) -- (10,-2);
\draw[thin, black!31] (3,-2) -- (3,-1);
\draw[thin, black!31] (3,-3) -- (10,-3);
\draw[thin, black!31] (4,-3) -- (4,-1);
\draw[thin, black!31] (4,-4) -- (10,-4);
\draw[thin, black!31] (5,-4) -- (5,-1);
\draw[thin, black!31] (5,-5) -- (10,-5);
\draw[thin, black!31] (6,-5) -- (6,-1);
\draw[thin, black!31] (6,-6) -- (10,-6);
\draw[thin, black!31] (7,-6) -- (7,-1);
\draw[thin, black!31] (7,-7) -- (10,-7);
\draw[thin, black!31] (8,-7) -- (8,-1);
\draw[thin, black!31] (8,-8) -- (10,-8);
\draw[thin, black!31] (9,-8) -- (9,-1);
\draw[thin, black!31] (9,-9) -- (10,-9);
\draw[thin, black!31] (10,-9) -- (10,-1);
%\draw[draw = none, fill = black!100] (3,-2) rectangle (4,-3);
% \draw[draw = none, fill = black!100] (5,-4) rectangle (6,-5);
% \draw[draw = none, fill = black!100] (6,-4) rectangle (7,-5);
% \draw[draw = none, fill = black!100] (6,-5) rectangle (7,-6);
% \draw[draw = none, fill = black!100] (8,-7) rectangle (9,-8);
% \draw[draw = none, fill = black!100] (9,-7) rectangle (10,-8);
% \draw[draw = none, fill = black!100] (9,-8) rectangle (10,-9);
 \draw[thin] (1,-1) -- (2,-1);
\draw[thin] (2,-1) -- (2,-2);
\draw[thin] (2,-2) -- (3,-2);
\draw[thin] (3,-2) -- (3,-3);
\draw[thin] (3,-3) -- (4,-3);
\draw[thin] (4,-3) -- (4,-4);
\draw[thin] (4,-4) -- (5,-4);
\draw[thin] (5,-4) -- (5,-5);
\draw[thin] (5,-5) -- (6,-5);
\draw[thin] (6,-5) -- (6,-6);
\draw[thin] (6,-6) -- (7,-6);
\draw[thin] (7,-6) -- (7,-7);
\draw[thin] (7,-7) -- (8,-7);
\draw[thin] (8,-7) -- (8,-8);
\draw[thin] (8,-8) -- (9,-8);
\draw[thin] (9,-8) -- (9,-9);
\draw[thin] (9,-9) -- (10,-9);
\draw[thin] (10,-9) -- (10,-10);
\draw[thick, densely dotted, orange] (1,-1) rectangle (2,-2);
\draw[thick, densely dotted, orange] (2,-2) rectangle (4,-4);
\draw[thick, densely dotted, orange] (4,-4) rectangle (7,-7);
\draw[thick, densely dotted, orange] (7,-7) rectangle (10,-10);
\end{tikzpicture}
\end{align*}
\vspace{-0.5cm}
\pause
 \begin{align*}
\Htild_\mu(X;0,t) & = \omega \sigmabold
\Big( \frac{z_1\cdots z_n}{\prod_{\alpha \in
B_\mu}(1 - t \zz^\alpha)} 
                    \Big),\\
   & =
 \omega \sigmabold
\Big( \frac{z_1\cdots z_n}{\prod_{\alpha \in
\colorb{R_\mu}}(1 - t \zz^\alpha)} 
                    \Big)
\end{align*}
\end{frame}
\begin{frame}{A Catalanimal formula for \(\Htild_\mu(X;q,t)\)}
  \begin{align*}
\quad \ \ \
\begin{tikzpicture}[scale = .49]
\begin{scope}
\draw[thick] (0,0) grid (1,5);
\draw[thick] (0,0) grid (2,3);
\node at (0.5, 4.5) { \footnotesize $b_1$};
\node at (0.5, 3.5) { \footnotesize $b_2$};
\node at (0.5, 2.5) { \footnotesize $b_3$};
\node at (1.5, 2.5) { \footnotesize $b_4$};
\node at (0.5, 1.5) { \footnotesize $b_5$};
\node at (1.5, 1.5) { \footnotesize $b_6$};
\node at (0.5, 0.5) { \footnotesize $b_7$};
\node at (1.5, 0.5) { \footnotesize $b_8$};
\node at (1,-.54) {\small row reading order };
\node at (1,-1.44) {\small $b_1 \prec b_2 \prec \cdots  \prec b_n$};
\end{scope}
\end{tikzpicture}
\quad  \quad \ \
\raisebox{14mm}{\parbox{7cm}{
 $R_\mu     :=     \big\{ \alpha_{ij} \in R_+ \mid  \south(b_i) \preceq  b_j \big\},$ \\[1.4mm]
$\widehat{R}_\mu  :=  \big\{ \alpha_{ij} \in R_+ \mid  \south(b_i) \prec b_j \big\}$.}}
\end{align*}

\vspace{-4.3mm}
\pause
\begin{theorem}[Blasiak-Haiman-Morse-Pun-S.]
The modified Macdonald polynomial $\Htild_\mu = \Htild_{\mu }(X;q,t)$ is given by
\vspace{-1mm}
{\small \begin{align*}
          \Htild_\mu =
          \omega \sigmabold \Bigg( z_1 \cdots z_n
\frac{
\displaystyle\prod_{\alpha_{ij} \in R_\mu \setminus \widehat{R}_\mu }
 \raisebox{-1.4mm}{$\big(1- q^{\arm(b_i)+1} t^{-\leg(b_i)} z_i/z_j \big)$}
\displaystyle\prod_{\alpha \in \widehat{R}_{\mu}}
 \raisebox{-1.4mm}{$\big(1-q  t\zz^\alpha \big)$} } {\prod_{\alpha \in R_+} \big(1-q  \zz^\alpha\big)
\prod_{\alpha \in R_\mu} \big(1-t  \zz^\alpha\big)} 
           \Bigg).
\end{align*}}
%Define the \cemph{Macdonald series}  $\Hbold_\mu(\zz;q,t) = \Hbold_\mu$ by
%\vspace{-2.4mm}
%\begin{multline*}
%
%{\small \text{$\Hbold_{\mu, \gamma} = \sigmabold \Bigg(
%\frac{\zz^\gamma
%\prod_{\alpha_{ij} \in R_\mu \setminus \widehat{R}_\mu }
% \big(1- q^{\arm(b_i)+1} t^{-\leg(b_i)} z_i/z_j \big)
%\prod_{\alpha_{ij} \in \widehat{R}_{\mu}}
% \big(1-q t z_i/z_j \big)}
%{\prod_{\alpha_{ij} \in R_+} \big(1-q  z_i/z_j\big)
%\prod_{\alpha_{ij} \in R_\mu} \big(1-t  z_i/z_j\big)}\Bigg).$}}
%\end{multline*}
%The modified Macdonald polynomial  $\Htild_\mu(X;q,t)$ given by
%\vspace{-2.4mm}
%\begin{align*}
%\Htild_\mu(X;q,t) =  \omega\pol_X  \big( z_1\cdots z_n  \Hbold_\mu \big).
%\end{align*}
%??l vs n use n
\end{theorem}
\end{frame}
\begin{frame}{Example}
  \begin{overlayarea}{\textwidth}{\textheight}
\setbeamercovered{transparent=0}

\vspace{-4.4mm}
\begin{align*}
\raisebox{-1.4mm}{
\begin{tikzpicture}[xscale = 1.5, yscale = 1.26]
\begin{scope}
%
\only<1>{
\draw[thick] (0,0) grid (1,5);
\draw[thick] (0,0) grid (2,3);
\node at (0.5, 4.5) { \large $b_1$};
\node at (0.5, 3.5) { \large $b_2$};
\node at (0.5, 2.5) { \large $b_3$};
\node at (1.5, 2.5) { \large $b_4$};
\node at (0.5, 1.5) { \large $b_5$};
\node at (1.5, 1.5) { \large $b_6$};
\node at (0.5, 0.5) { \large $b_7$};
\node at (1.5, 0.5) { \large $b_8$};
\node at (1,-.34) { partition $\mu = 22211$};
\node at (1.1,-.34) {\small \phantom{numerator factors  $1-q^{\rm arm+1}t^{-{\rm leg}} z_i/z_{j}$}};
}
\only<2->{
\draw[thick] (0,0) grid (1,5);
\draw[thick] (0,0) grid (2,3);
\node at (0.5, 4.5) {\footnotesize $1\shortminus  q \frac{z_1}{z_2}$};
\node at (0.5, 3.5) {\footnotesize $1\shortminus  q t^{\shortminus 1} \frac{z_2}{z_3}$};
\node at (0.5, 2.5) {\footnotesize $1\shortminus  q^2 t^{\shortminus 2} \frac{z_3}{z_5}$};
\node at (1.5, 2.5) {\footnotesize $1\shortminus  q  \frac{z_4}{z_6}$};
\node at (0.5, 1.5) {\footnotesize $1\shortminus q^2 t^{\shortminus 3} \frac{z_5}{z_7}$};
\node at (1.5, 1.5) {\footnotesize $1\shortminus q t^{\shortminus 1} \frac{z_6}{z_8}$};
\node at (0.5, 0.5) {\footnotesize $ $};
\node at (1.5, 0.5) {\footnotesize $ $};
\node at (1.1,-.34) {\small numerator factors  $1-q^{\rm arm+1}t^{-{\rm leg}} z_i/z_{j}$};
}
\end{scope}
\end{tikzpicture}}
\hspace{-.6cm}
\begin{tikzpicture}[scale = .79]
\draw[draw = none, fill = \qtrootcolor] (3,-1) rectangle (4,-2);
 \draw[draw = none, fill = \qtrootcolor] (4,-1) rectangle (5,-2);
 \draw[draw = none, fill = \qtrootcolor] (4,-2) rectangle (5,-3);
 \draw[draw = none, fill = \qtrootcolor] (5,-1) rectangle (6,-2);
 \draw[draw = none, fill = \qtrootcolor] (5,-2) rectangle (6,-3);
 \draw[draw = none, fill = \qtrootcolor] (6,-1) rectangle (7,-2);
 \draw[draw = none, fill = \qtrootcolor] (6,-2) rectangle (7,-3);
 \draw[draw = none, fill = \qtrootcolor] (6,-3) rectangle (7,-4);
 \draw[draw = none, fill = \qtrootcolor] (7,-1) rectangle (8,-2);
 \draw[draw = none, fill = \qtrootcolor] (7,-2) rectangle (8,-3);
 \draw[draw = none, fill = \qtrootcolor] (7,-3) rectangle (8,-4);
 \draw[draw = none, fill = \qtrootcolor] (7,-4) rectangle (8,-5);
 \draw[draw = none, fill = \qtrootcolor] (8,-1) rectangle (9,-2);
 \draw[draw = none, fill = \qtrootcolor] (8,-2) rectangle (9,-3);
 \draw[draw = none, fill = \qtrootcolor] (8,-3) rectangle (9,-4);
 \draw[draw = none, fill = \qtrootcolor] (8,-4) rectangle (9,-5);
 \draw[draw = none, fill = \qtrootcolor] (8,-5) rectangle (9,-6);
 \only<2>{
 \draw[draw = none, fill = gray!100] (2+0.5, -1-0.5) circle (.2);
\draw[draw = none, fill = gray!100] (3+0.5, -2-0.5) circle (.2);
\draw[draw = none, fill = gray!100] (5+0.5, -3-0.5) circle (.2);
\draw[draw = none, fill = gray!100] (6+0.5, -4-0.5) circle (.2);
\draw[draw = none, fill = gray!100] (7+0.5, -5-0.5) circle (.2);
\draw[draw = none, fill = gray!100] (8+0.5, -6-0.5) circle (.2);
}
\only<1->{
 \draw[draw = none, fill = gray!100] (2+0.5, -1-0.5) circle (.2);
\draw[draw = none, fill = gray!100] (3+0.5, -2-0.5) circle (.2);
\draw[draw = none, fill = gray!100] (5+0.5, -3-0.5) circle (.2);
\draw[draw = none, fill = gray!100] (6+0.5, -4-0.5) circle (.2);
\draw[draw = none, fill = gray!100] (7+0.5, -5-0.5) circle (.2);
\draw[draw = none, fill = gray!100] (8+0.5, -6-0.5) circle (.2);
}
 \only<2->{
 \draw[draw = none, fill = gray!36] (2+0.5, -1-0.5) circle (.2);
\draw[draw = none, fill = gray!36] (3+0.5, -2-0.5) circle (.2);
\draw[draw = none, fill = gray!36] (5+0.5, -3-0.5) circle (.2);
\draw[draw = none, fill = gray!36] (6+0.5, -4-0.5) circle (.2);
\draw[draw = none, fill = gray!36] (7+0.5, -5-0.5) circle (.2);
\draw[draw = none, fill = gray!36] (8+0.5, -6-0.5) circle (.2);
}
%\draw[draw = none, fill = red!68] (4,-3) rectangle (5,-4);
% \draw[draw = none, fill = red!68] (5,-4) rectangle (6,-5);
% \draw[draw = none, fill = red!68] (6,-5) rectangle (7,-6);
% \draw[draw = none, fill = red!68] (7,-6) rectangle (8,-7);
% \draw[draw = none, fill = red!68] (8,-7) rectangle (9,-8);
 \draw[thin, black!31] (1,-1) -- (9,-1);
\draw[thin, black!31] (2,-1) -- (2,-1);
\draw[thin, black!31] (2,-2) -- (9,-2);
\draw[thin, black!31] (3,-2) -- (3,-1);
\draw[thin, black!31] (3,-3) -- (9,-3);
\draw[thin, black!31] (4,-3) -- (4,-1);
\draw[thin, black!31] (4,-4) -- (9,-4);
\draw[thin, black!31] (5,-4) -- (5,-1);
\draw[thin, black!31] (5,-5) -- (9,-5);
\draw[thin, black!31] (6,-5) -- (6,-1);
\draw[thin, black!31] (6,-6) -- (9,-6);
\draw[thin, black!31] (7,-6) -- (7,-1);
\draw[thin, black!31] (7,-7) -- (9,-7);
\draw[thin, black!31] (8,-7) -- (8,-1);
\draw[thin, black!31] (8,-8) -- (9,-8);
\draw[thin, black!31] (9,-8) -- (9,-1);
\draw[thin] (1,-1) -- (2,-1);
\draw[thin] (2,-1) -- (2,-2);
\draw[thin] (2,-2) -- (3,-2);
\draw[thin] (3,-2) -- (3,-3);
\draw[thin] (3,-3) -- (4,-3);
\draw[thin] (4,-3) -- (4,-4);
\draw[thin] (4,-4) -- (5,-4);
\draw[thin] (5,-4) -- (5,-5);
\draw[thin] (5,-5) -- (6,-5);
\draw[thin] (6,-5) -- (6,-6);
\draw[thin] (6,-6) -- (7,-6);
\draw[thin] (7,-6) -- (7,-7);
\draw[thin] (7,-7) -- (8,-7);
\draw[thin] (8,-7) -- (8,-8);
\draw[thin] (8,-8) -- (9,-8);
\draw[thin] (9,-8) -- (9,-9);
\node at (3/2,-3/2) { $1 $};
\node at (5/2,-5/2) { $1 $};
\node at (7/2,-7/2) { $1 $};
\node at (9/2,-9/2) { $1 $};
\node at (11/2,-11/2) { $1 $};
\node at (13/2,-13/2) { $1 $};
\node at (15/2,-15/2) { $1 $};
\node (vv) at (17/2,-17/2) { $1 $};
\only<2>{
%\begin{scope}[yshift = -34*1, scale = .32]
%\draw[draw = none, fill = red!68] (2,-1) rectangle (3,-2);
%\node[anchor = west] at (3,-1.5) {\scriptsize  \  $R_q \setminus R_t$};
%\end{scope}
\begin{scope}[yshift = -224,xshift = 40, scale = .72]
\draw[thin, black!0] (2,-1) -- (3,-1);
\draw[thin, black!0] (2,-1) -- (2,-2);
\draw[thin, black!0] (2,-2) -- (3,-2);
\draw[thin, black!0] (3,-2) -- (3,-1);
\draw[draw = none, fill = gray!36] (2+0.5, -1-0.5) circle (.2/.72);
\node[anchor = west] at (3,-1.5) {\small \  $R_\mu \setminus \widehat{R}_\mu$ \  ($t$ factors)};
\end{scope}
\begin{scope}[yshift = -250,xshift = 40, scale = .72]
\draw[draw = none, fill = \qtrootcolor] (2,-1) rectangle (3,-2);
\node[anchor = west] at (3,-1.5) {\small \  $\widehat{R}_{\mu}$ \  ($t$ and  $qt$ factors)};
\end{scope}
\node[anchor = east] at (10/2,-11/2-.3) { $\Htild_{22211}$ \ \ \  };
}
\only<1>{
\begin{scope}[yshift = -224,xshift = 40, scale = .72]
\draw[thin, black!0] (2,-1) -- (3,-1);
\draw[thin, black!0] (2,-1) -- (2,-2);
\draw[thin, black!0] (2,-2) -- (3,-2);
\draw[thin, black!0] (3,-2) -- (3,-1);
\draw[draw = none, fill = gray!100] (2+0.5, -1-0.5) circle (.2/.72);
\node[anchor = west] at (3,-1.5) {\small \  $R_\mu \setminus \widehat{R}_\mu$ \  ($t$ factors)};
\end{scope}
\begin{scope}[yshift = -250,xshift = 40, scale = .72]
\draw[draw = none, fill = \qtrootcolor] (2,-1) rectangle (3,-2);
\node[anchor = west] at (3,-1.5) {\small \  $\widehat{R}_{\mu}$ \  ($t$ and  $qt$ factors)};
\end{scope}
}
\only<2->{
\node at (2.5,-1.5) {\footnotesize $q $};
\node at (3.5,-2.5) {\footnotesize $q t^{\shortminus 1} $};
\node at (5.5,-3.5) {\footnotesize $q^2 t^{\shortminus 2} $};
\node at (6.5,-4.5) {\footnotesize $q $};
\node at (7.5,-5.5) {\footnotesize $q^2 t^{\shortminus 3} $};
\node at (8.5,-6.5) {\footnotesize $q t^{\shortminus 1} $};
}
\end{tikzpicture}
\end{align*}
\end{overlayarea}
\end{frame}
\begin{frame}{\(q=t=1\) specialization}
  \begin{align*}
          & \omega \sigmabold \Bigg( z_1 \cdots z_n
\frac{
\displaystyle\prod_{\alpha_{ij} \in R_\mu \setminus \widehat{R}_\mu }
 \raisebox{-1.4mm}{$\big(1- q^{\arm(b_i)+1} t^{-\leg(b_i)} z_i/z_j \big)$}
\displaystyle\prod_{\alpha \in \widehat{R}_{\mu}}
 \raisebox{-1.4mm}{$\big(1-q  t\zz^\alpha \big)$} } {\prod_{\alpha \in R_+} \big(1-q  \zz^\alpha\big)
\prod_{\alpha \in R_\mu} \big(1-t  \zz^\alpha\big)} 
           \Bigg) \\
    \overset{q=t=1}{\to} & \omega \sigmabold \left( z_1\cdots z_n
    \frac{\prod_{\alpha \in R_\mu \setminus \widehat{R}_\mu}
    (1-\zz^\alpha) \prod_{\alpha \in \widehat{R}_\mu}(1-\zz^\alpha)}{\prod_{\alpha \in R_+}
    (1-\zz^\alpha) \prod_{\alpha \in R_\mu} (1-\zz^\alpha)}\right) \\
    = & \omega \sigmabold\left(\frac{z_1 \cdots z_n}{\prod_{\alpha \in R_+}
      (1-\zz^\alpha)}\right) \\
    = & \omega h_1^n \\
    = & e_1^n
  \end{align*}
\end{frame}
\begin{frame}{\(q=0\) specialization}
  \begin{align*}
          & \omega \sigmabold \Bigg( z_1 \cdots z_n
\frac{
\displaystyle\prod_{\alpha_{ij} \in R_\mu \setminus \widehat{R}_\mu }
 \raisebox{-1.4mm}{$\big(1- q^{\arm(b_i)+1} t^{-\leg(b_i)} z_i/z_j \big)$}
\displaystyle\prod_{\alpha \in \widehat{R}_{\mu}}
 \raisebox{-1.4mm}{$\big(1-q  t\zz^\alpha \big)$} } {\prod_{\alpha \in R_+} \big(1-q  \zz^\alpha\big)
\prod_{\alpha \in R_\mu} \big(1-t  \zz^\alpha\big)} 
           \Bigg) \\
    \overset{q=0}{\to} & \omega \sigmabold \left( \frac{z_1\cdots z_n}{\prod_{\alpha \in
                         R_\mu} (1-t \zz^\alpha)}  \right) \\
     = & \Htild_\mu(X;0,t)
  \end{align*}
\end{frame}
\begin{frame}{Proof of formula for \(\Htild_\mu\)}
  \begin{definition}
    $\nabla$ is the linear operator on symmetric functions satisfying
    $\nabla \Htild _{\mu } = t^{n(\mu)}q^{n(\mu ^{*})} \Htild _{\mu
    }$, where $n(\mu ) = \sum _{i} (i-1)\mu_{i}.$
  \end{definition}\pause
\begin{itemize}
%???add date?
\item Start with the Haglund-Haiman-Loehr formula for  $\Htild_\mu$ as a sum of LLT polynomials  $\Gcal_\nubold(X;q)$.\pause
\item Apply  $\omega\nabla$ to both sides.\pause
\item Use Catalanimal formula for  $\omega\nabla \Gcal_\nubold(X;q)$
  and collect terms.
\end{itemize}
\end{frame}
\begin{frame}{LLT Polynomials}
  \vspace{-1.4mm}
{\small
Let  $\nubold= (\nu_{(1)}, \dots, \nu_{(k)})$ be a tuple of skew shapes.
%\vspace{1mm}

\begin{itemize}
\onslide<2->{\item The \emph{content} of a box in row  $y$,  column  $x$ is  $x-y$.}
\onslide<3->{\item \emph{Reading order}: label boxes $b_1, \dots, b_n$ by scanning each diagonal from southwest to northeast, in order of increasing content.}
\onslide<4->{\item A pair  $(a,b) \in \nubold$ is \emph{attacking} if  $a$ precedes  $b$ in reading order and
\begin{itemize}
\item  ${\rm content}(b) = {\rm content}(a)$,  or
\item  ${\rm content}(b) = {\rm content}(a) + 1$ and $a \in \nu_{(i)}, b \in \nu_{(j)}$ with  $i > j$.
\end{itemize}}
\end{itemize}}

\vspace{-3mm}
\begin{equation*}
\begin{tikzpicture}[scale = .34]
\begin{scope}
\node[anchor=east] at (-0.1, 1) {$\nubold = \bigg( $};
%
\draw[thin, black!44]  (0,0) grid (1,1);
%\node at (0.5, 1.5) {\small $b_1$};
%\node at (1.5, 1.5) {\small $b_2$};
%\node at (1.5, 0.5) {\small $b_4$};
%\node at (2.5, 0.5) {\small $b_7$};
\node at (3.34, 0.02) { , };
\draw[very thick] (0,1) grid (2,2);
\draw[very thick] (1,0) grid (3,1);
\end{scope}
\begin{scope}[xshift = 117]
%
\draw[thin, black!44] (0,0) grid (1,2);
%\node at (1.5, 1.5) {\small $b_3$};
%\node at (2.5, 1.5) {\small $b_6$};
%\node at (1.5, 0.5) {\small $b_5$};
%\node at (2.5, 0.5) {\small $b_8$};
\draw[very thick] (1,0) grid (3,2);
\node at (3.58, 1) { $ \bigg)$};
\end{scope}
\end{tikzpicture}
\quad \quad \quad
\raisebox{-.94cm}{
\begin{tikzpicture}[scale = .51]
\begin{scope}
\only<5>{
\draw[draw = none, fill = red!47] (1,1) rectangle (2,2);
\draw[draw = none, fill = red!47] (4,4) rectangle (5,5);
}
\only<6>{
\draw[draw = none, fill = red!47] (4,4) rectangle (5,5);
\draw[draw = none, fill = red!47] (1,0) rectangle (2,1);
}
\only<7>{
\draw[draw = none, fill = red!47] (1,0) rectangle (2,1);
\draw[draw = none, fill = red!47] (4,3) rectangle (5,4);
}
\only<8>{
\draw[draw = none, fill = red!47] (1,0) rectangle (2,1);
\draw[draw = none, fill = red!47] (5,4) rectangle (6,5);
}
\only<9>{
\draw[draw = none, fill = red!47] (4,3) rectangle (5,4);
\draw[draw = none, fill = red!47] (2,0) rectangle (3,1);
}
\only<10>{
\draw[draw = none, fill = red!47] (5,4) rectangle (6,5);
\draw[draw = none, fill = red!47] (2,0) rectangle (3,1);
}
\only<11>{
\draw[draw = none, fill = red!47] (2,0) rectangle (3,1);
\draw[draw = none, fill = red!47] (5,3) rectangle (6,4);
}
%
\draw[help lines] (0,0) grid (6,5);
\draw[thick] (0,1) grid (2,2);
\draw[thick] (1,0) grid (3,1);
%
%
\draw[thick] (4,3) grid (6,5);
\only<3->{
\node at (0.5, 1.5) {\small $b_1$};
\node at (1.5, 1.5) {\small $b_2$};
\node at (1.5, 0.5) {\small $b_4$};
\node at (2.5, 0.5) {\small $b_7$};
%
\node at (3.0+1.5, 3.0+1.5) {\small $b_3$};
\node at (3.0+2.5, 3.0+1.5) {\small $b_6$};
\node at (3.0+1.5, 3.0+0.5) {\small $b_5$};
\node at (3.0+2.5, 3.0+0.5) {\small $b_8$};}
%
\only<2>{
\foreach \x in {0,...,5}
    \foreach \y in {0,...,4}
        {\pgfmathtruncatemacro{\myc}{\x - \y}
        \node at (\x+0.5,\y+0.5) {\footnotesize \myc };}
}
\end{scope}
\end{tikzpicture}}
\end{equation*}
{\small
\onslide<4->{
Attacking pairs:  $
{\color<5>{red}(b_2,b_3)},
{\color<6>{red}(b_3, b_4)},
{\color<7>{red}(b_4, b_5)},
{\color<8>{red}(b_4, b_6)},
{\color<9>{red}(b_5, b_7)},
{\color<10>{red}(b_6, b_7)},
{\color<11>{red}(b_7, b_8)} $}
}
\end{frame}
\begin{frame}{LLT Polynomials}
  \vspace{-2mm}
\small
\begin{itemize}
\item A \emph{semistandard tableau} on $\nubold  $ is a map
$T\colon \nubold  \rightarrow \Z _{+}$ which restricts to a
semistandard tableau on each $\nu_{(i)}$.
\item An \emph{attacking inversion} in $T$ is
an attacking pair $(a,b)$ such that~$T(a)>T(b)$.
\end{itemize}

The \emph{LLT polynomial} indexed by a tuple of skew shapes $\nubold
$ is
\begin{equation*}
\Gcal_{\nubold  }(\xx;q) = \sum _{T\in \SSYT (\nubold
)}q^{\inv (T)}\xx ^{T},
\end{equation*}
\vspace{-1mm}
where $\inv (T)$ is the number of attacking inversions in $T$ and
$\xx ^{T} = \prod _{a\in \nubold  } x_{T(a)}$.

\vspace{-2mm}
\begin{align*}
&
\begin{tikzpicture}[scale = .46]
\node[anchor = east] at (-1.4, 2.5) {$T \ \ = $};
\begin{scope}
\node[anchor = west] at (7,3) {\small \phantom{\colorb{non-inversion}}};
\only<2>{
\draw[draw = none, fill = blue!47] (1,1) rectangle (2,2);
\draw[draw = none, fill = blue!47] (4,4) rectangle (5,5);
\node[anchor = west] at (7,3) {\small \colorb{non-inversion}};
}
\only<3>{
\draw[draw = none, fill = red!47] (4,4) rectangle (5,5);
\draw[draw = none, fill = red!47] (1,0) rectangle (2,1);
\node[anchor = west] at (7,3) {\small \colorr{inversion}};
}
\only<4>{
\draw[draw = none, fill = red!47] (1,0) rectangle (2,1);
\draw[draw = none, fill = red!47] (4,3) rectangle (5,4);
\node[anchor = west] at (7,3) {\small \colorr{inversion}};
}
\only<5>{
\draw[draw = none, fill = blue!47] (1,0) rectangle (2,1);
\draw[draw = none, fill = blue!47] (5,4) rectangle (6,5);
\node[anchor = west] at (7,3) {\small \colorb{non-inversion}};
}
\only<6>{
\draw[draw = none, fill = blue!47] (4,3) rectangle (5,4);
\draw[draw = none, fill = blue!47] (2,0) rectangle (3,1);
\node[anchor = west] at (7,3) {\small \colorb{non-inversion}};
}
\only<7>{
\draw[draw = none, fill = red!47] (5,4) rectangle (6,5);
\draw[draw = none, fill = red!47] (2,0) rectangle (3,1);
\node[anchor = west] at (7,3) {\small \colorr{inversion}};
}
\only<8>{
\draw[draw = none, fill = red!47] (2,0) rectangle (3,1);
\draw[draw = none, fill = red!47] (5,3) rectangle (6,4);
\node[anchor = west] at (7,3) {\small \colorr{inversion}};
}
%
\draw[help lines] (0,0) grid (6,5);
\draw[thick] (0,1) grid (2,2);
\draw[thick] (1,0) grid (3,1);
\node at (0.5, 1.5) {\small $2$};
\node at (1.5, 1.5) {\small $4$};
\node at (1.5, 0.5) {\small $3$};
\node at (2.5, 0.5) {\small $5$};
\draw[thick] (4,3) grid (6,5);
\node at (3.0+1.5, 3.0+1.5) {\small $5$};
\node at (3.0+2.5, 3.0+1.5) {\small $6$};
\node at (3.0+1.5, 3.0+0.5) {\small $1$};
\node at (3.0+2.5, 3.0+0.5) {\small $1$};
\end{scope}
\end{tikzpicture}\\
&
\onslide<8>{\inv(T) = 4,  \quad \xx^T = x_1^2x_2x_3x_4x_5^2x_6}
\end{align*}
\end{frame}
\begin{frame}{Catalanimals}
The \emph{Catalanimal} indexed by $R_q, R_t, R_{qt} \subseteq R_+$ and $\lambda \in \Z^n$ is
\vspace{-.4mm}
\begin{align*}
H(R_q,R_t,R_{qt},\lambda)
%\defeq
 = \sigmabold
 \bigg(\frac{\zz ^\lambda \prod_{\alpha \in
R_{qt}} \big(1-q  t \zz ^\alpha \big)} {\prod_{\alpha \in R_q} \big(1-q \zz ^\alpha\big)
\prod_{\alpha \in R_t} \big(1-t \zz ^\alpha\big)} 
  \bigg).
 %
% \sum_{w \in \SS_n} w\bigg(\frac{\zz ^\lambda \prod_{\alpha \in
%R_{qt}} \big(1-q\, t\, \zz ^\alpha \big)} {\prod_{\alpha \in R_+}
%(1-\zz ^{-\alpha}) \prod_{\alpha \in R_q} \big(1-q \, \zz ^\alpha\big)
%\prod_{\alpha \in R_t} \big(1-t \, \zz ^\alpha\big)}\bigg).
 \\[-8.4mm]
\end{align*}
\pause
With  $n=3$,
%For  $n = 3$, $R_q = R_t = R_+ $, $R_{qt} = \{(1,3)\}$, and  $\lambda = 111$,
\vspace{-3mm}
\begin{align*}
& \displaystyle H(R_+,R_+,\{\alpha_{13}\}, (111)) =
\sigmabold \Big( \frac{\zz^{111} (1 - q t z_{1} /z_3)}
{\prod_{1\le i < j \le 3}(1 - q z_{i}/ z_{j})(1 - t z_{i}/ z_{j})} \Big)  \\
&  = s_{111} + (q+t+q^2+qt+t^2)s_{21}+ (qt + q^3+ q^2t + qt^2+t^3)s_{3} \\
& = \omega \nabla e_3.
\end{align*}
%say "a symmetric rational function"
\vspace{4mm}
\end{frame}
\begin{frame}{LLT Catalanimals}
  For a tuple of skew shapes $\nubold$, the \emph{LLT Catalanimal} $H_{\nubold} = H(R_q,R_t, R_{qt}, \lambda)$
is determined by
\vspace{1mm}
\begin{itemize}
\item $R_+ \supseteq R_q \supseteq R_t \supseteq R_{qt}$,
\item $R_+ \setminus R_q = $ pairs of boxes in the same diagonal,
\item $R_q \setminus R_t \ = $ the attacking pairs,
\item  $R_t \setminus R_{qt}  = $ pairs going between adjacent diagonals,
\item $\lambda$: fill each diagonal $D$ of $\nubold$ with
$1+\chi({\small \text{$D$ contains a row start}}) - \chi({\small \text{$D$ contains a row end}})$. \break
Listing this filling in reading order gives  $\lambda$.
%Then  $\lambda$ is the list of entries in the filling in reading order.
\end{itemize}
\end{frame}
% \begin{frame}{LLT Catalanimals}
%   \vspace{-2mm}

% {\small
% \begin{itemize}
% \setlength{\itemsep}{-0.4mm}
% \item[] \raisebox{-1mm}{
%  \begin{tikzpicture}[scale = .4]
% \draw[draw = none, fill = black!100] (2,-1) rectangle (3,-2);
% \end{tikzpicture}} \
% $R_+ \setminus R_q = $ pairs of boxes in the same diagonal,
% \item[] \raisebox{-1mm}{
%  \begin{tikzpicture}[scale = .4]
% \draw[draw = none, fill = red!68] (2,-1) rectangle (3,-2);
% \end{tikzpicture}} \
% $R_q \setminus R_t \ = $ the attacking pairs,
% \item[] \raisebox{-1mm}{
% \begin{tikzpicture}[scale = .4]
% \draw[thin, black!0] (2,-1) -- (3,-1);
% \draw[thin, black!0] (2,-1) -- (2,-2);
% \draw[thin, black!0] (2,-2) -- (3,-2);
% \draw[thin, black!0] (3,-2) -- (3,-1);
% \draw[draw = none, fill = gray!100] (2+0.5, -1-0.5) circle (.2);
% \end{tikzpicture}} \
%  $R_t \setminus R_{qt}  = $ pairs going between adjacent diagonals,
% \item[]
% \raisebox{-1mm}{
%  \begin{tikzpicture}[scale = .4]
% \draw[draw = none, fill = \qtrootcolor] (2,-1) rectangle (3,-2);
% \end{tikzpicture}} \
%  $R_{qt} = $ all other pairs,
% \item[] $\lambda$: fill each diagonal $D$ of $\nubold$ with
% $1+\chi({\small \text{$D$ contains a row start}}) - \chi({\small \text{$D$ contains a row end}})$.
% \end{itemize}
% }
% \vspace{-5mm}

% \begin{align*}
% \raisebox{5mm}{\begin{tikzpicture}[scale = .52]
% \begin{scope}
% \draw[thick] (0,0) grid (4,1);
% \draw[thick] (0,0) grid (3,3);
% %
% \only<1>{
% \node at (0.5, 2.5) {\small $b_1$};
% \node at (0.5, 1.5) {\small $b_2$};
% \node at (1.5, 2.5) {\small $b_3$};
% \node at (0.5, 0.5) {\small $b_4$};
% \node at (1.5, 1.5) {\small $b_5$};
% \node at (2.5, 2.5) {\small $b_6$};
% \node at (1.5, 0.5) {\small $b_7$};
% \node at (2.5, 1.5) {\small $b_8$};
% \node at (2.5, 0.5) {\small $b_9$};
% \node at (3.5, 0.5) {\small $b_{10}$};
% \node[anchor=west] at (-0.7,-1.34) {\small $\nubold = ((433))$};
% }
% \only<2->{
% \node at (0.5, 2.5) {\small $2$};
% \node at (0.5, 1.5) {\small $2$};
% \node at (1.5, 2.5) {\small $2$};
% \node at (0.5, 0.5) {\small $1$};
% \node at (1.5, 1.5) {\small $1$};
% \node at (2.5, 2.5) {\small $1$};
% \node at (1.5, 0.5) {\small $0$};
% \node at (2.5, 1.5) {\small $0$};
% \node at (2.5, 0.5) {\small $1$};
% \node at (3.5, 0.5) {\small $0$};
% \node[anchor=west] at (-0.7,-1.34) {\small $\lambda$, as a filling of $\nubold$};
% }
% \node[anchor=west] at (-0.7,-1.34) {\small \phantom{$\lambda$, as a filling of $\nubold$}};
% \end{scope}
% \end{tikzpicture}}
% \qquad
% \begin{tikzpicture}[scale = .43]
% \draw[draw = none, fill = \qtrootcolor] (4,-1) rectangle (5,-2);
%  \draw[draw = none, fill = \qtrootcolor] (5,-1) rectangle (6,-2);
%  \draw[draw = none, fill = \qtrootcolor] (6,-1) rectangle (7,-2);
%  \draw[draw = none, fill = \qtrootcolor] (7,-1) rectangle (8,-2);
%  \draw[draw = none, fill = \qtrootcolor] (7,-2) rectangle (8,-3);
%  \draw[draw = none, fill = \qtrootcolor] (7,-3) rectangle (8,-4);
%  \draw[draw = none, fill = \qtrootcolor] (8,-1) rectangle (9,-2);
%  \draw[draw = none, fill = \qtrootcolor] (8,-2) rectangle (9,-3);
%  \draw[draw = none, fill = \qtrootcolor] (8,-3) rectangle (9,-4);
%  \draw[draw = none, fill = \qtrootcolor] (9,-1) rectangle (10,-2);
%  \draw[draw = none, fill = \qtrootcolor] (9,-2) rectangle (10,-3);
%  \draw[draw = none, fill = \qtrootcolor] (9,-3) rectangle (10,-4);
%  \draw[draw = none, fill = \qtrootcolor] (9,-4) rectangle (10,-5);
%  \draw[draw = none, fill = \qtrootcolor] (9,-5) rectangle (10,-6);
%  \draw[draw = none, fill = \qtrootcolor] (9,-6) rectangle (10,-7);
%  \draw[draw = none, fill = \qtrootcolor] (10,-1) rectangle (11,-2);
%  \draw[draw = none, fill = \qtrootcolor] (10,-2) rectangle (11,-3);
%  \draw[draw = none, fill = \qtrootcolor] (10,-3) rectangle (11,-4);
%  \draw[draw = none, fill = \qtrootcolor] (10,-4) rectangle (11,-5);
%  \draw[draw = none, fill = \qtrootcolor] (10,-5) rectangle (11,-6);
%  \draw[draw = none, fill = \qtrootcolor] (10,-6) rectangle (11,-7);
%  \draw[draw = none, fill = \qtrootcolor] (10,-7) rectangle (11,-8);
%  \draw[draw = none, fill = \qtrootcolor] (10,-8) rectangle (11,-9);
%  \draw[draw = none, fill = gray!100] (2+0.5, -1-0.5) circle (.2);
% \draw[draw = none, fill = gray!100] (3+0.5, -1-0.5) circle (.2);
% \draw[draw = none, fill = gray!100] (4+0.5, -2-0.5) circle (.2);
% \draw[draw = none, fill = gray!100] (5+0.5, -2-0.5) circle (.2);
% \draw[draw = none, fill = gray!100] (6+0.5, -2-0.5) circle (.2);
% \draw[draw = none, fill = gray!100] (4+0.5, -3-0.5) circle (.2);
% \draw[draw = none, fill = gray!100] (5+0.5, -3-0.5) circle (.2);
% \draw[draw = none, fill = gray!100] (6+0.5, -3-0.5) circle (.2);
% \draw[draw = none, fill = gray!100] (7+0.5, -4-0.5) circle (.2);
% \draw[draw = none, fill = gray!100] (8+0.5, -4-0.5) circle (.2);
% \draw[draw = none, fill = gray!100] (7+0.5, -5-0.5) circle (.2);
% \draw[draw = none, fill = gray!100] (8+0.5, -5-0.5) circle (.2);
% \draw[draw = none, fill = gray!100] (7+0.5, -6-0.5) circle (.2);
% \draw[draw = none, fill = gray!100] (8+0.5, -6-0.5) circle (.2);
% \draw[draw = none, fill = gray!100] (9+0.5, -7-0.5) circle (.2);
% \draw[draw = none, fill = gray!100] (9+0.5, -8-0.5) circle (.2);
% \draw[draw = none, fill = gray!100] (10+0.5, -9-0.5) circle (.2);
% \draw[thin, black!31] (1,-1) -- (11,-1);
% \draw[thin, black!31] (2,-1) -- (2,-1);
% \draw[thin, black!31] (2,-2) -- (11,-2);
% \draw[thin, black!31] (3,-2) -- (3,-1);
% \draw[thin, black!31] (3,-3) -- (11,-3);
% \draw[thin, black!31] (4,-3) -- (4,-1);
% \draw[thin, black!31] (4,-4) -- (11,-4);
% \draw[thin, black!31] (5,-4) -- (5,-1);
% \draw[thin, black!31] (5,-5) -- (11,-5);
% \draw[thin, black!31] (6,-5) -- (6,-1);
% \draw[thin, black!31] (6,-6) -- (11,-6);
% \draw[thin, black!31] (7,-6) -- (7,-1);
% \draw[thin, black!31] (7,-7) -- (11,-7);
% \draw[thin, black!31] (8,-7) -- (8,-1);
% \draw[thin, black!31] (8,-8) -- (11,-8);
% \draw[thin, black!31] (9,-8) -- (9,-1);
% \draw[thin, black!31] (9,-9) -- (11,-9);
% \draw[thin, black!31] (10,-9) -- (10,-1);
% \draw[thin, black!31] (10,-10) -- (11,-10);
% \draw[thin, black!31] (11,-10) -- (11,-1);
% \draw[draw = none, fill = black!100] (3,-2) rectangle (4,-3);
%  \draw[draw = none, fill = black!100] (5,-4) rectangle (6,-5);
%  \draw[draw = none, fill = black!100] (6,-4) rectangle (7,-5);
%  \draw[draw = none, fill = black!100] (6,-5) rectangle (7,-6);
%  \draw[draw = none, fill = black!100] (8,-7) rectangle (9,-8);
%  \draw[thin] (1,-1) -- (2,-1);
% \draw[thin] (2,-1) -- (2,-2);
% \draw[thin] (2,-2) -- (3,-2);
% \draw[thin] (3,-2) -- (3,-3);
% \draw[thin] (3,-3) -- (4,-3);
% \draw[thin] (4,-3) -- (4,-4);
% \draw[thin] (4,-4) -- (5,-4);
% \draw[thin] (5,-4) -- (5,-5);
% \draw[thin] (5,-5) -- (6,-5);
% \draw[thin] (6,-5) -- (6,-6);
% \draw[thin] (6,-6) -- (7,-6);
% \draw[thin] (7,-6) -- (7,-7);
% \draw[thin] (7,-7) -- (8,-7);
% \draw[thin] (8,-7) -- (8,-8);
% \draw[thin] (8,-8) -- (9,-8);
% \draw[thin] (9,-8) -- (9,-9);
% \draw[thin] (9,-9) -- (10,-9);
% \draw[thin] (10,-9) -- (10,-10);
% \draw[thin] (10,-10) -- (11,-10);
% \draw[thin] (11,-10) -- (11,-11);
% \node at (3/2,-3/2) {\small $2 $};
% \node at (5/2,-5/2) {\small $2 $};
% \node at (7/2,-7/2) {\small $2 $};
% \node at (9/2,-9/2) {\small $1 $};
% \node at (11/2,-11/2) {\small $1 $};
% \node at (13/2,-13/2) {\small $1 $};
% \node at (15/2,-15/2) {\small $0 $};
% \node at (17/2,-17/2) {\small $0 $};
% \node at (19/2,-19/2) {\small $1 $};
% \node at (21/2,-21/2) {\small $0 $};
% \end{tikzpicture}
% \end{align*}
% \end{frame}
\begin{frame}{LLT Catalanimals}
  \vspace{-5mm}

{\small
\begin{itemize}
\setlength{\itemsep}{-0.4mm}
\item[] \raisebox{-1mm}{
 \begin{tikzpicture}[scale = .4]
\draw[draw = none, fill = black!100] (2,-1) rectangle (3,-2);
\end{tikzpicture}} \
$R_+ \setminus R_q = $ pairs of boxes in the same diagonal,
\item[] \raisebox{-1mm}{
 \begin{tikzpicture}[scale = .4]
\draw[draw = none, fill = red!68] (2,-1) rectangle (3,-2);
\end{tikzpicture}} \
$R_q \setminus R_t \ = $ the attacking pairs,
\item[] \raisebox{-1mm}{
\begin{tikzpicture}[scale = .4]
\draw[thin, black!0] (2,-1) -- (3,-1);
\draw[thin, black!0] (2,-1) -- (2,-2);
\draw[thin, black!0] (2,-2) -- (3,-2);
\draw[thin, black!0] (3,-2) -- (3,-1);
\draw[draw = none, fill = gray!100] (2+0.5, -1-0.5) circle (.2);
\end{tikzpicture}} \
 $R_t \setminus R_{qt}  = $ pairs going between adjacent diagonals,
\item[]
\raisebox{-1mm}{
 \begin{tikzpicture}[scale = .4]
\draw[draw = none, fill = \qtrootcolor] (2,-1) rectangle (3,-2);
\end{tikzpicture}} \
 $R_{qt} = $ all other pairs,
\item[] $\lambda$: fill each diagonal $D$ of $\nubold$ with
$1+\chi({\small \text{$D$ contains a row start}}) - \chi({\small \text{$D$ contains a row end}})$.
\end{itemize}
}
\vspace{-5mm}

\begin{align*}
\hspace{-1cm}
\raisebox{.3cm}{
\begin{tikzpicture}[scale = .47]
\begin{scope}
\draw[help lines] (0,0) grid (6,5);
\draw[thick] (0,1) grid (2,2);
\draw[thick] (1,0) grid (3,1);
%
\draw[thick] (4,3) grid (6,5);
\only<1>{
\node at (0.5, 1.5) {\small $b_1$};
\node at (1.5, 1.5) {\small $b_2$};
\node at (1.5, 0.5) {\small $b_4$};
\node at (2.5, 0.5) {\small $b_7$};
%
\node at (3.0+1.5, 3.0+1.5) {\small $b_3$};
\node at (3.0+2.5, 3.0+1.5) {\small $b_6$};
\node at (3.0+1.5, 3.0+0.5) {\small $b_5$};
\node at (3.0+2.5, 3.0+0.5) {\small $b_8$};
\node at (3,-1.34) {\small $\nubold$};
}
\only<2->{
\node at (0.5, 1.5) {\small $\colorb{2}$};
\node at (1.5, 1.5) {\small $\colorb{0}$};
\node at (1.5, 0.5) {\small $\colorb{2}$};
\node at (2.5, 0.5) {\small $\colorb{0}$};
%
\node at (3.0+1.5, 3.0+1.5) {\small $\colorg{2}$};
\node at (3.0+2.5, 3.0+1.5) {\small $\colorg{1}$};
\node at (3.0+1.5, 3.0+0.5) {\small $\colorg{1}$};
\node at (3.0+2.5, 3.0+0.5) {\small $\colorg{0}$};
\node at (3,-1.34) {\small $\lambda$, as a filling of $\nubold$};
}
\node at (3,-1.34) {\small \phantom{$\lambda$, as a filling of $\nubold$}};
\end{scope}
\end{tikzpicture}
}
\ \ \qquad
\begin{tikzpicture}[scale = .47]
\draw[draw = none, fill = \qtrootcolor] (3,-1) rectangle (4,-2);
 \draw[draw = none, fill = \qtrootcolor] (4,-1) rectangle (5,-2);
 \draw[draw = none, fill = \qtrootcolor] (5,-1) rectangle (6,-2);
 \draw[draw = none, fill = \qtrootcolor] (5,-2) rectangle (6,-3);
 \draw[draw = none, fill = \qtrootcolor] (6,-1) rectangle (7,-2);
 \draw[draw = none, fill = \qtrootcolor] (6,-2) rectangle (7,-3);
 \draw[draw = none, fill = \qtrootcolor] (7,-1) rectangle (8,-2);
 \draw[draw = none, fill = \qtrootcolor] (7,-2) rectangle (8,-3);
 \draw[draw = none, fill = \qtrootcolor] (7,-3) rectangle (8,-4);
 \draw[draw = none, fill = \qtrootcolor] (8,-1) rectangle (9,-2);
 \draw[draw = none, fill = \qtrootcolor] (8,-2) rectangle (9,-3);
 \draw[draw = none, fill = \qtrootcolor] (8,-3) rectangle (9,-4);
 \draw[draw = none, fill = \qtrootcolor] (8,-4) rectangle (9,-5);
 \draw[draw = none, fill = gray!100] (2+0.5, -1-0.5) circle (.2);
\draw[draw = none, fill = gray!100] (4+0.5, -2-0.5) circle (.2);
\draw[draw = none, fill = gray!100] (5+0.5, -3-0.5) circle (.2);
\draw[draw = none, fill = gray!100] (6+0.5, -3-0.5) circle (.2);
\draw[draw = none, fill = gray!100] (7+0.5, -4-0.5) circle (.2);
\draw[draw = none, fill = gray!100] (8+0.5, -5-0.5) circle (.2);
\draw[draw = none, fill = gray!100] (8+0.5, -6-0.5) circle (.2);
\draw[draw = none, fill = red!68] (3,-2) rectangle (4,-3);
 \draw[draw = none, fill = red!68] (4,-3) rectangle (5,-4);
 \draw[draw = none, fill = red!68] (5,-4) rectangle (6,-5);
 \draw[draw = none, fill = red!68] (6,-4) rectangle (7,-5);
 \draw[draw = none, fill = red!68] (7,-5) rectangle (8,-6);
 \draw[draw = none, fill = red!68] (7,-6) rectangle (8,-7);
 \draw[draw = none, fill = red!68] (8,-7) rectangle (9,-8);
 \draw[thin, black!31] (1,-1) -- (9,-1);
\draw[thin, black!31] (2,-1) -- (2,-1);
\draw[thin, black!31] (2,-2) -- (9,-2);
\draw[thin, black!31] (3,-2) -- (3,-1);
\draw[thin, black!31] (3,-3) -- (9,-3);
\draw[thin, black!31] (4,-3) -- (4,-1);
\draw[thin, black!31] (4,-4) -- (9,-4);
\draw[thin, black!31] (5,-4) -- (5,-1);
\draw[thin, black!31] (5,-5) -- (9,-5);
\draw[thin, black!31] (6,-5) -- (6,-1);
\draw[thin, black!31] (6,-6) -- (9,-6);
\draw[thin, black!31] (7,-6) -- (7,-1);
\draw[thin, black!31] (7,-7) -- (9,-7);
\draw[thin, black!31] (8,-7) -- (8,-1);
\draw[thin, black!31] (8,-8) -- (9,-8);
\draw[thin, black!31] (9,-8) -- (9,-1);
\draw[draw = none, fill = black!100] (6,-5) rectangle (7,-6);
 \draw[thin] (1,-1) -- (2,-1);
\draw[thin] (2,-1) -- (2,-2);
\draw[thin] (2,-2) -- (3,-2);
\draw[thin] (3,-2) -- (3,-3);
\draw[thin] (3,-3) -- (4,-3);
\draw[thin] (4,-3) -- (4,-4);
\draw[thin] (4,-4) -- (5,-4);
\draw[thin] (5,-4) -- (5,-5);
\draw[thin] (5,-5) -- (6,-5);
\draw[thin] (6,-5) -- (6,-6);
\draw[thin] (6,-6) -- (7,-6);
\draw[thin] (7,-6) -- (7,-7);
\draw[thin] (7,-7) -- (8,-7);
\draw[thin] (8,-7) -- (8,-8);
\draw[thin] (8,-8) -- (9,-8);
\draw[thin] (9,-8) -- (9,-9);
\node at (3/2,-3/2) {\small $\colorb{2} $};
\node at (5/2,-5/2) {\small $\colorb{0} $};
\node at (7/2,-7/2) {\small $\colorg{2} $};
\node at (9/2,-9/2) {\small $\colorb{2} $};
\node at (11/2,-11/2) {\small $\colorg{1} $};
\node at (13/2,-13/2) {\small $\colorg{1} $};
\node at (15/2,-15/2) {\small $\colorb{0} $};
\node at (17/2,-17/2) {\small $\colorg{0} $};
\end{tikzpicture}
\end{align*}
\end{frame}
\begin{frame}{LLT Catalanimals}
\begin{theorem}[Blasiak-Haiman-Morse-Pun-S.]
Let  $\nubold$ be a tuple of skew shapes 
and let $H_{\nubold} = H(R_q,R_t,R_{qt}, \lambda)$ be the associated LLT Catalanimal. Then
 %we have the following raising operator style formula for  $\nabla$ applied to the associated LLT polynomial:
 \vspace{-2mm}
\begin{align*}
\nabla \Gcal_{\nubold}(X;q)
& = c_\nubold\,\omega \pol_X (H_{\nubold})
\\
& = c_\nubold  \, \omega \pol_X \sigmabold
 \bigg(\frac{\zz ^\lambda \prod_{\alpha \in
R_{qt}} \big(1-q t\, \zz ^\alpha \big)} {\prod_{\alpha \in R_q} \big(1-q \, \zz ^\alpha\big)
\prod_{\alpha \in R_t} \big(1-t \, \zz ^\alpha\big)}\bigg)
\end{align*}
for some $c_\nubold \in \pm q^\Z t^{\Z}$.
%$A = A(\nubold)$ is the number of attacking pairs in  $\nubold$,%and $p = p(\nubold)$, $\gamma = \gamma(\nubold)$ are the magic number and diagonal lengths of $\nubold$.
\end{theorem}
\end{frame}
\begin{frame}{Haglund-Haiman-Loehr formula}
  \begin{theorem}[Haglund-Haiman-Loehr, 2005]
    \[
      \Htild_\mu(X;q,t) = \sum_D \left( \prod_{u \in D} q^{-\arm(u)}
        t^{\leg(u)+1}\right) \Gcal_{\nubold(\mu,D)}(X;q)\,,
    \]
    where
    \begin{itemize}
    \item the sum runs over all subsets
      \(D \subset \{(i,j) \in \mu \mid j > 1\}\), and
    \item \(\nubold(\mu,D) = (\nu^{(1)}, \ldots, \nu^{(k)})\) where
      \(k=\mu_1\) is the number of columns of \(\mu\), and
      \(\nu^{(i)}\) is a ribbon of size \(\mu_i^*\), i.e., box contents
      \(\{-1,-2,\ldots,-\mu_i^*\}\), and descent set
      \(Des(\nu^{(i)}) = \{-j \mid (i,j) \in D\}\).
    \end{itemize}
  \end{theorem}
\end{frame}
\begin{frame}{Haglund-Haiman-Loehr formula example}
  \begin{equation*}
        \begin{tikzpicture}[scale=0.65]
      \drawDg{3,2}{0}{0} \setcounter{boxnum}{1};
      \foreach \x\y in
      {1/3,1/2,2/2,1/1,2/1} { \node at (\x-0.5,\y-0.5) {\footnotesize
          $b_{\theboxnum}$}; \addtocounter{boxnum}{1}; };
      \node at (1,-.58) {$\mu$};
    \end{tikzpicture}
  \end{equation*}
\begin{equation*}
  % \vspace{1mm}
  %       \begin{tikzpicture}[scale=0.6]
  %     \drawDg{3,2}{0}{0} \foreach \x\y in {1/3,2/2} { \node at
  %       (\x-0.5,\y-0.5) {$0$}; } \node at (0.5,1.5) {$1$};
  %     \node at (1,-0.62) {\footnotesize \phantom{l}arms\phantom{g}};
  %   \end{tikzpicture}
    % \quad
    % \begin{tikzpicture}[scale=0.6]
    %   \drawDg{3,2}{0}{0} \foreach \x\y in {1/3,2/2} { \node at
    %     (\x-0.5,\y-0.5) {$0$}; } \node at (0.5,1.5) {$1$};
    %   \node at (1,-0.62) {\footnotesize legs};
    % \end{tikzpicture}
    \begin{tabular}{rccl}
    \begin{tikzpicture}[scale=0.3]
      \draw [dashed,gray] (-0.5,1.5)--(3.5,5.5); \draw [dashed,gray]
      (-0.5,0.5)--(3.5,4.5); \draw [dashed,gray]
      (-0.5,-0.5)--(3.5,3.5); \drawskewdg{0/1,0/1,0/1}{0}
      \drawskewdg{0/1,0/1}{2}
      \setcounter{boxnum}{1};
      \foreach \x\y in {1/3,1/2,3/4,1/1,3/3} {
        \node at (\x-0.5,\y-0.5) {\tiny \theboxnum};
        \addtocounter{boxnum}{1};
      }
      \node at (1.5,-1.5) {\footnotesize\(D = \{b_1,b_2,b_3\}\)};
      \node at (3.5,0.5) {\(q^{\shortminus 1}t^4\)};
    \end{tikzpicture}&
    \begin{tikzpicture}[scale=0.3]
      \draw [dashed,gray] (-0.5,1.5)--(2.5,4.5); \draw [dashed,gray]
      (-0.5,0.5)--(3.5,4.5); \draw [dashed,gray]
      (-0.5,-0.5)--(3.5,3.5); \drawskewdg{0/1,-1/1}{0}
      \drawskewdg{0/1,0/1}{2}
      \setcounter{boxnum}{1};
      \foreach \x\y in {0/2,1/2,3/4,1/1,3/3} {
        \node at (\x-0.5,\y-0.5) {\tiny \theboxnum};
        \addtocounter{boxnum}{1};
      }
      \node at (1.5,-1.5) {\footnotesize \(D = \{b_2,b_3\}\)};
      \node at (3.5,0.5) {\(q^{\shortminus 1}t^3\)};
    \end{tikzpicture}&
    \begin{tikzpicture}[scale=0.3]
      \draw [dashed,gray] (-0.5,1.5)--(3.5,5.5); \draw [dashed,gray]
      (-0.5,0.5)--(3.5,4.5); \draw [dashed,gray]
      (-0.5,-0.5)--(4.5,4.5); \drawskewdg{0/1,0/1,0/1}{0}
      \drawskewdg{-1/1}{3}
      \setcounter{boxnum}{1};
      \foreach \x\y in {1/3,1/2,3/4,1/1,4/4} {
        \node at (\x-0.5,\y-0.5) {\tiny \theboxnum};
        \addtocounter{boxnum}{1};
      }
      \node at (1.5,-1.5) {\footnotesize \(D = \{b_1,b_2\}\)};
      \node at (3.5,0.5) {\(q^{\shortminus 1}t^3\)};
    \end{tikzpicture}&
    \begin{tikzpicture}[scale=0.3]
      \draw [dashed,gray] (-1.5,0.5)--(3.5,5.5); \draw [dashed,gray]
      (-1.5,-0.5)--(3.5,4.5); \draw [dashed,gray]
      (-0.5,-0.5)--(3.5,3.5); \drawskewdg{-1/1,-1/0}{0}
      \drawskewdg{0/1,0/1}{2}
      \setcounter{boxnum}{1};
      \foreach \x\y in {0/2,0/1,3/4,1/1,3/3} {
        \node at (\x-0.5,\y-0.5) {\tiny \theboxnum};
        \addtocounter{boxnum}{1};
      }
      \node at (1.5,-1.5) {\footnotesize \(D = \{b_1,b_3\}\)};
      \node at (3.5,0.5) {\(t^2\)};
    \end{tikzpicture}\\
    \begin{tikzpicture}[scale=0.3]
      \draw [dashed,gray] (-0.5,1.5)--(2.5,4.5); \draw [dashed,gray]
      (-0.5,0.5)--(3.5,4.5); \draw [dashed,gray]
      (-0.5,-0.5)--(4.5,4.5); \drawskewdg{0/1,-1/1}{0}
      \drawskewdg{-1/1}{3}
      \setcounter{boxnum}{1};
      \foreach \x\y in {0/2,1/2,3/4,1/1,4/4} {
        \node at (\x-0.5,\y-0.5) {\tiny \theboxnum};
        \addtocounter{boxnum}{1};
      }
      \node at (1.5,-1.5) {\footnotesize \(D = \{b_2\}\)};
      \node at (3.5,0.5) {\(q^{\shortminus 1}t^2\)};
    \end{tikzpicture}&
    \begin{tikzpicture}[scale=0.3]
      \draw [dashed,gray] (-1.5,0.5)--(2.5,4.5); \draw [dashed,gray]
      (-1.5,-0.5)--(3.5,4.5); \draw [dashed,gray]
      (-0.5,-0.5)--(3.5,3.5); \drawskewdg{-2/1}{0}
      \drawskewdg{0/1,0/1}{2}
      \setcounter{boxnum}{1};
      \foreach \x\y in {-1/1,0/1,3/4,1/1,3/3} {
        \node at (\x-0.5,\y-0.5) {\tiny \theboxnum};
        \addtocounter{boxnum}{1};
      }
      \node at (1.5,-1.5) {\footnotesize \(D = \{b_3\}\)};
      \node at (3.5,0.5) {\(t\)};
    \end{tikzpicture}&
    \begin{tikzpicture}[scale=0.3]
      \draw [dashed,gray] (-1.5,0.5)--(2.5,4.5); \draw [dashed,gray]
      (-1.5,-0.5)--(2.5,3.5); \draw [dashed,gray] (-0.5,-0.5)--(3.5,3.5);
      \drawskewdg{-1/1,-1/0}{0} \drawskewdg{-1/1}{2}
      \setcounter{boxnum}{1};
      \foreach \x\y in {0/2,0/1,2/3,1/1,3/3} {
        \node at (\x-0.5,\y-0.5) {\tiny \theboxnum};
        \addtocounter{boxnum}{1};
      }
      \node at (1.5,-1.5) {\footnotesize \(D = \{b_1\}\)};
      \node at (3.5,0.5) {\(t\)};
    \end{tikzpicture}&
    \begin{tikzpicture}[scale=0.3]
      \draw [dashed,gray] (-1.5,0.5)--(1.5,3.5); \draw [dashed,gray]
      (-1.5,-0.5)--(2.5,3.5); \draw [dashed,gray] (-0.5,-0.5)--(3.5,3.5);
      \drawskewdg{-2/1}{0} \drawskewdg{-1/1}{2}
      \setcounter{boxnum}{1};
      \foreach \x\y in {-1/1,0/1,2/3,1/1,3/3} {
        \node at (\x-0.5,\y-0.5) {\tiny \theboxnum};
        \addtocounter{boxnum}{1};
      }
      \node at (1.5,-1.5) {\footnotesize \(D = \varnothing\)};
      \node at (3.5,0.5) {\(1\)};
    \end{tikzpicture}
    \end{tabular}
\end{equation*}
\end{frame}
\begin{frame}{Putting it all together}
  \begin{itemize}
  \item Take HHL formula \(\Htild_\mu = \sum_D a_{\mu, D}
    \Gcal_{\nubold(\mu, D)}\) and apply \(\omega \nabla\).\pause
  \item By construction, all the LLT Catalanimals 
    \(H_{\nubold(\mu,D)}\) appearing on the LHS will have the same
    root ideal data (\(R_q, R_t, R_{qt}\)). \pause
  \item Collect terms to get \(\prod_{\alpha_{ij} \in R_\mu \setminus
      \widehat{R}_\mu)}(1-q^{\arm(b_i)+1} t^{-\leg(b_i)} z_i/z_j)\)
      factor. 
  \end{itemize}
{\small \begin{align*}
          \Htild_\mu =
          \omega \sigmabold \Bigg( z_1 \cdots z_n
\frac{
\displaystyle\colorb{\prod_{\alpha_{ij} \in R_\mu \setminus \widehat{R}_\mu }
 \raisebox{-1.4mm}{$\big(1- q^{\arm(b_i)+1} t^{-\leg(b_i)} z_i/z_j \big)$}}
\displaystyle\prod_{\alpha \in \widehat{R}_{\mu}}
 \raisebox{-1.4mm}{$\big(1-q  t\zz^\alpha \big)$} } {\prod_{\alpha \in R_+} \big(1-q  \zz^\alpha\big)
\prod_{\alpha \in R_\mu} \big(1-t  \zz^\alpha\big)} 
           \Bigg).
\end{align*}}
\end{frame}
\begin{frame}
  \frametitle{Outline}
  \begin{enumerate}
  \item Background on symmetric functions and Macdonald polynomials
  \item A new formula for Macdonald polynomials
  \item {\bf LLT polynomials in the elliptic Hall algebra}
  \end{enumerate}
\end{frame}
\begin{frame}{Elliptic Hall Algebra}
  Burban and Schiffmann studied a subalgebra  $\mathcal{E}$
of the Hall algebra of coherent sheaves on an elliptic curve over
$\mathbb{F}_p$. \\

\ \\

The \emph{elliptic Hall algebra} $\Ecal$ is generated by subalgebras $\Lambda(X^{a,b})$
isomorphic to the ring of symmetric functions  $\Lambda$ over $\kk = \Q(q,t)$,
one for each coprime pair $(a,b) \in \Z^2$, along with an additional
central subalgebra.
\end{frame}
\begin{frame}{Shuffle algebra}
  Define a linear map
\vspace{-2.3mm}
\begin{align*}
\sigma_\Gamma \colon \bigoplus_n \kk(z_1,\ldots,z_n) \to \bigoplus_n \kk(z_1,\ldots,z_n)^{\SS_n} \\[-10mm]
\end{align*}
whose graded components  $\sigma_\Gamma^n$ are given by
\vspace{-1.4mm}
\begin{align*}
 \sigma_\Gamma^{n} \colon \kk (z_{1},\ldots,z_{n}) \to \kk (z_{1},\ldots,z_{n})^{S_n} \\[-11mm]
\end{align*}
\begin{align*}
 \sigma_\Gamma^{n}(f) = \sum _{w\in \SS_{n}}
w\bigl(f(z_{1},\ldots,z_{n})\prod _{1 \le i<j \le n}\Gamma
  (z_{i},z_{j})\bigr), \\
  \text{where }\Gamma(z_i,z_j) = \frac{1-qt z_i/z_j}{(1-z_j/z_i)(1-q z_i/z_j)(1-tz_i/z_j)}
\end{align*}
The \emph{shuffle algebra}  $\Scal_\Gamma$ is the  image of
$\bigoplus _{n} \kk[z_{1}^{\pm 1},\ldots,z_n^{\pm 1}]$ under the map $\sigma_\Gamma$,
equipped with a variant of the concatenation product.
\end{frame}
\begin{frame}{Shuffle to elliptic Hall isomorphism}
  \begin{itemize}
\item The \emph{right half-plane subalgebra} $\Ecal^{+}\subset \Ecal$
is generated by $\Lambda (X^{a,b})$ for $a>0$.
\item  $\Scal_\Gamma = \sigma_\Gamma\left( \bigoplus_n \kk[z_{1}^{\pm 1},\ldots,z_n^{\pm 1}] \right)$  ($\Gamma$-symmetrized Laurent polynomials).
\end{itemize}

\vspace{3mm}

\begin{theorem}[Schiffmann-Vasserot]
There is an algebra isomorphism
$\psi \colon \Scal_\Gamma \rightarrow \Ecal ^{+}$.
% given on the generators  $z_1 \in \kk[z_1^{\pm 1}] \subset \Scal_\Gamma$
%by $\psi (z_1^{a}) = p_{1}[-MX^{1,a}]$ for  $a \in \ZZ$.
\end{theorem}
\end{frame}



% \begin{frame}{Catalanimals in the shuffle algebra}
%   For  $\lambda \in \Z^n$,
% \vspace{-3mm}
% %"decode Gamma, and you get back Catalanimals"
% \begin{align*}
% \sigma_\Gamma^n(\zz^\lambda) &=
% \sum_{w \in S_n} w\bigg(\frac{\zz ^\lambda \prod_{\alpha \in R_+}
% \big(1-q t \zz ^\alpha \big)}
% {\prod_{\alpha \in R_+}\big(\big(1-\zz ^{-\alpha}\big) \big(1-q  \zz ^\alpha\big)
%  \big(1-t \zz ^\alpha\big)\big)}
% \bigg)
%  \\
% &= H(R_+,R_+,R_+,\lambda)  \in \Scal_\Gamma.
% \end{align*}
% Technicality: we have redefined \(\sigmabold(\zz^\gamma) = \sum_{w \in
% S_n}\left( \frac{\zz^\gamma}{\prod_{\alpha\in R_+}(1-\zz^{-\alpha})}
% \right) = \chi_\gamma\),
% the irreducible \(\GL_n\) character. 
% \end{frame}
% \begin{frame}{Catalanimals in the Shuffle algebra}
%  $\sigma_\Gamma^n(f)$ can lie in  $\Scal_\Gamma$ even when  $f$ is not a Laurent polynomial.
% \begin{theorem}[Negut]
% The following family of Catalanimals lie in the shuffle algebra:
% \vspace{-3mm}
% \begin{align*}
% & \sigma_\Gamma^n\Big(\frac{\zz^\lambda}{\prod_{i=1}^{n-1} (1-qt z_i/z_{i+1}) }\Big)
% = H(R_+,R_+,R_+',\lambda) \in \Scal_\Gamma,
% \end{align*}
% where  \ \, $R_+' = \{ \alpha_{ij} \in R_+ \mid i+1 < j \}.$
% \end{theorem}
% \end{frame}
% \begin{frame}{The wheel condition}
%   \begin{itemize}
%   \item A symmetric Laurent polynomial \(g(\zz)\) satisfies the
%     \emph{wheel condition} if it vanishes whenever any three of the
%     variables \(z_i,z_j,z_k\) are in the ratio
%     \((z_i:z_j:z_k) = (1:q:qt) =
%     (1:t:qt)\).
%     \item Let \(\Scal_{\tGamma} \isom \Scal_\Gamma\) for
%     \(\tGamma(z_i,z_j) = (1-z_i/z_j)(1-q z_j/z_i)(1-t z_j/z_i)(1 - qt
%     z_i/z_j)\).
%   \end{itemize}
% \begin{theorem}[Negut]
% A symmetric Laurent polynomial $g(z_{1},\ldots,z_{n})$ belongs to
% $\Scal _{\tGamma }$ if and only if it satisfies the wheel
% condition and vanishes whenever $z_{i} = z_{j}$ for $i\not =j$.
% \end{theorem}
% \end{frame}
\begin{frame}{Elliptic Hall algebra action}
  Schiffmann-Vasserot and Feigin-Tsymbauliuk constructed an action of $\Ecal$ on  $\Lambda$,
where $f(X^{0,1})$ acts by multiplication by $f(X)$.
\begin{prop}
  Conjugation by  $\nabla $ provides a symmetry of the
action of $\Ecal $ on $\Lambda$,
\begin{align*}
\\[-8mm]
\nabla\, f(X^{a,b})\, \nabla^{-1} = f(X^{a+b,b}).
\end{align*}
\end{prop}
\begin{corollary}
  $f(X^{1,1})\cdot 1 = \nabla f(X^{0,1}) \nabla^{-1} \cdot 1 = \nabla f$.
\end{corollary}
\begin{theorem}[Blasiak-Haiman-Morse-Pun-S.]
Let $H$ be a Catalanimal such that  $\psi (H) = f(X^{1,1})$.
Then
\begin{align*}
\nabla f = \omega \pol_X (H).
\end{align*}
where \(\pol_X\) sends \(\chi_\lambda \mapsto s_\lambda\) if
the last part of \(\lambda \geq 0\), otherwise \(\chi_\lambda \mapsto 0\).
\end{theorem}
\end{frame}
\begin{frame}
  \frametitle{Proof of \(\nabla \Gcal_\nubold\) formula}
  \begin{enumerate}
  \item LLT Catalanimals are tame.
  \item LLT Catalanimals lie in \(\psi^{-1}(\Lambda(X^{1,1}))\).
  \item Describe coproduct \(\Delta\) on \(\Ecal\) explicitly on tame
    Catalanimals and show \(\Delta H_\nubold\) matches \(\Delta
    \Gcal_\nubold\).
  \item Conclude \(\psi(H_\nubold) = c_\nubold^{-1}
    \Gcal_\nubold(X^{1,1}) \in \Ecal\).
  \item Apply previous theorem to conclude \(\nabla \Gcal_\nubold =
    c_\nubold \omega \pol_X(H_\nubold)\)
  \end{enumerate}
\end{frame}
\begin{frame}{A positivity conjecture}
  \begin{center}
What can this formula tell us that other formulas for Macdonald polynomials do not?
\end{center}
\pause
%say:raising op for HL so useful, redo theory of Mac's using this
%but that's not new, so here are some new things we can do with this


\vspace{-3mm}
\vspace{-1mm}
{\small \begin{align*}
  \Htild^{(s)}_{\mu} := \omega \sigmabold  \left( (z_1\cdots z_n)^s \,
\frac{
 \displaystyle\prod_{\alpha_{ij} \in R_\mu \setminus \widehat{R}_\mu }
 \raisebox{-1.4mm}{$\big(1- q^{\arm(b_i)+1} t^{-\leg(b_i)} z_i/z_j \big)$}
\displaystyle\prod_{\alpha \in \widehat{R}_{\mu}}
 \raisebox{-1.4mm}{$\big(1-q  t \zz^\alpha \big)$} } {\prod_{\alpha \in R_+} \big(1-q  \zz^\alpha\big)
\prod_{\alpha \in R_\mu} \big(1-t  \zz^\alpha\big)}\right)
\end{align*}}


\vspace{-1.4mm}
\begin{conjecture}[Blasiak-Haiman-Morse-Pun-S.]
For any partition  $\mu$ and positive integer $s$, the symmetric function
 $\Htild^{(s)}_{\mu}$ is Schur positive.
That is, the coefficients in
\vspace{-1mm}
\begin{align*}
\Htild^{(s)}_{\mu}
= \sum_{\nu } K^{(s)} _{\nu , \mu}(q,t)\, s_\nu(X)  \\[-10mm]
\end{align*}
satisfy $K^{(\bbb)}_{\nu , \mu}(q,t)\in \N[q,t]$.
%with non-negative integer coefficients.
\end{conjecture}
\end{frame}
\begin{frame}{Thank you!}
  \begin{bibdiv}
  \begin{biblist}

  \bib{blasiakLLT21}{article}{
  title = {{{LLT}} Polynomials in the {{Schiffmann}} Algebra},
  author = {Blasiak, Jonah},
  author = {Haiman, Mark},
  author = {Morse, Jennifer},
  author = {Pun, Anna},
  author = {Seelinger, George H.},
  year = {2021},
  month = {dec},
  journal = {arXiv e-prints},
  primaryclass = {math.CO},
  pages = {arXiv:2112.07063},
  adsnote = {Provided by the SAO/NASA Astrophysics Data System},
  adsurl = {https://ui.adsabs.harvard.edu/abs/2021arXiv211207063B},
  archiveprefix = {arxiv},
  eid = {arXiv:2112.07063},
}
\bib{BlasiakRaising23}{article}{
  title = {A Raising Operator Formula for {{Macdonald}} Polynomials},
  author = {Blasiak, Jonah},
  author = {Haiman, Mark},
  author = {Morse, Jennifer},
  author = {Pun, Anna},
  author = {Seelinger, George H.},
  year = {2023},
  month = {jul},
  pages = {arXiv:2307.06517},
  journal = {arXiv e-prints},
  primaryclass = {math},
  publisher = {{arXiv}},
}
    \bib{haglundCombinatorial05}{article}{
  title = {A Combinatorial Formula for {{Macdonald}} Polynomials},
  author = {Haglund, J.}
  author = {Haiman, M.}
  author = {Loehr, N.},
  year = {2005},
  volume = {18},
  number = {3},
  pages = {735--761 (electronic)},
  issn = {0894-0347},
}
\bib{MR1839919}{article}{
   author={Haiman, Mark},
   title={Hilbert schemes, polygraphs and the Macdonald positivity
   conjecture},
   journal={J. Amer. Math. Soc.},
   volume={14},
   date={2001},
   number={4},
   pages={941--1006},
   issn={0894-0347},
   review={\MR{1839919}},
   doi={10.1090/S0894-0347-01-00373-3},
}
\bib{MR1399754}{article}{
   author={Lascoux, Alain},
   author={Leclerc, Bernard},
   author={Thibon, Jean-Yves},
   title={Ribbon tableaux, Hall-Littlewood functions and unipotent
   varieties},
   journal={S\'{e}m. Lothar. Combin.},
   volume={34},
   date={1995},
   pages={Art. B34g, approx. 23},
   review={\MR{1399754}},
}
    \bib{shimozonoGraded00}{article}{
  title = {Graded {{Characters}} of {{Modules Supported}} in the {{Closure}} of a {{Nilpotent Conjugacy Class}}},
  author = {Shimozono, Mark}
  author = {Weyman, Jerzy},
  year = {2000},
  month = {feb},
  journal = {European Journal of Combinatorics},
  volume = {21},
  number = {2},
  pages = {257--288},
  issn = {0195-6698},
  doi = {10.1006/eujc.1999.0344},
}
\bib{weyman}{article}{
  title = {The Equations of Conjugacy Classes of Nilpotent Matrices},
  author = {Weyman, J.},
  year = {1989},
  month = {jun},
  journal = {Inventiones mathematicae},
  volume = {98},
  number = {2},
  pages = {229--245},
  issn = {1432-1297},
  doi = {10.1007/BF01388851},
  urldate = {2023-07-14},
  langid = {english},
  keywords = {Conjugacy Class,Nilpotent Matrice},
  file = {/Users/ghseeli/Dropbox (University of Michigan)/pdfs/weymanEquationsConjugacyClasses1989.pdf},
}
  \end{biblist}
  \end{bibdiv}
  \end{frame}
\end{document}
%%% Local Variables:
%%% mode: latex
%%% TeX-master: t
%%% End:
